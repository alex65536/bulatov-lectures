\makeatletter
\def\input@path{{../../}}
\makeatother
\documentclass[../../main.tex]{subfiles}

\graphicspath{
	{../../img/}
	{../img/}
	{img/}
}

\begin{document}
	
	\[du = dx - dy + 2\left(-\dfrac{x}{z} dx - \dfrac{y}{z}dy\right) = 
	\left(1 - \dfrac{2x}{2z}\right)dx - 2\left(2 + \dfrac{2y}{2z}\right)dy\]
	
	Рассматривая $du = 0$ и учитывая незав. $dx$ и $dy$ получаем систему:
	
	\[\begin{cases}
	\smallskip
	A_1 = 1 - \dfrac{2x}{z} = 0\\
	A_2 = 1 + \dfrac{y}{z} = 0\\
	x^2 + y^2 + z^2 = 0
	\end{cases} \implies \begin{cases}
	z = 2x\\
	y = -z = -2x\\
	4x^2 + 4x^2 + x^2 = 1
	\end{cases} \implies \begin{cases}
	\smallskip
	x = \pm \dfrac{1}{3}\\
	\smallskip
	y = \pm \dfrac{2}{3}\\
	\smallskip
	z = \pm \dfrac{2}{3}\\
	\end{cases}\]
	
	Тем самым мы нашли 2 стационарные точки условного экстремума 
	$M_1\left(\dfrac{1}{3},\; -\dfrac{2}{3},\; \dfrac{2}{3}\right);$
	
	$ M_2\left(-\dfrac{1}{3},\; \dfrac{2}{3}, \; -\dfrac{2}{3}\right)$.
	
	Для исследования их на экстремальностьвыразим $d^2u$ 
	через независимые дифференциалы $dx$ и $dy$.
	
	\[d^2u = d\left( \left(1 - \dfrac{2x}{3}\right)dx - 
	2\left(1 + \dfrac{y}{z}\right)dy\right) = 
	\left[dx, dy - f_ix\right] = -2 \cdot 
	\dfrac{zdx - xdz}{z^2}dx - 2 \cdot \dfrac{zdy - ydz}{z^2}dy\]
	
	\[\implies d^2u\left(M_{1, 2}\right) = \left[
	\begin{cases}
	y = -2x\\
	z = 2x 
	\end{cases} \implies dz = -\dfrac{x}{z}dx - \dfrac{y}{z}dy = 
	-\dfrac{1}{2}dx + dy 
	\right] =\]
	
	\[ = -2 \cdot \frac{2xdx - x\left(-\frac{1}{2}dx + dy\right)}{4x^2}dx - 
	2 \cdot \frac{2xdy + 2x \left(-\frac{1}{2}dx + dy\right)}{4x^2}dy = \]
	
	\[= -\dfrac{1}{2x}\left(2dx^2 + \dfrac{1}{2}dx^2 - dxdy + 2dy^2 - 
	dxdy + 2dy^2 \right) = -\dfrac{1}{2x} \left( \dfrac{5}
	{2} dx^2 - 2dxdy + 4dy^2 \right) \implies \]
	
	\[\implies d^2u\left(M_1\right) = \left[x_1 = \dfrac{1}{3}\right] = 
	-\dfrac{3}{2}\cdot(\dfrac{5}{2}dx^2 - 2dxdy + 4dy^2) < 0\ \; 
	\forall \left(dx, \; dy\right) \neq \vec{0} \implies 
	M_1 \text{~--- т. лок. max}\]
	
	Аналогично:
	
	\[d^2u \left(M_2\right) = \left[x_2 = -\dfrac{1}{3} \right] = 
	\dfrac{3}{2} \cdot \left( \dfrac{5}{2} dx^2 - 2dxdy + 4dy^2\right) < 0 
	\; \forall \left(dx, dy\right) \neq \vec{0} \implies M_2 
	\text{~--- т. лок. min}\]
	
	$u_{max} = u\left(\dfrac{1}{3}, \; -\dfrac{2}{3}, 
	\dfrac{2}{3}\right) = 3 \;\;\; 
	u_{min} = u\left(-\dfrac{1}{3}, \; 
	\dfrac{2}{3}, -\dfrac{2}{3}\right) = -3$ 
	
	\section{Метод множителей лагранжа} 
	
	Недостатком предыдущего метода является то, что используемые переменные 
	$x = $ $=\left(x_1, \ldots, x_n\right), \; y = \left(y_1,\ldots, y_n\right)$
	не равноправны между собой, т.к. $x_k \;\; k = \overline{1,\; n}$, 
	независимые переменные $y_j \;\; j = \overline{1,\; m}$, зависят и 
	должны быть выраженны через $x_k$ засчет уравнения связи. 
	Чтобы уравновесить эти переменные межд собой Лагранж расширил 
	число переменных и добавил к ним множество 
	$\lambda_1,\; \ldots,\; \lambda_m \;\; \forall 
	\lambda_i \in \mathbb{R}, \; i = \overline{1,\; m}$, 
	с помощью которых на рассматриваемой ФНП и уравнению 
	связи строится \emph{уравнение Лагранжа}:
	\begin{equation}
	L\left(x, \; y\right) = f\left(x, \; y\right) + \sum\limits_{k = 1}^m 
	\lambda_k F_k\left(x, \; y\right) \label{lec10.1:8}
	\end{equation}
	
	
	Покажем, что из исследования на локальный условный экстремум в исходной 
	функции равносильно исследованию функции Лагранжа 
	\eqref{lec10.1:8} на обычный локальный экстремум относительно $x, y,\lambda$
	
	Во-первых, из необходимого условия локального локального экстремума:
	\begin{equation}
	du = 0 \implies \dfrac{\partial f}{\partial x_1} dx_1 + \ldots + 
	\dfrac{\partial f}{\partial x_n} dx_n + \dfrac{\partial f}{\partial y_1}dy_1
	 + \ldots + \dfrac{\partial f}{\partial y_n} dy_n = 0 \label{lec10.1:9}
	\end{equation}
	
	Аналогично из уравнения связи поулчаем:
	\begin{equation}
	dF_x\left(x, \; y\right) = 0 \implies \dfrac{\partial 
	F_k}{\partial x_1}dx_1 + 
	\ldots + \dfrac{\partial F_k}{\partial x_n}dx_n + 
	\dfrac{\partial F_k}{\partial y_1}dy_1 + \ldots + 
	\dfrac{\partial F_k}{\partial y_m}dy_m = 0 \;\;\; x = \overline{1, \; m} 
	\label{lec10.1:10}
	\end{equation}
	
	Умножая каждое из соотношений \eqref{lec10.1:10} на соответствующий множитель 
	$\lambda_k$ и суммируя для $k = \overline{1,\;m}$ 
	вместе с \eqref{lec10.1:9}, получим
	\begin{equation}
	\left(\dfrac{\partial f}{\partial x_1} + \sum\limits_{k
	= 1}^m \lambda_k \dfrac{\partial F_k}{\partial x_1} 
	\right)dx_1 + \ldots + \left(\dfrac{\partial f}
	{\partial x_n} + \sum\limits_{k = 1}^m \lambda_k 
	\dfrac{\partial F_k}{\partial x_n} \right)dx_n + \ldots 
	+ \left(\dfrac{\partial f}{\partial y_m} + 
	\sum\limits_{k = 1}^m \lambda_k \dfrac{\partial F_k}
	{\partial y_m} \right)dy_m = 0 \label{lec10.1:11}
	\end{equation}
	
	Считая, что для связей выполняется условие существование 
	решений подберем $\lambda_k \in \R$ так, чтобы:
	\begin{equation}
	\dfrac{\partial f}{\partial y_i} + \sum\limits_{k = 1}^m \lambda_k 
	\dfrac{\partial F_k}{\partial y_i} = 0 \;\;\; i = \overline{1, \; m} 
	\label{lec10.1:12}
	\end{equation}
	
	Линейные отношения $\lambda_1, \lambda_2, \ldots, \lambda_m$ система 
	$\left( 12 \right) $ % сделай ссылкой  
	имеет единственное решение в силу того, что определитель
	
	\begin{equation}
	\label{QQQQQQQQQQQQQQQQQQQQQQQQQQQQQQQQQQkek_label_1}
	\begin{vmatrix}
	\frac{\partial F_1}{\partial y_1} & \frac{\partial F_1}{\partial y_2}
	& \cdots & \frac{\partial F_1}{\partial y_m} \\
	\frac{\partial F_2}{\partial y_1} & \frac{\partial F_2}{\partial y_2} 
	& \cdots & \frac{\partial F_2}{\partial y_m} \\
	\vdots  & \vdots  & \ddots & \vdots  \\
	\frac{\partial F_m}{\partial y_1} & \frac{\partial F_m}{\partial y_2}
	& \cdots & \frac{\partial F_m}{\partial y_m}
	\end{vmatrix} \ne 0 
	\end{equation}
	\smallskip
	
	~--- якобиан соответсвующих связей, который в силу теоремы о существовании
	и единственности решения ненулевой 
	
	
	В результате $\left( 11 \right)$ % сделай ссылкой 
	примет вид
	
	\begin{equation}
	\label{QQQQQQQQQQQQQQQQQQQQQQQQQQQQQQQQQQkek_label_2}
	\sum\limits_{j=1}^{n}\left(  \dfrac{\partial f }{\partial x_j} + 
	\sum\limits_{k=1}^{m} \lambda_k\dfrac{\partial F_k }{\partial x_j}  
	\right) \partial x_j =0 
	\end{equation}
	
	$\left( 13 \right) $ % сделай ссылкой 
	уже дифф. % что значит дифф. ?
	$d x_1,\ldots,d x_n$ являются независимыми величинами, поэтому
	
	\[\dfrac{\partial f }{\partial x_j} + \sum\limits_{k=1}^{m}
	\lambda_k \dfrac{\partial F_k }{\partial x_j} = 0,\; j = \overline{1, n}
	\]
	Присоед. к $\left( 14 \right) $ и уравнение $\left( 14 \right) $ по 
	уравн. связи получаем \smallskip систему из $\left( 12 \right) $, 
	$\left( 14 \right) $,
	$\left( 15 \right) $ из $ \left( n + 2m \right) $ уравнений:
	
	\begin{equation}
	\begin{cases}
	\dfrac{\partial f }{\partial x_j} + \sum\limits_{k=1}^{m}
	\lambda_k \dfrac{\partial F_k }{\partial x} = 0,\; j = \overline{1, n}\\
	F_k\left( x, y \right) = 0,\; k = \overline{1, m}
	\end{cases}
	\end{equation}
	
	от $ \left( 2m + n \right) $ неизвестных $x = \left( x_1, \ldots, x_n \right),
	y = \left( y_1, \ldots, y_m \right), \lambda = \left( \lambda_1,
	\ldots, \lambda_m \right) $.
	
	Покажем, что эта система соответсвует системе стационарных точек для
	функции Лангранжа $\left( 8 \right) $, где все переменные равноправны.
	
	Действительно, 
	\[dL\left( x, y, \lambda \right) = d\left( f\left( x, y\right) 
	+ \sum\limits_{k=1}^{m}\lambda_k F_k\left( x, y \right) \right) =\]
	\[
	\sum\limits_{j=1}^{n}\left( \dfrac{\partial f}{\partial x_j} 
	+ \sum\limits_{k=1}^{m}\lambda_k \dfrac{\partial F_k}{\partial x_j}  
	\right)d x_j + \sum\limits_{k=1}^{m} \left( \dfrac{\partial f}{\partial y_k} 
	+ \sum\limits_{i=1}^{m}\lambda_i \dfrac{\partial F_i}{\partial y_k}  \right)
	d 	y_k + \sum\limits_{i=1}^{m} F_k d \lambda_k = 0\]
	
	То есть $dL\left( x, y, z \right) = 0$, что соответствует уравнению 
	стационарной точки для функции Лагранжа. Найдя эту фукнцию дальнейшие
	исследования проводят обычным образом при поиске локального экстремума 
	находят $d^2L$ и избавляются от зависимых дифф. $d y_1, \ldots, d y_m $ 
	через $d x_1, \ldots, d x_n$ в силу уравнения связи, исследования КФ на 
	знакоопр. При этом в силу специфики функции Лагранжа $\left( 8 \right)$ 
	$\left( \text{Линейность по } \lambda \right) $ при вычислении $d^2 L$
	в стационарной точке величины $d \lambda_1, \ldots, d \lambda_m$ можно
	считать постоянными, что облегчает вычисления при решении практических
	примеров.
	
	\textbf{Пример}
	
	Решим предыдущий пример $u = x - 2y + 27$. В 
	соответствии с $\left( 8 \right)$ 
	имеем 
	
	$L \left( x, y, \lambda \right) = x - 2y + 27 + \lambda \left( x^2 + y^2 
	+ z^2 - 1 \right) \implies$ 
	
	\begin{equation}
	\begin{cases}
	L'_x = 1 + 2 \lambda x = 0 \\
	L'_y = -2 + 2 \lambda y = 0 \\
	L'_z = 2 + 2 \lambda z = 0 \\
	L'_\lambda = x^2 + y^2 + z^2 = 1
	\end{cases} \implies
	\begin{cases}
	x = -\dfrac{1}{2\lambda} \\
	y = \dfrac{1}{\lambda} \\
	z = -\dfrac{1}{\lambda} \\
	\dfrac{1}{4\lambda^2} + \dfrac{1}{\lambda^2} + \dfrac{1}{\lambda^2} = 1
	\end{cases} \implies
	\begin{cases}
	x_{1, 2} = \mp \dfrac{1}{3} \\
	y_{1, 2} = \pm \dfrac{2}{3} \\
	z_{1, 2} = \mp \dfrac{2}{3} 
	\end{cases}
	\end{equation}
	
	Этим стационарным точкам функции Лагранжа $M_{1, 2} \left( \mp \dfrac{1}{3}, 
	\pm \dfrac{2}{3}, \mp \dfrac{2}{3} \right) $ ~--- точки условия локального
	экстремума рассматриваемой функции. Считая, что $\lambda\; fix, 
	\lambda = \lambda_0$ для функции
	
	\[G = \left( x, y, z \right) = x - 2y + 27 + 
	\lambda_0\left( x^2 + y^2 + z^2 - 1 \right)\]
	$
	dG = \left( -1 + 2 \lambda_0 \right)dx + \left( -2 + 2\lambda_0 \right)dy 
	+ \left( 2 + 2 \lambda_0 \right) dz $.
	\smallskip
	
	Отсюда, в силу 	уравнения $d\left( x^2 + y^2 + z^2 - 1 
	\right) = 0 \implies$
	
	\[ dz = - \frac{xdx + ydy}{z}\]
	В рассмотренных стационарных точках получаем
	\[ dz = \left[
	\begin{array}{ccc}
	x & = & -\frac{1}{2\lambda} \\
	y & = & \frac{1}{\lambda} \\
	z & = & -\frac{1}{\lambda}
	\end{array}
	\right] = \frac{1}{2}dx - dy \implies d^2G = d\left(dG\right) = 
	d\left(\left(1 + 2\lambda_0x\right)d_x + \left(-2 
	+ 2\lambda_0y\right)dy\right) +\] 
	\[ + \left(2 - 2\lambda_0z\right)\left(-\frac{1}{2}dx + dy\right) \] 
	\[d^2G(M_1) = \left[\lambda_1 = \frac{3}{2}\right] > 0\]
	Если $f\left(x\right)$, где $x \in D \subset \R^n$, непрерывна на $D$,
	где $D$ ~--- компакт, то по т.Вейерштрасса получаем
	\[ \exists M = \max\limits_{x \in D} f(x) \in \R,
	m = \min\limits_{x \in D} f(x) \in \R \]
	Эти значения называются \emph{глобальными эктремумами} $f(x)$ на $D$.
	В общем случае, если $f(x)$ не является непрерывной, либо когда $D$ 
	не компакт, задача на глобальный экстремум может быть неразрешимой.
	\[ \exists M_0 = \sup\limits_{x \in D} f(x),
	m = \inf\limits_{x \in D} f(x) \]
	Если $M_0 \in \R$ и $M_0 \in E(f)$, то $M_0 = M = 
	\max\limits_{x \in D} f(x)$. \\
	Если $m_0 \in \R$ и $m_0 \in E(f)$, то $m_0 = m = 
	\min\limits_{x \in D} f(x)$. \\
	Для исследования функции на глобальный экстремум, во-первых, 
	находят критические точки, т.е. точки, 
	где либо функция разрывна, либо недифференцируема. Кроме того, с помощью 
	исследования на локальный условный экстремум(тем или иным методом)
	находят точки, подозрительный на экстремум на границе $D$.
	Если этих точек конечное число, то на практике не нужно проверять каждую из 
	этих точек на экстремум.
	Нужно только вычислить значения и выбрать наибольшее и наименьшее значения, 
	а далее сравнить, входят они в $E(f)$ или нет и 
	получить окончательный ответ.\\
	
	\begin{exmp}
		Рассмотрим функцию $D = |x - y| + |x + y| \leq 2$.
		\begin{center}
			\includegraphics[scale = 0.7]{square.jpg}
		\end{center}
		\[-x \leq y \leq x \]
		\[(x - y) + (x + y) \leq 2\]
		\[ x \leq 1 \implies y \leq 1  \implies x \geq -1\]
		Квадрат, который в этом случае является компактом.\\
		Рассмотрим задачу на локальный экстремум на этом 
		компакте для функции $u = x^2 + y^2$. 
		\begin{enumerate}
			\item[a)] внутри $D$\\
			$\begin{cases}
				u_{x}' = 2x = 0\\
				u_{y}' = 2y = 0\\
			\end{cases} \implies O(0; 0) \in D$.\\
			
			\item[б)] на $\partial D \\
			x = -1, u = y^2 + 1 \implies M_1(-1; 0)\\
			x = 1 \implies M_2(1; 0)\\
			y = 1 \implies M_3(0; 1)\\
			y = -1 \implies M_4(0; -1)$\\
			К этим точкам надо прибавить точки 
			недифференцируемости $\partial D$, т.е где к $
			\partial$ D нет касательных. 
			Этими точками будут $N_1(1; 1), N_2(1; -1), 
			N_3(-1; 1), N_4(-1; -1)$.
			Далее вычислим значения функции во всех этих 
			точках:
			$u(0; 0) = 0 - \min$\\
			$u(N_{1, 2, 3, 4}) = 2 - \max$\\
			$\exists M = \max f(x) = 2$\\
			$\exists m = \min f(x) = 0$
		\end{enumerate}
	\end{exmp}
	
	
\end{document}
