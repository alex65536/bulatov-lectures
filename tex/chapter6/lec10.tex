\makeatletter
\def\input@path{{../../}}
\makeatother
\documentclass[../../main.tex]{subfiles}

\graphicspath{
	{../../img/}
	{../img/}
	{img/}
}

\begin{document}
\section{Постановка задачи на условный экстремум}

В общем случае задача на условный экстремум рассматривается в 
следующем виде:

Требуется для Ф$(n+m)$П:
$\begin{cases}
	u = f\left( x, y \right) \\
	x = \left( x_1, \ldots, x_n \right) \in D \subset \R^n \\
	y = \left( y_1, \ldots, y_m \right) \in G \subset \R^m
\end{cases}$

Найти экстремум при наличии ограничения вида:
\begin{equation}
\begin{cases} \label{lec_10.num_1}
F_k(x, y) = 0 \\
k = \overline{1, m}
\end{cases} 
\end{equation}


\begin{example}
	\;
	
	Рассмотрим Ф2П
	$\begin{cases}
	u = x^2 + y^2 \\
	x + y = 1 \\
	(x, y) \in \R^2
	\end{cases} \text{~--- ограничена}
	$
	
	Учитывая, что $y = 1 - x$, получим:
	$u(x) = x^2 + (1-x)^2$. Исследуя её на локальный экстремум: \\
	$u'(x) = 2x - 2(1-x)$ \\
	$u'(x) = 0 \implies$ стационарная точка $x = \dfrac{1}{2}$ \\
	$u''\dfrac{1}{2} = 4 > 0 \implies x = \dfrac{1}{2} = x_{min} $ \\
	Тогда $y_{min} = 1 - x_{min} = 1/2 \implies \text{точка } 
	M_0\left( \dfrac{1}{2}, \dfrac{1}{2} \right)$ ~--- точка 
	условного локального минимума функции $u$, для которой 
	$u_{min} = u\left( \dfrac{1}{2}, \dfrac{1}{2} \right) = \dfrac{1}{2}$ 
	
	Рассмотренный пример имеет следующую геометрическую интерпретацию:
	если $u = c \ge 0 - fix$, то получаем семейство 
	окружностей с центром в точке $(0, 0)$ и радиусом $R = \sqrt{c}$.
	Это семейство будем рассматривать в сочетании с ограничением:
	$x + y = 1$, что даёт соответствующую прямую.
	Среди этих окружностей требуется выбрать ту, у которой $R$ ~--- 
	либо $max$, либо $min$
	
	В соответствующей ПДСК имеем:
	
	\includegraphics{family_of_circles.png}
	
	Если окружность с $R_{min} = \dfrac{1}{ \sqrt{2} }$, 
	которая имеет с нашей прямой одну общую точку касания: 
	$\left( \dfrac{1}{2}, \dfrac{1}{2} \right)$ \\
	$u_{min} = c_{min} = R^2_{min} = \dfrac{1}{2}$
	
	Если обозначить $F = \left( F_1, \ldots, F_m \right) \in \R^m$, 
	то тогда \ref{lec_10.num_1} принимает вид:
	
	\begin{equation} \label{lec_10.num_2}
		F(x, y) = \vec{0} \in \R^m
	\end{equation}
	
	Для функции \ref{lec_10.num_1} при ограничении \ref{lec_10.num_2}
	точку $(x_0, y_0) \in \left( D \times G \right) 
	\subset \R^{n+m}$
	т.е $x_0 \in D \subset \R^n , y_0 \in G \subset \R^m$ называют точкой 
	\emph{условного локального экстремума} (УЛЭ) функции 
	\ref{lec_10.num_1} при ограничении \ref{lec_10.num_2}, если
	$\exists V\left( x_0, y_0 \right) \in D \times G $ такая, что 
	$\forall \left( x, y \right) \in V\left( x_0, y_0 \right) \implies $
	
	(? ~--- есть ли здесь перечисление ниже)
	\begin{itemize}
		\item[a)] 
		
		$F(x_0, y_0) = 0$
		
		$\left[ \begin{gathered}
		f(x,y) \ge f(x_0, y_0) -  \text{локальный } min \\
		f(x, y) \le f(x_0, y_0) - \text{локальный } max 
		\end{gathered} \right.$
	
	\end{itemize}
		
Предположим, что используемые функции $F_k(x,y), k = \overline{1, m}$ 
непрерывно дифференцируемы по своим переменным. В этом случае для 
матрицы Якоби имеем:

\begin{equation} \label{lec_10.num_3}
	rank \begin{bmatrix}
	\pderiv{F_1}{x_1} & \cdots & \pderiv{F_1}{x_n} & 
	\pderiv{F_1}{y_1} & \cdots & \pderiv{F_1}{y_m} \\
	\pderiv{F_2}{x_1} & \cdots & \pderiv{F_2}{x_n} &
	\pderiv{F_2}{y_1} & \cdots & \pderiv{F_2}{y_m} \\
	\cdots & \cdots & \cdots & \cdots & \cdots & \cdots \\
	\pderiv{F_m}{x_1} & \cdots & \pderiv{F_m}{x_m} &
	\pderiv{F_m}{y_1} & \cdots & \pderiv{F_m}{y_m}
	\end{bmatrix}
	\Bigg|_{(x_0, y_0)} = m
\end{equation} 		
		
То тогда, без ограничения общности можно считать, что минор
$r-$ого порядка, расположенный в последних столбцах, ненулевой, т.е

\begin{equation} \label{lec_10.num_4}
	I_0 = det \begin{bmatrix}
	\pderiv{F_1}{y_1} & \cdots & \pderiv{F_1}{y_m} \\
	\cdots & \cdots & \cdots \\
	\pderiv{F_m}{y_1} & \cdots & \pderiv{F_m}{y_m}
	\end{bmatrix} \ne 0
\end{equation}

По теореме об однозначной разрешимости СФУ получаем, что 
соотношение \ref{lec_10.num_2} разрешимо относительно $y = \left( 
y_1, \ldots, y_m \right) $ в соответствующей $\widetilde{V} (x_0, y_0)$.
При этом получаемые решения $y_k = y_k \left( x_1, \ldots, x_n 
\right), k = \overline{1, m}$ будут удовлетворять начальному условию
$y(x_0) = \left( y_1(x_0), \ldots, y_m(x_0) \right) = y_0$

В связи с этим для полученного решения $y = \varphi(x) = \left( 
y_1(x), \ldots, y_m(x) \right)$ после подстановки в \ref{lec_10.num_1} 
приходим к $u(x) = f\left( x, \varphi(x) \right), x \in \widetilde{V}(x_0)$, 
исследования которой на обычный локальный экстремум будут 
соответствовать исследованию \ref{lec_10.num_1} 
при ограничениях \ref{lec_10.num_2} на условный локальный экстремум.
Т.е задача на условный локальный экстремум сводится, при возможности, 
к задаче на безусловный 
локальный экстремум.
\end{example}

\section{Метод дифференциалов исследования на условный локальный экстремум}

Предполагая непрерывную дифференцируемость функций связи \ref{lec_10.num_2} 
в $V(x_0, y_0)$ и считая выполенным \ref{lec_10.num_4}, как и выше,
получаем однозначную разрешимость \ref{lec_10.num_2}, \\ т.е
$\exists y = \varphi(x) \implies F\left( x, \varphi(x) \right) \equiv 0, 
y_0 = \varphi(x_0)$

После подстановки для
\begin{equation} \label{lec_10.num_5}
	u = u\left( x, \varphi(x) \right)
\end{equation}
в силу необходимого условия обыкновенного локального экстремума 
рассматриваемой точки $(x_0, y_0)$ имеем:
\[ du(x_0, y_0) = 0 \]
Отсюда в силу инвариантности формы первого дифференциала для 
$\phi(x) = f\left( x, \varphi(x) \right)$ :

\begin{equation} \label{lec_10.num_6}
	d\phi(x_0) = \sum_{k = 1}^{n} \pderiv{ f(x_0, y_0) }{x_k} 
	dx_k + \sum_{j = 1}^{m} \pderiv{ f(x_0, y_0) }{y_j} dy_j = 0
\end{equation}

Рассматривая \ref{lec_10.num_6} как линейную систему относительно неизвестных
$d\varphi_1, \ldots, d\varphi_m$ и учитывая, что якобиан этой системы $\ne 0$,
мы можем однозначно выразить линейным образом $\forall d\varphi_j$
через независимые дифференциалы $dx_1, \ldots, dx_n$

В результате получаем, что в рассматриваемой точке $(x_0, y_0)$
являющейся стационарной для $u = u\left( x, \varphi(x) \right)$
\[ du(x_0, y_0) = 0 \]
что после подстановки зависимых $d\varphi_j$ через независимые 
$dx_1, \ldots, dx_n$, приводит к виду:

\begin{equation} \label{lec_10.num_7}
	\sum_{k = 1}^{n} A_k dx_k = 0, A_k = A_k(x_0, y_0) \in \R
\end{equation}
Из \ref{lec_10.num_7} в силу независимости используемых дифференциалов 
$\implies \forall A_k = 0, k = \overline{1, n}$

Присоединяя уравнение связи точки $(x_0, y_0)$, получаем 
соответствующую систему для стационарной точки локального 
условного экстремума вида:
\[ \begin{cases}
	F_j(x_0, y_0) = 0, \; j = \overline{1, m} \\
	A_k(x_0, y_0) = 0, \; k = \overline{1, n}
\end{cases} \]
Здесь система из $(n + m)$ уравнений относительно $(n+m)$ неизвестных.

На практике основная трудность ~--- решение этой системы.
Решив её, полученную стационарную точку $(x_0, y_0)$ исследуют
на обыкновенный локальный экстремум обычным образом.

\begin{example}
	\;
	
	Рассмотрим функцию 
	\[ \begin{cases}
	u = x - 2y + 2z \\
	F = x^2 + y^2 + z^2 -1 = 0
	\end{cases} \]
	Имеем: \\
	$du = dx - 2dy + 2dz \\
	dF = 2xdx + 2ydy + 2zdz = 0$ \\
	Считая, что $z \ne 0$ : \\
	dz = $-\dfrac{x}{2}dx - \dfrac{y}{2}dy $ \\
	Отсюда: \\
	$du = dx - 2dy + 2\left( -\dfrac{x}{2}dx - 
	\dfrac{y}{2} dy \right) = 
	\left( 1 - \dfrac{2x}{z} \right)dx - 
	2\left( 1 + \dfrac{y}{z} \right)dy $ \\
	Решая уравнение $du = 0$ и учитывая независимость $dx$ и $dy$,
	получаем систему:
	\[ \begin{cases}
	A_1 = 1 - \dfrac{2}{z} = 0 \\
	\; \\
	A_2 = 1 + \dfrac{y}{z} = 0
	\end{cases} 
	\]
	\[ \;\;\;\; x^2 + y^2 + z^2 - 1 = 0 \]
	
	Находим:
	\[ \begin{cases}
	z = 2x \\
	y = -z = -2x
	\end{cases} \]
	\[ \;\;\;\; x^2 + 4x^2 + 4x^2 = 1 \]
	$x^2 = \dfrac{1}{9} \implies x = \pm \dfrac{1}{3} \implies \\
	y = -2x = \mp \dfrac{2}{3} \\
	z = 2x = \pm \dfrac{2}{3}$ 
	\\ \; \\
	$\implies$ мы нашли две точки локального экстремума:\\
	$M_{1,2} \left( \pm \dfrac{1}{3}, \mp \dfrac{2}{3}, 
	\pm \dfrac{2}{3} \right)$
	
	Для исследования их на экстремальность, выразим $d^2 u$ через
	независимые $dx$ и $dy$
	 
\end{example}



\end{document}