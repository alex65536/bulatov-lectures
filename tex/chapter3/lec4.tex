\makeatletter
\def\input@path{{../../}}
\makeatother
\documentclass[../../main.tex]{subfiles}

\begin{document}
	\section{Частные производные и дифференциал ФНП}
	
	Рассмотрим $f(x)$, определённую на множестве $D \subset \R^n$. Для
	внутренней точки $x_0 = (x_{01}, \ldots, x_{0n}) \in D$ и $\forall 
	\Delta x = (\Delta x_1, \ldots, \Delta x_n): x_0 + \Delta x \in D$ получим
	соответствующее приращение функции 
	\begin{equation}
		\label{increment}
		\Delta f(x_0) = f\left(x_0 + \Delta x\right) - f(x_0).
	\end{equation}
	\begin{defn}
		$f(x)$ \emph{дифференцируема} в рассматриваемой точке $x_0$, если
		\begin{equation}
			\label{differenc}
			\begin{array}{l}
			\exists \ p_k \in \R, \ k = \overline{1,n}:\ \Delta f(x_0)= p_1 
			\Delta x_1 + p_2 \Delta x_2 + \ldots + p_n \Delta x_n + \alpha,\ 
    		\alpha = o(|\Delta x|) = \\ =o(\sqrt{\Delta x_1^2 + \ldots + \Delta
    		x_n^2}),\ \Delta x \to 0 \ \left(\lim\limits_{\Delta x \to 0} 
    		\frac{\alpha}{|\Delta x|} = 0\right).
    	    \end{array}
		\end{equation}
	\end{defn}
	
	Первым необходимым условием дифференцируемости является непрерывность, так
	как из $\eqref{differenc} \implies \exists \lim\limits_{\Delta x \to 0}
	\Delta f(x_0) = \lim\limits_{\Delta x \to 0} (p_1 \Delta x_1 + \ldots + 
	p_n \Delta x_n + \alpha) = 0 \implies [\Delta f(x_0) = f\left(x_0 + 
	\Delta x\right) - f(x_0)] \implies 
	f\left(x_0 + \Delta x\right) \underset{\Delta x \to 0}{\to} f(x_0)$, что 
	соответствует  непрерывности $f(x)$ в точке $x_0$.
	
	Для получения второго необходимого условия рассмотрим для фиксированного 
	$k, \ 1 \leq k \leq n$, частные приращения, то есть $\Delta x_1 = \ldots = 
	\Delta x_{k-1} = \Delta x_{k + 1} = \ldots = \Delta x_n = 0$, а $\Delta x_k
	\ne 0$. В результате получим соответствующие частные приращения по каждой 
	координате:
	\[\Delta_k f(x_0) = f(x_{01}, \ldots, x_{0k-1}, x_{0k} + 
	\Delta x_k, x_{0k+1}, \ldots, x_{0n}).\]
	
	Из $\eqref{differenc} \implies
	\Delta_k f(x_0) = p_k \Delta x_k + o(|\Delta x_k|),\ k = \overline{1, n}
	\implies \frac{\Delta_k f(x_0)}{\Delta x_k} = p_k + 
	\frac{o(|\Delta x_k|)}{\Delta x_k} = p_k + \alpha_k.$
	
	Учитывая, что $
	\alpha_k = \frac{o(|\Delta x_k|)}{\Delta x_k} = 
	\frac{o(|\Delta x_k|)}{|\Delta x_k|} \cdot \frac{|\Delta x_k|}{\Delta x_k}
	\appr{\Delta x_k \to 0}0$, получим, переходя к пределу: 
	$\lim\limits_{\Delta x \to 0} \frac{\Delta_k f(x_0)}{\Delta x_k} = p_k.$
	
	Далее полученные пределы будем называть \emph{частными производными
	первого порядка} рассматриваемой ФНП в точке $x_0$ по $k$-ой координате и
	обозначать
	\begin{equation}
		\label{partial}
		\pderiv{f(x)}{x_k} = \lim\limits_{\Delta x_k \to 0}
		\frac{\Delta_k f(x_0)}{\Delta x_k} \in \R, \ k = \overline{1, n}.
	\end{equation} 
	
	Вторым необходимым условием дифференцируемости ФНП является существование 
	всех частных производных первого порядка $\eqref{partial}$ в точке $x_0$. В
	связи с этим определение дифференцируемости можно записать так:
	\begin{equation}
		\label{defn_of_diff}
		\Delta f(x_0) = \sum_{k = 1}^{n} \pderiv{f(x_0)}{x_k} 
		\Delta x_k + o\left(\sqrt{\Delta x_1^2 + \ldots + \Delta x_n^2}\right)
	\end{equation}
	
	Применяя $\eqref{defn_of_diff}$ для $f_k(x) = f_k(x_1, \ldots x_n) = x_k,\ 
	\text{fix } 1 \leq k \leq n$, получаем
	
	\begin{equation}
		\label{deriv}
		\pderiv{f_k(x)}{x_j} = 
		\begin{cases}
		 1,& \text{если } j = k \\
		 0,& \text{если } j \ne k \\
		\end{cases}
	\end{equation}
	
	Величина 
	\begin{equation}
	\label{differential}
	df(x_0) = \sum_{k = 1}^{n} \pderiv{f_k(x)}{x_k} \Delta x_k,
	\end{equation}
	являющаяся линейной функцией от $(\Delta x_1, \ldots, \Delta x_n)$ 
	называется $\emph{дифференциалом первого порядка}$ рассматриваемой ФНП в
	точке $x_0$. Используя $\eqref{deriv}$ в $\eqref{differential}$ 
	получим:
	\[dx_k = \Delta x_k \quad \forall \ \text{fix } 1\leq k \leq n\].
	
	В связи с этим,
	для независимых переменных $x_1, \ldots, x_n$ под их дифференциалом
	$dx_k$ понимается произвольное допустимое приращение
	$\Delta x_k,\ k = \overline{1, n}$.
	
	В результате $\eqref{differential}$ перепишется как 
	\begin{equation}
		\label{other_different}
		df(x_0) = \sum_{k = 1}^{n} \pderiv{f(x_0)}{x_k}dx_k.
	\end{equation}
	В дальнейшем будем обозначать $\displaystyle\pderiv{f(x_0)}{x_k} = 
	f_{x_k}'(x_0)$.
	Поэтому
	\[\eqref{other_different} \iff 
	df(x_0) = \sum_{k = 1}^{n} f_{x_k}'(x_0) dx_k.\]
\end{document}
