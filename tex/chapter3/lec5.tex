\makeatletter
\def\input@path{{../../}}
\makeatother
\documentclass[../../main.tex]{subfiles}
\graphicspath{
	{../../img/}
	{../img/}
	{img/}
}
\begin{document}
\section{Достаточное условие дифференцируемости ФНП}

Необходимыми условиями дифференцируемости ФНП в точке являются:
\begin{itemize}
\item непрерывность этой ФНП в рассматриваемой точке
\item существование конечных частных производных в этой точке
\end{itemize}
В случае функции одной переменной существование конечной производной было
не только необходимым, но и достаточным условием дифференцируемости, но
для ФНП это не так.
\begin{exmp}
Рассмотрим Ф2П $f(x,y) = \sqrt{\abs{xy}}$. $f$ непрерывна для $\forall x,y
\in \R^2$.

Рассмотрим частные производные в точке $M_0(0,0)$.

\[
f'_x(M_0) = 
\lim_{\D x \to 0} \frac{f(0+\D x, 0) -f(0,0)}{\D x}=
\lim_{\D x \to 0} \frac{\sqrt{\abs{\D x \cdot 0}}-\sqrt{\abs{0}}}{\D x}=
\lim_{\D x \to 0} \frac{0}{\D x}=
0
\]
\[
f'_y(M_0) = 
\lim_{\D y \to 0} \frac{f(0,0+\D y) -f(0,0)}{\D y}=
\lim_{\D y \to 0} \frac{\sqrt{\abs{0 \cdot \D y}}-\sqrt{\abs{0}}}{\D y}=
\lim_{\D y \to 0} \frac{0}{\D y}=
0
\]

Теперь проверим дифференцируемость функции.

\[\Delta f(0,0) = f'_x(0,0)\D x+f'_y(0,0)\D y + o(\sqrt{\D x^2+\D y^2})
\iff\]
\[\iff f(0+\D x,0+\D y)-f(0,0)=0\D x+0\D y+o(\sqrt{\D x^2+\D y^2})\iff\]
\[\iff \exists\lim_{\substack{\D x \to 0\\ \D y \to 0}}
\frac{\sqrt{\abs{\D x\D y}}}{\sqrt{\D x^2+\D y^2}}=0\]
Находим частный предел при $\D x = \D y$:
\[\lim_{\D x \to 0}\frac{\sqrt{\D x^2}}{\sqrt{2\D x^2}} = \sqrt{\frac{1}
{2}}\]
$\sqrt{\frac{1}{2}}\neq 0$, значит, двойного предела не существует и
функция не дифференцируема в $M_0$, несмотря на существование конечных
частных производных.
\end{exmp}



\begin{thm}[достаточное условие дифференцируемости ФНП]
Пусть $f(x)$ определена в области $D \subseteq \R^n$. Если её
частные производные в некоторой окрестности $V(M_0)\subset D$ точки
$M_0$ существуют и непрерывны, то $f(x)$ дифференцируема в $M_0$.
\end{thm}
\begin{proof}
Для простоты рассмотрим Ф2П $f(x,y)$ и $M_0(a,b)\in D$.

Пусть существуют непрерывные частные производные в $M_0$. Придавая $M_0$ 
произвольное приращение $\D M(\D x,\D y)\in \R^2$ так, чтобы $M_0+\D M\in 
V(M_0)$, и считая дополнительно, что $(a+\D x,b)\in V(M_0)$, имеем:
\[ g(t)=f(t,b), h(\tau) = f(a+\D x,\tau) \]
\[ \D g(a)=g(a+\D x) -g(a) = f(a+\D x, b)-f(a,b) \]
\[ \D h(b)=h(b+\D y) -h(b) = f(a+\D x, b+\D y)-f(a+\D x,b) \]

По формуле Лагранжа конечных приращений получаем:
\begin{enumerate}
\item $ \exists \theta_1\in\left]0;1\right[\mid
\D g(a) = g'_t(a+\theta_1\D x)\D x=f'_x(a+\theta_1\D x,b)\D x $
\item $ \exists \theta_2\in\left]0;1\right[\mid
\D h(b) = h'_\tau(b+\theta_2\D y)\D y=f'_y(a+\D x, b+\theta_2\D y)\D y $
\end{enumerate}

Поэтому
\[\D f(a,b) = \D g(a) +\D h(b) = f'_x(a+\theta_1\D x,b)\D x+
f'_y(a+\D x, b+\theta_2\D y)\D y =\]
\begin{equation}
= f'_x(a,b)\D x +f'_y(a,b)\D y + \alpha,
\label{l5:diff}
\end{equation}

где $\alpha = A\D x+B\D y$, где
\[A = f'_x(a+\theta_1\D x,b)-f'_x(a,b),\]
\[B = f'_y(a+\D x, b+\theta_2\D y) - f'_y(a,b)\]

Осталось показать, что
\begin{equation}
\alpha = o(\sqrt{\D x^2 + \D y^2}) \mid \D x \to 0, \D y \to 0
\label{l5:alpha_o}
\end{equation}

Используя неравенство Коши-Буняковского, получаем:
\[\abs{\alpha} = \abs{A\D x+B\D y} \leq \sqrt{(A^2+B^2)(\D x^2+\D y^2)}\]

Также в силу непрерывности производных имеем:
\[f'_x(a+\theta_1\D x,b)\underset{\D x\to 0}\longrightarrow f'_x(a,b)
\implies A \underset{\D x\to 0}\longrightarrow 0\]
\[f'_y(a+\D x, b+\theta_2\D y) \underset{\substack{\D x\to 0 \\
\D y \to 0}}\longrightarrow f'_y(a,b)
\implies B \underset{\substack{\D x\to 0 \\\D y \to 0}}\longrightarrow 0\]

Из вышеописанного получаем, что \eqref{l5:alpha_o} верно:
\[\abs{\frac{\alpha}{\sqrt{\D x^2 + \D y^2}}} \leq
\sqrt{A^2+B^2} \underset{\substack{\D x\to 0 \\
\D y \to 0}}\longrightarrow 0\]

Из \eqref{l5:diff} и \eqref{l5:alpha_o} получаем дифференцируемость
$f(x,y)$ в $M_0(a,b)$.

\end{proof}
\end{document}
