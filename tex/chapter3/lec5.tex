\makeatletter
\def\input@path{{../../}}
\makeatother
\documentclass[../../main.tex]{subfiles}
\graphicspath{
	{../../img/}
	{../img/}
	{img/}
}
\begin{document}
\section{Достаточное условие дифференцируемости ФНП}

Необходимыми условиями дифференцируемости ФНП в точке являются:
\begin{itemize}
\item непрерывность этой ФНП в рассматриваемой точке
\item существование конечных частных производных в этой точке
\end{itemize}
В случае функции одной переменной существование конечной производной было
не только необходимым, но и достаточным условием дифференцируемости, но
для ФНП это не так.
\begin{exmp}
Рассмотрим Ф2П $f(x,y) = \sqrt{\abs{xy}}$. $f$ непрерывна для $\forall x,y
\in \R^2$.

Рассмотрим частные производные в точке $M_0(0,0)$.

\[
f'_x(M_0) = 
\lim_{\D x \to 0} \frac{f(0+\D x, 0) -f(0,0)}{\D x}=
\lim_{\D x \to 0} \frac{\sqrt{\abs{\D x \cdot 0}}-\sqrt{\abs{0}}}{\D x}=
\lim_{\D x \to 0} \frac{0}{\D x}=
0
\]
\[
f'_y(M_0) = 
\lim_{\D y \to 0} \frac{f(0,0+\D y) -f(0,0)}{\D y}=
\lim_{\D y \to 0} \frac{\sqrt{\abs{0 \cdot \D y}}-\sqrt{\abs{0}}}{\D y}=
\lim_{\D y \to 0} \frac{0}{\D y}=
0
\]

Теперь проверим дифференцируемость функции.

\[\Delta f(0,0) = f'_x(0,0)\D x+f'_y(0,0)\D y + o(\sqrt{\D x^2+\D y^2})
\iff\]
\[\iff f(0+\D x,0+\D y)-f(0,0)=0\D x+0\D y+o(\sqrt{\D x^2+\D y^2})\iff\]
\[\iff \exists\lim_{\substack{\D x \to 0\\ \D y \to 0}}
\frac{\sqrt{\abs{\D x\D y}}}{\sqrt{\D x^2+\D y^2}}=0\]
Находим частный предел при $\D x = \D y$:
\[\lim_{\D x \to 0}\frac{\sqrt{\D x^2}}{\sqrt{2\D x^2}} = \sqrt{\frac{1}
{2}}\]
$\sqrt{\frac{1}{2}}\neq 0$, значит, двойного предела не существует и
функция недифференцируема в $M_0$, несмотря на существование конечных
частных производных.
\end{exmp}

\begin{thm}[достаточное условие дифференцируемости ФНП]
Пусть $f(x)$ определена в области $D \subset \R^n$. Если её
частные производные в некоторой окрестности $V(M_0)\subset D$ точки
$M_0$ существуют и непрерывны, то $f(x)$ дифференцируема в $M_0$.
\end{thm}
\begin{proof}
Для простоты рассмотрим Ф2П $f(x,y)$ и $M_0(a,b)\in D$.

Пусть существуют непрерывные частные производные в $M_0$. Придавая $M_0$ 
произвольное приращение $\D M(\D x,\D y)\in \R^2$ так, чтобы $M_0+\D M\in 
V(M_0)$, и считая дополнительно, что $(a+\D x,b)\in V(M_0)$, имеем:
\[ g(t)=f(t,b),\ h(\tau) = f(a+\D x,\tau) \]
\[ \D g(a)=g(a+\D x) -g(a) = f(a+\D x, b)-f(a,b) \]
\[ \D h(b)=h(b+\D y) -h(b) = f(a+\D x, b+\D y)-f(a+\D x,b) \]

По формуле Лагранжа конечных приращений получаем:
\begin{enumerate}
\item $ \exists \theta_1\in\left]0;1\right[\quad
\D g(a) = g'_t(a+\theta_1\D x)\D x=f'_x(a+\theta_1\D x,b)\D x $
\item $ \exists \theta_2\in\left]0;1\right[\quad
\D h(b) = h'_\tau(b+\theta_2\D y)\D y=f'_y(a+\D x, b+\theta_2\D y)\D y $
\end{enumerate}

Поэтому
\begin{equation}
\begin{array}{l}
\D f(a,b) = \D g(a) +\D h(b) = f'_x(a+\theta_1\D x,b)\D x+
f'_y(a+\D x, b+\theta_2\D y)\D y = \\
= f'_x(a,b)\D x +f'_y(a,b)\D y + \alpha,
\end{array}
\label{l5:diff}
\end{equation}

где $\alpha = A\D x+B\D y$, где
\[A = f'_x(a+\theta_1\D x,b)-f'_x(a,b),\]
\[B = f'_y(a+\D x, b+\theta_2\D y) - f'_y(a,b)\]

Осталось показать, что
\begin{equation}
\alpha = o(\sqrt{\D x^2 + \D y^2}) \mid \D x \to 0, \D y \to 0
\label{l5:alpha_o}
\end{equation}

Используя неравенство Коши-Буняковского, получаем:
\[\abs{\alpha} = \abs{A\D x+B\D y} \leq \sqrt{(A^2+B^2)(\D x^2+\D y^2)}\]

Также в силу непрерывности производных имеем:
\[f'_x(a+\theta_1\D x,b)\appr{\substack{\D x\to 0 \\
\D y \to 0}} f'_x(a,b)
\implies A \appr{\substack{\D x\to 0 \\
\D y \to 0}} 0\]
\[f'_y(a+\D x, b+\theta_2\D y) \appr{\substack{\D x\to 0 \\
\D y \to 0}} f'_y(a,b)
\implies B \appr{\substack{\D x\to 0 \\\D y \to 0}} 0\]

Из вышеописанного получаем, что \eqref{l5:alpha_o} верно:
\[\abs{\frac{\alpha}{\sqrt{\D x^2 + \D y^2}}} \leq
\sqrt{A^2+B^2} \appr{\substack{\D x\to 0 \\
\D y \to 0}} 0\]

Из \eqref{l5:diff} и \eqref{l5:alpha_o} получаем дифференцируемость
$f(x,y)$ в $M_0(a,b)$.
\end{proof}

\section{Дифференцирование сложных ФНП. Инвариантность формы.}
\begin{defn}
Функция $f(x)$, определённая на некотором множестве $D\subset\R^n$,
называется \emph{непрерывно дифференцируемой} на этом множестве, если в каждой 
точке $x\in D$ эта функция имеет непрерывные частные производные.
\end{defn}
\begin{thm}[о дифференцировании сложных ФНП]
Пусть $g_k(t),k=\overline{1,n}$~--- функции, определённые на 
$D\subset\R^m$ и непрерывно дифференцируемые в некоторой окрестности 
точки $t_0\in D$.

Пусть также $f(x)$~--- функция, определённая на $E\subset\R^n$ и
непрерывно дифференцируемая в соответствующей окрестности точки $x_0 = 
(g_1(t_0),g_2(t_0),\dots,g_n(t_0))\in E$.

Тогда, если существует их композиция $h(t) = f(g(t)) =
f(g_1(t),g_2(t),\dots,g_n(t))$, то она также является непрерывно 
дифференцируемой в окрестности $t_0$, причём
\[\pderiv{h(t_0)}{t_j}=\sum_{i=1}^{n}
\pderiv{f(x_0)}{x_i}\cdot
\pderiv{g_i(t_0)}{t_j},\ j=\overline{1,m}\]
\end{thm}

\begin{proof}
В силу непрерывной дифференцируемости $f(x)$ в $x_0$ на соответствущем 
приращении $\D x \in \R^n$ имеем
\begin{equation}
\D f(x_0) = \sum_{j=1}^{n} \pderiv{f(x_0)}{x_j}dx_j+\alpha
\mid \alpha=o(\abs{\D x})
\label{l5:df}
\end{equation}

В силу непрерывной дифференцируемости $g_k(x)$ в $t_0$ на соответствущем
приращении $\D t \in \R^m$ имеем
\begin{equation}
\D g_k(t_0) = \sum_{j=1}^{m} \pderiv{g_k(t_0)}{t_j}\D t_j+\beta_k
\mid \beta_k=o(\abs{\D t})
\label{l5:dg}
\end{equation}

Для $h(t)=f(g(t))$ в $t_0 \in D \subset \R^m$ имеем
\[\D h(t_0) = h(t_0+\D t)-h(t_0)=f(g(t_0+\D t))-f(g(t_0))=\]
\[=f\big((g(t_0+\D t)-g(t_0))+g(t_0)\big)-f(g(t_0))\]

В силу непрерывности $g_k(t)$ имеем, что $g(t_0+\D t) -g(t_0)=
\D g(t_0)=\D x \appr{\D t \to 0} 0$, тогда
\[\D h(t_0) = f(\D x + x_0) - f(x_0) = \D f(x_0)\]

В силу \eqref{l5:df} имеем
\[\D h(t_0) = \sum_{j=1}^{n} \pderiv{f(x_0)}{x_j}dx_j+\alpha\]

В силу \eqref{l5:dg} и того, что $dx_j = \D g_j(t_0)$, имеем 

\[\D h(t_0) = \sum_{j=1}^{n} \pderiv{f(x_0)}{x_j}\left(
\sum_{k=1}^{m} \pderiv{g_j(t_0)}{t_k}\D t_k+\beta_j
\right)+\alpha=\]
\[=\sum_{k=1}^{m}\sum_{j=1}^{n} \pderiv{f(x_0)}{x_j}\cdot
\pderiv{g_j(t_0)}{t_k}\D t_k+\sum_{j=1}^{n}\pderiv{f(x_0)}{x_j}\beta_j
+\alpha=\]
\[=\sum_{k=1}^{m}\pderiv{h(t_0)}{t_k}\D t_k+
\sum_{j=1}^{n}\pderiv{f(x_0)}{x_j}\beta_j+\alpha\]

Пусть $\gamma=\sum\limits_{j=1}^{n}\pderiv{f(x_0)}{x_j}\beta_j+\alpha$.
Тогда:
\begin{equation}
\D h(t_0)=\sum_{k=1}^{m}\pderiv{h(t_0)}{t_k}\D t_k+\gamma
\label{l5:dh}
\end{equation}
Для доказательства теоремы осталось лишь показать, что
$\gamma = o(\abs{\D t})$.
\begin{itemize}
\item
	Во-первых, имеем $\alpha = o(\abs{\D x})$. Во-вторых:
	\[\abs{\D x_k} = \abs{\D g_k(t_0)} = \abs{\sum_{j=1}^{m}
	\pderiv{g_k(t_0)}{t_j}\D t_j+o(\abs{\D t})}\leq
	\sum_{j=1}^{m}
	\abs{\pderiv{g_k(t_0)}{t_j}}\abs{\D t_j}+\abs{o(\abs{\D t})}\leq\]
	\[\leq\text{[нер-во Коши-Буняковского]}\leq
	\sqrt{\left(\sum_{j=1}^{m}\left(
	\pderiv{g_k(t_0)}{t_j}\right)^2\right)\cdot
	\left(\sum_{j=1}^{m}\D t_j^2\right)}=\]
	\[=\left[c=\sum_{j=1}^{m}\left(
	\pderiv{g_k(t_0)}{t_j}\right)^2=const\right]=\sqrt c\abs{\D t}+
	\abs{o(\abs{\D t})}=\]
	\[=\abs{\D t}\cdot\left(\sqrt c + \abs{\frac{o(\abs{\D t})}
	{\abs{\D t}}}\right)=\abs{\D t}\cdot O(1)\]
	
	Мы получили, что при $\D t \neq 0 \quad \abs{\frac{\D x_k}{\D t}}
	\leq O(1)$, следовательно, $\frac{\D x_k}{\abs{\D t}}=O(1)$ для
	всех $k=\overline{1,n}$. Тогда также выполняется:
	\[\frac{\abs{\D x}}{\abs{\D t}}=\sqrt{\sum_{k=1}^{n}\left(
	\frac{\D x_k}{\abs{\D t}}\right)^2}=O(1)\]
	
	Отсюда получаем:
	\[\lim_{\D t \to 0}\frac{\alpha}{\abs{\D t}}=
	\lim_{\D t \to 0}\left(\frac{\alpha}{\abs{\D x}}\cdot
	\frac{\abs{\D x}}{\abs{\D t}}\right)=\lim_{\D t \to 0}0\cdot O(1) = 0\]
	
	Т.е. $\alpha = o(\abs{\D t})$.
\item
	Аналогично ограничим первую часть $\gamma$:
	\[\abs{\sum_{j=1}^{n}\pderiv{f(x_0)}{x_j}\beta_j}\leq\text{
	[нер-во Коши-Буняковского]}\leq\]
	\[\leq\sqrt{\left(\sum_{j=1}^{n}\left(\pderiv{f(x_0)}{x_j}\right)^2
	\right)\cdot\left(\sum_{j=1}^{n}\beta_j^2\right)}=
	O(1)\cdot\sqrt{\sum_{j=1}^{n}o(\abs{\D t})^2}\]
	
	Получим порядок суммы:
	\[\lim_{\D t \to 0}\frac{\abs{\sum_{j=1}^{n}\pderiv{f(x_0)}{x_j}
	\beta_j}}{\abs{\D t}}\leq
	\lim_{\D t \to 0}O(1)\cdot\sqrt{\sum_{j=1}^{n}
	\left(\frac{o(\abs{\D t})}{\abs{\D t}}\right)^2} =\]
	\[=\lim_{\D t \to 0}O(1)\cdot\sqrt{\sum_{j=1}^{n}
	0^2}=0\]
	
	Т.е. $\sum\limits_{j=1}^{n}\pderiv{f(x_0)}{x_j}\beta_j=o(\abs{\D t})$.
\end{itemize}
Из двух пунктов выше получаем $\gamma = o(\abs{\D t})$, что в сочетании
с \eqref{l5:dh} даёт доказательство теоремы.
\end{proof}

\begin{crl}[инвариантность формы 1-го дифференциала ФНП]
Для $h(t) = f(g(t))$ в случае непрерывно дифференцируемых в 
соответствующих окрестностях $f(x), g(x)$ имеем
\[dh=\sum_{j=1}^m\pderiv{h(t)}{t_j}dt_j=
\sum_{i=1}^n\pderiv{f(x)}{x_i}dx_i,\]
где $dt_j$~--- приращение независимой переменной, а $dx_i=dg_i(t)$ --
произвольное приращение $g_i(t)$.
\end{crl}
\begin{proof}
В случае независимой переменной $t$ равенство следует из определения
дифференциала:
\[dh = \sum_{j=1}^m\pderiv{h(t)}{t_j}\D t_j = 
\sum_{j=1}^m\pderiv{h(t)}{t_j}dt_j\]

В случае $x_i = g_i(t)$ имеем:
\[dh = \sum_{j=1}^m\pderiv{h(t)}{t_j}dt_j = 
\sum_{i=1}^{n}\sum_{j=1}^m
\pderiv{f(x)}{x_i}\cdot\pderiv{g_i(t)}{t_j}dt_j=\]
\[=\sum_{i=1}^{n}\left(\pderiv{f(x)}{x_i}\cdot
\sum_{j=1}^m\pderiv{g_i(t)}{t_j}dt_j\right)=\sum_{i=1}^{n}
\pderiv{f(x)}{x_i}dx_i\]
\end{proof}

\end{document}
