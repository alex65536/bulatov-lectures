\makeatletter
\def\input@path{{../../}}
\makeatother
\documentclass[../../main.tex]{subfiles}
\graphicspath{
	{../../img/}
	{../img/}
	{img/}
}

\begin{document}
	\begin{exmp}
		Рассмотрим сложную Ф3П $u=f(x+y+z,xyz)$, полученную из Ф2П $f(s,t)$ 
		подстановкой $s=x+y+z$, $t=xyz$.
		
		Считая, что $f(s,t)$~--- дифференцируемая Ф2П для рассматриваемой 
		функции $u(x,y,z)$, действуя сначала с промежуточными переменными, имеем
		\begin{gather*}
			du = df(s,t) = f'_sds + f'_tdt = \left[\begin{aligned}
				& ds = d(x + y + z) = dx + dy + dz\\
				& dt = d(xyz) = yzdx + xzdy + xydz
			\end{aligned}\right] = \\
			= f'_s\left(dx + dy + dz\right) + 
			f'_t\left(yzdx + xzdy + xydz\right) = 
			\left(f'_s + yzf'_t\right)dx + 
			\left(f'_s + xzf'_t\right)dy + 
			\left(f'_s + xyf'_t\right)dz.
		\end{gather*}
		
		Отсюда, в частности, ввиду инвариантности формулы первого 
		дифференциала, находим частные производные первого порядка
		\[\begin{cases}
			u'_x = f'_s + yzf'_t,\\
			u'_y = f'_s + xzf'_t,\\
			u'_z = f'_s + xyf'_t.
		\end{cases}\]
		При использовании этих формул подразумевается, что сначала 
		вычисляют $f'_s$, $f'_t$, а потом подставляют 
		$t = xyz$, $s = x + y + z$.
	\end{exmp}

	\section{Производные и дифференциалы высших порядков ФНП}
	
	Для простоты ограничимся дифференцируемыми Ф2П $u=f(x,y)$, $(x,y)\in 
	D\subset \R^2$.
	
	В силу дифференцируемости получаем, что
	\[\exists g(x,y) = f'_x,\ \exists h(x,y) = f'_y.\]
	Если эти функции также являются дифференцируемыми, то их производные 
	первого порядка $g'_x$, $g'_y$, $h'_x$, $h'_y$ называют \emph{производными 
	второго порядка} исходной Ф2П и обозначают 
	\[f''_{x^2} = \left(f'_x\right)'_x = g'_x = \pderiv{f'_x}{x} =
	 \pderiv{}{x}\left(\pderiv{f}{x}\right) = \pderiv{^2f}{x^2},\]
	\[f''_{xy} = \left(f'_x\right)'_y = g'_y = 
	\pderiv{}{y}\left(\pderiv{f}{x}\right) = \pderiv{^2f}{x\partial y},\]
	\[f''_{yx} = \left(f'_y\right)'_x = h'_x = 
	\pderiv{}{x}\left(\pderiv{f}{y}\right) = \pderiv{^2f}{y\partial x},\]
	\[f''_{y^2} = \left(f'_y\right)'_y = h'_y = 
	\pderiv{}{y}\left(\pderiv{f}{y}\right) = \pderiv{^2f}{y^2}.\]
	Среди этих производных второго порядка производные $f''_{xy}$ и 
	$f''_{yx}$ называются \emph{смешанными}.
	
	По аналогичной схеме вводятся дифференциалы более высоких порядков. 
	Например:
	\[f'''_{xy^2} = \left(f'_x\right)''_{y^2} =
	 \pderiv{^2}{y^2}\left(\pderiv{f}{x}\right) = 
	 \pderiv{^3f}{y^2\partial x}.\]
	Для этой смешанной производной третьего порядка 
	\emph{соответствующими смешанными производными} будут $f'''_{y^2x}$, 
	$f'''_{yxy}$.
	
	Аналогично и для функции трёх переменных.
	
	\begin{exc}
		Расписать производные третьего порядка для Ф3П $u=f(x,y,z)$.
	\end{exc}

	Далее ФНП, у которой существуют производные $n$-го порядка будем 
	называть $n$ раз дифференцируемой.
	
	\begin{exmp}
		$u = x^{\left(y^z\right)}$.
		\[u'_x = \left[\begin{aligned}
			& y - \text{fix}\\
			& z - \text{fix}
		\end{aligned}\right] = y^zx^{\left(y^z - 1\right)},\]
		\[u'_y = \left[\begin{aligned}
			& x - \text{fix}\\
			& z - \text{fix}
		\end{aligned}\right] = \left(e^{y^z\ln x}\right)'_y = 
		e^{y^z\ln x}\cdot\left(y^z\ln x\right)'_y = x^{y^z}\ln x\cdot 
		z\cdot y^{z-1},\]
		\[u'_z = \left[\begin{aligned}
			& x - \text{fix}\\
			& y - \text{fix} 
		\end{aligned}\right] = \left(e^{y^z\ln x}\right)'_z = 
		e^{y^z\ln x}\cdot\left(y^z\ln x\right)'_z = x^{y^z}y^z\ln y\ln x.\]
		
		Для смешанных производных получаем
		\begin{gather*}
			u''_{xz} = \left(u'_x\right)'_z = 
			\left(y^zx^{y^z - 1}\right)'_z = 
			y^z\ln y\cdot x^{y^z - 1} + 
			y^{z}\left(e^{(y^z - 1)\ln x}\right)'_z =\\
			= y^z\ln y\cdot x^{y^z - 1} + y^zx^{y^z - 1}\cdot y^z\ln y\ln x,
		\end{gather*}
		
		\begin{gather*}
			u''_{zx} = \left(u'_z\right)'_x =
			 \left(x^{y^z}y^z\ln y\ln x\right)'_x
			 = y^zx^{y^z - 1}\ln y\ln x + x^{y^z - 1}y^z\ln y.
		\end{gather*}
		
		Получили, что $u''_{xz}=u''_{zx}$.
	\end{exmp}

	\begin{thm}[О равенстве смешанных производных второго порядка ФНП]
		\label{secdiffeq}
		Если ФНП $u = f(x)$, $x = \left(x_1, x_2, \dots, x_n\right)\in 
		D\subset\R^n$, дифференцируема в некоторой окрестности внутренней 
		точки $x_0 = \left(x_{01}, x_{02}, \dots, x_{0n}\right)\in D$ и имеет 
		непрерывные смешанные производные $u''_{x_ix_j}(x_0)$, 
		$u''_{x_jx_i}(x_0)$,\ \ $k\ne j$,\ \ $k, j = \overline{1, n}$, то тогда, 
		в силу непрерывности этих производных в окрестности $x_0$, выполняется
		\begin{equation}
			u''_{x_ix_j} = u''_{x_jx_i}
			\label{11:diff}
		\end{equation}
	\end{thm}

	\begin{proof}
		Для простоты ограничимся Ф2П $u=f(x,y)$, где $(x,y)\in D\subset\R^2$, 
		у которой существуют непрерывные смешанные производные второго порядка 
		$u''_{xy}$ и $u''_{yx}$ в некоторой окрестности $V\left(M_0\right)
		\subset D$ внутренней точки $M_0 = (a,b)\in D$.
		
		Выбирая произвольное приращение $\left(\D x, \D y\right)\in\R^2$ 
		так, чтобы $\left(a + \D x, b + \D y\right)\in V(M_0)$, и считая, 
		что при этом $\left(a + \D x, b\right)\in V(M_0)$, 
		$\left(a, b + \D y\right)\in V(M_0)$, рассмотрим выражение
		\begin{equation}
			F=f\left(a + \D x, b + \D y\right) - f\left(a + \D x, b\right) - 
			f\left(a, b + \D y\right) + f\left(a, b\right).
			\label{12:diff}
		\end{equation}
		
		При указанных выше предположениях \eqref{12:diff} корректно определено 
		для достаточно малых $\D x$, $\D y$.
		
		Для фиксированных допустимых $\D y$ для функции $g(t)=f(t, b + \D y) - 
		f(t, b)$
		в точке $t=a$ на приращении $\D x$ имеем
		\[\D g(a)=g(a + \D x) - g(a) = f(a + \D x, b + \D y) - f(a + \D x, b) - 
		f(a, b + \D y) + f(a, b) = F.\]
		
		Далее, применяя формулу Лагранжа конечных приращений для функции 
		одной переменной, получаем, что $\exists\Theta_1\in\left]0;1\right[$ 
		такое, что
		\begin{equation}
			F = \D g(a) = g'_t\left(a + \Theta_1\D x\right)\D x = 
			\left(f'_x\left(a + \Theta_1\D x, b + \D y\right) - 
			f'_x\left(a + \Theta_1\D x, b\right)\right)\D x.
			\label{13:diff}
		\end{equation}
		
		Фиксируя дальше $\D x$ и $\Theta_1$ для Ф1П $h(\tau) = 
		f'_x\left(a+\Theta_1\D x, \tau\right)$ на соответствующем приращении 
		$\D y$ в точке $\tau = b$ получаем
		\begin{equation}
			\D h(b) = h(b + \D y) - h(b) = 
			f'_x\left(a + \Theta_1\D x, b + \D y\right) - 
			f'_x\left(a + \Theta_1\D x, b\right).
			\label{14:diff}
		\end{equation}
		
		Поэтому 
		\begin{equation}
			F\overset{\eqref{13:diff},\eqref{14:diff}}{=} \D h(b)\D x
			\label{15:diff}
		\end{equation}
		
		Применяя снова формулу Лагранжа конечных приращений для Ф1П в точке 
		$\tau = b$ на приращении $\D y$ имеем 
		$\exists \Theta_2\in\left]0;1\right[$ такое, что
		\begin{equation}
		\begin{array}{c}
            \D h(b) = h'_\tau\left(b + \Theta_2\D y\right)\D y = 
            f''_{xy}\left(a + \Theta_1\D x, b + \Theta_2\D y\right)\D 
            y\implies \\
			\implies F\overset{\eqref{15:diff}}{=} f''_{xy}\left(A\right)\D x
			\D y,
			\label{16:diff}
		\end{array}
		\end{equation}
		\begin{equation}
			A = \left(a + \Theta_1\D x, b + \Theta_2\D y\right)
			\underset{\substack{\D x\rightarrow 0\\
			\D y\rightarrow 0}}{\overset{\Theta_1, \Theta_2 = O(1)}
			{\longrightarrow}} M_0\left(a, b\right)
			\label{17:diff}
		\end{equation}
		
		Аналогично показывается, что если действовать в другом порядке, то 
		$\exists \Theta_3, \Theta_4\in\left]0;1\right[$ такие, что
		\begin{equation}
			F = f''_{yx}\left(B\right)\D y\D x
			\label{18:diff}
		\end{equation}
		\begin{equation}
			B = \left(a + \Theta_3\D x, b + \Theta_4\D y\right)
			\underset{\substack{\D x\rightarrow 0\\
			\D y\rightarrow 0}}{\overset{\Theta_3,\Theta_4 = O(1)}
			{\longrightarrow}} M_0\left(a, b\right)
			\label{19:diff}
		\end{equation}
		
		Сравнение \eqref{16:diff} и \eqref{18:diff} для $\D x\ne0$, 
		$\D y\ne0$ даёт равенство
		\[f''_{xy}\left(A\right)\D x\D y = f''_{yx}\left(B\right)\D y\D x
		\implies f''_{xy}\left(A\right) = f''_{yx}\left(B\right).\]
		
		Отсюда, используя непрерывность рассматриваемых смешанных 
		производных и в силу \eqref{17:diff}, \eqref{19:diff}, получим
		\[\lim_{
			\substack{\D x\rightarrow0\\
					\D y\rightarrow 0}
				}f''_{xy}\left(A\right)\overset{\eqref{17:diff}}{=}
		f''_{xy}\left(M_0\right)\overset{\eqref{19:diff}}{=}
		\lim_{
			\substack{\D x\rightarrow0\\
					\D y\rightarrow0}
				}f''_{yx}\left(B\right).\]
	\end{proof}

	\begin{remark}
		Аналогичная теорема имеет место в случае соответствующих 
		произвольных смешанных производных для достаточного числа 
		раз непрерывно дифференцируемой ФНП, т.е. если для 
		$u = f\left(x_1, x_2,\dots, x_n\right)$ во внутренней 
		точке $x_0 = \left(x_{01}, x_{02},\dots, x_{0n}\right)
		\in D\subset \R^n$ существуют соответствующие смешанные 
		производные $\pderiv{^{m+k}u\left(x_0\right)}{x_j^k\partial x_i^m}$ и 
		$\pderiv{^{m+k}u\left(x_0\right)}{x_i^m\partial x_j^k}$, $i\ne j$, 
		то в соответствующей окрестности $V\left(x_0\right)\subset D$ эти 
		смешанные производные будут равны между собой. Аналогично и для более 
		общего случая всех смешанных производных. Например, для трижды 
		непрерывно дифференцируемой Ф3П $u=f\left(x, y, z\right)$ в точке 
		$M_0\left(x_0, y_0, z_0\right)$ имеем
		$f''_{x^2y}\left(M_0\right) = f''_{yx^2}\left(M_0\right) = 
		f''_{xyx}\left(M_0\right)$ и т.д.
	\end{remark}

	Чтобы определить дифференциалы высших порядков ФНП, рассмотрим для 
	дифференцируемой функции $u=f\left(x\right)$, $x\in D\subset \R^n$ 
	её первый дифференциал
	\begin{equation}
		F = df\left(x\right) = \sum_{j = 1}^nf'_{x_j}\left(x\right)dx_j,
		\label{20:diff}
	\end{equation}
	вычисленный на соответствующих $dx_j = \D x_j$, $j = \overline{1, n}$.
	
	Если в \eqref{20:diff} зафиксировать $dx_j$, $j = \overline{1, n}$, то 
	получим опять новую ФНП $F = F(x)$. Используя для независимой переменной 
	$x$ новые приращения $\delta x_j$ получим выражение
	\[\delta F(x) = \sum_{i = 1}^nF'_{x_i}\delta x_i.\]
	
	Отсюда, после использования \eqref{20:diff}, имеем
	\begin{equation}
		\delta F(x) = \sum_{i = 1}^n\left(\sum_{j = 1}^nf'_{x_j}dx_j\right)
		\delta x_i = \sum_{i, j = 1}^nf''_{x_jx_i}dx_j\delta x_i
		\label{21:diff}
	\end{equation}
	
	\eqref{21:diff} представляет собой \emph{билинейную форму} относительно 
	старых приращений $d_j$ и новых $\delta x_i$. Если в этой билинейной форме 
	в качестве новых приращений использовать старые, т.е. 
	$\forall k = \overline{1, n}$ $\delta x_k = dx_k$, то получим 
	\emph{квадратичную форму}
	\begin{equation}
		dF(x) = \delta F(x) \overset{\eqref{21:diff}}{=} 
		\sum_{i, j = 1}^nf''_{x_jx_i}dx_jdx_i
		\label{22:diff}
	\end{equation}
	относительно старых приращений. Эта квадратичная форма \eqref{22:diff} 
	называется \emph{дифференциалом второго порядка} исходной ФНП $f(x)$ 
	и обозначается
	\[d^2f(x) = dF(x) \overset{\eqref{22:diff}}{=} 
	\sum_{i, j = 1}^nf''_{x_jx_i}dx_jdx_i.\]
		
	Если у нас в рассматриваемых точках все смешанные частные производные 
	непрерывны, то, по теореме о равенстве смешанных частных производных 
	\eqref{secdiffeq}, имеем $f''_{x_ix_j} = f''_{x_jx_i}$. Поэтому у $d^2f$ 
	матрица квадратичной формы относительно $dx_1, dx_2,\dots, dx_n\in\R$ в 
	\eqref{22:diff} будет симметричной, т.е. имеем 
	\emph{симметричную квадратичную форму}.
	
	В этом случае на практике удобно использовать дифференциальный оператор
	\begin{equation}
		d = \pderiv{\left(.\right)}{x_1}dx_1 + \dots + 
		\pderiv{\left(.\right)}{x_n}dx_n,
		\label{23:diff}
	\end{equation}
	действие которого производится по формуле
	\[du = \pderiv{u}{x_1}dx_1 + \dots + \pderiv{u}{x_n}dx_n.\]
	
	Формальное использование алгебраической формулы для возведения в квадрат 
	даёт
	\begin{gather*}
		d^2 = \left(\sum_{k = 1}^n\pderiv{\left(.\right)}{x_k}dx_k\right)^2 = 
		\left(\sum_{k = 1}^n\pderiv{\left(.\right)}{x_k}dx_k\right)
		\left(\sum_{k = 1}^n\pderiv{\left(.\right)}{x_k}dx_k\right) = \\
		= \sum_{i, j = 1}^n\pderiv{\left(.\right)}{x_i}\cdot
		\pderiv{\left(.\right)}{x_j}dx_idx_j = \left[\begin{aligned}
			\pderiv{\left(.\right)}{x_i}\cdot\pderiv{\left(.\right)}{x_j} = 
			\pderiv{^2\left(.\right)}{x_i\partial x_j}
		\end{aligned}\right] = 
		\sum_{i, j = 1}^n\pderiv{^2\left(.\right)}{x_i\partial x_j}dx_idx_j.
	\end{gather*}
	
	В результате для дважды непрерывно дифференцируемой ФПН получаем
	\[d^2u = \sum_{i, j = 1}^n\pderiv{^2u}{x_i\partial x_j}dx_idx_j.\]
	
	Дифференциалы более высоких порядков определяются рекуррентным образом 
	$d^pf(x) = d\left(d^{p - 1}f(x)\right)$, $p\in\N$, при этом 
	$d^0f(x) = f(x)$. В данном случае опять для $p$ раз непрерывно 
	дифференцируемой ФНП имеем следующую операторную формулу
	\begin{equation}
		d^pf(x) = \left(\pderiv{\left(.\right)}{x_1}dx_1 + \dots + 
		\pderiv{\left(.\right)}{x_n}dx_n\right)^p f.
		\label{24:diff}
	\end{equation}
	
	Применение этой формулы на практике основано на следующем обобщении бинома 
	Ньютона
	\[\left(a_1 + a_2 + \dots + a_n\right)^p = \sum_{
		\substack{
			\left(i_1, i_2, \dots, i_n\right):\\
			i_1 + i_2 + \dots + i_n = p
	}}\frac{p!}{i_1!\cdot i_2!\cdot \dots i_n!}\cdot a_1^{i_1}a_2^{i_2}
	\dots a_n^{i_n},\]
	где суммирование проводится по всевозможным индексам $\forall i_k\ge 0$, 
	$k = \overline{1, n}$ таким, что $\sum\limits_{k = 1}^ni_k = p$.
\end{document}
