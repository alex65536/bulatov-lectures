\makeatletter
\def\input@path{{../../}}
\makeatother
\documentclass[../../main.tex]{subfiles}

\graphicspath{
  {../../img/}
  {../img/}
  {img/}
}

\begin{document}
	\begin{exmp}
		Для трижды непрерывно дифференцируемой Ф3П
		$u = \left(x, y, z\right)$ имеем:
		\begin{gather*}
			d^3 u \stackrel{\left(23\right)}{=}
			\left(\pderiv{\left(.\right)}{x} dx
			+ \pderiv{\left(.\right)}{y} dx
			+ \pderiv{\left(.\right)}{z} dx\right)^3 f
			= \\ = \left[
			\begin{aligned}
				& n = 3, p = 3 \\
				& \encircle{1}~3
				= 3 + 0 + 0
				= 0 + 3 + 0
				= 0 + 0 + 3 \\
				& \text{Коэффициент: }
				\dfrac{3!}{3!0!0!} = 1 \\
				& \encircle{2}~3
				= 2 + 1 + 0
				= 2 + 0 + 1
				= 1 + 2 + 0
				= 1 + 0 + 2
				= 0 + 1 + 2
				= 0 + 2 + 1 \\
				& \text{Коэффициент: }
				\dfrac{3!}{2!1!0!} = 3 \\
				& \encircle{3}~3
				= 1 + 1 + 1 \\
				& \text{Коэффициент: }
				\dfrac{3!}{1!1!1!} = 6 \\
			\end{aligned}
			\right] = \\ =
			\left[
				\left(a_1 + a_2 + a_3\right)^3
				= a_1^3 + a_2^3 + a_3^3
				+ 3 \left(
					a_1^2 a_2 + a_1^2 a_3 + a_1 a_2^2
					+ a_1 a_3^2 + a_2 a_3^2 + a_2^2 a_3
				\right)
				+ 6 a_1 a_2 a_3
			\right]
			= \\ = \ldots
			= f'''_{x^3} dx^3 + f'''_{y^3} dy^3 + f'''_{z^3} dz^3
			+ 3 \left(
				f'''_{x^2 y} dx^2 dy + f'''_{x^2 z} dx^2 dz
				+ f'''_{x y^2} dx dy^2 + f'''_{y z^2} dy dz^2
				\right. \\ \left.
				+ f'''_{y^2 z} dy^2 dz + f'''_{x z^2} dx dz^2
			\right)
			+ 6 f'''_{x y z} dx dy dz.
		\end{gather*}
	\end{exmp}
	
	\section{Формула Тейлора для ФНП}
	Рассмотрим ФНП $u = f\left(x\right),
	x = \left(x_1, \ldots, x_n\right) \in D \subset \R^n$,
	являющуюся $\left(m + 1\right)$ непрерывно дифференцируемой
	в некоторой окрестности $V \left(x_0\right)$ внутренней точки $x_0 \in D$.
	Рассматривая произвольные $\forall \D x \in \R^n$,
	такие что $\left(x_0 + t \D x\right) \in V\left(x_0\right),
	\forall t \in \left[0, 1\right]$,
	рассмотрим $F \left(t\right)
	= f\left(x_0 + t \D x\right), t \in \left[0, 1\right]$.
	Во-первых, имеем
	\[
	\begin{cases}
		F \left(0\right) = f \left(x_0\right), \\
		F \left(1\right) = f \left(x_0 + \D x\right); \\
	\end{cases}
	\implies
	\D F \left(0\right)
	= f \left(x_0 + \D x\right) - f \left(x_0\right)
	= \D f \left(x_0\right).
	\]
	Используя для сложной функции $F \left(t\right)$ формулу дифференцирования
	в силу инвариантности 1-го дифференциала при $\D t = 1 - 0 = 1$
	имеем
	\[
		\D F \left(0\right)
		= \D f \left(x_0\right)
		= \sum_{k = 1}^{m} \dfrac{d^k F \left(0\right)}{k!} + R_m.
	\]
	Отсюда, учитывая, что $\D t = 1$
	и $d^k F \left(0\right)
	= d^k f \left(x_0\right), k = \overline{1, m}$
	(доказывается по индукции), имеем
	\[
		\D f \left(x_0\right)
		= \sum_{k = 1}^{m} \dfrac{d^k f \left(x_0\right)}{k!} + R_m.
	\]
	Для остаточного члена $R_m$ в форме Лагранжа имеем
	\[
		R_m
		= \dfrac{d^{m+1} f \left(x_0 + \Theta \D x\right)}
		{\left(m + 1\right)!},
		\exists \Theta \in \left]0, 1\right[.
	\]
	В результате получили формулу Тейлора-Лагранжа для ФНП:
	\[
		\D f \left(x_0\right)
		= \sum_{k + 1}^{m} \dfrac{d^k f \left(x_0\right)}{k!}
		+ \dfrac{d^{m+1} f \left(x_0\right)}{\left(m + 1\right)},
		\Theta \in \left]0, 1\right[.
	\]
	Аналогичным образом из соответствующей формулы Тейлора-Пеано для Ф1П
	выводится формула Тейлора-Пеано для ФНП:
	\[
		\D f \left(x_0\right)
		= \sum_{k = 1}^{m} \dfrac{d^k f \left(x_0\right)}{k!}
		+ o \left(\abs{dx}^m\right),
		\text{ где } \abs{dx}
		= \sqrt{x_1^2 + \ldots + x_n^2}.
	\]
	В дальнейшем, эти формулы будут использоваться
	при исследовании ФНП на экстремум.
\end{document}
