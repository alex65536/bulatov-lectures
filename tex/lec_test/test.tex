\makeatletter
\def\input@path{{../../}}
\makeatother
\documentclass[../../main.tex]{subfiles}

\usepackage{verbatim}

\begin{document}
\section{Тестовый раздел}

\begin{defn}[Коши]
    $$\forall \eps > 0 \quad \exists \delta > 0 \quad \forall x \in D(f) \quad 0 < \abs{x-x_0} < \delta \implies \abs{f(x) - a} \le \eps$$
\end{defn}

\begin{thm}[Про \LaTeX]
    \LaTeX\ крут :)
\end{thm}

\begin{proof}
    По определению.
\end{proof}

\begin{crl}[\texttt{alex65536}]
    Электронный конспект стоит вести в \LaTeX'е.
\end{crl}

\begin{crl}
    \LaTeX-документы текстовые,
    поэтому можно использовать \texttt{git} для конроля версий.
\end{crl}

\begin{rem}
    Числовые множества в \LaTeX: $\R\C\Q\N\Z$.
\end{rem}

\begin{exmp}
    Пример примера :)
\end{exmp}


\begin{lem}[О единственном следствии из теоремы]
 Следствие может быть без номера (если оно одно)
\end{lem}

\begin{proof}
 Используйте \texttt{crl*}!
\end{proof}

\begin{crl*}
    Получится как-то так
\end{crl*}

\begin{exc}
    За\LaTeX'ать весь матан.
\end{exc}

\bigskip

Заметим, что строчки лучше делать короткими.
Так будет проще смотреть изменения.

\begin{exmp}[Маяковский]

    Такой текст

\begin{verbatim}
    Послушайте!
    Ведь, если звезды зажигают~--
    значит~-- это кому-нибудь нужно?
    Значит~-- кто-то хочет, чтобы они были?
    Значит~-- кто-то называет эти плевочки
                                  жемчужиной?
\end{verbatim}

    отформатируется как

    \medskip

    Послушайте!
    Ведь, если звезды зажигают~--
    значит~-- это кому-нибудь нужно?
    Значит~-- кто-то хочет, чтобы они были?
    Значит~-- кто-то называет эти плевочки
                                  жемчужиной?

\end{exmp}

\end{document}
