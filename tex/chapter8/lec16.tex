\makeatletter
\def\input@path{{../../}}
\makeatother
\documentclass[../../main.tex]{subfiles}

\graphicspath{
	{../../img/}
	{../img/}
	{img/}
}

\begin{document}
\section{Основные геометрические приложения n-кратных интегралов}

В основном будем использовать общую формулу
\begin{equation}
\label{lec16:1}
\mes H = \int\limits_Hdx = \underset{H}{\int \ldots \int}dx_1 \ldots dx_n,
\end{equation}
где $H$ -- измеримое множество из $\R^n$.

В дальнейшем ограничимся
использованием \eqref{lec16:1} для вычисления площадей квадрируемых плоских
фигур, а также площадей поверхностей и кубируемых тел в $\R^{3}$.

Для площади квадрируемой плоской фигуры $D \subset \R^{2}$ в ПДСК
$Oxy$ имеем:
\begin{equation}
\label{lec16:2}
S = \text{Пл. } D = \iint\limits_Ddxdy.
\end{equation}
В случае, когда $D \subset \R^{2}$ - криволинейная трапеция т. е. 
\begin{equation}
\label{lec16:3}
	D=\{
	(x,y)\in \R^2 \mid c(x)\leq y\leq d(x), a\leq x \leq b
	\},
\end{equation}
где $c(x)$ и $d(x)$ непрерывны на $\left[a; b\right]$
 используется вычесление 2И через повторные интегралы:
\begin{equation}
\label{lec16:4}
S = \int\limits_a^bdx\int\limits_{c(x)}^{d(x)}dy = 
\int\limits_a^b\left[y\right]^d(x)_c(x)
=\int\limits_a^b(d(x) - c(x))dx.
\end{equation}
Если заранее неизвестно, график какой функции  $c(x)$ или $d(x)$ 
расположен выше, то формула \eqref{lec16:1} принимает вид:
\begin{equation}
\label{lec16:5}
S =  \int\limits_a^b\abs{c(x) - d(x)}dx.
\end{equation}
Ранее формула использовалась для вычисления
 площади через ОИ. Т. е. использование
  указанного метода вычисления {ничего нового}
в сравнении с ОИ не даёт. На практике
 более эффективно использование 2И для
  вычисления площадей с помощью соответствующей замены переменных в 2И.


В общем случае если между квадрируемыми фигурами 
$D\subset \R^2$ и $G$ имеется диффеоморфизм:
\begin{equation*}
\begin{cases}
x = x(u, v), \\
y = y(u, v),
\end{cases}
\text{ где }  (u, v) \in G, (x, y) \in D,
\end{equation*}
где используются соответствующие ПДСК $Ouv$ и $Oxy$, 
то тогда, вычисляя якобиан:
\begin{equation*}
J(u,v) = \det
\begin{bmatrix}
x'_u & x'_v\\
y'_u & y'_v\\
\end{bmatrix}
\neq 0
\end{equation*}
имеем
\begin{equation}
\label{lec16:6}
S \overset{\ref{lec16:2}}{=}  \abs{ J(u,v)} dudv.
\end{equation}


\begin{example}
	Рассмотрим плоскую квадрируемую  фигуру
	 $D \subset \R{2}$, ограниченную линиями второго
	порядка
	\begin{equation*}
	(a_1 x + b_1 y + c_1)^2 + (a_2 x + b_2 y + c_2)^2 = R^2,
	\end{equation*}
	где
	\begin{equation*}
	\Delta =
	\begin{vmatrix}
	a_1 & b_1\\
	a_2 & b_2\\
	\end{vmatrix} =
	a_1b_2-b_1a_2\neq 0.
	\end{equation*}
	В этом случае для обратного диффеоморфизма
	\begin{equation*}
	\begin{cases}
	u = a_1 x + b_1 y + c_1, \\
	v = a_2 x + b_2 y + c_2,
	\end{cases}
	\end{equation*}
	имеем
	\begin{equation*}
	J=\det 
	\begin{bmatrix}
	u_x'&u_y'\\
	v_x'&v_y'
	\end{bmatrix}
	=
	\begin{pmatrix}
	a_1&b_1\\
	a_2&b_2
	\end{pmatrix}=\Delta\neq0
	\end{equation*}	
	тогда для исходного якобиана получаем
	\begin{equation*}
	J=\cfrac{1}{J}=\cfrac{1}{\Delta}.
	\end{equation*}
	\begin{equation*}
	S  \overset{\ref{lec16:6}}{=} \iint\limits_{u^2 + v^2 \leqslant R^2}
	\dfrac{1}{\abs{\Delta}}dudv = \dfrac{1}{\abs
		{\Delta}}\iint\limits_{u^2 + v^2 \leqslant R^2}dudv=\left[
	S_{\text{пр.}}=\pi R^2\right]=
	\cfrac{\pi R^2}{\abs{\Delta}}.
	\end{equation*}
	При использовании в  \eqref{lec16:6}  полярной замены
	\begin{equation*}
	\begin{cases}
		x=r\cos\phi,\\
		y=r\sin\phi,
	\end{cases}
	(x,y)\in G \subset \R^2,
	\end{equation*}
	учитывая, что якобиан $J=r$, для площади имеем 
	\begin{equation}
	\label{lec16:7}
	S = \iint\limits_Grdrd\phi.
	\end{equation}
\end{example}

\begin{exercise}
	Доказать, что при использовании обобщенной полярной замены
	\begin{equation*}
	\begin{cases}
	x=a\cos^\alpha\phi,\\
	y=b\sin^\alpha\phi,
	\end{cases}
	\end{equation*}
	где $
	a,b=const>0,
	\alpha>0, (x,y)\in D \subset \R^2, (rm\phi)\in G\subset \R^2$
	якобиан $J=\alpha a b r \cos^{\alpha-1}\phi\sin^{\alpha-1}\phi$ и значит для площади имеем
	\begin{equation}
	\label{lec16:8}
	S \overset{\ref{lec16:6}}{=} \alpha a b \iint\limits_G r\cos^{\alpha - 1}\phi
	\sin^{\alpha - 1}\phi dr d\phi.
	\end{equation}
\end{exercise}
	\begin{example}
	Для площади $S_{\text{элл}}$ плоской фигуры, ограниченной эллипсом
	\[
	\dfrac{x^2}{a^2} + \dfrac{y^2}{b^2} = 1
	\], 
	из \eqref{lec16:8}, при
	$ \alpha = 1 $ имеем:
	\begin{equation*}
	S_{\text{элл}} = 
	ab\iint\limits_
	{\begin{gathered}
		0\leq r\leq1\\
		0\leq \phi \leq 2\pi
		\end{gathered}
	 } rdrd \phi =
	ab \int\limits_0^{2\pi}d\phi\int\limits_0^1rdr =
	ab \left[\phi \right]^{2\pi}_0
	\left[\dfrac{r^2}{2}\right]_0^1 = \pi ab.
	\end{equation*}
	В частности, при $a = b = r > 0$ получаем известную формулу плозади круга $S_{\text{кр}} = \pi R^2$.
\end{example}

Позднее будет показано, что если в $\R^{3}$ имеется гладкая поверхность
$\text{П}\subset \R^3$, заданная явным уравнением 
\begin{equation*}
\begin{cases}
z = f(x, y),\\
(x, y) \in D \subset \R^{2}\\
D -- \text{кв.} 
\end{cases}
\end{equation*} 
где $f(x, y)$ -
непрервына и дифференцируема на квадрируемом компакте $D$, то тогда
\begin{equation}
\label{lec16:9}
S_\text{П} = \iint\limits_D\sqrt{1 + (f'_x)^2 + (f'_y)^2}dxdy.
\end{equation}
\begin{example}
	Вычислим площадь сферы $x^2+y^2+z^2=R^2$
\begin{equation*}
	z=\pm\sqrt{R^2-x^2-y^2}\Rightarrow 
	x^2+y^2\leq R^2
\end{equation*}	
Учитывая симметричность относительно $Oxy$ имеем
\begin{gather*}
S_{\text{сф}}=\iint\limits_{x^2+y^2 \leq R^2}
\sqrt{
1+\left(-\cfrac{x}{\sqrt
	{
		R^2-x^2-y^2
	}}
\right)^2
+
\left(
\cfrac{-y}
{\sqrt{R^2-x^2-y^2}}\right)^2
}dxdy
=\\
=
\iint\limits_{x^2+y^2\leq R^2}
\cfrac{Rdxdy}{\sqrt{R^2-x^2-y^2}}
=
\left[
\begin{gathered}
x=r\cos \phi\\
y=r\sin \phi\\
I=r^2
\end{gathered}
\right]
=
2\iint\limits_{G}
\cfrac{Rrdrd\phi}
{\sqrt{1-r^2}}
=
\left[
\begin{gathered}
0\leq r \leq R\\
0\leq \phi \leq 2 \pi
\end{gathered}
\right]=\\
=
2\int\limits_0^{2\pi}d\phi\int\limits_0^{R}
\cfrac{r}{\sqrt{R^2-r^2}}\,dr=
\end{gather*}
имеем несобственный интеграл, но формально действуя как с собственным
\begin{gather*}
=
2\left[\phi\right]^{2\pi}_0
\left[
-\sqrt{R^2-r^2}^R_0
\right]
=
R\cdot2\cdot2\pi(0-(-R))=4\pi R^2
\end{gather*}
\end{example}
Рассмотрим случай кубируемых тел $T\subset \R^3$. Ипользуя ПДСК $Oxyz$ в силу \ref{lec16:1}:
\begin{equation}
\label{lec16:10}
V=\text{объем }T=\iiint dxdydz.
\end{equation}
В случае когда кубируемое тело $T\subset \R^3$ представляет собой цилиндроид:
\begin{equation*}
T=\{
(x,y,z)\in\R^3 \mid 
p(x,y)\leq z \leq q(x,y) x,y\in D\subset \R^3
\}
\end{equation*}
где $D\subset \R^2$ -- квадрируемая проекция $T$ на $Oxy$, а $p(x,y)$ и $q(x,y)$ предполагается непрерывной на $D$.

Переходя к повторным интегралам получаем, что объем, что объем ограниченной цилиндрической функции:
\begin{equation*}
V_{\text{цил}}=\iint\limits_D\int\limits_{p(x,y)}^{q(x,y)}dz\,dxdy
\end{equation*}
\end{document}
