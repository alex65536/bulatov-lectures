\makeatletter
\def\input@path{{../../}}
\makeatother
\documentclass[../../main.tex]{subfiles}

\graphicspath{
	{../../img/}
	{../img/}
	{img/}
}

\begin{document}
\section{Основные геометрические приложения n-кратных интегралов}

В основном будем использовать общую формулу
\begin{equation}
\label{lec16:1}
\mes H = \int\limits_Hdx = \idotsint\limits_H dx_1 \ldots dx_n,
\end{equation}
где $H$ --- измеримое множество из $\R^n$.

В дальнейшем ограничимся
использованием \eqref{lec16:1} для вычисления площадей квадрируемых плоских
фигур, а также площадей поверхностей и кубируемых тел в $\R^{3}$.

Для площади квадрируемой плоской фигуры $D \subset \R^{2}$ в ПДСК
$Oxy$ имеем:
\begin{equation}
\label{lec16:2}
S = \text{площадь } D = \iint\limits_Ddxdy.
\end{equation}
В случае, когда $D \subset \R^{2}$~--- криволинейная трапеция, т.~е. 
\begin{equation}
\label{lec16:3}
	D=\{
	(x,y)\in \R^2 \mid c(x)\leq y\leq d(x), a\leq x \leq b
	\},
\end{equation}
где $c(x)$ и $d(x)$ непрерывны на $\left[a; b\right]$,
 используется вычисление 2И через повторные интегралы:
\begin{equation}
\label{lec16:4}
S = \int\limits_a^bdx\int\limits_{c(x)}^{d(x)}dy = 
\int\limits_a^b\left[y\right]^{d(x)}_{c(x)}dx
=\int\limits_a^b(d(x) - c(x))dx.
\end{equation}
Если заранее неизвестно, график какой из функций $c(x)$ или $d(x)$ 
расположен выше, то формула \eqref{lec16:4} принимает вид:
\begin{equation}
\label{lec16:5}
S =  \int\limits_a^b\abs{c(x) - d(x)}dx.
\end{equation}
Ранее формула использовалась для вычисления
 площади через ОИ. Т.~е. использование
  указанного метода вычисления {ничего нового}
в сравнении с ОИ не даёт. На практике
 более эффективно использование 2И для
  вычисления площадей с помощью соответствующей замены переменных в 2И.


В общем случае, если между квадрируемыми фигурами 
$D\subset \R^2$ и $G$ имеется диффеоморфизм
\begin{equation*}
\begin{cases}
x = x(u, v), \\
y = y(u, v),
\end{cases}
\text{ где }  (u, v) \in G, (x, y) \in D,
\end{equation*}
где используются соответствующие ПДСК $Ouv$ и $Oxy$, 
то тогда, вычисляя якобиан
\begin{equation*}
J(u,v) = \det
\begin{bmatrix}
x'_u & x'_v\\
y'_u & y'_v\\
\end{bmatrix}
\neq 0,
\end{equation*}
имеем
\begin{equation}
\label{lec16:6}
S \overset{\ref{lec16:2}}{=} \iint\limits_G \abs{ J(u,v)} dudv.
\end{equation}


\begin{example}
	Рассмотрим плоскую квадрируемую  фигуру
	 $D \subset \R^{2}$, ограниченную линиями второго
	порядка
	\begin{equation*}
	(a_1 x + b_1 y + c_1)^2 + (a_2 x + b_2 y + c_2)^2 = R^2,
	\end{equation*}
	где
	\begin{equation*}
	\Delta =
	\begin{vmatrix}
	a_1 & b_1\\
	a_2 & b_2\\
	\end{vmatrix} =
	a_1b_2-b_1a_2\neq 0.
	\end{equation*}
	В этом случае для обратного диффеоморфизма
	\begin{equation*}
	\begin{cases}
	u = a_1 x + b_1 y + c_1, \\
	v = a_2 x + b_2 y + c_2,
	\end{cases}
	\end{equation*}
	имеем
	\begin{equation*}
	J=\det 
	\begin{bmatrix}
	u_x'&u_y'\\
	v_x'&v_y'.
	\end{bmatrix}
	=
	\begin{vmatrix}
	a_1&b_1\\
	a_2&b_2
	\end{vmatrix}=\Delta\neq0.
	\end{equation*}	
	Тогда для исходного якобиана получаем
	\begin{equation*}
	I=\dfrac{1}{J}=\dfrac{1}{\Delta}.
	\end{equation*}
	\begin{equation*}
	S  \overset{\ref{lec16:6}}{=} \iint\limits_{u^2 + v^2 \leq R^2}
	\dfrac{1}{\abs{\Delta}}dudv = \dfrac{1}{\abs
		{\Delta}}\iint\limits_{u^2 + v^2 \leq R^2}dudv=\left[
	S_{\text{круга}}=\pi R^2\right]=
	\dfrac{\pi R^2}{\abs{\Delta}}.
	\end{equation*}
	При использовании в  \eqref{lec16:6}  полярной замены
	\begin{equation*}
	\begin{cases}
		x=r\cos\phi,\\
		y=r\sin\phi,
	\end{cases}
	(x,y)\in G \subset \R^2,
	\end{equation*}
	учитывая, что якобиан $J=r$, для площади имеем 
	\begin{equation}
	\label{lec16:7}
	S = \iint\limits_Grdrd\phi.
	\end{equation}
\end{example}

\begin{exercise}
	Доказать, что при использовании обобщенной полярной замены
	\begin{equation*}
	\begin{cases}
	x=ar\cos^\alpha\phi,\\
	y=br\sin^\alpha\phi,
	\end{cases}
	\end{equation*}
	где $
	a,b=const>0,\
	\alpha>0,\ (x,y)\in D \subset \R^2,\ (r,\phi)\in G\subset \R^2$,
	якобиан $J=\alpha a b r \cos^{\alpha-1}\phi\sin^{\alpha-1}\phi$ и, значит, для 
	площади имеем
	\begin{equation}
	\label{lec16:8}
	S \overset{\ref{lec16:6}}{=} \alpha a b \iint\limits_G r\cos^{\alpha - 1}\phi
	\sin^{\alpha - 1}\phi \: dr d\phi.
	\end{equation}
\end{exercise}
	\begin{example}
	Для площади $S_{\text{элл}}$ плоской фигуры, ограниченной эллипсом
	\begin{equation*}
	\dfrac{x^2}{a^2} + \dfrac{y^2}{b^2} = 1,
	\end{equation*}
	из \eqref{lec16:8}, при
	$ \alpha = 1 $ имеем:
	\begin{equation*}
	S_{\text{элл}} = 
	ab\iint\limits_
	{\substack{
		0\leq r\leq1\\
		0\leq \phi \leq 2\pi
	}} rdrd \phi =
	ab \int\limits_0^{2\pi}d\phi\int\limits_0^1rdr =
	ab \left[\phi \right]^{2\pi}_0
	\left[\dfrac{r^2}{2}\right]_0^1 = \pi ab.
	\end{equation*}
	В частности, при $a = b = r > 0$ получаем известную формулу плозади круга 
	$S_{\text{кр}} = \pi R^2$.
\end{example}

Позднее будет показано, что если в $\R^{3}$ имеется гладкая поверхность
$\Pi\subset \R^3$, заданная явным уравнением 
\begin{equation*}
\begin{cases}
z = f(x, y),\\
(x, y) \in D \subset \R^{2},
\end{cases}
\end{equation*} 
где $f(x, y)$
непрерывна и дифференцируема на квадрируемом компакте $D$, то тогда
\begin{equation}
\label{lec16:9}
S_\Pi = \iint\limits_D\sqrt{1 + (f'_x)^2 + (f'_y)^2}\:dxdy.
\end{equation}
\begin{example}
	Вычислим площадь сферы $x^2+y^2+z^2=R^2$.
\begin{equation*}
	z=\pm\sqrt{R^2-x^2-y^2}\implies 
	x^2+y^2\leq R^2.
\end{equation*}	
Учитывая симметричность сферы относительно $Oxy$, имеем
\begin{gather*}
S_{\text{сф}}=\iint\limits_{x^2+y^2 \leq R^2}
\sqrt{
1+\left(\dfrac{-x}{\sqrt
	{
		R^2-x^2-y^2
	}}
\right)^2
+
\left(
\dfrac{-y}
{\sqrt{R^2-x^2-y^2}}\right)^2
}dxdy
=\\
=
\iint\limits_{x^2+y^2\leq R^2}
\dfrac{Rdxdy}{\sqrt{R^2-x^2-y^2}}
=
\left[
\begin{gathered}
x=r\cos \phi\\
y=r\sin \phi\\
I=r^2
\end{gathered}
\right]
=
2\iint\limits_{G}
\dfrac{Rrdrd\phi}
{\sqrt{1-r^2}}
=
\left[
\begin{gathered}
0\leq r \leq R\\
0\leq \phi \leq 2 \pi
\end{gathered}
\right]=\\
=
2\int\limits_0^{2\pi}d\phi\int\limits_0^{R}
\dfrac{r}{\sqrt{R^2-r^2}}\,dr=
\\ =[
\text{имеем несобственный интеграл, но формально действуем как с собственным}
]= \\ =
2\left[\phi\right]^{2\pi}_0\cdot
\left[
-\sqrt{R^2-r^2}
\right]^R_0
=
R\cdot2\cdot2\pi(0-(-R))=4\pi R^2.
\end{gather*}
\end{example}
Рассмотрим случай кубируемых тел $T\subset \R^3$. Используя ПДСК $Oxyz$, в силу 
\ref{lec16:1}:
\begin{equation}
\label{lec16:10}
V=\text{объем }T=\iiint\limits_T dxdydz.
\end{equation}
В случае, когда кубируемое тело $T\subset \R^3$ представляет собой цилиндроид
\begin{equation*}
T=\{
(x,y,z)\in\R^3 \mid 
p(x,y)\leq z \leq q(x,y),\ (x,y)\in D\subset \R^2
\},
\end{equation*}
где $D\subset \R^2$~--- квадрируемая проекция $T$ на $Oxy$, а $p(x,y)$ и 
$q(x,y)$ предполагаются непрерывными на $D$.

Переходя к повторным интегралам, получаем, что объем ограниченной 
цилиндрической функции:
\begin{equation}
\label{lec16:11}
V_{\text{цил}}=\iint\limits_D\int\limits_{p(x,y)}^{q(x,y)}dz\,dxdy=
\iint\limits_D(q(x,y)-p(x,y))\:dxdy.
\end{equation}

Как и в случае площадей, если заранее не известно, какая из поверхностей 
$q(x,y)$ и $p(x,y)$ расположена выше, \ref{lec16:11} обобщается в виде
\begin{equation}
\label{lec16:12}
	V_{\text{цил}}=\iint\limits_D\abs{q(x,y)-p(x,y)}dxdy.
\end{equation}

\begin{example}
	Вычислим объем шара $x^2+ y^2 + z^2 \leq R^2$ с использованием 
	\ref{lec16:11}. Его можно рассматривать как цилиндроид
	\begin{equation*}
		p(x, y) = -\sqrt{-(x^2 + y^2) + R^2} \leq z \leq
		\sqrt{R^2 - x^2 - y^2} = q(x, y),
	\end{equation*}
	где $D: x^2 + y^2 \leq R^2$.
	
	В результате, в силу \eqref{lec16:11}, следует:
	\begin{gather*}
			V_{\text{ш}} = 2 \iint\limits_{x^2 + y^2 \leq R^2}\sqrt{R^2 - x^2 - 
			y^2}\:dxdy =
			\left[
			\begin{gathered}
				x = Rr \cos \phi\\
				y = Rr \sin \phi\\
				J = R^2 r
			\end{gathered}
		\right] =2R^3\iint\limits_Gr\sqrt{1-r^2}\:drd\phi=\\
		=
			\left[
			\begin{gathered}
				0 \leq \phi \leq 2 \pi\\
				0 \leq r \leq 1
			\end{gathered}
			\right]
			=2R^3 \int\limits_0^{2\pi}d\phi \int\limits_0^1r\sqrt{1 - r^2} \; d(r^2) =
			2R^3\left[-\dfrac{2}3(1-\phi^2)^{\frac{3}2}\right]^1_0 = 
			R^3\cdot 2\pi \cdot \dfrac{2}{3}=\frac{4}{3}\pi R^3.
	\end{gather*}
\end{example}


В общем случае, кроме вычисления объёма по формуле \ref{lec16:12} по
2И, эффективным методом является использование замены в 3И. 

Пусть имеется диффеоморфизм
\begin{equation*}
\begin{cases}
	x = x(u, v, \omega), \\
	y = y(u, v, \omega), \\ 
	z = z(u, v, \omega), 
\end{cases}
(u, v, \omega) \in G  \subset \R^{3} , \;\; (x, y, z) \in T \subset \R^{3},
\end{equation*}
где кубируемые тела $G$ и $T$ в $\R^{3}$ рассматриваются в соответствующих ПДСК
$Ouv\omega$ и $Oxyz$. Если используемые функции непрерывно дифференцируемы и 
для якобиана этого диффеоморфизма выполняется
\begin{equation*}
J(u, v, \omega) = \det
\begin{bmatrix}
	x'_u&x'_v&x'_\omega\\
	y'_u&y'_v&y'_\omega\\
	z'_u&z'_v&z'_\omega
\end{bmatrix} \implies 
J(u,v,w)\neq
0 \forall(u,v,\omega)\in G \subset \R^3.
\end{equation*}
то тогда, в силу формулы замены переменных в 3И, имеем:
\begin{equation}
\label{lec16:13}
V = \text{объём} T = \iiint\limits_tdxdydz=\iiint\limits_G \abs{J(u, v, 
\omega)}dudv d\omega.
\end{equation}

\begin{example}
Рассмотрим параллелепипед, ограниченный поверхностями :
\begin{equation*}
		-h_k \leq a_kx + b_ky + c_kz + d_k \leq h_k, k =\overline{ 1, 3}.
\end{equation*}
Используя обратный диффеоморфизм
\begin{equation*}
	\begin{cases}
		&u = a_1x + b_1y + c_1z + d_1\\
		&v = a_2x + b_2y + c_2z + d_2\\
		&\omega = a_3x + b_3y + c_3z + d_3.
	\end{cases}, 
\end{equation*}
и для его якобиана
\begin{equation*} 
	\mathfrak{J} = \det
	\begin{bmatrix}
		u'_x&u'_y&u'_z\\
		v'_x&v'_y&v'_z\\
		\omega'_x&\omega'_y&\omega'_z
	\end{bmatrix}=
\begin{vmatrix}
	u'_x&u'_y&u'_z\\
	v'_x&v'_y&v'_z\\
	\omega'_x&\omega'_y&\omega'_z
\end{vmatrix}
	 = \Delta.
\end{equation*}
Если $\Delta\neq0$, то
 $J = \dfrac{1}{\mathfrak{J} } = \dfrac{1}{\Delta} \neq 0$. В связи с этим, на 
 основании
\eqref{lec16:13} получаем, что объем ограниченного параллелепипеда:
\begin{equation*}
	V_{\text{парол.}} = \iiint\limits_{
		\substack{
			-h_1 \leq u \leq h_1\\
			-h_2 \leq v \leq h_2\\
			-h_3 \leq \omega \leq h_3\\
	}} \dfrac{du \; dv \; d\omega}{\abs{\Delta}} =
	\dfrac{1}{\abs{\Delta}}\int\limits_{-h_1}^{h_1}du\int\limits_{-h_2}^{h_2}dv
	\int\limits_{-h_3}^{h_3}d\omega =
	\dfrac{1}\Delta (2h_1)(2h_2)(2h_3) =\dfrac{8h_1h_2h_3}{\abs{\Delta}}.
\end{equation*}
\end{example}

При использовании сферической замены
\begin{equation*}
	\begin{cases}
		x = r \cos \phi \cos \psi,\\
		y = r \sin \phi \cos \psi,\\
		z = r \sin \psi,\\
		I = r\cos \psi, 
	\end{cases}
	\begin{gathered}
		(x, y, z) \in T, \\
		(r, \phi, \psi) \in G,\\
		r\geq 0,\\
		\phi\in \left[0,2\pi\right],\psi\in\left[-\dfrac{\pi}2,\dfrac{\pi}2\right].
	\end{gathered}
\end{equation*}
то в этом случае \ref{lec16:13} имеет вид:
\begin{equation}
\label{lec16:14}
V = \text{объём} T = \iiint\limits_Gr^2\cos\psi dr d\phi d\psi.
\end{equation}
\begin{exercise}
	Найти якобиан общей сферической замены 
	\begin{equation*}
		\begin{cases}
			x = ar \cos^{\alpha}\phi \cos^{\beta}\psi,\\
			y = br \sin^{\alpha}\phi \cos^{\beta}\psi,\\
			z = cr \sin^{\beta}\psi,
		\end{cases}
		a, b, c =const>0, \;\; \alpha,\beta=const \in \R{}.
	\end{equation*}
и формулу для объема.
\end{exercise}

\section{Механические приложения 2И и 3И}
Механические приложения 2И и 3И используются при вычислении масс плоских фигур 
и пространственных тел, а также нахождении центров
тяжести и соответствующих моментов инерции материальных плоских фигур из 
$\R^2$ и тел из $\R^3$.

Рассмотрим квадрируемый плоский материальный компакт
$D \subset \R{2}$, у которого в ПДСК в любой точке $M(x, y) \in D$ известна
плотность $\rho = \rho(M) = \rho(x,y)$, являющаяся непрерывной Ф2П на $D$. 
Тогда
масса $m_0$ рассматриваемой материальной плоской фигуры $D \subset \R^{2}$
будет вычисляться по формуле
\begin{equation}
\label{lec16:15}
m_0 = \iint\limits_D \rho(x, y) \; dxdy,
\end{equation}
а координаты центра тяжести $c(x_0, y_0)$ (центра масс) фигуры $D$ по формулам
\begin{equation}
\label{lec16:16}
\begin{cases}
	&x_0 = \frac{1}{m_0}\iint\limits_D\rho x dx dy,\\
	&y_0 = \frac{1}{m_0}\iint\limits_D\rho y dx dy.
	\end{cases}
\end{equation}
В случае, когда фигура однородна $\rho=const>0$,
 формулы \ref{lec16:17} упрощается:
 \begin{equation*}
 \begin{cases}
 	&x_0 = \frac{1}{S_0}\iint\limits_D x dx dy,\\
 	&y_0 =\frac{1}{S_0}\iint\limits_D y dx dy.
 	\end{cases}
 \text{, где} S_0=\iint\limits_D dxdy=\text{Пл}D.
 \end{equation*}
В общем случае для моментов инерции $I_x$ и $I_y$ относительно соответствующих 
координатных
осей $Ox$ и $Oy$ имеем
\begin{equation}
\label{lec16:17}
\begin{cases}
	&I_x = \iint\limits_Dy^2\rho dx dy,\\
	&I_y = \iint\limits_Dx^2\rho dx dy.\\
	\end{cases}
\end{equation}
Рассматривая также центробежный момент,
\begin{equation}
\label{lec16:18}
I_{xy} = \iint\limits_Dxy\rho dx dy.
\end{equation}
Аналогичная формула и в случае материального пространственного тела
$T \subset \R^{3}$, $T$ --- кубируемый компакт в $\R^{3}$.
Если в ПДСК $Oxyz$ $\forall M(x, y, z) \in T$, определена плотность $\rho (M) =
\rho(x, y, z)$ --- непрерывная Ф3П на $T$, то тогда для массы
\begin{equation}
\label{lec16:19}
m_0 = \iiint\limits_T \rho(x, y, z) dx dy dz,
\end{equation}
и для координат центра тяжести $c(x_0, y_0, z_0)$ получаем
\begin{equation}
\label{lec16:20}
\begin{cases}
&x_0 = \frac{1}{m_0}\iiint\limits_Tx \rho dx dy dz,\\
&y_0 = \frac{1}{m_0}\iiint\limits_Ty \rho dx dy dz,\\
&z_0 = \frac{1}{m_0}\iiint\limits_Tz \rho dx dy dz.\\
\end{cases}
\end{equation}
В случае, когда тело однородное, т.~е. $\rho=const>0$
\begin{equation*}
	\ref{lec16:20}\iff
	\begin{cases}
		&x_0 = \frac{1}{V_0}\iiint\limits_Tx  dx dy dz,\\
		&y_0 = \frac{1}{V_0}\iiint\limits_Ty  dx dy dz,\\
		&z_0 = \frac{1}{V_0}\iiint\limits_Tz  dx dy dz.\\
	\end{cases}
\text{ где } V_0=\text{объем }T=\iiint\limits_T dxdydz.
\end{equation*}
Моменты инерции рассмотренного тела $T \subset \R{3}$ относительно 
соответствующих
координатных плоскостей $Oxy$, $Oyz$, $Oxz$:
\begin{equation}
\label{lec16:21}
\begin{cases}
&I_{xy} = I_{yx} = \iiint\limits_Tz^2 \rho dx dy dz,\\
&I_{yz} = I_{zy} = \iiint\limits_Tx^2 \rho dx dy dz,\\
&I_{zx} = I_{xz} = \iiint\limits_Ty^2 \rho dx dy dz.\\
\end{cases}
\end{equation}
Для моментов инерции $I_x$, $I_y$, $I_z$ относительно координатных осей
$Ox$, $Oy$, $Oz$ соответственно получаем:
\begin{equation}
\label{lec16:22}
\begin{cases}
&I_x = I_{xy} + I_{xz} = \iiint\limits_T (y^2 + z^2) \rho dx dy dz,\\
&I_y = I_{yx} + I_{yz} = \iiint\limits_T (x^2 + z^2) \rho dx dy dz,\\
&I_z = I_{xz} + I_{yz} = \iiint\limits_T (x^2 + y^2) \rho dx dy dz.\\
\end{cases}
\end{equation}
\section{Интеграл Эйлера-Пуассона}
\end{document}
