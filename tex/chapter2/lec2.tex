\makeatletter
\def\input@path{{../../}}
\makeatother
\documentclass[../../main.tex]{subfiles}

\begin{document}
\section{Предел последовательностей и функций в $\R^n$.}

	\emph{Последовательностью} в $\R^n$ будем называть произвольное отображение
	\[
		f : \N \to \R^n
	\]
  	ставящее в соответствие для $\forall k \in \N$ единственное значение $M_k = f(k) \in \R^n.$ 
  	Эти n-мерные последовательности будем также называть \emph{точечными последовательностями} в $\R^n$ и обозначать $(M_k), k \in \N$.
  	Используя координатную запись $\forall (M_k):$
  	\[
  		M_k = (x_{k1}, x_{k2}, \dots, x_{kn}),
 	\] 
 	где $\forall x_{kj} \in \R, j = \overline{1, n}.$  
	То есть задание n-мерных последовательностей равносильно заданию $n$ числовых последовательностей $(x_{kj}) k \in \N, j = \overline{1,n}$.
	Последовательность $(M_k), k \in \N$ называется сходящейся к точке $M_0 \in \R^n$ в пространстве $(\R^n, \rho)$, если
	\[
		\lim_{k \to \infty}{\rho(M_k, M_0)} = 0
	\]
	т.е. 
	\begin{equation}
	\label{eq1}
		\forall \varepsilon > 0\ \exists\ \nu(\varepsilon) \in \R\ |\ \forall k > \nu(\varepsilon) \Rightarrow \rho(M_k, M_0) \leqslant \varepsilon 
	\end{equation}
	На языке окрестностей имеем:
	\[
		\eqref{eq1} \Leftrightarrow \forall \varepsilon > 0\ \exists\ \nu \in \R\ |\ \forall k > \nu  \Rightarrow M_k \in \bar B_\varepsilon(M_0).
	\]
	По аналогии с M-леммой для сходимости числовых последовательностей доказывается
	\begin{lem}[N-лемма]
		сходимости n-мерных последовательностей:
		
		Если $\exists N = const \geqslant 0:$
		\[
			\forall \varepsilon > 0\ \exists\ \nu_\varepsilon \in \R\ |\ \forall k > \nu_\varepsilon \Rightarrow \rho(M_k, M_0) \leqslant N \varepsilon,
		\]
		то $M_k \underset{k\rightarrow\infty}{\longrightarrow}  M_0.$
	\end{lem}

	Как и для числовой последовательности, для \eqref{eq1} точку $M_0$ будем называть пределом рассматриваемой последоваетельности $(M_k)$ и кроме записи $M_k \underset{k\rightarrow\infty}{\longrightarrow}  M_0$ будем использовать $ \lim_{k \to \infty}{M_k} = M_0$
	\begin{thm}[Критерий сходимости n-мерных последовательностей]
		Последовательность $M_k = (x_{k1}, \dots, x_{kn}) \in \R^n, k \in \N$ сходится в линейном метрическом пространстве $(\R^n, d)$ к точке $M_0 = (x_{01}, \dots, x_{0n}) \in \R^n \Leftrightarrow \forall\ fixed\ j = \overline{1, n} \Rightarrow$
		\begin{equation}
		\label{krit}
			x_{kj}\underset{k\rightarrow\infty}{\longrightarrow} x_{0j}.
		\end{equation}
	\end{thm}
	\begin{proof}
		
		\nec 
		Пусть $M_k \underset{k\rightarrow\infty}{\longrightarrow}  M_0$.
		
		Тогда, используя евклидово расстояние 
		\[
			d(M_k, M_0) = \sqrt{\sum_{j = 0}^{n}{(x_{ki} - x_{0j})^2}}
		\]
		получаем, что
		\[
			d(M_k, M_0) \underset{k\rightarrow\infty}{\longrightarrow} = 0 \Rightarrow
			\forall \varepsilon > 0\ \exists \nu \in \R\ |\ \Rightarrow d(M_k, M_0) \leqslant \varepsilon.
		\]
		Отсюда, учитывая, что для произвольного фиксированного $j = \overline{1, n}$
		\[
			|x_{kj} - x_{0j}| \leqslant \sqrt{(x_{1j} - x_{0j})^2 + \dots + (x_{kj} - x_{0j})^2 + \dots + (x_{nj} - x_{0j})^2} = d(M_k, M_0) 
		\]
		Получаем, что 
		\[
			\forall \varepsilon > 0\ \exists\ \nu \in \R\ |\ \forall k > \nu \Rightarrow |x_{kj} - x_{0j}| \leqslant d(M_k, M_0) \leqslant \varepsilon
		\]
		То есть выполняется \eqref{krit}
		
		\suff Пусть для каждого фиксированного $j = \overline{1, n}$ выполняется \eqref{krit}. Тогда
		\[
			\forall \varepsilon > 0\ \exists\ \nu_j \in \R\ |\ \forall k > \nu_j \Rightarrow |x_{kj} - x_{0j}| \leqslant \varepsilon
		\]
		Выбирая $\nu = max\{\nu_i\}$, получаем, что для $\forall k > \nu \Rightarrow |x_{kj} - x_{0j}| \leqslant \varepsilon$
		 и 
		 \[
		 	d(M_k, M_0) \leqslant \sqrt{\sum_{j = 1}^{n}{max(x_{kj} - x_{0j})^2}} \leqslant \sqrt{\sum_{j = 1}^{n}{\varepsilon^2}} = \varepsilon\sqrt{n}, 
		 \]
		 Т.е. $\exists N = \sqrt{n} \leqslant 0$ такое, что $d(M_k, M_0) \leqslant N \varepsilon,$ а значит по N-лемме сходимости n-мерных последовательностей имеем:
		 \[
		 	M_k \underset{k\rightarrow\infty}{\longrightarrow}  M_0.
		 \]
	\end{proof}

	\begin{exc}
		Доказать, что для $\forall \widetilde M, \overline{M} \in \R^n$ eвклидово расстояние $d(\widetilde M, \overline{M})$, октаэдрическое расстояние $\rho_1(\widetilde M, \overline{M}))$ и кубическое расстояние $\rho_2(\widetilde M, \overline{M}))$ удовлетворяют неравенствам
		\[
		\frac{d(\widetilde M, \overline{M})}{\sqrt{n}} \leqslant \rho_2(\widetilde M, \overline{M}) \leqslant \rho_1(\widetilde M, \overline{M})) \leqslant \sqrt{n}\cdot d(\widetilde M, \overline{M})
		\]
 		И вывести, что если в одном из метрических пространств $(\R^n, d)$, $(\R^n, \rho_1)$ или $(\R^n, \rho_2)$ имеем $M_k \underset{k\rightarrow\infty}{\longrightarrow}  M_0,$ то в силу N-леммы для n-мерных последовательностей то же самое будет и в остальных рассматриваемых метрических пространствах.
	\end{exc}
	
	\begin{rem}
		Доказанная теорема сводит исследование сходимости точечных последовательностей в метрическом пространстве $(\R^n, d)$ к исследованию на сходимость соответствующих координатных числовых последовательностей.
		В связи с этим большинство основных свойств сходящихся числовых последовательностей естественным образом переносятся на n-мерные последовательности в $(\R^n, d)$
		(единственность предела, предел линейной комбинации, принцип выбора, критерий Коши сходимости n-мерных последовательностей и т. д.)
	\end{rem}
\end{document}
