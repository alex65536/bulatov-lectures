\makeatletter
\def\input@path{{../../}}
\makeatother
\documentclass[../../main.tex]{subfiles}

\begin{document}
\section{Предел последовательностей в $\R^n$.}

  \emph{Последовательностью} в $\R^n$ будем называть произвольное 
  отображение
  \[
    f : \N \to \R^n,
  \]
  ставящее в соответствие для $\forall k \in \N$ единственное 
  значение $M_k = f(k) \in \R^n$. 
  Эти $n$-мерные последовательности будем также называть 
  \emph{точечными последовательностями} в $\R^n$ и обозначать $
  (M_k), k \in \N$.
  Используя координатную запись, для $\forall (M_k)$ имеем:
  \[
    M_k = (x_{k1}, x_{k2}, \dots, x_{kn}),
  \] 
  где $\forall x_{kj} \in \R, j = \overline{1, n}$.  
  То есть, задание $n$-мерных последовательностей равносильно 
  заданию $n$ действительных числовых последовательностей 
  $(x_{kj}) k \in \N, j = \overline{1,n}$.
  
  Последовательность $(M_k), k \in \N$ называется \emph{сходящейся} 
  к точке $M_0 \in \R^n$ в пространстве $(\R^n, \rho)$, если
  \[
    \lim_{k \to \infty}{\rho(M_k, M_0)} = 0
  \]
  т.е. 
  \begin{equation}
  \label{seq-lim}
    \forall \eps > 0\ \exists\ \nu(\eps) \in \R\ |\ \forall k \geq 
    \nu(\eps) \implies \rho(M_k, M_0) \leq \eps 
  \end{equation}
  На языке окрестностей имеем:
  \[
    \eqref{seq-lim} \iff \forall \eps > 0\ \exists\ \nu \in \R\ |\ 
    \forall k \geq \nu  \implies M_k \in \overline B_\eps(M_0)
  \]
  По аналогии с M-леммой для сходимости числовых последовательностей 
  доказывается
  \begin{lem}[N-лемма сходимости $n$-мерных последовательностей]
    Если 
    \[
      \exists N = const \geq 0 \text{ такое, что } 
      \forall \eps > 0\ \exists\ \nu_\eps \in \R\ |\ \forall k \geq 
      \nu_\eps \implies \rho(M_k, M_0) \leq N \eps,
    \]
    то $M_k \underset{k\rightarrow\infty}{\longrightarrow}  M_0$.
  \end{lem}

  Как и для числовой последовательности, для \eqref{seq-lim} точку 
  $M_0$ будем называть пределом рассматриваемой последовательности 
  $(M_k)$ и кроме записи $M_k \underset{k\rightarrow\infty}
  {\longrightarrow}  M_0$ будем использовать запись 
  $ \lim\limits_{k \to \infty}{M_k} = M_0$.
  \begin{thm}[Критерий сходимости n-мерных последовательностей]
    Последовательность $M_k = (x_{k1}, \dots, x_{kn}) \in \R^n, k 
    \in \N$, сходится в линейном метрическом пространстве $(\R^n, d)$
    с евклидовой метрикой $d$ к точке $M_0 = (x_{01}, \dots, x_{0n})
    \in \R^n$ тогда и только тогда, когда
    \begin{equation}
    \label{krit-posl}
      x_{kj}\underset{k\rightarrow\infty}
      {\longrightarrow} x_{0j}, \quad \forall j = \overline{1, n}.
    \end{equation}
  \end{thm}
  \begin{proof}
    \quad
    
    \nec: 
    Пусть $M_k \underset{k\rightarrow\infty}{\longrightarrow}  M_0$.
    
    Тогда, используя евклидово расстояние 
    \[
      d(M_k, M_0) = \sqrt{\sum_{j = 0}^{n}{(x_{kj} - x_{0j})^2}},
    \]
    получаем, что
    \[
      d(M_k, M_0) \underset{k\rightarrow\infty}{\longrightarrow} 0 
      \implies
      \forall \eps > 0\ \exists \nu \in \R\ |\ \forall k \geq \nu 
      \implies d(M_k, M_0) \leq \eps.
    \]
    Отсюда, учитывая, что для произвольного фиксированного $j = 
    \overline{1, n}$ имеем
    \[
      |x_{kj} - x_{0j}| \leq \sqrt{(x_{k1} - x_{01})^2 + (x_{k2} - 
      x_{02})^2 + \dots + (x_{kj} - x_{0j})^2 + \dots + (x_{kn} - 
      x_{0n})^2} = d(M_k, M_0) 
    \]
    получаем, что 
    \[
      \forall \eps > 0\ \exists\ \nu \in \R\ |\ \forall k \geq \nu 
      \implies |x_{kj} - x_{0j}| \leq d(M_k, M_0) \leq \eps,
    \]
    то есть выполняется \eqref{krit-posl}.
    
    \bigskip
    
    \suff: Пусть для каждого фиксированного $j = \overline{1, n}$ 
    выполняется \eqref{krit-posl}. Тогда
    \[
      \forall \eps > 0\ \exists\ \nu_j \in \R\ |\ \forall k \ge \nu_j 
      \implies |x_{kj} - x_{0j}| \leq \eps
    \]
    Выбирая $\displaystyle\nu = \max_{1 \le i \le n}(\nu_i)$, 
    получаем, что для $\forall k \ge \nu \implies |x_{kj} - x_{0j}| 
    \leq \eps$
     и 
     \[
       d(M_k, M_0) \leq \sqrt{\sum_{j = 1}^{n}{max(x_{kj} - 
       x_{0j})^2}} \leq \sqrt{\sum_{j = 1}^{n}{\eps^2}} = 
       \eps\sqrt{n}, 
     \]
     Т.е. $\exists N = \sqrt{n} \geq 0$ такое, что $d(M_k, M_0) \leq 
     N \eps,$ а значит, по N-лемме сходимости $n$-мерных 
     последовательностей, имеем:
     \[
       M_k \underset{k\rightarrow\infty}{\longrightarrow}  M_0.
     \]
  \end{proof}

  \begin{exc}
    Доказать, что для $\forall \widetilde M, \overline{M} \in \R^n$ 
    eвклидово расстояние $d(\widetilde M, \overline{M})$, 
    октаэдрическое расстояние $\rho_1(\widetilde M, \overline{M}))$ 
    и кубическое расстояние $\rho_2(\widetilde M, \overline{M}))$ 
    удовлетворяют неравенствам
    \[
    \frac{d(\widetilde M, \overline{M})}{\sqrt{n}} \leq 
    \rho_2(\widetilde M, \overline{M}) \leq \rho_1(\widetilde M, 
    \overline{M})) \leq \sqrt{n}\cdot d(\widetilde M, \overline{M})
    \]
     и вывести на основании этого, что если в одном из метрических 
     пространств $(\R^n, d)$, $(\R^n, \rho_1)$ или $(\R^n, \rho_2)$ 
     имеем $M_k \underset{k\rightarrow\infty}{\longrightarrow}  M_0,$
     то в силу N-леммы для $n$-мерных последовательностей 
     то же самое будет и в остальных рассматриваемых метрических пространствах.
  \end{exc}
  
  \begin{rem}
    Доказанная теорема сводит исследование сходимости точечных 
    последовательностей в метрическом пространстве $(\R^n, d)$ к 
    исследованию на сходимость соответствующих координатных числовых 
    последовательностей.
    В связи с этим большинство основных свойств сходящихся числовых 
    последовательностей (кроме использующих неравенства) естественным
    образом переносятся на $n$-мерные последовательности в $(\R^n, d)$
    (единственность предела, предел линейной комбинации, принцип 
    выбора, критерий Коши сходимости $n$-мерных последовательностей и 
    т. д.)
  \end{rem}
  
  \begin{exmp}
    Рассмотрим пространство $\R^2$ с элементами
    $M=(x,y) \in \R^2, x,y \in \R$. 
    Покажем, что если последовательность $M_k=(x_k,y_k)$ ограничена 
    при $k \in \N$, т. е.
    \[
    \exists r>0 \; | \; \forall k \in \N 
    \implies d(M_k; \vec0)=\sqrt{x^2_k+y^2_k} \leq r,
    \]
    то из нее можно выбрать сходящуюся последовательность.
    
    Из неравенств
    \[ 
    |x_k| \leq \sqrt{x^2_k+y^2_k}\leq r, \; \forall k \in \N 
    \]
    \[ 
    |y_k| \leq \sqrt{x^2_k+y^2_k}\leq r, \; \forall k \in \N 
    \]
    следует в силу ограниченности $M_k=(x_k,y_k) \in \R^2$, 
    что ограничены и числовые последовательности
    $(x_k)$ и $(y_k)$. 
    Используя дальше принцип выбора для ограниченной 
    числовой последовательности $(x_k)$,
    получим, что у нее существует сходящаяся подпоследовательность
    $x_{k_j}\underset{k_j \longrightarrow \infty}
    {\longrightarrow}x_0 \in \R$.
    Тогда из ограниченной подпоследовательности $y_{k_j}$ 
    можно выбрать некоторую подпоследовательность
    $y_{m_i}\underset{m_i\longrightarrow \infty}
    {\longrightarrow}y_0 \in \R$
    
    Для получаемых индексов $1 \leq m_1 \leq \ldots 
    \leq m_i \leq \ldots $ из множества
    $\{k_{1_j}, k_{2_j}, \dots, k_{i_j}, \dots\}$
    последовательность $ (x_{m_i}) $ будет являться 
    подпоследовательностью $ (x_{k_j}) $ и, значит, 
    \[  \lim_{m_i \to \infty}{x_{m_i}}= 
    \lim_{k_i \to \infty}{x_{k_i}}=  x_0\]
    В результате получим двумерную подпоследовательность 
    $(x_{m_i}, y_{m_i}) \underset{m_i\longrightarrow \infty}
    {\longrightarrow} (x_0, y_0) \in \R^2$,
    то есть в этом случае справедлив принцип выбора.
    
    Аналогичным образом принцип выбора доказывается в пространстве $
    (\R^n, d)$
  \end{exmp}

  \begin{exc}
    По аналогии с числовой последовательностью
    доказать свойства:
    \begin{enumerate}
      \item единственность  предела $n$-мерных последовательностей;
      \item предел линейной комбинации;
      \item критерий Коши сходимости.
    \end{enumerate}
  \end{exc}

  Далее, по аналогии с числовой последовательностью, 
  $n$-мерную последовательность $(M_k),  k \in \N $
  будем называть \emph{бесконечно малой последовательностью}, если
  \[
    M_k \underset{k \longrightarrow \infty}
    {\longrightarrow}\vec{0} = (0,0,\dots,0) \in \R^n.
  \]
  Как и для ЧП, показывается, что любая линейная 
  комбинация бесконечно малых $n$-мерных последовательностей с 
  ограниченными коэффициентами является бмп, то есть, если
  \[
    \forall \lambda_k, \mu_{k} = O(1),
  \]
  то в случае, когда $\widetilde{M}_k \appr{k \to \infty}\vec0$ и
  $\overline{M}_k \appr{k \to \infty}\vec0$ имеем
  \[
    \lambda_k \widetilde{M}_k + \mu_k \overline{M}_k =
    (\lambda_k \widetilde{x}_{k1} + \mu_k \overline{x}_{k1}, 
    \dots,
    \lambda_k \widetilde{x}_{kn} + \mu_k \overline{x}_{kn}
    )
    \underset{k \longrightarrow \infty}{\longrightarrow}
    \vec{0} = (0,0,\dots,0).
  \]
\end{document}
