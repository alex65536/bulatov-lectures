\makeatletter
\def\input@path{{../../}}
\makeatother
\documentclass[../../main.tex]{subfiles}

\begin{document}
	
	\section{Непрерывные ФНП}	
	
	Рассмотрим $u = f(x) = f(x_1, x_2, \ldots, x_n)$, которая определена на 
	$D \subset \R^n$ (открытое связное множество). Эта ФНП называется
	\emph{непрерывной} в точке $M_0 = (x_{01}, x_{02}, \ldots, x_{0n}) \in D$
	, если
	\begin{equation}
	\label{continuity}
	\forall M = (x_1, \ldots, x_n) \in D \iff \exists 
	\lim\limits_{M \to M_0}f(M) = f(M_0)
	\end{equation}
	%\eqref{eq42}
	\begin{equation}
	\label{contin_e/d}
	\eqref{continuity} \iff \forall \epsilon > 0 \exists \delta > 0 \ : \ 
	\forall M \in D
	, d(M, M_0) \leq \delta \implies |f(M) - f(M_0)| \leq \epsilon
	\end{equation}
	Отличие \eqref{contin_e/d} от предела состоит в том, что :
	\begin{enumerate}
		\item  Рассмотренная точка $M$ должна принадлежать области 
		       определения $D$;
		\item Заранее известно предельное значени $p = f(M_0)$.
	\end{enumerate}
    В дальнейшем будем использовать ,,,,,,,\\
    В случае, когда множесто $D$ будет иметь более сложную структуру 
    (например, изолированные точки), то по определению ФНП непрерывны в этих
    точках. Если у $D$ есть граничные точки, то рассматривается непрерывность 
    вдоль множества $D$, то есть :
    \[
       f(x) \text{ непрерывна в точке } M_0 \iff \exists \lim\limits_{M \to M_0}
       f(M) = f(M_0), \ M \in D.
    \]
    Множество всех непрерывных на $D \subset \R^n$ будем обозначать $C(D)$.
    $f \in C(D) \implies f$ непрерывна $\forall M \in D$. Как и для Ф1П 
    показывается, что для ФНП, непрерывной на $D \subset \R^n$, имеем :
    \begin{enumerate}
    	\item $f, g \in C(D),$ то $\forall \lambda, \mu \in R \iff 
    	      \lambda f + \mu g \in C(D)$;
    	\item $f, g \in C(D),$ то $f \cdot g \in C(D)$;
    	\item $f, g \in C(D),$ то $\frac{f}{g}$ непрерывна  во всех точках 
    	      $M_0 \in D$, таких, что  $g(M_0) != 0$.
    \end{enumerate}
    Имеют место локальные свойства непрерывных ФНП:
    \begin{enumerate}
    	\item Локальная ограниченность: \\
    	Если $f(x)$ непрерывна в точке $M_0 \in D(f)$, то $\exists V(M_0)
    	\subset D : \forall M \in V(M) \iff |f(M)| \leq const$
    	\item Стабилизация знака: \\
    	Если $f(x)$ непрерывна в точке $M_0 \in D(f)$ и $f(M_0) \ne 0$, то 
       $\exists V(M_0) \subset D \ : \ \forall M \in V(M_0) \iff \text{sign}f(M)
    	= \text{sign }f(M_0)$
    \end{enumerate}
    \begin{thm}
    	\emph{(о непрерывности композиции ФНП)} Если $n$ функций $g_k(t), k = 
    	\overline{1,n}$ от $m$ переменных $t = (t_1, \ldots, t_m) \in D(g) 
    	\subset \R^n$ непрерывна в точке $t_0 = \left( t_{01}, \ldots, t_{0m} 
    	\right) \in D(g),$ где $g = (g_1, \ldots, g_n$, a $f(x), 
    	x = (x_1, \ldots, x_n)$ непрерывна в точке $x_0 = g(t_0) \in D(f)$,
    	то, в случае существования композиции 
    	\begin{equation}
    	\label{composition}
    		h(t) = f(g(t)) = f(g_1(t), g_2(t), \ldots, g_n(t))
    	\end{equation}
        сложная функция \eqref{composition} будет непрерывна в точке 
        $t_0 \in D(g)$.
    \end{thm}
    \begin{exc}
    	Доказать данную теорему по аналогии с теоремой для Ф1П.
    \end{exc}
    \begin{crl*}[о пределе композиции ФНП]
    	Пусть $\exists \lim\limits_{t \to t_0}g_k(t) = p_k, \ 
    	\forall k = \overline{1, n}, \  g(t) = (g_1(t), \ldots, g_m(t)) t = (t_1,
    	 \ldots, t_m) \in D(g) \subset \R^m, t_0 = (t_{01}, \ldots, t_{0m}) - $
    	предельная точка для $D(g)$. Если функция $f(x)$ непрерывна в точке 
    	$x_0 = (p_1, \ldots, p_n) \in D \subset \R^n$, то в случае 
    	существования композиции $h(t) = f(g(t))$ соответствующих окрестностей
    	$t_0$ и $x_0$ 
    	$\exists \lim\limits_{t \to t_0}h(t) = \lim\limits_{t \to t_0}f(g(t)) =
    	\left[f - \text{непрерывна}\right] = $
    \end{crl*}
	
\end{document}