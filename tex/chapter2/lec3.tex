\makeatletter
\def\input@path{{../../}}
\makeatother
\documentclass[../../main.tex]{subfiles}

\graphicspath{
	{../../img/}
	{../img/}
	{img/}
}

\begin{document}

Последовательность $M_k \in \R^n,$ будем называть \emph{ББП} в метрическом
пространстве $(\R^n, d)$, если     
\[ \exists \lim_{k \to \infty}{M_k} = \infty, \text{ т.~е. } 
\forall \eps > 0\ \exists \nu \in \R\ |\ \forall k \geq \nu \implies
d(M_k, \vec{0}) \geq \eps\]

В отличие от числовых последовательностей для $n$-мерной ББП не определяют
знак бесконечности, т.~е. не рассматривают отдельно случаи 
$+\infty$ и $-\infty$.

В общем случае в метрическом пространстве $(\R^n, d)$, когда
$\forall M_k \in D \subset \R^n, k \in \N$ в случае, когда $M_k 
\underset{k \to \infty}{\longrightarrow} M_0 \in \R^n$, тогда
$M_0 \in \overline{D}$. В частности, для любого компакта 
$(\overline{D} = D)$ в случае, когда $\forall M_k \in D \implies M_0 =
\displaystyle \lim_{k \to \infty}{M_k} \in D$. По аналогии с ЧП 
последовательность $M_k \underset{k \to \infty}{\longrightarrow} M_0$,
где $\forall M_k \ne M_0, k \in \N$
будем называть \emph{последовательностью Гейне} точки $M_0$.

\begin{exc}
	Показать, что точка $M_0 \in \R^n$ будет предельной для множества 
	$D \subset \R^n \iff \exists$ последовательность Гейне
	$M_k \in D, k \in \N$ этой точки.
\end{exc}
 
Как и для числовых последовательностей будем придерживаться терминологии:

Последовательность $(M_k)$ = $\left\{\begin{aligned}
	&\text{сходится} \iff \text{имеет конечный предел } M_0 \\
	&\text{имеет предел} \iff \left[\begin{aligned}
		\text{сходится} \\
		M_k \to \infty
	\end{aligned}\right. \\
	&\text{не имеет предела} \iff \begin{aligned}
		&\text{не имеет ни конечного, ни} \\
		&\text{бесконечного предела}
	\end{aligned}
\end{aligned}\right.$

\section{Предел функций нескольких переменных (ФНП)}

Пусть $D \subset \R^n$~--- область (открытое связное множество из $\R^n$).
Функцией нескольких переменных (ФНП) будем называть
произвольное отображение: 
\begin{equation}
\label{FNP}
	f: D \to \R,
\end{equation}
ставящее в соответствие для $\forall x = (x_1, \ldots, x_n) \in D$
единственное значение $\exists! u = f(x) = f(x_1, \ldots, x_n) \in \R$.
Множество $D$ в \eqref{FNP} называется \emph{областью определения} и
обозначается $D = D(f) = D(u)$. 
\emph{Множеством значений} для \eqref{FNP} будем называть множество
$E = \{ u = f(x) \in \R\ |\ \forall x \in D\}$.

\emph{Графиком} $\text{Г}_f$ для \eqref{FNP} будем называть множество в
$\R^{n + 1}$
\[ \text{Г}_f = \{ (x, u) \in \R^{n + 1} | u = f(x), \forall x \in D(f) \}. \]
Для $n=1$ $\text{Г}_f$ будет представлять некоторую линию в $\R^2$, а для $n=2$
$\text{Г}_f$ будет являться некоторой поверхностью в $\R^3$. 
Для больших размерностей для геометрического изучения ФНП
используются линии (поверхности) уровня, т.~е.
$(n-1)$-мерные множества, определяющиеся в силу \eqref{FNP} неявным уравнением
$f(x) = C = const \in \R, x \in D$.

Например, для Ф3П
(функции от трёх переменных)
$u = x_1^2 + x_2^2 + x_3^2$ для линии уровня имеем уравние 
$x_1^2 + x_2^2 + x_3^2 = const = C$. Если $C < 0$, то получаем $\varnothing$.
Если $C = 0$, то поверхность уровня вырождается в точку $(0, 0, 0) \in \R^3$.
Для $C > 0$ поверхностью уровня будет сфера с центром в $\vec{0}\ (0, 0, 0)$
и радиуса $R = \sqrt{C} > 0$. Поэтому при $C \in ]0; +\infty[$ получаем
систему концентрических сфер. Кроме ФНП \eqref{FNP}, рассматриваемых на
области $D \subset \R^n$,
будем также исследовать ФНП с более сложной областью
определения, например, когда
у $D$ есть отдельные изолированные точки. Сходимость ФНП \eqref{FNP} будем
изучать лишь для точек $M_0 \in \R^n$, являющихся предельными для множества
$D$ из $\R^n$, т.~е. $M_0$~--- либо граничная точка для $D$, либо внутренняя.

Будем говорить, что ФНП $u = f(x)$, определенная в некоторой выколотой
окрестности $\dot{V}(x_0) \subset D(f)$ точки $x_0$, где $x_0$~--- 
предельная для $D$, сходится
к числу $p_0 \in \R$, если
\begin{equation}
	\label{FNP-lim}
	\forall \eps > 0\ \exists\ \delta > 0\ |\ \forall x \in \dot{V}(x_0),
	d(x, x_0) \leq \delta \implies |f(x) - p_0| \leq \eps
\end{equation}

В этом случае также используют запись
$f(x) \underset{x \to x_0}{\longrightarrow} p_0 \in \R$, а само число $p_0$ 
называют \emph{пределом} $f(x)$ при $x \to x_0$ и пишут
$\displaystyle \lim_{x \to x_0}{f(x)} = p_0 \in \R$.

По аналогии с $M$-леммой для Ф1П доказывается \emph{$C$-лемма для ФНП}:

Если $\exists\ C = const \ge 0\ |\ \forall\ \eps > 0, \exists\ \delta > 0\ |\
\forall x \in \dot{V}(x_0), d(x, x_0) \leq \delta \implies |f(x) - p_0|
\leq C \cdot \eps$,
то тогда $f(x)\underset{x \to x_0}{\longrightarrow}p_0$. 

Так же, как и для Ф1П,
доказывается \emph{критерий Гейне} сходимости ФНП в пространстве $\R^n$:

Для того, чтобы $f(x)\underset{x \to x_0}{\longrightarrow}p_0$
необходимо и достаточно, чтобы
$\forall\ M_k \in D$~--- последовательности Гейне точки 
$M_0 \in \R^n \implies$ 
\[f(M_k)\underset{k \to \infty}
{\longrightarrow}p_0 \in \R\]   

\begin{proof}
	\quad 

	\nec
	$f(x) \underset{x \to x_0}{\longrightarrow} p_0\iff \forall \eps > 0 \quad \exists 
	\delta \geq 0, \quad \forall x \in \dot{V}(x_0)$, $d(x,x_0) \leq \delta \implies 
	|f(x)-p_0| \leq \eps.$ Пусть $M_k$~--- последовательность Гейне точки 
	$M_0$, т.e.
	\begin{equation}
	\label{heine-1}
		M_k\underset{k \to \infty}{\longrightarrow} M_0 \iff \forall \tilde \eps > 0 
		\quad \exists \nu \in \R \;|\; \forall k \geq \nu \implies d(M_k, x_0) \leq 
		\tilde \eps
	\end{equation} 
	Возьмём в \eqref{heine-1} $\tilde\eps = \delta$. 
	\[ \forall \eps > 0 \quad\exists \nu \in \R \;|\; \forall k \geq \nu \implies d(M_k,x_0) 
	\leq \tilde \eps = \delta \implies f(M_k)-p_0\leq \eps \implies f(M_k) 
	\underset{k \to \infty}{\longrightarrow}p_0.\]

	\suff Положим, что это неверно, тогда:
	\begin{equation}
	\label{heine-2}
		\exists \eps_0>0 \quad\forall \delta \geq 0 \quad\exists x \in  \dot{V}(x_0), 
		\text{ т.ч. } d(x,x_0)\leq\delta, \text{ и } |f(x)-p_0|>\eps_0,
	\end{equation}
	т.е. для $\delta = \frac1n \quad \exists x(\frac1n) : \delta = \frac1n 
	\implies d(x(\frac1n),x_0)\leq \delta$
	При $n \to \infty \implies \delta \to 0$ и $d(x(\frac1n),x_0) \to 0 \implies 
	x(\frac1n) \underset{n \to \infty}{\longrightarrow}x_0$, т.е. последовательность 
	$y_n = x(\frac1n)$~-- последовательность Гейне, тогда $\forall \eps_0 \quad 
	\exists \nu \in \R \;|\; \forall k \geq \nu \implies |f(y_k)-p_0|\leq \eps_0$, 
	что противоречит \eqref{heine-2}.
\end{proof}

На основании этого критерия и соответствующих свойств сходящихся $n$-мерных
последовательностей доказываются основные свойства сходящихся ФНП
в пространстве $(\R^n, d)$: единственность предела, предел линейной
комбинации, произведения и частного сходящихся ФНП.

Кроме того, в $(\R^n, d)$ справедлива \emph{теорема о сжатой ФНП}:

Если $\forall x \in \dot{V}(x_0) \subset D(f) \cap D(g) \cap D(h) \implies
g(x) \leq f(x) \leq h(x)$, то тогда в случае, когда
$g(x)\underset{x \to x_0}{\longrightarrow}p_0
$ и $h(x)\underset{x \to x_0}{\longrightarrow}p_0$,
имеем также, что $f(x)\underset{x \to x_0}{\longrightarrow}p_0$.

\begin{exmp}

Пусть \[ \left\{\begin{aligned}
	f(x) = \dfrac{x_1^4 + x_2^4 + \ldots + x_n^4}{x_1^2 + x_2^2 +
		\ldots + x_n^2}, \\
	\forall x \ne \vec{0} = (0, 0, \ldots 0) \in \R^n.
\end{aligned}\right. \]

\begin{gather*}
	\forall x \ne \vec{0}\ \text{имеем}\ 0 \leq f(x) =
	\displaystyle\sum_{k=1}^{n}{\dfrac{x_k^4}{x_1^2 + \dots + x_k^2 +
			\ldots + x_n^2}} \le
	\displaystyle\sum_{k=1}^{n}{\dfrac{x_k^4}{x_k^2}} = \\ =
	(x_1^2 + \ldots + x_k^2 + \ldots + x_n^2)
	\underset{x \to \vec{0}}{\longrightarrow}0, \text{поэтому }
	\exists\displaystyle\lim_{x \to \vec{0}}f(x) = 0.
\end{gather*}

\end{exmp}

По аналогии с Ф1П в случае, когда $f(x)\underset{x \to x_0}{\longrightarrow}0$
говорим, что имеем бесконечно малую ФНП (бмФНП). В частности,
в разобранном выше примере имеем бмФНП в окрестности нулевого вектора.
Бесконечно большой ФНП (ББФНП) будем называть функцию,
для которой $f(x)\underset{x \to x_0}{\longrightarrow}\infty$, т.~е.

\[\forall\ \eps > 0, \exists\ \delta > 0\ |\ 
\forall x \in \dot{V}(x_0) \subset D(f), d(x, x_0) \leq \delta
\implies |f(x)| \geq \eps\]

Здесь также можно рассматривать случай ББФНП, стремящейся к $\pm \infty$.
\begin{itemize}
	\item[а)] $f(x)\underset{x \to x_0}{\longrightarrow} -\infty \iff
	\forall\ \eps > 0, \exists\ \delta > 0\ |\ 
	\forall x \in \dot{V}(x_0) \subset D(f), d(x, x_0) \leq \delta
	\implies f(x) \leq -\eps$
	\item[б)] $f(x)\underset{x \to x_0}{\longrightarrow} +\infty \iff
	\forall\ \eps > 0, \exists\ \delta > 0\ |\ 
	\forall x \in \dot{V}(x_0) \subset D(f), d(x, x_0) \leq \delta
	\implies f(x) \geq \eps$
\end{itemize}

Кроме бесконечных пределов в конечных точках определяют предел ФНП
на бесконечности: $f(x)\underset{x \to \infty}{\longrightarrow}
p_0 \in \R \iff
\forall\ \eps > 0, \exists\ \delta > 0\ |\ 
\forall x \in D(f), d(x, \vec{0}) \geq \delta
\implies |f(x) - p_0| \leq \eps$.

В данном случае знак бесконечности не уточняется.

Используя бесконечные пределы в конечных точках и 
конечные пределы на бесконечности,
естественным образом рассматриваются смешанные пределы:

\begin{itemize}
	\item[а)] $f(x)\underset{x \to \infty}{\longrightarrow} +\infty \iff
	\forall\ \eps > 0, \exists\ \delta > 0\ |\ 
	\forall x \in D(f), d(x, \vec{0}) \geq \delta
	\implies f(x) \geq \eps$
	\item[б)] $f(x)\underset{x \to \infty}{\longrightarrow} -\infty \iff
	\forall\ \eps > 0, \exists\ \delta > 0\ |\ 
	\forall x \in D(f), d(x, \vec{0}) \geq \delta
	\implies f(x) \leq -\eps$
	\item[в)] $f(x)\underset{x \to \infty}{\longrightarrow} \infty \iff
	\forall\ \eps > 0, \exists\ \delta > 0\ |\ 
	\forall x \in D(f), d(x, \vec{0}) \geq \delta
	\implies |f(x)| \geq \eps$
\end{itemize}
	
Если $f(x)$ является бмФНП при $x \to x_0$, где $x_0$ может быть как конечной,
так и бесконечной, то в случае, когда $\forall x \in \dot{V}{(x_0)} \implies
f(x) \ne 0$, то $\dfrac{1}{f(x)}$~--- ББФНП при $x \to x_0$ и наоборот.
	
\section{Реализация для Ф2П и Ф3П}	

Рассмотрим сначала случай $n=2$, т.~е. когда мы имеем двумерное евклидово
пространство $(\R^2, d)$ с евклидовым расстоянием. Если 
имеется ПДСК $Oxy$, где для
$\forall M \in \R^2 \implies$
$M = (x, y),\ x, y \in \R$, то получаем
соответствующую функцию двух переменных 
$u = f(x, y), (x, y) \in D \subset \R^2$ ~---
плоская область определения.

В данном случае $\text{Г}_f$ будет представлять некую поверхность в $\R^3$,
а линии уровня~--- это соответствующие неявно заданные плоские кривые
$\left\{\begin{aligned}
	&f(x, y) = const \\
	&(x, y) \in D
\end{aligned}\right.$

Если у нас Ф2П $f(x, y)$ определена в некоторой выколотой окрестности 
$\dot{V}(M_0) \subset D$ точки $M_0(x_0, y_0) \in \R^2$,
являющейся предельной для $D$, то в этом случае  предел
$p_0 = \displaystyle \lim_{M \to M_0}{f(M)}$ называется
\emph{двойным пределом} и записывается 
\begin{equation}
\label{double-lim}
p_0 = \underset{\substack{
	x \to x_0 \\
	y \to y_0
}}{\lim}f(M)
\end{equation}

Кроме \eqref{double-lim} рассматривают \emph{повторные пределы}
\begin{equation}
\label{rep-lim}
	\left\{\begin{aligned}
		A = \displaystyle\lim_{x \to x_0}{\left(\lim_{y \to y_0}{f(x, y)}\right)}, \\
		B = \displaystyle\lim_{y \to y_0}{\left(\lim_{x \to x_0}{f(x, y)}\right)}, 
	\end{aligned}\right.
\end{equation}

которые можно также называть \emph{пределами вдоль координатных осей}.
Поясняющие рисунки: \\
a) 
\includegraphics[scale=1.0]{rep-lim-case-a.png}
б) 
\includegraphics[scale=1.0]{rep-lim-case-b.png}

Кроме \eqref{double-lim} и \eqref{rep-lim} рассматривают частные
пределы Ф2П, когда независимые переменные стемятся к $x_0, y_0$
не произвольным образом, а специальным, например, по некоторой плоской
кривой $l \subset{D(f)} \subset \R^2$. Можно показать, что если для Ф2П
$\exists$ двойной предел \eqref{double-lim} и $\exists$ повторные
пределы \eqref{rep-lim}, то все они равны между собой. Аналогично и
для частных пределов.

\begin{exmps}
\begin{itemize}
	\item[1)] 
	\[\left\{\begin{aligned}
		f(x) = \dfrac{x^2 + y^2}{|x| + |y|} \\
		(x, y) \ne (0, 0)
	\end{aligned}\right.\]
	В данном случае $\exists A = \displaystyle\lim_{x \to 0}
	{\left(\lim_{y \to 0}{ \dfrac{x^2 + y^2}{|x| + |y|} }\right)} =
	\displaystyle \lim_{x \to 0}{\dfrac{x^2}{|x|}} =
	\displaystyle \lim_{x \to 0}{|x|} = 0$. В силу симметрии
	$\exists B = \displaystyle\lim_{y \to 0}
	{\lim_{x \to 0}{ \dfrac{x^2 + y^2}{|x| + |y|} }} = 0$.
	Поэтому, \underline{если} существует двойной предел $p_0$,
	то $p_0=A=B=0$. В данном случае имеем:
	\[
		0 \leq f(x, y) = 
		\dfrac{x^2}{|x| + |y|} + \dfrac{y^2}{|x| + |y|} \leq
		\dfrac{x^2}{|x|} + \dfrac{y^2}{|y|} = (|x| + |y|) 
		\underset{\substack{x \to 0 \\ y \to 0}}{\longrightarrow}0,
	\]
	поэтому действительно $\exists\ p_0 = \underset{\substack{x \to 0 \\ y \to 0}}
	{\lim}f(x, y) = 0$.
	
	\item[2)] Пусть \[\left\{\begin{aligned}
		&f(x, y) = x \cos{\dfrac{1}{y}} + y \cos{\dfrac{1}{x}} \\
		&xy \ne 0
	\end{aligned}\right.\]
	В данном случае в точке $M_0(0, 0)$, являющейся предельной для
	$D = \R^2 \backslash M_0$, имеем
	\[|f(x, y) \leq |x| \cdot \left|\cos{\dfrac{1}{y}} \right| + |y| \cdot 
	\left|\cos{\dfrac{1}{x}} \right| \leq (|x| + |y|) 
	\underset{\substack{x \to 0 \\ y \to 0}}{\longrightarrow}0 \implies
	f(x, y) \underset{\substack{x \to 0 \\ y \to 0}}{\longrightarrow}0,\] т.~е. 
	$\exists$ двойной предел, но в то же время 
	\[\text{при } x \ne 0 \implies \nexists \displaystyle \lim_{y \to 0}
	x \cos{\dfrac{1}{y}}, \text{а при } y \ne 0 \implies
	\nexists \displaystyle \lim_{x \to 0} y \cos{\dfrac{1}{x}}\] 
	
	В связи с этим здесь, несмотря на то, что существует двойной
	предел, повторные пределы не существуют.
	
	\item[3)] Для $f(x, y) = \dfrac{x^2 y}{x^4 + y^2}, (x, y) \ne (0, 0)$
	имеем \[\exists A = \displaystyle \lim_{x \to 0}
	{\left(\lim_{y \to 0}{\left( \dfrac{x^2 y}{x^4 + y^2} \right)}\right)} =
	\lim_{x \to 0}{\left(\dfrac{0}{x^4}\right)} = 0\]
	\[\exists B = \displaystyle \lim_{y \to 0}
	{\left(\lim_{x \to 0}{\left( \dfrac{x^2 y}{x^4 + y^2} \right)}\right)} =
	\lim_{y \to 0}{\left(\dfrac{0}{y^2}\right)} = 0\]
	
	В данном случае, несмотря на то, что $\exists A = B = 0$, но
	двойной предел не существует:
	\[\nexists p_0 = 
	\lim_{\substack{x \to 0 \\ y \to 0}}{\dfrac{x^2 y}{x^4 + y^2}}\]	
	Для обоснования этого рассмотрим частный предел по параболе
	$y = x^2$.
	
	\[\exists p = \lim_{\substack{x \to 0 \\ y = x^2 \to 0}}
	{\dfrac{x^2 \cdot x^2}{x^4 + (x^2)^2}} = \dfrac{1}{2} \ne 0
	\implies \nexists \lim_{\substack{x \to 0 \\ y \to 0}}{f(x, y)}\]
\end{itemize}
\end{exmps}

Рассмотренная выше реализация для Ф2П естественным образом
переностится на Ф3П. Здесь в $\R^3$ в ПДСК $Oxyz:
\left\{\begin{aligned}
	&u = f(x, y, z) \\
	&(x, y, z) \in D \subset \R^3	\end{aligned}\right.$
	Аналогично, как и выше, рассматривается тройной предел
	$p_0 = \displaystyle \lim_{\substack{x \to x_0 \\ y \to y_0 \\ z \to z_0}}
{f(x, y, z)}$, где $M_0(x_0, y_0, z_0) \in \R^3$~--- предельная точка для $D$.
А также рассматриваются повторные пределы:
\begin{gather*}
	A = \lim_{x \to x_0}{(\lim_{y \to y_0}{(\lim_{z \to z_0}{f(x, y, z)})})} \\
	B = \lim_{x \to x_0}{(\lim_{z \to z_0}{(\lim_{y \to y_0}{f(x, y, z)})})} \\
	C = \lim_{y \to y_0}{(\lim_{x \to x_0}{(\lim_{z \to z_0}{f(x, y, z)})})} \\
	D = \lim_{y \to y_0}{(\lim_{z \to z_0}{(\lim_{x \to x_0}{f(x, y, z)})})} \\
	E = \lim_{z \to z_0}{(\lim_{x \to x_0}{(\lim_{y \to y_0}{f(x, y, z)})})} \\
	F = \lim_{z \to z_0}{(\lim_{y \to y_0}{(\lim_{x \to x_0}{f(x, y, z)})})}
\end{gather*}
	
Здесь также, если существуют повторные и тройные
пределы, то все они равны между собой. Аналогично и для
частных пределов Ф3П.
	
\begin{exmps}
\begin{itemize}
	\item[1)] Пусть $f(x, y, z) = \dfrac{x^2 + y^2 + z^2}{e^{x + y + z}}$.
	Рассмотрим предел на бесконечности, когда 
	$\left\{
	\substack{x \to +\infty \\ y \to +\infty \\ z \to +\infty}\right.$.
	Разделяя переменные, получаем:
	\begin{gather*}
		p_0 = 
		\lim_{x \to +\infty}{\dfrac{x^2}{e^x}}
		\lim_{y \to +\infty}{\dfrac{1}{e^y}}
		\lim_{z \to +\infty}{\dfrac{1}{e^z}} +
		\lim_{x \to +\infty}{\dfrac{1}{e^x}}
		\lim_{y \to +\infty}{\dfrac{y^2}{e^y}}
		\lim_{z \to +\infty}{\dfrac{1}{e^z}} +
		\lim_{x \to +\infty}{\dfrac{1}{e^x}}
		\lim_{y \to +\infty}{\dfrac{1}{e^y}}
		\lim_{z \to +\infty}{\dfrac{z^2}{e^z}} = \\ =
		\left[
			\dfrac{1}{e^t} \underset{t \to +\infty}{\longrightarrow}0,
			\dfrac{t^2}{e^t} 
			\underset{t \to +\infty}{
				\overset{{\lopital}}{\longrightarrow}} 0			
		\right] = 0.
	\end{gather*}
	
	Пусть
	\item[2)] \[
		p_0 = \lim_{\substack{x \to +0 \\ y \to +0 \\ z \to +0}}
		{(x + y + z)^{xyz}}
	\]
	
	Чтобы получить предполагаемое значение $p_0$, рассмотрим один
	из повторных пределов: 
	\[A = \displaystyle \lim_{x \to +0}
	{\left(\lim_{y \to +0}{\left(\lim_{z \to +0}{(x + y + 
	z)^{xyz}}\right)}\right)} =
	\lim_{x \to +0}{\left(\lim_{y \to +0}{(x + y)^0}\right)} = \lim_{x \to +0}{1} 
	= 1\]
	
	Поэтому, \underline{если} $\exists p_0$, то $p_0 = A = 1$.
	Пользуясь пределом сжатой ФНП, докажем, что действительно $p_0 = 1$.
	Во-первых, т.~к. $x + y + z
	\underset{\substack{x \to +0 \\ y \to +0 \\ z \to +0}}
	{\longrightarrow}0$, то в достаточно малой соответствующей выколотой
	окрестности предельной точки $O(0, 0, 0)$ имеем 
	$0 < x + y + z \leq 1$, а тогда для $x > 0$, $y > 0$, $z > 0$
	из этой окрестности $\implies (x + y + z)^{xyz}{\leq 1}$.
	Во-вторых, используя неравенство между средним арифметическим
	и средним геометрическим трёх положительных чисел, получаем
	$\dfrac{x + y + z}{3} \geq \sqrt[3]{xyz}$, поэтому
	\[ {\left(3\sqrt[3]{xyz}\right)}^{xyz} \leq {(x + y + z)}^{xyz} \leq 1. \]
	
	Cейчас достаточно показать, что ${(3\sqrt[3]{xyz})}^{xyz}
	\underset{\substack{x \to +0 \\ y \to +0 \\ z \to +0}}{\longrightarrow} 1$.
	Полагая $t = xyz \to +0$, получаем, что 
	\[ {(3\sqrt[3]{xyz})}^{xyz} = {(27t)}^{t/3} = e^{\frac{t}{3}\ln{27t}}
	= e^{\frac{\ln{(27t)}}{3t^{-1}}}. \]
	Отсюда, учитывая, что 
	\[ \lim_{t \to +0}{\dfrac{\ln{(27t)}}{3t^{-1}}}
	\overset{\lopital}{=} \lim_{t \to +0}{\dfrac{1/t}{ (-3/t^2) }} =
	-\dfrac{1}{3} \displaystyle \lim_{t \to +0}{t} = 0\]
	 имеем 
		\[{\left(\lim_{\substack{x \to +0 \\ y \to +0 \\ z \to +0}}
			{3\sqrt[3]{xyz}}
		\right)}^{xyz} = e^0 = 1\]
\end{itemize}
\end{exmps}

Отметим, что в общем случае для ФНП при $n \geq 2$, в отличие от Ф1П,
правила Лопиталя вычисления пределов нет.

\end{document}
