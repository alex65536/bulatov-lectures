\makeatletter
\def\input@path{{../../}}
\makeatother
\documentclass[../../main.tex]{subfiles}

\begin{document}

Последовательность $M_k \in \R^n$ будем называть ББП в метрическом
пространстве $(\R^n, d)$, если     
\[ \exists \lim_{k \to \infty}{M_k} = \infty, \text{ т.е. } 
\forall \eps > 0\ \exists \nu \in \R\ |\ \forall k \geq \nu \implies
d(M_k, \vec{0}) \leq \eps\]

В отличие от числовых последовательностей для $n$-мерной ББП не определяют
знак бесконечности, т.е. не рассматривают отдельно случаи 
$(+\infty)$ и $(-\infty)$.

В общем случае в метрическом пространстве $(\R^n, d)$, когда
$\forall M_k \in D \subset \R^n, k \in \N$ в случае, когда $M_k 
\underset{k \to \infty}{\longrightarrow} M_0 \in \R^n$, то тогда
$M_0 \in \overline{D}$. В частности, для любого компакта $(\overline{D} = D)$
в случае, когда $\forall M_k \in D \implies M_0 = \displaystyle \lim_{k \to \infty}{M_k} \in D$. По аналогии с чп последовательность 
$M_k \underset{k \to \infty}{\longrightarrow} M_0$, где $\forall M_k \ne M_0, k \in \N$
будем называть последовательностью Гейне точки $M_0$.

\begin{exc}
	Показать, что точка $M_0 \in \R^n$ будет предельной для множества $D \subset \R^n$
	$\iff \exists$ последовательность Гейне $M_k \in D, k \in \N$ этой точки.
\end{exc}
 
Как и для числовых последовательностей будем придерживаться терминологии.

Последовательность $M_k$ = $\left\{\begin{aligned}
	&\text{сходится} \iff \text{имеет конечный предел} M_0 \ne \infty \\
	&\text{имеет предел} \iff \left[\begin{aligned}
		\text{сходится} \\
		M_k \to \infty
	\end{aligned}\right. \\
	&\text{не имеет предела} \iff \begin{aligned}
		&\text{не имеет ни конечно, ни} \\
		&\text{ни бесконечного предела}
	\end{aligned}
\end{aligned}\right.$

\section{Предел функций нескольких переменных (ФНП)}

Пусть $D \subset \R^n$ - область (открытое связное множество из $\R^n$). ФНП будем
называть произвольное отображение: 
\begin{equation}
\label{FNP}
	D \to \R
\end{equation}
ставящее в соответствие $\forall x = (x_1, \ldots, x_n) \in D$ единственное значение
$\exists! u = f(x) = f(x_1, \ldots, x_n) \in \R$. Множество $D$ в \eqref{FNP}
называется \emph{областью определения} и обозначается $D = D(f) = D(u)$. 
\emph{Множеством значений} для \eqref{FNP} будем называть множество
$E = \{ u = f(x) \in \R\ |\ \forall x \in D\}$.

\emph{Графиком} $\text{Г}_f$ для \eqref{FNP} будем называть множество в $\R^{n + 1}$.
\[ \text{Г}_f = \{ (x, u) \in \R^{n + 1} | u = f(x), \forall x \in D(f) \} \]
Для $n$=1 $\text{Г}_f$ будет представлять некоторую линию в $\R^2$, а для $n$=2
$\text{Г}_f$ будет являться некоторой поверхностью в $\R^3$. 
Для больших размерностей для геометрического изучения ФНП
искользуется линии (поверхности) уровня, т.е.
$(n-1)$-мерные множества, определяющиеся в силу \eqref{FNP} неявным уравнением
$f(x) = C = const \in \R, x \in D$. Например, для Ф3П (функция от трёх переменных)
$u = x_1^2 + x_2^2 + x_3^2$ для линии уровня имеем уравние 
$x_1^2 + x_2^2 + x_3^2 = const = C$. Если $C < 0$, то имеем $\emptyset$.
Если $C = 0$, то поверхность уровня вырождается в 1 точку $(0, 0, 0) \in \R^3$.
Для $C > 0$ поверхностью уровня будет сфера с центром в $\vec{0}\ (0, 0, 0)$
и радиусом $R = \sqrt{C} > 0$. Поэтому при $C \in ]0; +\infty[$ получаем систему
концентрических сфер. Кроме ФНП \eqref{FNP}, рассматриваемых на множестве $D$,
будем также исследовать ФНП с более сложной областью определение, например, когда
у $D$ есть отдельные изолированные точки. Сходимость ФНП \eqref{FNP} будем
изучать лишь для точек $M_0 \in \R^n$, являющихся предельными для множества
$D$ из $\R^n$, т.е. $M_0$ - либо граничная точка для $D$, либо внутренняя.

Будем говорить, что ФНП $u = f(x)$, определенная в некоторой выколотой окрестности 
$\dot{V}(x_0) \subset D(f)$ точки $x_0$, где $x_0$ - предельная для $D$, сходится
к числу $p_0 \in \R$, если
\begin{equation}
	\label{FNP-lim}
	\forall \eps > 0\ \exists\ \delta > 0\ |\ \forall x \in \dot{V}(x_0),
	d(x, x_0) \leq \delta \implies |f(x) - p_0| \leq \eps
\end{equation}

В этом случае также будем использовать запись
$f(x) \underset{x \to x_0}{\longrightarrow} p_0 \in \R$, а само число $p_0$ называется
пределом $f(x)$ при $x \to x_0$ и записывается 
$\displaystyle \lim_{x \to x_0}{f(x)} = p_0 \in \R$.


\end{document}
