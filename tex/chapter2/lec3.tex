\makeatletter
\def\input@path{{../../}}
\makeatother
\documentclass[../../main.tex]{subfiles}

\begin{document}

Последовательность $M_k \in \R^n$ будем называть ББП в метрическом
пространстве $(\R^n, d)$, если     
\[ \exists \lim_{k \to \infty}{M_k} = \infty, \text{ т.е. } 
\forall \eps > 0\ \exists \nu \in \R\ |\ \forall k \geq \nu \implies
d(M_k, \vec{0}) \leq \eps\]

В отличие от числовых последовательностей для $n$-мерной ББП не определяют
знак бесконечности, т.е. не рассматривают отдельно случаи 
$(+\infty)$ и $(-\infty)$.

В общем случае в метрическом пространстве $(\R^n, d)$, когда
$\forall M_k \in D \subset \R^n, k \in \N$ в случае, когда $M_k 
\underset{k \to \infty}{\longrightarrow} M_0 \in \R^n$, то тогда
$M_0 \in \overline{D}$. В частности, для любого компакта $(\overline{D} = D)$
в случае, когда $\forall M_k \in D \implies M_0 = \displaystyle \lim_{k \to \infty}{M_k} \in D$. По аналогии с чп последовательность 
$M_k \underset{k \to \infty}{\longrightarrow} M_0$, где $\forall M_k \ne M_0, k \in \N$
будем называть последовательностью Гейне точки $M_0$.

\begin{exc}
	Показать, что точка $M_0 \in \R^n$ будет предельной для множества $D \subset \R^n$
	$\iff \exists$ последовательность Гейне $M_k \in D, k \in \N$ этой точки.
\end{exc}
 
Как и для числовых последовательностей будем придерживаться терминологии.

Последовательность $M_k$ = $\left\{\begin{aligned}
	&\text{сходится} \iff \text{имеет конечный предел} M_0 \ne \infty \\
	&\text{имеет предел} \iff \left[\begin{aligned}
		\text{сходится} \\
		M_k \to \infty
	\end{aligned}\right. \\
	&\text{не имеет предела} \iff \begin{aligned}
		&\text{не имеет ни конечно, ни} \\
		&\text{ни бесконечного предела}
	\end{aligned}
\end{aligned}\right.$


\end{document}
