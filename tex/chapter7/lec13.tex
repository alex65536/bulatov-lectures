\makeatletter
\def\input@path{{../../}}
\makeatother
\documentclass[../../main.tex]{subfiles}

\graphicspath{
	{../../img/}
	{../img/}
	{img/}
}

\begin{document}
\section{Двойной интеграл ( 2 И ) }

Рассмотрим Ф2П $ f \left( x, y \right) $  , где 
$ \left( x, y \right) \in D \subset \R^n $ и $ D $~--- измеримое множество 
в $ \R^2 $ (плоская квадрируемая фигура). 
В этом случае имеем \emph{2И}:

\begin{equation}
	\label{lec_13, num_12}
	I = \iint\limits_{D} f \left( x, y \right)  dxdy
\end{equation}

\begin{thm}[О вычислении 2И по прямоугольнику для 
непрерывных Ф2П]
    Пусть $ f \left( x, y \right)  $~--- непрерывна на прямоугольнике
    $ \text{П} = \left[\ a, b\ \right] \times \left[\ c, d\ \right] =  
    \left\lbrace \ \left( x, y \right) \in \R^2\ |\ 
    a \leq x \leq b, \ c \leq y \leq d\ \right\rbrace $.
    Тогда для \eqref{lec_13, num_12} $ \implies $
    \begin{equation}
    \label{lec_13, num_13}
    I = \int\limits_a^b 
    \left( \int\limits_c^d f \left( x, y \right) dy  \right) dx =
     \int\limits_c^d \left( \int\limits_a^b 
     f \left( x, y \right) dx  \right) dy
    \end{equation}
\end{thm}

\begin{proof}
	Рассмотрим произвольное разбиение $ \widetilde{P} = 
	\left\lbrace x_i \right\rbrace  $, $i = \overline{1,m} $ 
	отрезка $ \left[ a, b\right]  $. T.e $a = x_0 < x_1 < \dots 
	< x_{i-1} < x_i < \dots < x_m = b $. Пусть $ \overline{P} $~--- 
	разбиение точками $ \left\lbrace y_j \right\rbrace  $, 
	 где $j = \overline{1, l} $ отрезка $ \left[ c, d\right]  $. 
	 T.e $c = y_0 < y_1 < \dots < y_{j-1} < y_j < \dots < y_l = d $.
	 
	 С помощью вертикальных прямых $ x = x_i $, $ i = \overline{1, m} $
	 и горизонтальных прямых $ y = y_j $, $ j = \overline{1, l} $
	 прямоугольник $ \text{П} $ разобъется на части: $ \text{П}_{ij} = 
	 \left[ x_{i - 1}, x_i \right] \times \left[ y_{i - 1}, y_i \right] $, 
	 образующим соответствующее разбиение 
	 $ P = \left\lbrace \text{П}_{ij} \right\rbrace $,
	 $ i = \overline{1, m} $, $ j = \overline{1, l} $ 
	 исходного $ \text{П} $. Т.е. $ \text{П} = 
	 \bigcup\limits_{i = 1}^m \bigcup\limits_{j = 1}^l \text{П}_{ij}$, 
	 которые могут иметь общими лишь граничные точки
	 
	 \begin{center}
	 	\includegraphics[width=0.6\textwidth]{lec13_rectangle.png}
	 \end{center}
 
 	Т.к $ f \in C( \text{П} )$ (т.е непрерывна), то $ f \in C( \text{П}_{ij} )$,
 	$ \forall i = \overline{1, m} , \forall j = \overline{1, l}$. Для каждого 
 	$ fix\ i = \overline{1, m} $ рассмотрим функцию $ F_i \left( y \right) = 
 	\int\limits_{ x_{i - 1} } ^ {x_i} f \left( x, y \right) dx $, 
 	$ i = \overline{1, m} $
 	
 	Эта функция корректно определена, 
 	т.к $ \forall fix\ y \implies f \left( x, y\right) $
 	непрерывна по $ x \implies $ интегрируема по $ x $. По теореме о среднем
 	для однократных интегралов $ \implies \exists t_{ij} \in 
 	\left[ x_{i - 1}, x_i \right] $, т.ч $ F_j \left( y_j \right) = 
 	\int\limits_{x_{i - 1} } ^ {x_i} f \left( x, y_j \right) dx = 
 	f \left(  t_{ij}, y_j \right) \Delta x_i$, 
 	$ \Delta x_i = x_i - x_{i - 1}$, где 
 	$ i = \overline{1, m} $
 	
 	В соответствии с этим получаем множество 
 	$ Q = \left\lbrace M_{ij} \right\rbrace $,
 	где $ M_{ij} = \left( t_{ij}, y_j \right) \in \text{П}_{ij} $, 
 	$ i = \overline{1, m} $, 
 	$ j = \overline{1, l} $
 	
 	Т.к все используемые множества измеримы (квадрируемы в $ \R^2 $), 
 	то для полной \emph{специальной} интегральой суммы
 	\begin{equation}
 	\label{lec_13, num_14}
 	\sigma = \sigma \left( t, \left( P, Q \right) \right) = 
 	\sum\limits_{i = 1}^m 
 	\sum\limits_{j = 1}^l f \left( M_{ij} \right) \Delta \text{П}_{ij}
 	\end{equation}
 	где $\Delta \text{П}_{ij} = \Delta x_i \Delta y_j$~---  
 	площадь $\text{П}_{ij}$ 
 	( $ \Delta x_i = x_i - x_{i-1}, \Delta y_j = y_j - y_{j - 1} $ )
 	
 	Имеем:
 	\[ \sigma = \sum\limits_{i = 1}^m 
 	\sum\limits_{j = 1}^l f \left( M_{ij} \right) \Delta x_i \Delta y_j = \]
 	
 	\[ = \sum\limits_{j = 1}^l 
 	\sum\limits_{i = 1}^m F_i \left( y_j \right) \Delta y_j = 
 	\sum\limits_{j = 1}^l \left( 
 	\int\limits_{x_0 = a}^{ x_1 } + \int\limits_{x_1}^{ x_2 } + \dots + 
 	\int\limits_{ x_{m-1} }^{x_m = b} \right) \Delta y_j = \]
 	
 	\[ = \sum\limits_{j = 1}^l \left( \int\limits_a^b f 
 	\left( x, y_j\right) dx \right) \Delta y_j =  \]
 	
 	\begin{equation}
 	\label{lec_13, num_15}
 	= \sum\limits_{j = 1}^l F \left( y_j \right) \Delta y_j 
 	\end{equation}
 	
 	где
 	\begin{equation}
 	\label{lec_13, num_16}
 	\begin{cases}
 	F \left( y \right) = \int\limits_a^b f \left( x, y \right) dx \\
 	y \in \left[ c, d \right] 
 	\end{cases}
 	\end{equation}
 	
 	Из \eqref{lec_13, num_15} $ \implies $ что рассматриваемая $\sigma$
 	является одной из интегральных сумм для Ф1П  \eqref{lec_13, num_16}.
 	Отсюда при 
 	\[ \widetilde{d} = diam \widetilde{P} = 
 	\max \left\lbrace \Delta y_j \right\rbrace 
 	\underset{1 \leq j \leq l}{\longrightarrow} 0 \]
 	В силу непрерывности, а значит интегрируемости получаем: 
 	\[ \exists \lim\limits_{ \widetilde{d} \to 0 } \sigma = 
 	\lim\limits_{ \widetilde{d} \to 0 } 
 	\sum\limits_{j = 1}^l F \left( y_j \right) \Delta y_j =
 	\int\limits_c^d F \left( y \right)dy  \]
 	
 	В силу непрерывности и интегрируемости с использованием
 	специальных интегральный сумм за счет выбора точек, предел 
 	$ \sigma $ будет такой же, как и предел общих интегральных сумм 
 	в силу критерия Гейне
 	
 	Таким образом получим, что
 	\[ I = \lim\limits_{ \widetilde{d} \to 0 } \sigma = 
 	\int\limits_c^d F \left( y \right) dy =  \]
 	\[ = \int\limits_c^d \left( \int\limits_a^b 
 	f \left( x, y\right) dx \right) dy \]
 	что соответсвует правой части \eqref {lec_13, num_13} 
 	
 	Аналогично доказывается равенство и в левой части \eqref {lec_13, num_13} 
\end{proof}

\begin{rems} 
	
	\quad
	
	\begin{enumerate}
		\item  На практике формулы (???), 
		\eqref {lec_13, num_13} записываются в виде 
		\[ I = \int\limits_a^b dx \int\limits_c^d f \left( x, y \right) dy = 
		\int\limits_c^d dy \int\limits_a^b f \left( x, y \right) dx \]
		
		Такое представление 2И называется представлением через 
		повторные однократные интегралы.
		
		\item Полученный результат естественным образом обобщается
		общим случаем $ f \in \R(\text{П}) $ и когда
		\[
		\begin{cases}
		G \left( x \right) = \int\limits_c^d f \left( x, y \right) dy \\
		\left[ a, b \right] 
		\end{cases}
		\]
		также интегрируема на $ \left[ a, b \right] $
		
		Здесь двойной интеграл
		\begin{equation}
		\label{lec_13, num_17}
		\underset{\substack{
				a \leq x \leq b \\
				c \leq y \leq d
		}}{\iint} f \left( x, y \right) dx dy = 
		\int\limits_a^b G \left( x \right) dx  =
		\int\limits_a^b dx \int\limits_c^d f \left( x, y \right) dy
		\end{equation}
		
		Аналогично, когда вместо $ G \left( x \right) $ используется
		$ U \left( y \right) = \int\limits_a^b f \left( x, y \right) dx $, 
		которая предполагается также интегрируемой на 
		$ \left[ c, d \right]  $. Здесь
		
		\begin{equation}
		\label{lec_13, num_18}
		\underset{\substack{
				a \leq x \leq b \\
				c \leq y \leq d
		}}{\iint} f \left( x, y \right) dx dy = 
		\int\limits_c^d U \left( y \right) dy  =
		\int\limits_c^d dy \int\limits_a^b f \left( x, y \right) dx
		\end{equation}
		
	\end{enumerate}
	
\end{rems}

\begin{crl}[о вычислении 2И по криволинейной трапеции]
	Если $ f \in C \left( \widetilde{T} \right) $, 
	где $ \widetilde{T} \subset \R^2 $~---  
	криволинейная трапеция вдоль $ Oy $, т.е 
	\begin{equation}
	\label{lec_13, num_19}
	\widetilde{T} = \left\lbrace 
	\left( x, y\right) \in \R^2\ | 
	\ c \left( x \right) \leq y \leq d \left( x \right),
	 a \leq x \leq b \right\rbrace
	 \end{equation}
	 То когда $c \left( x \right), d \left( x \right) $~---  непрерывны на 
	 $ \left[ a, b \right] $, то имеем
	 \begin{equation}
	 \label{lec_13, num_20}
	 \iint\limits_{\widetilde{T}} f \left( x, y \right)  dxdy = 
	 \int\limits_a^b dx \int\limits_{c \left( x \right) }^{d \left( x \right)} 
	 f \left( x, y \right) dy
	 \end{equation}
	 
	 Аналогично в случае, когда $ f \in C \left( \overline{T} \right) $, 
	 где $ \overline{T} $~---  криволинейная трапеция вдоль $ Ox $, т.е 
	 \begin{equation}
	 \label{lec_13, num_21}
	 \overline{T} = \left\lbrace 
	 \left( x, y\right) \in \R^2\ | 
	 \ a \left( y \right) \leq x \leq b \left( y \right),
	 c \leq y \leq d \right\rbrace
	 \end{equation}
	 
	 Когда $ a \left( y \right), b \left( y \right) $
	 ~--- непрерывны на $ \left[ c, d \right] $, получаем:
	 \begin{equation}
	 \label{lec_13, num_22}
	 \iint\limits_{\overline{T}} f \left( x, y \right) dx dy = 
	 \int\limits_c^d dy 
	 \int\limits_{a \left( y \right) } ^ 
	 {b \left( y \right) } f \left( x, y \right) dx
	 \end{equation}
\end{crl}

\begin{proof}
	Докажем \eqref {lec_13, num_20}. Обоснование \eqref {lec_13, num_22}
	аналогично.
	
	Рассмотрим 
	\[ 
	f_0 \left( x, y \right) =  
	\begin{cases}
	f \left( x, y \right),  \left( x, y \right) \in \widetilde{T} \\
	0, \left( x, y \right) \notin \widetilde{T}
	\end{cases}
	\]
	Для непрерывных на $ \left[ a, b \right] $ функций 
	$ c \left( x \right), d \left( x \right) $ из теоремы Вейерштрасса
	$ \implies \exists c_0 = 
	\min\limits_{\left[ a, b \right] } c \left( x \right) \in \R $ и 
	$ \exists d_0 = \max\limits_{\left[a, b \right] } d \left(x\right) \in \R $.
	
	Рассматривая прямоугольник $ \text{П}_0 = \left[ a, b \right] \times 
	\left[ c_0, d_0 \right]  $ и учитывая, что $ f_0 \left( x, y \right) $ для 
	$ \left( x, y \right) \in \text{П}_0 $ вне трапеции $\widetilde{T}$ дает
	$ f_0 \left( x, y \right) = 0 $ в силу аддитивности ОИ и 2И имеем:
	
	\[
	\iint\limits_{ \widetilde{T} } f \left( x, y \right) dx dy = 
	\iint\limits_{ \widetilde{T} } \underset{ = f_0 \left( x, y \right) } 
	{f \left( x, y \right)} dx dy + 
	\iint\limits_{ \text{П}_0 \smallsetminus \widetilde{T} } 
	\underset{ = f_0 \left( x, y \right) }  0 dx dy =
	\]
	\[
	= \iint\limits_{ \text{П}_0} f_0 \left( x, y \right) dx dy = 
	\int\limits_a^b dx
	\int\limits_{c_0}^{d_0} f_0 \left( x, y \right) dy = 
	\]
	\[
	\left[ \int\limits_{c_0}^{d_0} f_0 dy =
	 \int\limits_{c_0} ^ {c \left( x \right) } 
	\underset{ = 0} {f_0} dy + 
	\int\limits_{c \left( x \right) } ^ {d \left( x \right) } 
	\underset{ = f} {f_0} dy +
	 \int\limits_{d \left( x \right) } ^ {d_0} 
	\underset{ = 0} {f_0} dy \right] = 
	\int\limits_a^b dx \int\limits_{c \left( x \right) } ^
	{d \left( x \right) } f \left( x, y \right) dy
	\]
\end{proof}

\begin{exmp}
	Рассмотрим $ I = \iint\limits_{x^2 + y^2 \leq 2x} 
	f \left(x, y \right) dxdy $, 
	где для простоты считаем, что $ f \left( x, y \right) $~---  непрерывна. 
	Получим представление этого 2И через повторные.
	
	Выделяя полные квадраты для области интегрирования $ \implies $
	\[
	x^2 -2x + y^2 \leq 0 \implies (x-1)^2 + y^2 \leq 1
	\]
	\[
	C \left( 1, 0 \right), R = 1
	\]
	Рассмотрим 2 случая:
	\begin{enumerate}
		\item   \[ I = \int\limits_a^b dx \int\limits_{c \left( x \right) } ^ 
		 {d \left( x \right) } f \left( x, y \right) dy \]
		 Для этого изображаем область интегрирования
		\item Foo
	\end{enumerate}
\end{exmp}

\end{document}






















