\makeatletter
\def\input@path{{../../}}
\makeatother
\documentclass[../../main.tex]{subfiles}

\graphicspath{
	{../../img/}
	{../img/}
	{img/}
}

\begin{document}
	
	\begin{theorem}[о сведении вычетов 3И к 2И]
		Пусть $H$ - цилиндроид вдоль $Oz$, т.е.
		\begin{equation}
			H = \left\{(x, y, z) \in \R^3 \left(\left(x, y\right) \in D \subset \R^2, p \leq z \leq q\right) \right\}
			\label{lec14:1}
		\end{equation}
		
		 $p, q \in \R, D$ ~--- квадрируемая область в $\R^2$. Если $f \in C(H)$, то в случае, когда $H$ ~--- компакт в $\R^3$, имеем:
		\begin{equation}
		\label{lec14:2}
		I = \int\limits_{D} \int \left(\int\limits_p^q f\left(x, y, z\right)dz\right)dxdy 
		\end{equation}
	\end{theorem}
		
	\begin{proof}
		Доказательство проводится по той же схеме, что и при вычислении 2И по прямоугольнику. Вначале рассмотрим $\forall P_0 = \left\{ z_k \right\}$, k = $\overline{0, m}$ отрезка $\left[p, q\right], p = z_0 < z_1 < \ldots < z_{m - 1} < z_m = 1.$ Далее в силу непрерывности $f$ на $H$ имеем:
		
		\begin{equation}
		\exists F\left(x, y\right) = \int\limits_{p}^{q} f\left(x, y, z\right) dz
		\label{lec14:3}
		\end{equation}\\
		Далее рассмотрим $\forall \widetilde{P_0}$ квадрируемого компакта $D \subset \R^2$, являющегося проекцией $H \subset \R^3$ на $Oxy$ получим разбиение $T_0 = \left\{D_j \right\}, j = \overline{1, i}, \text{ где } \bigcup\limits_{j = 1}^i D_j = D$ и $\mes(D_i \cap D_j) = 0 , \forall i \neq j$, т.е. квадрируемые части разбиения $T_0$ между собой могут иметь общими лишь может быть граничные точки.
		
		Выбирая произвольным образом множество отмеченных точек $\widetilde{Q_0} = \left\{ M_j \right\},\\ M_j = \left(x_j, y_j\right) \in D, j = \overline{1, l}$ в соответствии с \eqref{lec14:3} рассмотрим интеграл Римана:
		\begin{equation} 
		I_{x_j} = \int\limits_{z_k - 1}^{z_k} f(x_j, y_j, z)dz = F_k (M_j),\; j = \overline{1, l}  \label{lec14:4}
		\end{equation}
		
		где $F_k(x, y) = \int\limits_{z_k - 1}^{z_k} f\left(x, y, z_k\right)dz, k = \overline{1, m}$
		 
		По теореме о среднем для ОИ $\exists t_{k_{j}} \in \left[z_{k - 1}, z_k \right], k = \overline{1, m}$ такие, что:
		 
		 \begin{equation}
		 	I_{K_j} = f\left(x_j, y_j, t_{k_j}\right)\left(z_k - z_{k - 1}\right) = F\left(N_{k_j}\right) \Delta z_k \label{lec14:5}
		 \end{equation}
		
		где $\begin{cases}
		N_{k_j} = (x_j, y_j, t_{j_j}) = (M_j, t_{k_j})\\
		\Delta z_k = z_k - z_{k - 1}, k = \overline{1, m}, j = \overline{1, l}
		\end{cases}$
		
		Отсюда $\widetilde{d_0} = \diam \widetilde{P_0} \to 0$ имеем: 
		
		\begin{equation}
		\int\limits_D \int F\left(x, y\right) dxdy = \lim\limits_{\widetilde{d_0} \to 0} \widetilde{\sigma_0} \label{lec14:6}
		\end{equation}
		
		где $\widetilde{\sigma_0} = \sum\limits_{j = 1}^{l} F\left(M_j\right) \Delta D_j = \left(\int\limits_{z_0 = p}^{z_1} + \ldots + \int\limits_{z_m - 1}^{z_m = q} \right) = \sum\limits_{j = 1}^{l} \sum\limits_{k = 1}^{m} \left(\int\limits_{z_k - 1}^{z_k} f(M_j, z)dz\right) \Delta D_j = $
		
		\begin{equation}
		= \sum\limits_{j = 1}^{l} \sum\limits_{k = 1}^{m} I_{k_j} \Delta D_j = \sum\limits_{j = 1}^{l} \sum\limits_{k = 1}^{m} f\left(N_{k_j}\right) \Delta z_k \Delta D_j \label{lec14:7}
		\end{equation}
		
		где $\Delta D_j = \mes D_j = S_{D_j}$.\\
		
		\eqref{lec14:7} является специальной интегральной суммой для $\sigma_0$ для \eqref{lec14:2}, построенной на разбиении $P = \left\{ H_k \right\}, k = \overline{1, m}$ на множестве специально отмеченных точек $Q = \left\{ M_{k_j}\right\}$ где 
		\[
		\begin{cases}
		H_{k_j} = D_j x \left[t_{k - 1}, z_k\right]\\
		N_{k_j} = (x_j, y_j, t_{k_j})
		\end{cases} k = \overline{1, m}, \ j = \overline{1, l}
		\]
		
		Из непрерывности исходной функции предел этих специальных интегральных сумм рассмотренного 3И будет такой же, что и у общей интегральной суммы, поэтому при $d = \diam P \to 0 \implies 0 \leq \diam \widetilde{P} < d \implies d \to 0$ для рассматриваемого кубируемого компакта $H \subset \R^3 \implies$
		
		\begin{equation}
		I = \lim\limits_{d \to 0} \sigma_0 = \int_{D} \int F\left(x, y\right) dxdy = \int\limits_D \int \left(\int\limits_{p}^{q} F\left(x, y\right) dz\right) dxdy \label{lec14:9}
		\end{equation}
		
	\end{proof}
	
	\begin{rems}
		\quad
		\begin{enumerate}
			\item По аналогии с 2И, 3И на практике записывается в виде повторных интеграллов:
			
			\begin{equation}
			\iiint\limits_{H} f \left( x, y, z \right) dx dy dz = \int\limits_{D}\int dx dy \int\limits_{p\left(x, y\right)}^{a\left(x, y \right)} f\left(x, y, z\right) dz \label{lec_14, num_1}
			\end{equation}
			
			\item Если кубируемый компакт $H \subset \R$ является цилиндром вдоль $Oz$, т.е. 
			\[H = \left\{\left(x, y, z \right) \in \R^3 \; \vline \; p \left(x, y\right) \leq z \leq q \left(x, y \right) ; \left(x, y\right) \in \R^2 \right\} \]
			где $p\left(x, y\right)$ и $q\left(x, y\right)$ непрерывны на квадрируемом компакте $D \subset \R^2$, является проекцией $H$ на $Oxy$, то тогда для $f \in C\left(H\right)$, как и для 2И получаем:
			\[\iiint\limits_{H} f \left( x, y, z \right) dx dy dz = \int\limits_{D}\int dx dy \int\limits_{p\left(x, y\right)}^{a\left(x, y \right)} f\left(x, y, z\right) dz\]
			
			\item Если кубируемый компакт $H \subset \R^3$ является цилиндром вдоль $Oz$, то в случае \eqref{lec_14, num_1} для квадрируемого компакта $D \subset \R^2$, является криволинейное трапецией вдоль $Oy$, т.е.
			\[D = \left\{\left(x, y\right) \in \R^2 \; \vline \; c \left(x\right) \leq y \leq d \left(x \right), a \leq x \leq b \right\} \]
			где $c(x)$ и $d(x)$ непрерывна на $\left[a, b\right]$ имеем:
			\begin{equation}
			\iiint\limits_{\begin{cases}
				p\left(x, y\right) \leq z \leq q\left(x, y\right)\\
				c\left(x\right) \leq y \leq d\left(x\right)\\
				a \leq x \leq b
				\end{cases}} f \left( x, y, z \right) dx dy dz = \int\limits_{a}^{b} dx \int\limits_{d\left(x\right)}^{c\left(x\right)} dy \int\limits_{q\left(x, y\right)}^{c\left(x, y\right)} f(x, y, z) dz \label{lec_14, num_2}
			\end{equation}
			где также предполагается, что $p\left(x, y\right)$ и $q\left(x, y\right)$ непрерывны на $D$.
			Получили представление 3И через соответствующие повторные интегралы.
			
			\item Формула \eqref{lec_14, num_2} естественным образом обобщается на случай соответствующих цилиндроидов и их проекций вдоль других координатных осей (всего получаем $3! = 6$ возможностей). В общем случае, когда кубируемый компакт $H \subset R^3$ является объединением конечного числа таких цилиндроидов, вычисление 3И сводится к вычислению соответсвующих интегралов по элементарным цилиндроидам вдоль каких-либо осей.  
		\end{enumerate}
	\end{rems}

	\begin{example}
		Для непрерывной на $\left[0, 1\right]$ функции $g\left(z \right)$ рассмотрим 3И, заданный в виде повторных интегралов:
			\[I_0 = \int\limits_{-1}^1dx\int\limits_{-\sqrt{1 - x^2}}^{\sqrt{1 - x^2}}dy\int\limits_{x^2 + y^2}^1 g\left(z\right)dz.\]
		
		В данном случае имеем
		
		\[I_0 = \iiint\limits_{H} g(z) dxdydz\]
		
		Где $H$ компакт в $\R^3$, ограниченный в пространстве параболлой $z = x^2 + y^2$ и плоскостью $z = 1$, проекцией которой на плоскость $Oxy$ является круг $D: x^2 + y^2 \leq 1$
		
		\begin{center}
		\includegraphics[scale = 0.7]{lec14.jpg}
		\end{center}
		
		Рассмотрим $H$ как составную часть цилиндроида вдоль $Oy$, имеем:
		
		\[H = \left\{\left(x, y, z\right) \in \R^3 \vline -\sqrt{z - x^2} \leq y \leq \sqrt{z - x^2}, -\sqrt{z} \leq x \leq \sqrt{z} \right\}\]
		
		Переходя к повторным пределам к повторным пределам имеем:
		
		\[I_0 = \int\limits_0^1dz\int\limits_{-\sqrt{z}}^{\sqrt{z}}dx\int\limits_{-\sqrt{z - x^2}}^{\sqrt{z - x^2}}g\left(z\right)dy = 2\int\limits_0^1dz\int\limits_{-\sqrt{z}}^{\sqrt{z}}g\left(z\right)\sqrt{z - x^2}dx \]
		
		\[\left[\int\sqrt{z - x^2}dx = \frac{x}{2}\sqrt{z - x^2} + \frac{z}{2}\arcsin{\frac{x}{\sqrt{z}}} + C\right]\]
		
		\[I_0 = 4 \int\limits_0^1 g\left(z\right) \left[ \frac{x}{2}\sqrt{z - x^2} + \frac{z}{2}\arcsin{\frac{x}{\sqrt{z}}}\right]_{x=0}^{x=\sqrt{z}}dz= 2 \int\limits_0^1 g\left(z\right) \left[x\sqrt{z - x^2} + z\arcsin{\frac{x}{\sqrt{z}}}\right]_{x=0}^{x=\sqrt{z}}dz\]
		
		\[I_0 = 2 \int\limits_0^1 g\left(z\right) \left(\sqrt{z} \cdot \sqrt{z-z} + z \cdot \arcsin{\frac{\sqrt{z}}{\sqrt{z}}}\right) dz = 2 \int\limits_0^1 g\left(z\right) \cdot z \cdot \frac{\pi}{2} dz = \pi \int\limits_0^1 z \cdot g\left(z\right)dz\]  
		
	\end{example}

	\section{Замена переменных в $n$-кратном интеграле}
	
	Рассмотрим $D \subset \R^n$ и $G \subset \R^n$. Множество $D$ будем 
	рассматривать в ПДСК\\ $Ox_1 \ldots x_n$, а $G$ ~--- в $Ot_1 \ldots t_n$.
	Отображение 
	\begin{equation}
	\label{lec14:11}
	f : G \to D
	\end{equation} при котором 
	$\forall t = \left( t_1, \ldots, t_n \right) \in G$ $\exists ! 
	x = f\left(t\right) = \left(f_1\left( t\right) , \ldots, f_n\left( t\right)  \right)  \in D$
	будет взаимно однозначным, если оно биективно. В этом случае 
	\begin{equation}
	\label{lec14:12}
	\exists ! g :  D \to G
	\end{equation}
	при котором $\forall x \in D \Rightarrow  \exists ! t = g\left( x\right) $ такое, что 
	$f\left( t\right)  = x,\; g = f^{-1}$.
	
	Если во взаимно однозначных отображениях \eqref{lec14:11}, \eqref{lec14:12}
	используется непрерывно дифференцируемая функция, то говорят что множества $D$ и $G$ ~--- \emph{дифеоморфны}, и в этом случае 
	\eqref{lec14:11}, \eqref{lec14:12} называются \emph{дифеоморфными} отображениями
	или \emph{дифеоморфизмами}.
	
	Для дифеоморфизма \eqref{lec14:11} его якобианом является 
	
	\begin{equation}
	\label{lec14:13}
	I(t) = \det \frac{\partial f}{\partial t}=
	\begin{vmatrix}
	\smallskip
	\frac{\partial f_1\left(t\right)}{\partial t_1} & \frac{\partial f_1\left(t\right)}{\partial t_2}
	& \cdots & \frac{\partial f_1\left(t\right)}{\partial t_n} \\
	\frac{\partial f_2\left(t\right)}{\partial y_1} & \frac{\partial f_2\left(t\right)}{\partial t_2} 
	& \cdots & \frac{\partial f_2\left(t\right)}{\partial t_n} \\
	\vdots  & \vdots  & \ddots & \vdots  \\
	\frac{\partial f_n\left(t\right)}{\partial t_1} & \frac{\partial f_n\left(t\right)}{\partial t_2}
	& \cdots & \frac{\partial f_n\left(t\right)}{\partial t_n}
	\end{vmatrix} \ne 0
	\end{equation}
	
	
	Для \eqref{lec14:12} его якобиан будем обозначать 
	
	\begin{equation}
	\label{lec14:14}
	J(x) = \det \frac{\partial g}{\partial x}=
	\begin{vmatrix}
	\smallskip
	\frac{\partial g_1\left(x \right)}{\partial x_1} & \frac{\partial g_1\left(x \right)}{\partial x_2}
	& \cdots & \frac{\partial g_1\left(x \right)}{\partial x_n} \\
	\frac{\partial g_2\left(x \right)}{\partial x_1} & \frac{\partial g_2\left(x \right)}{\partial x_2} 
	& \cdots & \frac{\partial g_2\left(x \right)}{\partial x_n} \\
	\vdots  & \vdots  & \ddots & \vdots  \\
	\frac{\partial g_n\left(x \right)}{\partial x_1} & \frac{\partial g_n\left(x \right)}{\partial x_2}
	& \cdots & \frac{\partial g_n\left(x \right)}{\partial x_n}
	\end{vmatrix} \ne 0
	\end{equation}
	
	Используя свойства определителя и Теоремы о дифференцировании сложных ФНП, можно
	показать, что якобианы \eqref{lec14:13}, \eqref{lec14:14} для дифеоморфизмов \eqref{lec14:11}, \eqref{lec14:12}
	удовлетворяют условию $I(t) J(x) = 1$.
	Отсюда следует, что для дифеоморфизмов \eqref{lec14:13}, \eqref{lec14:14} 
	их якобианы ненулевые. Можно показать, что 
	при дифеоморфизме \eqref{lec14:11} для мер образа $G$ и прообраза $D\Rightarrow$
	
	\begin{equation}
	\label{lec14:15}
	mes D = \int\limits_G \left| 
	I(t) \right|dt = \iint\limits_G \ldots \int\limits \left| 
	I\left( t_1, \ldots, t_n\right) \right| dt_1\ldots dt_n
	\end{equation} 
	
	Аналогично при дифеоморфизме \eqref{lec14:12} :
	
	\begin{equation}
	\label{lec14:16}
	mes D = \int\limits_D \left| 
	I(x) \right| dx = \iint\limits_D \ldots \int\limits \left| 
	I\left( x_1, \ldots, x_n\right) \right| dx_1\ldots dx_n
	\end{equation}
	
	Обоснование этих формул в случае $n = 2$ и $n = 3$ будет 
	сделано позже. На основании этих формул справедлива :
	
	\begin{thm}[о замене неравенство в $n$-кратном интеграле]
		Если для дифеоморфизма \eqref{lec14:11} и функции $h(x), x\in D\;
		\exists$ непрерывная композиция
		\[
		\left( F \circ f \right) (t) = F(f(t)) = h(f_1(t), \ldots, f_n(t)), 
		t \in G \subset \R^n
		\]
		то тогда интеграл по $D$:
		
		$\int\limits h(x) dx = \left[\begin{array}{l}
		x = f(t) - \text{непрерывно дифференцирумая}\\
		\text{на } D \Rightarrow
		I(t) \neq 0 = dx = I(t) dx 
		\end{array} \right] =$
		
		\begin{equation}
		\int\limits_G h(f(t)) \left| I(t) \right| dt = \iint\limits_{G}
		\ldots \int\limits h(f_1(t),\ldots f_n(t))\left| I(t) \right| dt_1\ldots dt_n
		\text{, где } t = \left( t_1, \ldots, t_n \right)\in G \subset \R^n 
		\label{lec14:17}
		\end{equation}
		
	\end{thm}


\end{document}