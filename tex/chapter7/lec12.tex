\makeatletter
\def\input@path{{../../}}
\makeatother
\documentclass[../../main.tex]{subfiles}

\begin{document}
\section{Измеримые множества в $\R^n$\\}

Для $a_i, b_i \in \R, a_i < b_i, i = \overline{1,n}$ 
прямоугольным параллелепипедом
$\Pi \subset \R^n$ будем называть множеств~---о точек: 

\[
	\{x = (x_1, x_2, \ldots, x_n) \in \R^n | a_i < x_i < b_i, i =
	 \overline{1,n}\} = [a_1; b_1] \cdot [a_2; b_2] \cdot \dots \cdot [a_m; b_m]
\]

Для таких параллелепипедов $\Pi\in \R^n$ за меру $mes\Pi$ примем число:\\
\begin{equation}
\label{611}
mes\Pi = (b_1 - a_1) \cdot (b_2 - a_2) \cdot \ldots \cdot (b_n - a_n).
\end{equation}

\begin{example}
	\begin{enumerate}
		\item Для $\Pi \subset \R \implies \Pi = \{x | a \le x \le b \} = [a, b] $\\
		$mes\Pi = b - a$~--- длина. 
		\item Для $\Pi \subset \R^2 \implies \Pi = 
		\{(x,y)| a \le x \le b, c \le y \le d \} = [a, b] \cdot [c, d]$\\
		$mes\Pi = (b - a)(d - c)$~--- площадь. 
		\item Для $\Pi \subset \R^3 \implies \Pi = 
		\{(x,y,z) a \le x \le b, c \le y \le d, e \le z \le h \} = 
		[a, b] \cdot [c, d] \cdot [e, h] $\\
		$mes\Pi = (b - a)(d - c)(h - e)$~--- объем. 
	\end{enumerate}
\end{example}

Многогранником $H \subset \R^n$ будем называть произвольное 
конечное объединение параллелепипедов в $\R^n$.

Если этот многогранник разбит на конечное число  составляющих параллелепипедов 
$H_k, k = \overline{1,m}$, у каждого из которых общими могут быть лишь
граничные точки, т.е. $\linebreak$
$H = \bigcup\limits_{k=1}^m H_k$, то в этом случае полагают:

\begin{equation}
\label{mes_sum}
mesH = \sum\limits_{k=1}^{m}  mesH_k.
\end{equation}

Можно показать, что \eqref{mes_sum} не зависит от способа разбиения H 
на составляющие его параллелепипеды.

Дальнейшая теория меры для множеств из $\R^n$ строится по тем же
схемам, что и квадрирование фигур в $\R^2$ и кубирование тел в $\R^3$.

Для произвольного $D \subset \R^n$ многогранник P считается вписанным в $D$, 
если $P \subset D$. Аналогично, многогранник Q 
считается  описанным около $D$, если $D \subset Q$.

Величины
\begin{equation}
\begin{cases}
m_* = \underset{\forall P \subset D}{\sup} \{ mesP \}, \\
m^* = \underset{\forall Q \supset D}{\inf} \{ mesQ \},
\end{cases}
\end{equation}
называются соответственно нижней мерой для D и верхней мерой для D.
По теореме о гранях, эти меры существуют (конечные или бесконечные).

В случае, когда $ m^* = m_{*} \in \R $,
множество $ D $ называется измеримым по Жордану в пространстве $\R^n$ 
и за его меру принимается $mesD = m^* = m_{*}$.
\\\\
Необходимым условием существования конечной меры (измеримости) по Жордану 
является ограниченность рассматриваемого множества $D \subset \R^n$. 
В дальнейшем считаем, что для пустого множества $mes\emptyset = 0$.
В дальнейшем множество $D_0 \subset \R^n$ будем называть множеством меры нуль, 
если для $\forall\; \varepsilon > 0 \exists P_\varepsilon \subset D_0$ 
и $\exists Q_\varepsilon \supset D_0$, такие, что

\begin{equation}
\label{614}
mesQ_\varepsilon - mesP_\varepsilon \le \varepsilon,
\end{equation}

Для множеств меры нуль имеем следующие свойства:
\begin{enumerate}
	\item Пустое множество, одноточечное множество и множество, 
	состоящее из конечного числа точек в $\R^n$, является множеством меры нуль.
	\item Объединение конечного или счётного числа множеств меры нуль 
	в $\R^n$ будем множеством меры нуль.
	\item Любое подмножество множества меры нуль имеет нулевую меру.
\end{enumerate}

Из этих свойств следует, что критерием измеримости множества $D \subset \R^n$ 
является его ограниченность и условие, что мера границы равна нулю
 ($ mes\partial D = 0 $).

В общем случае для $\forall D_1, D_2 \subset \R^n$ (любых измеримых) имеем:

\begin{equation}
\label{gg}
mes (D_1 \cup D_2) = mes D_1 + mesD_2 - mes(D_1 \cap D_2).
\end{equation}

Если в \eqref{gg} мера пересечения $mes(D_1 \cap D_2) = 0$, то имеем:
\[
	mes(D_1 \cup D_2) = mesD_1 + mesD_2,
\]
что выражает свойство конечной аддитивности меры Жордана в $\R^n$.

Аналогично, если мера $mes(D_i \cap D_j) = 0, 
\forall i \ne j, i = \overline{1,m}, j = \overline{1,m}$, 
то по ММИ доказывается, что

\begin{equation}
\label{mes_equity}
mes\left(\bigcup_{k=1}^m D_k \right) = \sum_{k=1}^{m} mesD_k.
\end{equation}

Равенство \eqref{mes_equity} справедливо для измеримых множеств 
в $D_k \in \R^n, k = \overline{1,m}$ в случае, 
когда у них общими могут быть лишь граничные точки.

\section{Интегральные суммы и интеграл ФНП}

Рассмотрим произвольное разбиение $P = \{P_k\}$, ${k = \overline{1, m}}$,
где $\forall P_k \subset D$ измеримы в $D$, и для которых:

\begin{enumerate}
	\item $\bigcup\limits_{k = 1}^m P_k = D$
	\item $mes(P_i \cap P_j) = 0, i \neq j; i, j = \overline{1, m}$
\end{enumerate}

Выбирая произвольным образом отмеченное множество точек 
$Q = \{M_k\}, \forall M_k \in P_k, k = \overline{1, m}$,
для ФНП $f(x) \forall x \in D \subset \R^n $ рассмотрим интегральную сумму:

\begin{equation}
\sigma = \sigma(f, \{P; Q\}) = \sum\limits_{k = 1}^mf(M_k)\Delta{P_k},
\end{equation}
где $\Delta{P_k} = mes P_k, k = \overline{1, m}$.

В дальнейшем величину $d_k = \max d(A; B)$, где $ \forall A, B \in P_k$,
будем называть диаметром множества $P_k \subset D$. 
Геометрически $d_k$ представляет собой диаметр $n$-мерного шара
наименьшего размера, содержащего $P_k$.

Величину ${d_0 = \max r_k}$, ${1 \le k \le m}$, 
будем называть диаметром используемого разбиения и
обозначать $d_0 = diam P$.

Функция $f(x), x \in D \subset \R^n$- измеримое
множество, считается интегрируемой на $D$, если
\begin{equation}
\label{7.8}
\exists I \subset \R^n,
\forall \varepsilon > 0, \exists \delta > 0, \forall \{P; Q\},
diam P \le \delta \implies \abs{I  - \delta} \le \varepsilon.
\end{equation}

Число $ I \in \R $ в \eqref{7.8}
будем называть значением $n$-кратного интеграла от 
$f(x)$ по $D \subset \R^n$ и записывать:

\begin{equation}
	I = \underset{D}{\iint\ldots\int} f(x_1, \ldots, x_n) d x_1 \ldots dx_n =
	 \int\limits_{D}f(x)dx
\end{equation}


\begin{example}
	Пусть $f(x) = 1, \forall x \in D \subset \R^n$. В этом случае \\
	\begin{equation}
	\label{7.10}
	\sigma = \sum\limits_{k = 1}^m\Delta{P_k} = 
	\sum\limits_{k=1}^mmesP_k = mes(\bigcup\limits_{k = 1}^m P_k) = mesD
	\implies \int\limits_D dx = mesD = \lim\limits_{d_0 \to 0} \sigma
	\end{equation}
	
	Доказательство \eqref{7.10} соответствует 
	геометрическому смыслу n-кратного интеграла.\\
	
	В случае $n = 1$, \eqref{7.10} дает длину отрезка на $[a, b]$.\\
	В случае, $n = 2$, \eqref{7.10} соответствует площади $S = mesD$~--- 
	измеримого множества $D \subset \R^2$.\\
	В случае, $n = 3$, \eqref{7.10} соответствует объему $V = mesD$~---
	кубируемого тела из $D \subset \R^3$.\\
	
	Далее, в случае $n=2$, для измеримого $D \subset \R^2$ в ДПСК 
	$O_{xy}$ для Ф2П $f(x, y)$ получаем двойной интеграл:
	\[
		\underset{D}{\int\int}f(x, y)dxdy
	\]
	В частности, $\underset{D}{\int\int}f(x,y)dxdy = mesD$~--- площадь $D$ 
	
	Далее, в случае $n=3$, для Ф2П $f(x, y, z)$ получаем тройной интеграл:
	\[
	\underset{D}{\int\int\int}f(x,y,z)dxdydz
	\]
	В частности, $\underset{D}{\int\int\int}f(x,y,z)dxdydz = mesD$~--- объем $D$ 
\end{example}


Как и для однократного интеграла ($n = 1$), множество интегрируемых на 
$D \subset \R^n$ функций будем обозначать
$\R(D)$. Имеем следующие основные свойства $n$~--- кратных интегралов:
\begin{enumerate}
	\item Линейность\\
	Если $f, g \in \R(D)$для измеримого $D \subset \R^n$, то
	
	\[
			\forall \lambda, \mu \in \R \implies (\lambda f + \mu g) \in \R(D)
	\]
	\[
			\int\limits_D (\lambda f(x) + \mu g(x))dx = \lambda \int\limits_D f(x)dx +
			\mu \int\limits_D g(x)dx.
	\]
	
	\item Аддитивность\\
	Если $f \subset \R(D), D = D_1 \bigcup D_2$, 
	где $D_1$, $D_2$~--- измеримые множества в $\R^n$,
	общими у которых могут быть лишь только граничные точки 
	(${ mes(D_1 \cap D_2) = 0 }$), то:
	
	\[
		\int\limits_{D}fdx = \int\limits_{D_1}fdx + \int\limits_{D_2}fdx.
	\]
		
	\item Монотонность и неотрицательность\\
	Пусть $f, g \in \R(D)$ и 
	$\forall x \in D \subset \R^n \implies f(x) \ge g(x)$. Тогда:
	
	\[
	\int\limits_Df(x)dx \le \int\limits_Dg(x)dx.
	\]
	
	Отсюда, учитывая, что $\int\limits_D0dx=0$ получаем, что в случае, когда 
	$\forall x \in D \implies g(x) \ge 0$, то $\int\limits_Dg(x)dx \ge 0$, 
	что соответствует значению n-кратного интеграла.
	
	\item Основная оценка\\
	Пусть $\forall x \in D \subset \R^n \implies m_0 \le f(x) \le M_0$, где
	$m_0, M_0 \in \R$, а $D$~--- измеримое множество в $\R^n$. Тогда:
	
	\[
		m_0 \cdot mesD \le \int\limits_D f(x) dx \le M_0 \cdot mesD.
	\]
\end{enumerate}


Обоснование всех этих свойств проводится по той же схеме, 
что и для однократного интеграла Римана.

Аналогично Ф1П обосновываются критерий Гейне 
и критерий Коши интегрируемости ФНП,
а также $N$-лемма для критерия Коши интегрируемости ФНП. 
На основании этого, так же, как и для
однократных интегралов, доказывается
\begin{thm}[об интегрируемости непрерывных ФНП]
	Если $f \in C(D)$, т.е. непрерывна на измеримом компакте
	 $D \subset \R^n$, то $f \in \R(D)$,
	т.е. интегрируема на $D$.
\end{thm}

\begin{crl*}[теорема о среднем для ФНП]
	Пусть $f \in C(D)$, а $g \subset \R(D)$, где $D \subset \R^n$~---
	 измеримый компакт. Если функция $g(x)$ сохраняет один и 
	 тот же знак $\forall x \in D $, то $\exists x_0 \in D \subset \R^n \implies$
	
	\begin{equation}
	\label{integral-formula}
	\int\limits_D f(x)g(x)dx =
	f(x_0)\int\limits_Dg(x)dx.
	\end{equation}
	
	Доказательство по той же схеме, что и для Ф1П.
\end{crl*}


\begin{rem}
	Если взять в \eqref{integral-formula} 
	$g(x) \equiv 1, \forall x \in D$, то получаем:
	\[
		\text{если } f \in C(D), 
		\text{ то }\exists x_0 \in D 
		\text{ такое, что } \int\limits_Df(x)dx = f(x_0) mes D,
	\]
	Это равенство соответствует простейшей теореме о 
	среднем для непрерывной ФНП.
\end{rem}
	
\end{document}
