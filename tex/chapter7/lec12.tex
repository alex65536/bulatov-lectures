\makeatletter
\def\input@path{{../../}}
\makeatother
\documentclass[../../main.tex]{subfiles}

\begin{document}
\section{Измеримые множества в $\R^n$}

Для $a_i, b_i \in \R,\ a_i < b_i,\ i = \overline{1,n}$ 
\emph{прямоугольным параллелепипедом}
$\Pi \subset \R^n$ будем называть множество точек
\[
	\Pi = \{x = (x_1, x_2, \ldots, x_n) \in \R^n\ \vline\ 
	a_i \le x_i \le b_i,\ i =
	 \overline{1,n}\} = [a_1; b_1] \times [a_2; b_2] 
	 \times \dots \times [a_n; b_n]
\]

Для таких параллелепипедов $\Pi\subset \R^n$ за 
\emph{меру} $\mes\Pi$ примем число
\begin{equation}
\label{611}
\mes\Pi = (b_1 - a_1) \cdot (b_2 - a_2) \cdot \ldots \cdot (b_n - a_n).
\end{equation}

(\emph{Примечание редактора:} эту меру называют \emph{мерой Жордана}).

\begin{example}
    ~
    
	\begin{enumerate}
		\item Для $\Pi \subset \R \implies \Pi = \{x\ |\ 
		a \le x \le b \} = [a, b] $\\
		$\mes\Pi = b - a$~--- длина. 
		\item Для $\Pi \subset \R^2 \implies \Pi = 
		\{(x,y)\ |\ a \le x \le b,\ c \le y \le d \} = [a, b] \times [c, d]$\\
		$\mes \Pi = (b - a)(d - c)$~--- площадь. 
		\item Для $\Pi \subset \R^3 \implies \Pi = 
		\{(x,y,z)\ |\ a \le x \le b,\ c \le y \le d,\ e \le z \le h \} = 
		[a, b] \times [c, d] \times [e, h] $\\
		$\mes\Pi = (b - a)(d - c)(h - e)$~--- объем. 
	\end{enumerate}
\end{example}

\emph{Многогранником} $H \subset \R^n$ будем называть произвольное 
конечное объединение параллелепипедов в $\R^n$.
Если этот многогранник разбит на конечное число  составляющих параллелепипедов 
$H_k,\ k = \overline{1,m}$, т.~е.
\[H = \bigcup\limits_{k=1}^m H_k,\]
у каждого из которых общими могут быть лишь
граничные точки, то в этом случае полагают:

\begin{equation}
\label{mes_sum}
\mes H = \sum\limits_{k=1}^{m}  \mes H_k.
\end{equation}

Можно показать, что \eqref{mes_sum} не зависит от способа разбиения $H$ 
на составляющие его параллелепипеды.

Дальнейшая теория меры для множеств из $\R^n$ строится по тем же
схемам, что и квадрирование фигур в $\R^2$ и кубирование тел в $\R^3$.

Для произвольного $D \subset \R^n$ многогранник $P$ 
считается \emph{вписанным} в $D$, 
если $P \subset D$. Аналогично, многогранник Q 
считается \emph{описанным} около $D$, если $D \subset Q$.

Величины
\begin{equation}
\begin{cases}
m_* = \underset{P \subset D}{\sup} \{ \mes P \}, \\
m^* = \underset{Q \supset D}{\inf} \{ \mes Q \}
\end{cases}
\end{equation}
называются соответственно \emph{нижней мерой} для $D$ и 
\emph{верхней мерой} для $D$.
По теореме о гранях, эти меры существуют (конечные или бесконечные).

В случае, когда $ m^* = m_* \in \R$,
множество $D$ называется \emph{измеримым по Жордану} в пространстве $\R^n$,
и за его меру принимается $\mes D = m^* = m_*$.

Необходимым условием существования конечной меры (измеримости) 
является ограниченность рассматриваемого множества $D \subset \R^n$. 
Для пустого множества по определению полагают $\mes\emptyset = 0$.
В общем случае, $D_0 \subset \R^n$ будем называть \emph{множеством меры нуль}, 
если для $\forall \eps > 0\ \exists H$~--- многогранник, $D_0\subset H$ и
\begin{equation}
\label{614}
\mes H\leq\eps
\end{equation}

Для множеств меры нуль имеем следующие свойства:
\begin{enumerate}
	\item Пустое множество, одноточечное множество и множество, 
	состоящее из конечного числа точек в $\R^n$, является множествами меры нуль.
	\item Объединение конечного или счётного числа множеств меры нуль 
	в $\R^n$ является множеством меры нуль.
	\item Любое подмножество множества меры нуль является множеством меры нуль.
\end{enumerate}

Из этих свойств следует, что критерием измеримости множества $D \subset \R^n$ 
является его ограниченность и условие, что мера его границы равна нулю
 ($ \mes\partial D = 0 $).

В общем случае для любых измеримых $D_1, D_2 \subset \R^n$ имеем

\begin{equation}
\label{lec7-1-5}
\mes (D_1 \cup D_2) = \mes D_1 + \mes D_2 - \mes(D_1 \cap D_2),
\end{equation}

Если в \eqref{lec7-1-5} мера пересечения $\mes(D_1 \cap 
D_2) = 0$, что, например, имеет место, когда у $D_1$ и 
$D_2$ общими могут быть лишь граничные точки, имеем
\[
	\mes(D_1 \cup D_2) = \mes D_1 + \mes D_2,
\]
что выражает свойство \emph{конечной аддитивности} меры Жордана в $\R^n$.

Аналогично, если $\mes(D_i \cap D_j) = 0\quad 
\forall i \ne j,\ i = \overline{1,m},\ j = \overline{1,m}$, 
то по ММИ доказывается, что

\begin{equation}
\label{mes_equity}
\mes\left(\bigcup_{k=1}^m D_k \right) = \sum_{k=1}^{m} \mes D_k.
\end{equation}

Равенство \eqref{mes_equity} справедливо для измеримых множеств 
$D_k \subset \R^n,\ k = \overline{1,m}$ в случае, 
когда у них общими являются лишь, может быть, граничные точки.

\section{Интегральные суммы и интеграл ФНП}

Рассмотрим произвольное разбиение $P = \{P_k\}$, ${k = 
\overline{1, m}}$ измеримого множества $D$
на измеримые части $P_k \subset D,\ k = 
\overline{1, m}$, для которых:

\begin{enumerate}
	\item $\bigcup\limits_{k = 1}^m P_k = D$,
	\item $\mes(P_i \cap P_j) = 0,\ i \neq j,\ i = 
	\overline{1, m},\ j = \overline{1, m}$.
\end{enumerate}

Выбирая произвольным образом отмеченное множество точек 
$Q = \{M_k\}\quad \forall M_k \in P_k,\ k = \overline{1, m}$,
для ФНП $f(x),\ \forall x \in D \subset \R^n $ 
рассмотрим \emph{интегральную сумму}

\begin{equation}
\sigma = \sigma(f, \{P; Q\}) = \sum\limits_{k = 1}^mf(M_k)\Delta{P_k},
\end{equation}
где \[\Delta{P_k} = \mes P_k,\ k = \overline{1, m}.\]

В дальнейшем величину $d_k = \underset{\forall A, B \in 
P_k}\sup \{d(A; B)\}$
будем называть \emph{диаметром} множества $P_k \subset 
D$, $k = \overline{1, m}$. 
Геометрически $d_k$ соответствует диаметру $n$-мерного шара
наименьшего размера, содержащего $P_k$.

Величину ${d_0 = \underset{1 \le k \le m}\max \{d_k\}}$ 
будем называть \emph{диаметром} используемого разбиения и
обозначать $d_0 = \diam P$.

Функция $f(x)$, $x \in D \subset \R^n$, где $D$~--- измеримое
множество, считается \emph{интегрируемой} на $D$, если
\begin{equation}
\label{7.8}
\exists I \in \R \quad
\forall \eps > 0 \quad \exists \delta > 0 \quad 
\forall \{P; Q\} \ 
\diam P \le \delta \implies \abs{I  - \sigma} \le \eps.
\end{equation}

Число $ I \in \R $ в \eqref{7.8}
будем называть \emph{значением $n$-кратного интеграла} от 
$f(x)$ по $D \subset \R^n$ и записывать:

\begin{equation}
	I \stk{7.8} = \lim_{\diam P\to 0} \sum_{k=1}^m f(M_k) \Delta P_k
	= \underset{D}{\idotsint} f(x_1, \ldots, x_n) d x_1 \ldots dx_n =
	 \int\limits_{D}f(x)dx
\end{equation}


\begin{example}
	Пусть $f(x) = 1\ \forall x \in D \subset \R^n$. В этом случае
	\begin{equation}
	\label{7.10}
	\sigma = \sum\limits_{k = 1}^m\Delta{P_k} = 
	\sum\limits_{k=1}^m \mes P_k =
	\mes\left(\bigcup\limits_{k = 1}^m P_k\right) = \mes D
	\implies \int\limits_D dx = \mes D = \lim\limits_{d_0 \to 0} \sigma
	\end{equation}
	
	Доказательство \eqref{7.10} соответствует 
	простейшему геометрическому смыслу $n$-кратного 
	интеграла.
	
	В случае $n = 1$ \eqref{7.10} дает длину отрезка на $[a, b]$.
	
	В случае $n = 2$ \eqref{7.10} соответствует площади $S = \mes D$ 
	измеримого множества $D \subset \R^2$.
	
	В случае $n = 3$ \eqref{7.10} соответствует объему $V = \mes D$
	кубируемого тела $D \subset \R^3$.
	
	Далее, в случае $n=2$, для измеримого $D \subset \R^2$ в ДПСК 
	$Oxy$ для Ф2П $f(x, y)$ получаем \emph{двойной интеграл}:
	\[
		\iint\limits_D f(x, y)\:dxdy.
	\]
	В частности, $\iint\limits_D dxdy = \mes D$~--- площадь $D$.
	
	Далее, в случае $n=3$, для Ф3П $f(x, y, z)$ получаем 
	\emph{тройной интеграл}:
	\[
	\iiint\limits_D f(x,y,z)\:dxdydz
	\]
	В частности, $\iiint\limits_D dxdydz = \mes D$~--- объем $D$.
\end{example}


Как и для однократного интеграла ($n = 1$), множество 
интегрируемых на измеримом множестве 
$D \subset \R^n$ функций будем обозначать
$R(D)$. Имеем следующие основные свойства $n$-кратных интегралов:

\begin{enumerate}
	\item Линейность:
	
	Если $f, g \in R(D)$ для измеримого $D \subset \R^n$, то
	
	\[
			\forall \lambda, \mu \in \R \implies (\lambda f + \mu g) \in R(D)
	\]
	\[
			\int\limits_D (\lambda f(x) + \mu g(x))dx = \lambda \int\limits_D f(x)dx +
			\mu \int\limits_D g(x)dx.
	\]
	
	\item Аддитивность:
	
	Если $f \subset R(D)$, где $D = D_1 \cup D_2$, 
	а $D_1$, $D_2$~--- измеримые множества в $\R^n$,
	$\mes(D_1~\cap~D_2) = 0$, то
	
	\[
		\int\limits_{D}f(x)dx = \int\limits_{D_1}f(x)dx + \int\limits_{D_2}f(x)dx.
	\]
		
	\item Монотонность и неотрицательность:
	
	Пусть $f, g \in R(D)$ на измеримом множестве $D$, то 
	в случае, когда 
	$\forall x \in D \subset \R^n \implies f(x) \le g(x)$, имеем
	
	\[
	\int\limits_Df(x)dx \le \int\limits_Dg(x)dx.
	\]
	
	Отсюда, учитывая, что $\int\limits_D0\:dx=0$ получаем, что в случае, когда 
	$\forall x \in D \implies g(x) \ge 0$, то $\int\limits_Dg(x)dx \ge 0$, 
	что соответствует неотрицательности $n$-кратного интеграла.
	
	\item Основная оценка:
	
	Пусть $\forall x \in D \subset \R^n \implies m_0 \le f(x) \le M_0$, где
	$m_0, M_0 \in \R$, а $D$~--- измеримое множество в $\R^n$. Тогда
	
	\[
		m_0 \cdot \mes D \le \int\limits_D f(x) dx \le M_0 \cdot \mes D.
	\]
\end{enumerate}


Обоснование всех этих свойств проводится по той же схеме, 
что и для однократного интеграла Римана.

Аналогично Ф1П обосновываются \emph{критерий Гейне} 
и \emph{критерий Коши} интегрируемости ФНП,
а также \emph{$N$-лемма} для критерия Коши интегрируемости ФНП. 
На основании этого, так же, как и для
интегралов Римана, доказывается

\begin{thm}[об интегрируемости непрерывных ФНП]
	Если $f \in C(D)$, т.~е. $f$ непрерывна на измеримом компакте
	 $D \subset \R^n$, то $f \in R(D)$,
	т.~е. интегрируема на $D$.
\end{thm}

\begin{crl*}[теорема о среднем для ФНП]
	Пусть $f \in C(D)$, а $g \in R(D)$, где $D \subset \R^n$~---
	 измеримый компакт. Если функция $g(x)$ сохраняет один и 
	 тот же знак $\forall x \in D $, то тогда $\exists 
	 x_0 \in D \subset \R^n \implies$
	
	\begin{equation}
	\label{integral-formula}
	\int\limits_D f(x)g(x)dx =
	f(x_0)\int\limits_Dg(x)dx.
	\end{equation}
\end{crl*}

Обоснование \eqref{integral-formula} проводится по той же 
схеме, что и для интеграла Римана.

\begin{rem}
	Если взять в \eqref{integral-formula} 
	$g(x) \equiv 1\ \forall x \in D$, то получаем
	\[
		\text{если } f \in C(D), 
		\text{ то }\exists x_0 \in D 
		\text{ такое, что } \int\limits_Df(x)dx = f(x_0) \mes D,
	\]
	Это равенство соответствует простейшей теореме о 
	среднем для непрерывных ФНП.
\end{rem}
	
\end{document}
