\makeatletter
\def\input@path{{../}}
\makeatother
\documentclass[../main.tex]{subfiles}

\begin{document}
 \chapter*{Вопросы для подготовки к экзамену}
 \addcontentsline{toc}{chapter}{Вопросы для подготовки к экзамену}
 
 \begin{enumerate}
  \item Пространство $\R^n$. Расстояние в $\R^n$. Неравенство Буняковского.
  \item Последовательности в $\R^n$. Сходимость. Основной критерий сходимости. 
  \item Функции $n$ переменных (ф.~$n$~п.). Предел. Критерий Гейне. Повторные 
  пределы.
  \item Непрерывность ф.~$n$~п. Непрерывность композиции.
  \item Основные теоремы о свойствах непрерывных ф.~$n$~п.
  \item Непрерывность по одной переменной. Равномерная непрерывность ф.~$n$~п. 
  Теорема Кантора.
  \item Дифференцируемость ф.~$n$~п. Частные производные (ЧП).
  \item Необходимые условия дифференцируемости. Достаточные условия 
  дифференцируемости.
  \item Дифференцируемость композиции. ЧП сложной функции. Инвариантность 
  формы первого дифференциала.
  \item Производные старших порядков. Теорема Шварца о равенстве смешанных 
  производных.
  \item Дифференциалы. Оператор $d$.
  \item Формула Тейлора.
  \item Теорема о неявной функции. Дифференцирование неявной функции. 
  \item Системы функциональных уравнений (СФУ). Теорема об однозначной 
  разрешимости СФУ. Алгоритм дифференцирования.
  \item Задача о локальном экстремуме. Необходимые условия. Достаточные 
  условия для ф.~$2$~п.
  \item Достаточные условия локального экстремума ф.~$n$~п.
  \item Зависимость функций. Признак независимости. Признак зависимости.
  \item Условный экстремум. Сведение к безусловному. Функция Лагранжа.
  \item Алгоритм решения задачи на условный экстремум. Глобальный экстремум.
  \item Квадратичные формы. Знакоопределенные квадратичные формы. Критерий 
  Сильвестра.
  \item Измеримые множества в $\R^n$.
  \item Интеграл в $\R^n$. Свойства. Замена переменных в интеграле.
  \item Определение 2И.  Вычисление 2И по прямоугольнику.
  \item 2И по трапеции. Замена переменных в 2И. Примеры.
  \item 3И. Определение. Свойства. Переход к сферическим и к цилиндрическим 
  координатам. Примеры.
  \item Приложения 2И и 3И. Примеры.
  \item Огибающая семейства плоских кривых. Дискриминантная кривая. Примеры.
  \item КрИ-1. Определение. Геометрический и механический смысл КрИ-1.
  \item Вычисление КрИ-1. Примеры.
  \item КрИ-2. Физический смысл КрИ-2. Связь с КрИ-1.
  \item Вычисление КрИ-2. Свойства КрИ-2. Примеры.
  \item Формула Грина.
  \item Основная теорема о КрИ.
  \item Вычисление первообразной для $Pdx+Qdy$. Формула Ньютона-Лейбница для 
  КрИ. Примеры.
  \item Поверхности в $\R^3$. Координатные кривые. Касательная плоскость. 
  Вычисление вектора нормали. Односторонние и двусторонние поверхности.
  \item Первая квадратичная форма поверхности. Площадь поверхности. Вычисление 
  площади поверхности.
  \item ПовИ-1. Вычисление ПовИ-1. Примеры.
  \item Свойства ПовИ-1. Приложение ПовИ-1.
  \item ПовИ-2. Связь ПовИ-1.
  \item Связь ПовИ-2 с 2И.
  \item Формула Остроградского. Построение формул для вычисления объема. 
  Примеры.
  \item Связь КрИ с ПовИ.
  \item Основная теорема о КрИ в $\R^3$. Вычисление первообразной для 
  $Pdx+Qdy+Rdz$.
  \item Числовой ряд. Сходимость. Сумма. Примеры. Необходимое условие 
  сходимости. Критерий Коши.
  \item Положительные ряды. Критерий сходимости положительного ряда. 
  Геометрический и гармонический ряды.
  \item Признаки сравнения. Примеры.
  \item Признаки Коши и Даламбера. Примеры.
  \item Интегральный признак. Обобщенный гармонический ряд. Степенной признак. 
  Примеры.
  \item Признаки Дюамеля и Гаусса. Примеры.
  \item Формула Валлиса.
  \item Формула Стирлинга.
  \item Знакопеременные ряды. Признак Лейбница. Примеры.
  \item Преобразование Абеля. Лемма Абеля.
  \item Признак Дирихле. Признак Абеля. Примеры.
  \item Абсолютная и условная сходимость ряда. Примеры. Критерий абсолютной 
  сходимости. Необходимое условие условной сходимости.
  \item Линейная комбинация рядов. Группировка в рядах.
  \item Перестановка членов ряда. Теоремы о допустимости перестановки.
  \item Перемножение рядов. Теорема Коши.
  \item Перемножение рядов по правилу Коши. Теорема Мертенса. Пример 
  расходимости произведения сходящихся рядов.
  \item Обобщенное суммирование рядов. Теорема о обобщенной сумме сходящегося 
  ряда. Необходимое условие сходимости ряда по методу Чезаро. Примеры 
  обобщенного суммирования.
 \end{enumerate}

\end{document}
