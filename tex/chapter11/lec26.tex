\makeatletter
\def\input@path{{../../}}
\makeatother
\documentclass[../../main.tex]{subfiles}

\graphicspath{
	{../../img/}
	{../img/}
	{img/}
}

\begin{document}
\section{Числовые ряды}
Рассмотрим последовательность
\[a_1,\  a_2,\  a_3,\  \ldots,\  a_n \in \R.\]
Используя её, построим последовательность
\[
\begin{array}{l}
		S_1 = a_1, \\
		S_2 = a_1 + a_2,\\
		S_3 = a_1 + a_2 + a_3,\\
		\dots\\
		S_n = \sum\limits_{k = 1}^na_k
\end{array}		
\]
Эту последовательность удобно записывать в виде:
\begin{equation}
	\label{lec_26, num_1}
	 \sum\limits_{k = 1}^\infty a_k
\end{equation}
Выражение вида \eqref{lec_26, num_1} называется
\emph{числовым рядом} (или просто \emph{рядом}).
Числа  $a_1, a_2,  \ldots, a_k,  \ldots$ называются \emph{элементами\slashчленами ряда}.

Рассмотрим $\lim\limits_{n \to \infty} S_n$. Если он существует и конечен, то говорят, 
что ряд \eqref{lec_26, num_1}  \emph{сходится}. 
В противном случае ---\emph{ расходится}.

Если предел конечен, т.~е. $\lim\limits_{n \to \infty} S_n = S \in \R$, то $S$ 
называют \emph{суммой ряда} и пишут:
\[
	\sum\limits_{k = 1}^\infty a_k=S
.\]

Сумму $S_n$ называют \emph{частной суммой ряда}. 
\emph{Сходимость ряда} означает,
что сходится последовательность его частных сумм.

\begin{example}
	 \[
		 \sum\limits_{k = 1}^\infty\frac{1}{k(k+1)} = 
		 \frac{1}{1\cdot 2} + \frac{1}{2\cdot 3} + \ldots
	\]
	\begin{gather*}
		S_n = \frac{1}{1\cdot 2} + \frac{1}{2\cdot 3} + \ldots +
		\frac{1}{n (n+1)} = \\
		= \left(1 - \frac{1}{2}\right) + 
		\left(\frac{1}{2} - \frac{2}{3}\right) + \ldots + 
		\left(\frac{1}{n} - \frac{1}{n+1}\right) =
		 1 - \frac{1}{n+1}
		\xrightarrow[n \to \infty]{}1
	\end{gather*}
	\[
		\lim\limits_{n \to \infty} S_n = 1 \implies \text{
		ряд сходится, его сумма --- 1}
	\]
\end{example}
\begin{example}
	 \[
		 \sum\limits_{k = 1}^\infty\frac{1}{2^{k-1}}
	\]
	\begin{gather*}
		S_n =  \sum\limits_{k = 1}^n\frac{1}{2^{k-1}} =
		1 + \frac{1}{2} + \frac{1}{4} + \ldots + \frac{1}{2^{n-1}} =\\
		=  \frac{ 1 - \frac{1}{2^n}}{1 - \frac{1}{2}}
		\xrightarrow[n \to \infty]{}2
		\implies \text{
		ряд сходится, его сумма --- 2}
	\end{gather*}
\end{example}
\begin{example}
	 \[
		 \sum\limits_{k = 1}^\infty(-1)^{k-1}
	\]
	\begin{gather*}
		S_1 = 1, \ S_2 = 0, \ S_3 = 1, \ S_4 = 0, \ \ldots
		\implies \nexists \lim\limits_{n \to \infty} S_n \implies \\
		\implies \text{ряд расходится}
	\end{gather*}
\end{example}

Сходимость ряда означает, что для достаточно большого $n$ выполняется
$\mid S -S_n \mid \leq \epsilon$ т.~е.
\[
	\forall\epsilon > 0\enspace \exists\nu = \nu_\epsilon \enspace
	\forall n\geq \nu \implies \  \mid S -S_n \mid \ \leq \epsilon, \ \text{ т.~е.}\ 
	\left|\sum\limits_{k = n_1}^\infty a_k\right| \leq \epsilon
.\]

Если отбросить конечное число первых членов ряда, например  
первые  $m$ членов, то получится новый ряд:

\[
	\sum\limits_{k = m + 1}^\infty a_k = a_{m+1} + a_{m+2} +  \ldots
	 \text{\ --- \emph{остаток ряда \eqref{lec_26, num_1} } }
.\]
Для $n > m$:\enspace $S_n = S_m + a_{m+1} + a_{m+2} + \ldots + a_n = 
S_m + \sum\limits_{k = m+1}^n a_k$

При $n\longrightarrow{}\infty$ предел слева существует и конечен $\iff$
 существует и конечен предел справа. 
 Т.~е. ряд \eqref{lec_26, num_1}  сходится $\iff$  
 сходится любой его остаток.
Также, если сходится ряд, то сходится и любой его остаток. 
Если сходится какой-либо остаток ряда, то сходится  и сам  ряд.

НЕОБХОДИМОЕ УСЛОВИЕ СХОДИМОСТИ РЯДА(!!)

Пусть ряд (1) сходится и $S$ --- его сумма. Тогда:

....

При .... получаем:

...

Если ряд сходится, то ... --- необходимое условие сходимости ряда(!!)

ПРИМЕР

Рассмотрим ряд:

... --- гармонический ряд(!!)
...
Перейдём к пределу при ...:

Предположим, что ряд сходится и имеет сумму $S$: 
... --- противоречие -> гармонический ряд расходится.

КРИТЕРИЙ КОШИ СХОДИМОСТИ ЧИСЛОВОГО РЯДА

Сходимость ряда -- это сходимость последовательностей его частных
сумм т.~е. выполняется критерий Коши для последовательности:

....


.... --- получаем критерий Коши для рядов:

... сходится ....

Пусть .... и .... --- два сходящихся ряда.

Ряд .... называют линейной комбинацией рядов(!!). 
Если ряды
сходятся, то сходится и их линейная комбинация.
 
 ДОК-ВО...
 
 Линейная комбинация расходящихся рядов может сходиться.
 Если один из рядов сходится, а другой расходится, то 
 сумма расходится.

Если ... сходится, то

... --- сходится

ПОЛОЖИТЕЛЬНЫЕ РЯДЫ

Будем рассматривать ряд ... . Такие ряды
называют положительными (!!).

Тогда .... -> ...->можно использовать критерий
 сходимости монотонной последовательности
 
 КРИТЕРИЙ СХОДИМОСТИ ПОЛОЖИТЕЛЬНЫХ РЯДОВ
 
 Положительный ряд сходится <-> последовательность 
 его частных сумм ограничена сверху
 
 ЗАМЕЧАНИЕ:
 
 Если  $S_n$ не ограничена сверху, то ... и пишут
 
 ....
 
 Рассмотрим ряд:
 
.... --- геометрический ряд
... .... если $q \geq 1$, то
.... -> ряд расходится
Если $0 < q < 1$, то ряд сходится
Если $q \geq 1$, то ряд  расходится

ПРИЗНАКИ СРАВНЕНИЯ

Мы рассматриваем положительные ряды.
Сделать вывод об поведении изучаемого ряда
можно, сравнивая его элементы с  элементами
другого (эталонного) ряда,  поведение (сходимость\slash
расходимость) которого известно.

ПРИЗНАК 1...:

Пусть ..., ... --- положительные ряды и пусть$a \leq cb \, \forall k$,
тогда:

a) если сходится $b_k$, то сходится и $a_k$

б) если расходится $a_k$, то рассходится и $a_k$

ДОК-ВО
а) Пусть $b_k$ сходится. Тогда ... 
(критерий сходимости положительных рядов(Link))
-> $a_k$ сходится (из критерия сходимости положительных рядов).

б) Пусть $a_k$ расходится, предположим, что $b_k$
сходится, но тогда $a_k$ сходится --- противоречие 
-> $b_k$ расходится.

ПРИМЕР

1) ... --- расходится по 1 призн.б)(!!)

2) ... ... ... ... сходится -> 1 призн. а) (!!)
...

ЗАМЕЧАНИЕ:

В признаке достаточно, чтобы $a_k \leq b_k \forall k \geq N$, т.~е.
признак работает для  остатков рядов $a_k$ и $b_k$

....
...

ПРИЗНАК 2:
 Пусть существует предел  ....
 
 a) ... сходится -> $a_k$ сходится.
 
 б) ... $b_k$ расходится -> ряд $a_k$ расходится
 
 ДОК-ВО(!!)
 
 a) Пусть ..., тогда для достаточно больших $k$ будем иметь ...,....  
 по признаку 1а (!!) из сходимости $b_k$ следует сходимость $a_k$.
 
 б) Пусть ..., тогда ...
 
 Пусть $b_k$ расходится, если предположить, что $a_k$ 
 сходидтся и по доказанному призн. 2а(!!) $b_k$ сходится ---
 противоречие  -> $a_k$ --- расходится.
 
 ПРИМЕР
 
 1) ... Эталонный ряд: ...
 ...-> т.~к. $b_k$ сходится, то $a_k$ сходится.
 
 2) ... Эталонный ряд: ...
...-> т.~к. $b_k$ рассходится, т и о $a_k$ сходится.

ПРИЗНАК 3

Пусть .... Тогда:

а) Если $b_k$ сходится, то $a_k$ сходится 

б) Если $a_k$ расходится, то  $b_k$ расходится

ДОК-ВО

...

Перемножим их почленно. Имеем:
...
\end{document}
