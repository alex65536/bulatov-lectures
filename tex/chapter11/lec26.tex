\makeatletter
\def\input@path{{../../}}
\makeatother
\documentclass[../../main.tex]{subfiles}

\graphicspath{
	{../../img/}
	{../img/}
	{img/}
}

\begin{document}
\section{Числовые ряды}
Рассмотрим числовую последовательность $a_n$:
\[a_1,\  a_2,\  a_3,\  \ldots,\  a_n \in \R.\]
Используя её, построим новую
последовательность $S_n$ следующим образом:
\[
\begin{array}{l}
		S_1 = a_1, \\
		S_2 = a_1 + a_2,\\
		S_3 = a_1 + a_2 + a_3,\\
		\dots\\
		S_n = \sum\limits_{k = 1}^na_k.
\end{array}		
\]
Эту последовательность удобно изучать, записывая ее в виде:
\begin{equation}
	\label{lec_26, num_1}
	 \sum\limits_{k = 1}^\infty a_k
\end{equation}
\begin{definition}
Выражение вида \eqref{lec_26, num_1} называется
\emph{числовым рядом} (или просто \emph{рядом}).
Числа  $a_1, a_2,  \ldots, a_k,  \ldots$ 
называются \emph{элементами} (\emph{членами}) ряда.
\end{definition}

Рассмотрим $\lim\limits_{n \to \infty} S_n$. Если он существует 
и конечен, то говорят, 
что ряд \eqref{lec_26, num_1}  \emph{сходится}. 
В противном случае говорят, что ряд \emph{расходится}.
Если предел конечен, т.~е. 
$\lim\limits_{n \to \infty} S_n = S \in \R$, то $S$ 
называют \emph{суммой ряда} и пишут:
\[
	\sum\limits_{k = 1}^\infty a_k=S
.\]
Сумму первых $m$ элементов ряда 
$S_m$ называют \emph{частной суммой ряда}. 
Сходимость ряда означает,
что сходится последовательность его частных сумм.

\begin{examples}

~

\begin{enumerate}
 \item 
	Рассмотрим ряд
	 \[
		 \sum\limits_{k = 1}^\infty\frac{1}{k(k+1)} = 
		 \frac{1}{1\cdot 2} + \frac{1}{2\cdot 3} + \ldots
	\]
	\begin{gather*}
		S_n = \frac{1}{1\cdot 2} + \frac{1}{2\cdot 3} + \ldots +
		\frac{1}{n (n+1)} = \\
		= \left(1 - \frac{1}{2}\right) + 
		\left(\frac{1}{2} - \frac{2}{3}\right) + \ldots + 
		\left(\frac{1}{n} - \frac{1}{n+1}\right) =
		 1 - \frac{1}{n+1}
		\appr{n \to \infty}{1}. \\
		\lim\limits_{n \to \infty} S_n = 1 \implies \text{
		ряд сходится, его сумма равна 1.}
	\end{gather*}
\item
	Рассмотрим ряд
	 \[
		 \sum\limits_{k = 1}^\infty\frac{1}{2^{k-1}}.
	\]
	\begin{gather*}
		S_n =  \sum\limits_{k = 1}^n\frac{1}{2^{k-1}} =
		1 + \frac{1}{2} + \frac{1}{4} + \ldots + \frac{1}{2^{n-1}} =\\
		=  \frac{ 1 - \frac{1}{2^n}}{1 - \frac{1}{2}}
		\xrightarrow[n \to \infty]{}2
		\implies \text{ряд сходится, его сумма --- 2.}
	\end{gather*}
\item
	Рассмотрим ряд
	 \[
		 \sum\limits_{k = 1}^\infty(-1)^{k-1}.
	\]
	\begin{gather*}
		S_1 = 1, \ S_2 = 0, \ S_3 = 1, \ S_4 = 0, \ \ldots
		\implies \nexists \lim\limits_{n \to \infty} S_n \implies \text{ряд 
		расходится.}
	\end{gather*}
\end{enumerate}
\end{examples}

Сходимость ряда означает, что для достаточно большого
$n$ выполняется $\abs{S - S_n} \le \eps$, т.~е.
\[
	\forall\eps > 0\enspace \exists\nu = \nu_\eps \enspace
	\forall n\geq \nu \implies \abs{S - S_n} \leq \eps, \ \text{ 
	т.~е.}\ 
	\left|\sum\limits_{k = n_1}^\infty a_k\right| \leq \eps
.\]

Если отбросить конечное число первых членов ряда, например,  
первые  $m$ членов, то получится новый ряд
\[
	\sum\limits_{k = m + 1}^\infty a_k = a_{m+1} + a_{m+2} +  \ldots,
\]
который называется \emph{остатком} ряда \eqref{lec_26, num_1}.

Рассмотрим частные суммы $S_n$ и $S_m$ 
ряда   \eqref{lec_26, num_1}, $n > m$:
\[
	S_n = S_m + a_{m+1} + a_{m+2} + \ldots + a_n = 
	S_m + \sum\limits_{k = m+1}^n a_k
.\]
Перейдём к пределу при $n\to\infty$:
\[
	\lim\limits_{n \to \infty}S_n = 
	\lim\limits_{n \to \infty}\left(S_m + \sum\limits_{k = m+1}^n a_k\right) =
	S_m + \lim\limits_{n \to \infty}\sum\limits_{k = m+1}^n a_k
.\]
$\lim\limits_{n \to \infty}S_n$
существует и конечен тогда и только тогда, когда
 существует и конечен 
 $ \lim\limits_{n \to \infty} \sum\limits_{k = m+1}^n a_k$. 
Таким образом, если ряд \eqref{lec_26, num_1}  сходится,
то сходится и любой его остаток. 
Если же сходится какой-либо остаток ряда, то сходится  и сам  ряд. Если 
сходится какой-либо остаток ряда, то сходится и любой другой его остаток.

\subsection{Необходимое условие сходимости ряда}

\begin{thm}[необходимое условие сходимости ряда]
	Если ряд \eqref{lec_26, num_1} сходится,
	то \[\lim\limits_{n \to \infty} a_n = 0.\]
\end{thm}
\begin{proof}
	Пусть ряд \eqref{lec_26, num_1} сходится и $S$ --- его сумма. Тогда для его 
	частных сумм выполняется
	$S_n - S_{n-1} = a_n$. При $n\to\infty$ получаем:
	\[
		\lim\limits_{n \to \infty} a_n = \lim\limits_{n \to \infty}
		(S_n - S_{n - 1}) = \lim\limits_{n \to \infty}S_n - 
		 \lim\limits_{n \to \infty}S_{n - 1}  = S - S = 0. \qedhere
	\]
\end{proof}
\begin{example}
	Рассмотрим \emph{гармонический ряд}
	\[
		 \sum\limits_{k = 1}^\infty\frac{1}{k} = 
		 1+ \frac{1}{2} + \frac{1}{3} + \ldots + 
		  \frac{1}{n} + \ldots
	.\]
	Для него выполнено необходимое условие
	\[a_n = \frac{1}{n} \appr{n \to \infty}0,\]
	но при этом
	\[
		S_{2n} = S_n + \frac{1}{n + 1} + \frac{1}{n + 2} + 
		 \ldots + \frac{1}{2n} > S_n + \frac{1}{2n}\cdot n =
		 S_n + \frac{1}{2}
	.\]
	Предположим, что ряд сходится и имеет сумму $S$.
	Тогда перейдём к пределу при $n\to\infty$:
	\[
		S_{2n} > S_{n} + \frac12 \implies S \geq S + 
		 \frac{1}{2}
	.\]
	Получили противоречие, т.~е. гармонический ряд расходится.
\end{example}

\subsection{Критерий Коши сходимости числовых рядов}

Сходимость ряда ---
 это сходимость последовательности его частных
сумм, т.~е. выполняется критерий Коши для последовательности:
\begin{gather*}
	\forall\eps > 0 \quad \exists\nu = \nu_\eps \quad
	\forall m,n \geq \nu: \abs{S_n - S_m} \leq \eps.
\end{gather*}
Без ограничения общности предположим, что $n > m$. Тогда:
\begin{gather*}
	\enspace \left| S_n - S_m \right| =
	\left| \sum\limits_{k = 1}^n a_k -  \sum\limits_{k = 1}^m a_k \right| = 
	\left| \sum\limits_{k = m + 1}^n a_k \right| \leq \eps.
\end{gather*}
Таким образом, получаем критерий Коши для рядов:
\begin{thm}[критерий Коши сходимости числового ряда]
	  \[
	  \sum\limits_{k = 1}^\infty a_k\  \text{сходится}
	  \iff 
	 \forall\eps > 0 \quad \exists\nu = \nu_\eps \quad 
	 \forall m, n \in \N \enspace \text{т.~ч.} \enspace n > m \geq \nu \implies
	 \left| \sum\limits_{k = m + 1}^n a_k \right| \leq \eps
	 .\]
\end{thm}

Пусть $\sum\limits_{k = 1}^\infty a_k$  и $\sum\limits_{k = 1}^\infty b_k$
 --- два сходящихся ряда, а $\alpha, \beta \in \R$~--- некоторые числа.
Ряд $\sum\limits_{k = 1}^\infty(\alpha a_k + \beta b_k)$
 называют \emph{линейной комбинацией} рядов. 
\begin{thm}
Если ряды сходятся, то сходится и их линейная комбинация.
 \end{thm}
\begin{proof}
	Пусть  $\sum\limits_{k = 1}^\infty a_k = A$  
	и $\sum\limits_{k = 1}^\infty b_k = B$, \enspace
	 $\alpha, \beta \in \R$. Соответствующие частные суммы обозначим $A_k$ и 
	 $B_k$. Тогда
 	\[
 		\sum\limits_{k = 1}^\infty(\alpha a_k + \beta b_k) =
 		\lim\limits_{k \to \infty} (\alpha A_k + \beta B_k) =
 		\alpha\lim\limits_{k \to \infty} A_k +
 		\beta\lim\limits_{k \to \infty} B_k =
 		\alpha\sum\limits_{k = 1}^\infty a_k + 
 		\beta\sum\limits_{k = 1}^\infty b_k = \alpha A + 
 		\beta B. \qedhere
 	\]
 \end{proof}
 Линейная комбинация расходящихся рядов может сходиться.
 Если один из рядов сходится, а другой --- расходится, то их линейная 
 комбинация
 расходится. Если ряд \eqref{lec_26, num_1} сходится, 
 то $\sum\limits_{k=1}^\infty \alpha a_k = \alpha \sum\limits_{k=1}^\infty 
 a_k$.
 \section{Положительные ряды}
\begin{definition}
Если все члены ряда \eqref{lec_26, num_1} положительны
($\forall k \in \N \ a_k > 0$),
то такой ряд называют \emph{положительным} рядом.
\end{definition}
Для положительных рядов имеем:
\[
	S_{n+1} = S_n + a_{n+1}  > S_n \implies S_n \uparrow.
\]
А это значит, что к положительным рядам применим 
критерий сходимости монотонной последовательности: для сходимости 
последовательности $S_n$ необходимо и достаточно, чтобы она была ограничена 
сверху. Отсюда получаем
 \begin{thm} [критерий сходимости положительных рядов]
 	\label{lec26:pos_series}
 	 Для сходимости положительного ряда 
 	 необходимо и достаточно, чтобы 
 	 последовательность 
 	 его частных сумм была ограничена сверху.
 \end{thm}
 \begin{rem}
 	 Если  последовательность $S_n$ не ограничена сверху, 
 	 то  $S_n\appr{n \to \infty}{+\infty}$. В таком случае иногда записывают:
 	 \[
 		\sum\limits_{k = 1}^\infty a_k= +\infty
	.\]
	Рассмотрим \emph{геометрический} ряд
	 \[
 		\sum\limits_{k = 1}^\infty q^{k-1}=  1 + q + q^2 + \ldots
 		\ (q > 0).
	\]
	Частная сумма его первых $n$ членов:
	\[S_n = \frac{1 - q^n}{1 - q}.\]
	Если $0 \le q < 1$, то ряд сходится:
	\[
	\sum\limits_{k = 1}^\infty q^{k-1} = \lim\limits_{n \to \infty} S_n =
	\frac{1}{1 - q} \in \R
	.\]
	Если же $q \geq 1$, то ряд расходится:
	\[
	\sum\limits_{k = 1}^\infty q^{k-1} = \lim\limits_{n \to \infty} S_n =
	+ \infty
	.\]
 \end{rem}
 \subsection{Признаки сравнения рядов}

Рассматриваем положительные ряды.
Сделать вывод об поведении изучаемого ряда
можно, сравнивая его элементы с  элементами
другого (эталонного) ряда,  поведение (сходимость/расходимость) которого 
известно.
 \begin{thm}[признак сравнения 1\textdegree]
 	\label{lec26:comp_test_1}
 	Пусть $\sum\limits_{k = 1}^\infty a_k,\enspace \sum\limits_{k = 1}^\infty 
 	b_k$
 	 --- положительные ряды\\ 
 	 и пусть  $\exists c = const > 0\ \text{т.~ч.}\enspace  
 	 a_k \leq c\cdot b_k \ \: \forall k\in \N$. Тогда:
	\begin{enumerate}[label={\alph*)}]
	\item если сходится $\sum\limits_{k = 1}^\infty 
 	b_k$, то сходится и $\sum\limits_{k = 1}^\infty a_k$;
	\item если расходится $\sum\limits_{k = 1}^\infty a_k$,
	то расходится и $\sum\limits_{k = 1}^\infty b_k$.
	\end{enumerate}
 \end{thm}
\begin{proof}
	
	~
	
	\begin{enumerate}[label={\alph*)}]
	\item Пусть $b_k$ сходится.
	Тогда по 
	\hyperref[lec26:pos_series]{критерию сходимости положительных рядов}:
	 \begin{gather*}
	 	\exists M \in \R \quad \forall n \in \N \quad 
	 	\sum\limits_{k = 1}^n b_k \leq M
	 	\implies
		\left[
			a_k \leq c\cdot b_k \enspace \forall k \in \N
		\right]
		\implies \\
		\implies
		\sum\limits_{k = 1}^n a_k \leq
		\sum\limits_{k = 1}^n c\cdot b_k =
		c\sum\limits_{k = 1}^n b_k \leq
		c \cdot M = M_1 \implies \\
		\implies
		\text{$a_k$ сходится по 
		\hyperref[lec26:pos_series]{критерию сходимости положительных рядов}}	
	 .\end{gather*}
	\item Пусть $a_k$ расходится. Предположим, что $b_k$
	сходится, но тогда по пункту a)
	сходится и $a_k$ --- противоречие. 
	Отсюда получаем, что $b_k$ расходится. \qedhere
	\end{enumerate}
\end{proof}
\begin{examples}

~

\begin{enumerate}
\item
	Рассмотрим ряд
	\[
		 \sum\limits_{k = 1}^\infty\frac{1}{\sqrt{k}}
	.\]
	Так как \  $\forall k \in \N \enspace
	 \dfrac{1}{\sqrt{k}} \geq \dfrac{1}{k}$ и ряд 
	 $\displaystyle\sum\limits_{k = 1}^\infty \frac{1}{k}$ расходится,
	то по \hyperref[lec26:comp_test_1]{признаку  1\textdegree b)} ряд
	$\displaystyle\sum\limits_{k = 1}^\infty\frac{1}{\sqrt{k}}$ расходится.
\item
	Рассмотрим ряд
	\[
		 \sum\limits_{k = 1}^\infty\frac{1}{k + 3^k}
	.\]
	Так как \  $\forall k \in \N \enspace
	 \dfrac{1}{{k + 3^k}} \leq \dfrac{1}{3^k}$ \  и ряд 
	\ $\displaystyle\sum\limits_{k = 1}^\infty \frac{1}{3^k}$\   сходится,
	то по \hyperref[lec26:comp_test_1]{признаку  1\textdegree a)} ряд
	$\displaystyle \sum\limits_{k = 1}^\infty\frac{1}{k + 3^k}$ сходится.
\end{enumerate}
\end{examples}
\begin{rem}
Для использования \hyperref[lec26:comp_test_1]{признака  1\textdegree} 
достаточно, чтобы $a_k \leq b_k\quad \forall k\geq N$, т.~е.
признак работает в этом случае для остатков рядов $a_k$ и $b_k$. 
 \end{rem}
\begin{example}
\begin{gather*}
	\sum\limits_{k = 0}^\infty\frac{1}{k!} = \frac{1}{1} + 
	\frac{1}{1} +  \frac{1}{2} +  \frac{1}{6} +
	\frac{1}{24} +  \frac{1}{120} + \ldots\\
	\sum\limits_{k = 0}^\infty\frac{1}{2^k} = \frac{1}{1} + 
	\frac{1}{2} +  \frac{1}{4} +  \frac{1}{8} +
	\frac{1}{16} +  \frac{1}{32} + \ldots\\
\end{gather*}
	В силу того, что
	$\displaystyle\frac{1}{k!} < \frac{1}{2^k} \  \forall k \geq 4$
	и ряд $\displaystyle\sum\limits_{k = 0}^\infty\frac{1}{2^k}$ сходится,
	ряд $\displaystyle\sum\limits_{k = 0}^\infty\frac{1}{k!}$ сходится по
	 \hyperref[lec26:comp_test_1]{признаку  1\textdegree a)}.
\end{example}
  \begin{thm}[признак сравнения 2\textdegree]
 	\label{lec26:comp_test_2}
 	Пусть $\sum\limits_{k = 1}^\infty a_k,
 	\enspace \sum\limits_{k = 1}^\infty b_k$
 	 --- положительные ряды,\\ 
 	а также существует предел
 	\[
 	\lim\limits_{n \to \infty}\frac{a_k}{b_k} = l,\enspace 0 \leq l \leq +\infty
 	.\]
	Тогда:
	\begin{enumerate}[label={\alph*)}]
	\item если $l < +\infty$ и $b_k$ сходится, то сходится и $a_k$;
	\item если $l > 0$ и $b_k$ расходится, то расходится и $a_k$.
	\end{enumerate}
 \end{thm}
 \begin{proof}
 
 ~
 
	\begin{enumerate}[label={\alph*)}]
	\item Пусть $l< +\infty$.
	Тогда для достаточно больших $k$ будем иметь:
	 \[
	 	\frac{a_k}{b_k} \leq l + 1  \implies a_k \leq (l+1)b_k
	 .\]
	Тогда по \hyperref[lec26:comp_test_1]{признаку  1\textdegree a)} из 
	 сходимости $b_k$ будет следовать сходимость $a_k$.
	\item Пусть $l > 0$, тогда:
	\[
	\frac{b_k}{a_k} \xrightarrow[k \to \infty]{} \frac{1}{l} < +\infty
	.\]
	Пусть $b_k$ расходится. Если предположить,
	что $a_k$  сходится, то по доказанному пункту
	2\textdegree a) сходится и $b_k$. Противоречие.
	Отсюда получаем, что $a_k$ расходится. \qedhere
	\end{enumerate}
\end{proof}
\begin{examples}

~

\begin{enumerate}
\item
Рассмотрим ряд:
	\[
		 \sum\limits_{k = 1}^\infty a_k=  \sum\limits_{k = 1}^\infty\frac{1}{3^k + 5}
	.\]
	За эталонный возьмём следующий ряд:
	\[
		\sum\limits_{k = 1}^\infty b_k=   \sum\limits_{k = 1}^\infty\frac{1}{3^k}
	.\]
	Тогда \[\frac{a_k}{b_k} = \frac{3^k}{3^k + 5} \xrightarrow[k \to \infty]{} 
	1.\]
	Т.~к. $b_k$ сходится, то по  
	 \hyperref[lec26:comp_test_2]{признаку  2\textdegree a)}\ $a_k$ также 
	 сходится.
\item
	Рассмотрим ряд:
	\[
		 \sum\limits_{k = 1}^\infty a_k=  \sum\limits_{k = 1}^\infty\frac{1}{10k + 7}
	.\]
	За эталонный возьмём следующий ряд:
	\[
		\sum\limits_{k = 1}^\infty b_k=   \sum\limits_{k = 1}^\infty\frac{1}{k}
	.\]
	Тогда \[\frac{a_k}{b_k} = \frac{k}{10^k + 7} \xrightarrow[k \to \infty]{} 
	\frac{1}{10}.\] 
	Т.~к. $b_k$ расходится, то по  
	 \hyperref[lec26:comp_test_2]{признаку  2\textdegree b)}
	 $a_k$ также расходится.
\end{enumerate}
\end{examples}

\begin{thm}[признак сравнения 3\textdegree]
 	\label{lec26:comp_test_3}
 	Пусть $\sum\limits_{k = 1}^\infty a_k,\enspace \sum\limits_{k = 1}^\infty 
 	b_k$
 	 --- положительные ряды,\\ 
 	а также $\displaystyle\frac{a_{k+1}}{a_k} \leq \frac{b_{k+1}}{b_k} \quad
 	\forall k \in \N.$ Тогда:
	\begin{enumerate}[label={\alph*)}]
	\item если  $b_k$ сходится, то сходится и $a_k$;
	\item если $a_k$ расходится, то расходится и $b_k$.
	\end{enumerate}
\end{thm}
\begin{proof}
	По условию выполняются следующие неравенства:
 	\begin{gather*}
			\frac{a_2}{a_1} \leq \frac{b_2}{b_1}, \\
			\frac{a_3}{a_2} \leq \frac{b_3}{b_2}, \\
			\dots\\
			\frac{a_n}{a_{n-1}} \leq \frac{b_n}{b_{n-1}}.
	\end{gather*}
	Перемножим их почленно. Получаем
	\[
		\frac{a_2}{a_1}\cdot\frac{a_3}{a_2}\cdot\ldots
		\cdot\frac{a_n}{a_{n-1}} \leq
		\frac{b_2}{b_1}\cdot\frac{b_3}{b_2}\cdot\ldots
		\cdot\frac{b_n}{b_{n-1}}
	.\]
	Сократив дроби в обеих частях неравенства, получим:
	\[
		\frac{a_n}{a_1} \leq \frac{b_n}{b_1} \implies
		a_n \leq \frac{a_1}{b_1}\cdot b_n\quad \forall n \in \N
	.\]
	Так как $\dfrac{a_1}{b_1} = const$, 
	доказательство признака 3\textdegree \ следует из
	\hyperref[lec26:comp_test_1]{признака  1\textdegree}.
\end{proof}
\end{document}
