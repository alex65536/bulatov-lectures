\makeatletter
\def\input@path{{../../}}
\makeatother
\documentclass[../../main.tex]{subfiles}

\graphicspath{
	{../../img/}
	{../img/}
	{img/}
}

\begin{document}
\section{Действия над рядами}
\subsection{Линейная комбинация рядов}
Рассмотрим два ряда:
$\sum\limits_{k = 1}^{\infty} \alpha_k$ и
$\sum\limits_{k = 1}^{\infty} \beta_k$.
Их линейной комбинацией называется:
\[x \sum\limits_{k = 1}^{\infty} \alpha_k +
y \sum\limits_{k = 1}^{\infty} \beta_k,\]
где $x,y = \fix$.
Если ряды
$\sum\limits_{k = 1}^{\infty} \alpha_k$ и
$\sum\limits_{k = 1}^{\infty} \beta_k$
сходятся, то и их линейная комбинация сходится:
\[x \sum\limits_{k = 1}^{n} \alpha_k + y \sum\limits_{k = 1}^{n} \beta_k
\underset{n \to \infty} {\longrightarrow}
x \sum\limits_{k = 1}^{\infty} \alpha_k +
y \sum\limits_{k = 1}^{\infty} \beta_k\]
\[x \sum\limits_{k = 1}^{n} \alpha_k + y \sum\limits_{k = 1}^{n} \beta_k
= \sum\limits_{k = 1}^{n} (x \alpha_k + y \beta_k)
\underset{n \to \infty} {\longrightarrow}
\sum\limits_{k = 1}^{\infty} (x \alpha_k + y \beta_k)\]
Рассмотрим ряд справа $\sum\limits_{k = 1}^{\infty} (x \alpha_k + y \beta_k)$
и его частные суммы
$x \sum\limits_{k = 1}^{n} \alpha_k + y \sum\limits_{k = 1}^{n} \beta_k
= \sum\limits_{k = 1}^{n} (x \alpha_k + y \beta_k)$
как линейную комбинацию рядов $\sum\limits_{k = 1}^{\infty} \alpha_k$ и
$\sum\limits_{k = 1}^{\infty} \beta_k$.

\subsection{Группировка членов ряда}
Рассмотрим ряд
$\sum\limits_{k = 1}^{\infty} = a_1 + a_2 + a_3 + \ldots + a_n + \ldots$
Если сгруппировать слагаемые в этой сумме (т.~е. рассавить скобки), то получим
другой ряд-группировку:
\begin{equation}
\begin{gathered}
	\label{lec30, series_groups}
	\sum\limits_{k = 1}^{\infty} c_k \\
	c_1 = a_1 + \ldots + a_{k_1} \\
	c_2 = a_{k_1+1} + \ldots + a_{k_2} \\
	\ldots \\
	c_n = a_{k_{n-1}+1} + \ldots + a_{k_n} \\
	\ldots
\end{gathered}
\end{equation}
\begin{thm}
	Если ряд $\sum\limits_{k = 1}^{\infty} a_k$ сходится,
	то группировка членов этого ряда не меняет его суммы.
\end{thm}
\begin{proof}
Рассмотрим частную сумму ряда \eqref{lec30, series_groups}
$\sigma_m = \sum\limits_{n = 1}^{m} c_n$.
Пусть $\sum\limits_{k = 1}^{\infty} a_k$ сходится, т.~е. сходится
$ S_n = \sum\limits_{k = 1}^{n} a_k$, тогда
\[\sigma_m = c_1 + c_2 + \ldots + c_n = (a_1 + \ldots a_{k_1}) +
(a_{k_1 + 1} + \ldots a_{k_2}) +
\ldots + (a_{k_{m-1} + 1} + \ldots a_{k_m}) = S_{k_m}\]
т.~е. последовательность частичных сумм ряда-группировки
\eqref{lec30, series_groups} является подпоследовательностью
последовательности $S_n$ и сходится к тому же пределу что и $S_n$.
\end{proof}
Группировка членов расходящегося ряда, вообще говоря, недопустима.
\begin{example}
	$\sum\limits_{k = 1}^{n} (-1)^{k - 1} = 1 - 1 + 1 - 1  + \ldots \neq
	(1 - 1) + (1 - 1) + \ldots = 0$
\end{example}
\subsection{Перестановка членов ряда}
Если члены ряда $\sum\limits_{k = 1}^{\infty} a_k$
переставить местами каким-либо образом, то получим новый ряд-перестановку:
$\sum\limits_{n = 1}^{\infty} p_n$, \
$p_n = a_{k_n}$ ~--- какой-то член ряда $\sum\limits_{k = 1}^{\infty} a_k$.
\begin{thm}
	Если ряд $\sum\limits_{k = 1}^{\infty} a_k$ сходится и его члены положительны,
	то перестановка допустима, т.~е. при перестановке членов сходящегося
	положительного ряда его сумма не меняется.
\end{thm}
\begin{proof}
	Пусть $\sum\limits_{k = 1}^{\infty} a_k$  сходится и
	$\sum\limits_{k = 1}^{\infty} a_k = A$, \ \
	$p_1 + p_2 + \ldots + p_m = a_{k_1} + a_{k_2} + \ldots + a_{k_m} = *$.
	Среди номеров $k_1, k_2, \ldots , k_m \ \exists$ наибольший, пусть это
	$N = \max\{k_1, k_2, \ldots k_n\}$, \ тогда $* \leq a_1 + a_2 + \ldots +
	a_N \leq A$, \ т.~е. частные суммы ряда \eqref{lec30, series_groups}
	ограничены числом $A$. Поскольку ряд \eqref{lec30, series_groups}
	положительный, он сходится. При этом его сумма:
	$P = \lim\limits_{m \to \infty}(p_1 + p_2 + \ldots + p_m) \leq A$. \
	Но с другой стороны можно считать, что $\sum\limits_{k = 1}^{\infty} a_k$
	получен перестановкой членов ряда \eqref{lec30, series_groups}, тогда:
	\[A \leq P \ \implies \ P = A\]
\end{proof}

\begin{crl*}
	При перестановке членов абсолютно сходящегося ряда его сумма не меняется.
\end{crl*}
\begin{proof}
		Пусть $\sum\limits_{k = 1}^{\infty} a_k$ сходится абсолютно,
		значит сходятся ряды $\sum\limits_{k = 1}^{\infty} b_k$ и
		$\sum\limits_{k = 1}^{\infty} c_k$, где\[b_k = \begin{cases}
			a_k, \ \ a_k > 0\\
			0, \ \ a_k < 0
		  \end{cases} \ \
		 c_k = \begin{cases}
			0, \ \ a_k > 0\\
			-a_k, \ \ a_k < 0
		  \end{cases}\]
  Перестановка в $\sum\limits_{k = 1}^{\infty} a_k$ порождает перестановки
  в рядах  $\sum\limits_{k = 1}^{\infty} b_k$ и
  $\sum\limits_{k = 1}^{\infty} c_k$, \ Но это положительные ряды,
  значит их сумма не меняется. А т.~к. $\sum\limits_{k = 1}^{\infty} a_k
   = \sum\limits_{k = 1}^{\infty} b_k + \sum\limits_{k = 1}^{\infty} c_k$,
   \ то не меняется и сумма ряда $\sum\limits_{k = 1}^{\infty} a_k$.
\end{proof}
\begin{thm}[Теорема Римана]
	Если $\sum\limits_{k = 1}^{\infty} a_k$ сходится условно, то
	$\forall \ A, \ - \infty \leq A \leq + \infty \ \exists$
	перестановка членов ряда $\sum\limits_{k = 1}^{\infty} a_k$ т.ч.
	$\sum\limits_{n = 1}^{\infty} p_n = A$.
	\begin{proof}
		Схема доказательства: Пусть $A$ ~--- число, т.~к. ряд
		$\sum\limits_{k = 1}^{\infty} a_k$ сходится условно, то оба ряда
		$\sum\limits_{k = 1}^{\infty} b_k$ и
		$\sum\limits_{k = 1}^{\infty} c_k$, где
		\[b_k = \begin{cases}
			a_k, \ \ a_k > 0\\
			0, \ \ a_k < 0
		  \end{cases} \ \
		 c_k = \begin{cases}
			0, \ \ a_k > 0\\
			-a_k, \ \ a_k < 0
		  \end{cases}\]
		  имеют бесконечные суммы, т.~е. у нас есть <<банк>> положительных
		  элементов	ряда $\sum\limits_{k = 1}^{\infty} a_k$ и <<банк>> \
		  отрицательных элементов ряда $\sum\limits_{k = 1}^{\infty} a_k$.
		  Будем брать элементы ряда $\sum\limits_{k = 1}^{\infty} b_k$
		  подряд стоящие так, чтобы их сумма стала чуть больше чем $A$.
		  Затем добавляем отрицательные элементы, представленые в ряде
		  $\sum\limits_{k = 1}^{\infty} c_k$ до тех пор, пока набранная
		  сумма станет мнньше, чем $A$. Затем снова добавляе положительные
		  элементы, чтобы сумма стала больше $A$. Поскольку $a_k \to 0$,
		  то полученная последовательность cумм будет стремиться к $A$.

		  Если $A = +\infty$, то набираем, например, положительные
		  слагаемые ряда $\sum\limits_{k = 1}^{\infty} a_k$ до тех пор
		  пока их сумма станет больше 10, затем один отрицательный элемент,
		  затем положительные, пока сумма станет больше 20, затем снова один
		  отрицательный элемент и т.д. Полученная перестановка стремится к
		  $+ \infty$.
	\end{proof}
\end{thm}
\subsection{Перемножение рядов}
Произведением рядов $\sum\limits_{k = 1}^{\infty} a_k$ и
$\sum\limits_{k = 1}^{\infty} b_k$ называют ряд:
$\sum\limits_{\forall \ i,j}^{\infty} a_i b_j$, т.~е. ряд, содержащий все
возможные произведения элементов рядов $\sum\limits_{k = 1}^{\infty} a_k$ и
$\sum\limits_{k = 1}^{\infty} b_k$. \ Рассмотрим ряд:
\begin{equation}
	\label{lec30, series_multiplication}
	\sum\limits_{n = 1}^{\infty} c_n
\end{equation}
где $c_n$ ~--- произведения чисел $a_i$ и $b_j$.
\begin{thm}[Теорема Коши]
	Пусть ряды $\sum\limits_{k = 1}^{\infty} a_k$ и
	$\sum\limits_{k = 1}^{\infty} b_k$ сходятся абсолютно и
	$\sum\limits_{k = 1}^{\infty} a_k = A$ и
	$\sum\limits_{k = 1}^{\infty} b_k = B$, тогда ряд-произведение
	\eqref{lec30, series_multiplication} сходится абсолютно и его сумма:
	\[\sum\limits_{n = 1}^{\infty} c_n = A \cdot B \]
	\[\sum\limits_{k = 1}^{\infty} a_k \cdot
	\sum\limits_{k = 1}^{\infty} b_k =
	\sum\limits_{\forall \ i,j}^{\infty} a_i b_j\]
\end{thm}
\begin{proof}
Покажем, что $\sum\limits_{n = 1}^{\infty} |c_n|$ сходится. Рассмотрим
его частную сумму:
\[|c_1| + |c_2| + \ldots + |c_m| = |a_{i_1} \cdot b_{j_1}| +
|a_{i_2} \cdot b_{j_2}| + \ldots + |a_{i_m} \cdot b_{j_m}| \leq *\]
Пусть $l = \max\{i_1, j_1, \ldots i_m, j_m\}$, тогда:
\[* \leq (|a_1| + \ldots + |a_l|)(|b_1| + \ldots + |b_l|)
\leq\ \sum\limits_{k = 1}^{\infty} |a_k| \cdot
\sum\limits_{k = 1}^{\infty} |b_k| = M\]
Частная сумма $\sum\limits_{n = 1}^{\infty} |c_n| \leq M \implies
\sum\limits_{n = 1}^{\infty} |c_n|$ сходится, т.~е.
$\sum\limits_{n = 1}^{\infty} c_n$ сходится абсолютно.
Так как $\sum\limits_{n = 1}^{\infty} c_n$ сходится абсолютно, то допустимы
любые перестановки его членов, т.~е. суммы при этом не меняются. Поэтому при
вычислени суммы этого ряда можно использовать любую удобную нам
перестановку. Например можно в качестве последовательности его частичных
сумм рассмотреть $(a_1 + \ldots + a_l)(b_1 + \ldots + b_l)$. Здесь
содержится любое произведение $a_i b_j$ при достаточно большом $l$, т.~е.
при $l \to \infty$ сюда попадают все элементы из
$\sum\limits_{\forall \ i,j}^{\infty} a_i b_j$. То есть:
\[\sum\limits_{\forall \ i,j}^{\infty} a_i b_j =
\lim\limits_{l \to \infty} (a_1 + \ldots + a_l)(b_1 + \ldots + b_l) =
A \cdot B\]
\end{proof}
Существует и часто используется и специальный способ перемножения рядов.
Перемножение рядов по методу Коши:
\[(a_1 + a_2 + a_3 + \ldots)(b_1 + b_2 + b_3 + \ldots) = a_1 b_1 +
(a_1 b_2 + a_2 b_1) + (a_1 b_3 + a_2 b_2 + a_3 b_1) + \ldots +
\sum\limits_{i + j = n + 1} a_i b_j + \ldots\]
\begin{thm}[Теорема Мертенса]
Если ряды $\sum\limits_{k = 1}^{\infty} a_k$ и
$\sum\limits_{k = 1}^{\infty} b_k$ сходятся и хотя бы один из них
сходится абсолютно, то сходится ряд \eqref{lec30, series_multiplication}
и его сумма равна $A \cdot B$. Без доказательства.
\end{thm}

Если $\sum\limits_{k = 1}^{\infty} a_k$ и
$\sum\limits_{k = 1}^{\infty} b_k$ условно сходятся, то их произведение
может быть расходящимся.
\begin{example}
	\[\sum\limits_{k = 1}^{\infty} \frac{(-1)^{k-1}}{\sqrt{k}} \cdot
	\sum\limits_{k = 1}^{\infty} \frac{(-1)^{k-1}}{\sqrt{k}} =
	\left(1 - \frac{1}{\sqrt{2}} + \frac{1}{\sqrt{3}} + \cdots +
	\frac{(-1)^{n-1}}{\sqrt{n}} + \cdots \right) \cdot \]
	\[\cdot \left(1 - \frac{1}{\sqrt{2}} + \frac{1}{\sqrt{3}} + \cdots +
	\frac{(-1)^{n-1}}{\sqrt{n}} + \cdots \right) = 1 +
	\left( - \frac{1}{\sqrt{2}} - \frac{1}{\sqrt{2}} \right) +
	\left( \frac{1}{\sqrt{3}} + \frac{1}{\sqrt{2}} \cdot
	\frac{1}{\sqrt{2}} + \frac{1}{\sqrt{3}} \right) + \] 
	\[ + \left( -1 \right) ^ {n - 1}
	\underbrace{\left(1 \cdot \frac{1}{\sqrt{n}} + \frac{1}{\sqrt{2}} \cdot
	\frac{1}{\sqrt{n - 1}} + \frac{1}{\sqrt{3}} \cdot
	\frac{1}{\sqrt{n - 2}} + \ldots + \frac{1}{\sqrt{n}}
	\cdot 1\right)}_{\tilde{c}_n} + \cdots \]
	Cумма $\tilde{c}_n$ содержит n слагаемых и каждое из них не меньше,
	чем $\frac{1}{\sqrt{n}}$. Поэтому $|c_n| \geq 1, \ c_n \not\to 0$
	~--- ряд расходится.
\end{example}
\begin{remark}
	Произведение расходящихся рядов может оказаться сходящимся.
\end{remark}
\section{Ряды в $\C$}
Рассмотрим ряд:
\[\sum\limits_{k = 1}^{\infty} z_k , \ z_k =
\alpha_k + i\beta_k \in \C\]
Сходимость ряда определяется по той же схеме, что и ранее.
\[ \sigma_n = \sum\limits_{k = 1}^{\infty} z_k =
\sum\limits_{k = 1}^{\infty} (\alpha_k + i\beta_k)\]
Предел $\lim\limits_{n \to \infty} \sigma_n = \sigma$,
если он существует, называют суммой ряда, если он конечен, то ряд сходится.
Поскольку $\sigma_n = \sum\limits_{k = 1}^{\infty} (\alpha_k + i\beta_k) =
\sum\limits_{k = 1}^{\infty} \alpha_k +
i \sum\limits_{k = 1}^{\infty} \beta_k$,
то это приводит к рядам $\sum\limits_{k = 1}^{\infty} \alpha_k$ и
$\sum\limits_{k = 1}^{\infty} \beta_k$.
Конечный предел $\sigma_n$ существует $\iff$ когда
существуют конечные пределы $\sum\limits_{k = 1}^{n} \alpha_k$ и
$\sum\limits_{k = 1}^{n} \beta_k$.
Т.е. $\sum\limits_{k = 1}^{\infty} z_k$ сходится $\iff$
сходятся оба ряда $\sum\limits_{k = 1}^{\infty} \alpha_k , \
\sum\limits_{k = 1}^{\infty} \beta_k$.
\[\sum\limits_{k = 1}^{\infty} z_k = \sum\limits_{k = 1}^{\infty}
(\alpha_k + i\beta_k) = \sum\limits_{k = 1}^{\infty} \alpha_k +
i\sum\limits_{k = 1}^{\infty} \beta_k\]
Необходимое условие сходимости: $\lim\limits_{k \to \infty} z_k = 0$.

Критерий Коши:
\[ \forall \ \varepsilon > 0 \ \exists \ N = N(\varepsilon), \ \forall \
n \geq N(\varepsilon) \ \forall \ p > 0 \implies
|S_{n + p} - S_n| \leq \varepsilon \]

\end{document}
