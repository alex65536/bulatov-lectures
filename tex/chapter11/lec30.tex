\makeatletter
\def\input@path{{../../}}
\makeatother
\documentclass[../../main.tex]{subfiles}

\graphicspath{
	{../../img/}
	{../img/}
	{img/}
}

\begin{document}
\section{Действия над рядами}
\subsection{Линейная комбинация рядов}
Рассмотрим два ряда:
\begin{equation}
\label{30:3}
\sum\limits_{k = 1}^{\infty} \alpha_k,\quad
\sum\limits_{k = 1}^{\infty} \beta_k. 
\end{equation}
Их \emph{линейной комбинацией} называется ряд
\begin{equation}
\label{30:4}
x \sum\limits_{k = 1}^{\infty} \alpha_k +
y \sum\limits_{k = 1}^{\infty} \beta_k, 
\end{equation}
где $x,y - \fix$, $x, y \in\R$.

Если ряды \eqref{30:3}
сходятся, то и их линейная комбинация \eqref{30:4} сходится:
\[x \sum\limits_{k = 1}^{n} \alpha_k + y \sum\limits_{k = 1}^{n} \beta_k
\appr{n \to \infty}
x \sum\limits_{k = 1}^{\infty} \alpha_k +
y \sum\limits_{k = 1}^{\infty} \beta_k.\]

Чаще приходится рассматривать ряд 
\begin{equation}
\label{30:4.1}
\sum\limits_{k = 1}^{\infty} (x \alpha_k + y \beta_k). 
\end{equation}
Его частная сумма равна
\[\sum\limits_{k = 1}^{n} (x \alpha_k + y \beta_k) = x \sum\limits_{k = 1}^{n} 
\alpha_k + y \sum\limits_{k = 1}^{n} \beta_k,\] т.~е. линейной комбинации 
частных сумм рядов \eqref{30:3}. Если ряды \eqref{30:3} сходятся, то сходится 
и \eqref{30:4.1}:

\[x \sum\limits_{k = 1}^{n} \alpha_k + y \sum\limits_{k = 1}^{n} \beta_k
= \sum\limits_{k = 1}^{n} (x \alpha_k + y \beta_k)
\appr{n \to \infty}
\sum\limits_{k = 1}^{\infty} (x \alpha_k + y \beta_k).\]

\subsection{Группировка членов ряда}
Рассмотрим ряд
\[\sum\limits_{k = 1}^{\infty} a_n = a_1 + a_2 + a_3 + \ldots + a_n + \ldots.\]
Если сгруппировать слагаемые в этой сумме (т.~е. расставить скобки), то получим
другой ряд (ряд-группировку):
\begin{equation}
\begin{gathered}
	\label{lec30, series_groups}
	\sum\limits_{k = 1}^{\infty} c_k \\
	c_1 = a_1 + \ldots + a_{k_1} \\
	c_2 = a_{k_1+1} + \ldots + a_{k_2} \\
	\ldots \\
	c_n = a_{k_{n-1}+1} + \ldots + a_{k_n} \\
	\ldots
\end{gathered}
\end{equation}
\begin{thm}
	Если ряд $\sum\limits_{k = 1}^{\infty} a_k$ сходится,
	то группировка членов этого ряда не меняет его суммы.
\end{thm}
\begin{proof}
Рассмотрим частную сумму ряда \eqref{lec30, series_groups}
$\sigma_m = \sum\limits_{n = 1}^{m} c_n$.
Пусть $\sum\limits_{k = 1}^{\infty} a_k$ сходится, т.~е. сходится
$ S_n = \sum\limits_{k = 1}^{n} a_k$, тогда
\[\sigma_m = c_1 + c_2 + \ldots + c_n = (a_1 + \ldots a_{k_1}) +
(a_{k_1 + 1} + \ldots a_{k_2}) +
\ldots + (a_{k_{m-1} + 1} + \ldots a_{k_m}) = S_{k_m},\]
т.~е. последовательность частичных сумм ряда-группировки
\eqref{lec30, series_groups} является подпоследовательностью
последовательности $S_n$ и сходится к тому же пределу, что и $S_n$.
\end{proof}
Группировка членов расходящегося ряда, вообще говоря, недопустима.
\begin{example}
	\[\sum\limits_{k = 1}^{n} (-1)^{k - 1} = 1 - 1 + 1 - 1  + \ldots \neq
	(1 - 1) + (1 - 1) + \ldots = 0.\]
\end{example}
\subsection{Перестановка членов ряда}
Если члены ряда $\sum\limits_{k = 1}^{\infty} a_k$
переставить местами каким-либо образом, то получим новый ряд (ряд-перестановку)
\begin{equation}
\label{30:6}
\sum\limits_{n = 1}^{\infty} p_n,
\end{equation}
где $p_n = a_{k_n}$ ~--- какой-либо член ряда $\sum\limits_{k = 1}^{\infty} 
a_k$.
\begin{thm}
	Если ряд $\sum\limits_{k = 1}^{\infty} a_k$ сходится и его члены положительны,
	то перестановка допустима, т.~е. при перестановке членов сходящегося
	положительного ряда его сумма не меняется.
\end{thm}
\begin{proof}
	Пусть $\sum\limits_{k = 1}^{\infty} a_k$  сходится и
	$\sum\limits_{k = 1}^{\infty} a_k = A$. Рассмотрим
	частичную сумму ряда \eqref{30:6}:
	\[p_1 + p_2 + \ldots + p_m = a_{k_1} + a_{k_2} + \ldots + a_{k_m} = *\]
	Среди номеров $k_1, k_2, \ldots, k_m$ существует 
	наибольший, пусть это
	$N = \max\{k_1, k_2, \ldots k_m\}$, тогда \[* \leq a_1 + a_2 + \ldots +
	a_N \leq A,\] т.~е. частные суммы ряда \eqref{lec30, series_groups}
	ограничены числом $A$. Поскольку ряд \eqref{lec30, series_groups}
	положительный, то он сходится. При этом его сумма равна
	\[P = \lim\limits_{m \to \infty}(p_1 + p_2 + \ldots + p_m) \leq A.\]
	Но с другой стороны можно считать, что $\sum\limits_{k = 1}^{\infty} a_k$
	получен перестановкой членов ряда \eqref{lec30, series_groups}. Тогда
	\[A \leq P \implies P = A. \qedhere\]
\end{proof}
\begin{crl*}
	При перестановке членов абсолютно сходящегося ряда его сумма не меняется.
\end{crl*}
\begin{proof}
		Пусть ряд $\sum\limits_{k = 1}^{\infty} a_k$ сходится абсолютно,
		значит, сходятся ряды $\sum\limits_{k = 1}^{\infty} b_k$ и
		$\sum\limits_{k = 1}^{\infty} c_k$, где\[b_k = \begin{cases}
			a_k,& a_k > 0\\
			0,& a_k \le 0
		  \end{cases} \qquad
		 c_k = \begin{cases}
			0,& a_k > 0\\
			-a_k,& a_k \le 0
		  \end{cases}\]
  Перестановка членов в ряде $\sum\limits_{k = 1}^{\infty} a_k$ порождает 
  перестановки членов
  в рядах  $\sum\limits_{k = 1}^{\infty} b_k$ и
  $\sum\limits_{k = 1}^{\infty} c_k$. Но эти ряды являются положительными,
  т.~е. их сумма не меняется. А т.~к. $\sum\limits_{k = 1}^{\infty} a_k
   = \sum\limits_{k = 1}^{\infty} b_k - \sum\limits_{k = 1}^{\infty} c_k$,
   то не меняется и сумма ряда $\sum\limits_{k = 1}^{\infty} a_k$.
\end{proof}
\begin{thm}[Риман]
	Если $\sum\limits_{k = 1}^{\infty} a_k$ сходится условно, то
	\[\forall A,\ - \infty \leq A \leq + \infty \quad \exists\;
	\text{перестановка членов ряда } \sum\limits_{k = 1}^{\infty} a_k \text{ 
	т.~ч. }
	\sum\limits_{n = 1}^{\infty} p_n = A.\]
	\begin{proof}
		Доказательство проводится по следующей схеме.
		
		Пусть $A$~--- число. Т.~к. ряд
		$\sum\limits_{k = 1}^{\infty} a_k$ сходится условно, то оба ряда
		$\sum\limits_{k = 1}^{\infty} b_k$ и
		$\sum\limits_{k = 1}^{\infty} c_k$, где
		\[b_k = \begin{cases}
			a_k,& a_k > 0\\
			0,& a_k \le 0
		  \end{cases} \qquad
		 c_k = \begin{cases}
			0,& a_k > 0\\
			-a_k,& a_k \le 0
		  \end{cases}\]
		  имеют бесконечные суммы. Т.~е. у нас есть <<банк>> положительных
		  элементов	ряда $\sum\limits_{k = 1}^{\infty} a_k$ и <<банк>> \
		  отрицательных элементов ряда $\sum\limits_{k = 1}^{\infty} a_k$.
		  Будем брать элементы ряда $\sum\limits_{k = 1}^{\infty} b_k$
		  по порядку так, чтобы их сумма стала чуть больше, чем $A$.
		  Затем добавляем отрицательные элементы, представленные в ряде
		  $\sum\limits_{k = 1}^{\infty} c_k$ до тех пор, пока набранная
		  сумма не станет меньше, чем $A$. Затем снова добавляем положительные
		  элементы, чтобы сумма стала больше $A$. Поскольку $a_k \to 0$,
		  то полученная последовательность cумм будет стремиться к $A$.

		  Если $A = +\infty$, то набираем, например, положительные
		  слагаемые ряда $\sum\limits_{k = 1}^{\infty} a_k$ до тех пор
		  пока их сумма не станет больше $10$, затем один отрицательный элемент,
		  затем снова положительные элементы, пока сумма не станет больше $20$, 
		  затем снова один
		  отрицательный элемент и т.~д. Полученная перестановка стремится к
		  $+ \infty$.
		  
		  Аналогичные рассуждения можно применить в случае $A = -\infty$.
	\end{proof}
\end{thm}
\subsection{Перемножение рядов}
\begin{defn}
Рассмотрим два ряда
\begin{equation}
 \label{30:7}
 \sum\limits_{k = 1}^{\infty} a_k
\end{equation}
и
\begin{equation}
 \label{30:8}
 \sum\limits_{k = 1}^{\infty} b_k.
\end{equation}

\emph{Произведением} рядов \eqref{30:7} и
\eqref{30:8} называют ряд
\[\sum\limits_{\forall i,j}^{\infty} a_i b_j,\] т.~е. ряд, содержащий все
возможные произведения элементов рядов \eqref{30:7} и
\eqref{30:8}. Обозначим этот ряд как
\begin{equation}
	\label{lec30, series_multiplication}
	\sum\limits_{n = 1}^{\infty} c_n.
\end{equation}
\end{defn}
\begin{thm}[Коши]
	Пусть ряды $\sum\limits_{k = 1}^{\infty} a_k$ и
	$\sum\limits_{k = 1}^{\infty} b_k$ сходятся абсолютно, причем
	$\sum\limits_{k = 1}^{\infty} a_k = A$ и
	$\sum\limits_{k = 1}^{\infty} b_k = B$. Тогда ряд-произведение
	\eqref{lec30, series_multiplication} сходится абсолютно и его сумма равна
	\[\sum\limits_{n = 1}^{\infty} c_n = A \cdot B, \]
	т.~е.
	\[\sum\limits_{k = 1}^{\infty} a_k \cdot
	\sum\limits_{k = 1}^{\infty} b_k =
	\sum\limits_{\forall i,j}^{\infty} a_i b_j.\]
\end{thm}
\begin{proof}
Покажем, что $\sum\limits_{n = 1}^{\infty} |c_n|$ сходится. Рассмотрим
его частную сумму
\[|c_1| + |c_2| + \ldots + |c_m| = |a_{i_1} \cdot b_{j_1}| +
|a_{i_2} \cdot b_{j_2}| + \ldots + |a_{i_m} \cdot b_{j_m}| \leq *\]
Пусть $l = \max\{i_1, j_1, \ldots i_m, j_m\}$. Тогда
\[* \leq (|a_1| + \ldots + |a_l|)(|b_1| + \ldots + |b_l|)
\leq\ \sum\limits_{k = 1}^{\infty} |a_k| \cdot
\sum\limits_{k = 1}^{\infty} |b_k| = M\]
Частная сумма $\sum\limits_{n = 1}^{\infty} |c_n|$ не превосходит $M$. Тогда 
$\sum\limits_{n = 1}^{\infty} |c_n|$ сходится, т.~е.
$\sum\limits_{n = 1}^{\infty} c_n$ сходится абсолютно.

Так как ряд $\sum\limits_{n = 1}^{\infty} c_n$ сходится абсолютно, то допустимы
любые перестановки его членов, и сумма при этом не меняется. Поэтому при
вычислении суммы этого ряда можно использовать любую удобную
перестановку. Например, можно в качестве последовательности его частичных
сумм рассмотреть \[(a_1 + \ldots + a_l)(b_1 + \ldots + b_l).\] В ней
содержится любое произведение $a_i b_j$ при достаточно большом $l$, т.~е.
при $l \to \infty$ в эту последовательность попадут все элементы из
$\sum\limits_{\forall i,j}^{\infty} a_i b_j$. Т.~е.
\[\sum\limits_{\forall i,j}^{\infty} a_i b_j =
\lim\limits_{l \to \infty} (a_1 + \ldots + a_l)(b_1 + \ldots + b_l) =
A \cdot B. \qedhere\]
\end{proof}
Существует и часто используется специальный способ перемножения рядов по 
методу Коши:
\[(a_1 + a_2 + a_3 + \ldots)(b_1 + b_2 + b_3 + \ldots) = a_1 b_1 +
(a_1 b_2 + a_2 b_1) + (a_1 b_3 + a_2 b_2 + a_3 b_1) + \ldots +
\sum\limits_{i + j = n + 1} a_i b_j + \ldots\]
\begin{thm}[Мертенс]
Если ряды $\sum\limits_{k = 1}^{\infty} a_k$ и
$\sum\limits_{k = 1}^{\infty} b_k$ сходятся и хотя бы один из них
сходится абсолютно, то ряд \eqref{lec30, series_multiplication} сходится
и его сумма равна $A \cdot B$.
\end{thm}

Если $\sum\limits_{k = 1}^{\infty} a_k$ и
$\sum\limits_{k = 1}^{\infty} b_k$ условно сходятся, то их произведение
может быть расходящимся.
\begin{example}
	\[\sum\limits_{k = 1}^{\infty} \frac{(-1)^{k-1}}{\sqrt{k}} \cdot
	\sum\limits_{k = 1}^{\infty} \frac{(-1)^{k-1}}{\sqrt{k}} =
	\left(1 - \frac{1}{\sqrt{2}} + \frac{1}{\sqrt{3}} + \cdots +
	\frac{(-1)^{n-1}}{\sqrt{n}} + \cdots \right) \cdot \]
	\[\cdot \left(1 - \frac{1}{\sqrt{2}} + \frac{1}{\sqrt{3}} + \cdots +
	\frac{(-1)^{n-1}}{\sqrt{n}} + \cdots \right) = 1 +
	\left( - \frac{1}{\sqrt{2}} - \frac{1}{\sqrt{2}} \right) +
	\left( \frac{1}{\sqrt{3}} + \frac{1}{\sqrt{2}} \cdot
	\frac{1}{\sqrt{2}} + \frac{1}{\sqrt{3}} \right) + \] 
	\[ + \left( -1 \right) ^ {n - 1}
	\underbrace{\left(1 \cdot \frac{1}{\sqrt{n}} + \frac{1}{\sqrt{2}} \cdot
	\frac{1}{\sqrt{n - 1}} + \frac{1}{\sqrt{3}} \cdot
	\frac{1}{\sqrt{n - 2}} + \ldots + \frac{1}{\sqrt{n}}
	\cdot 1\right)}_{\widetilde{c}_n} + \cdots \]
	Cумма $\widetilde{c}_n$ содержит $n$ слагаемых, каждое из которых не меньше,
	чем $\dfrac{1}{n}$. Поэтому $|c_n| \geq 1$, $c_n \not\to 0$, т.~е.
	ряд расходится.
\end{example}
\begin{remark}
	Произведение расходящихся рядов может оказаться сходящимся.
\end{remark}
\section{Ряды в $\C$}
Рассмотрим ряд
\begin{equation}
\label{30:?}
\sum\limits_{k = 1}^{\infty} z_k , \quad z_k =
\alpha_k + i\beta_k \in \C. 
\end{equation}
Сходимость ряда определяется по той же схеме, что и ранее. Рассмотрим 
частичные суммы
\[ \sigma_n = \sum\limits_{k = 1}^{\infty} z_k =
\sum\limits_{k = 1}^{\infty} (\alpha_k + i\beta_k).\]
Предел $\lim\limits_{n \to \infty} \sigma_n = \sigma$,
если он существует, называют \emph{суммой ряда}. Если этот предел конечен, то 
ряд сходится.
Поскольку $\sigma_n = \sum\limits_{k = 1}^{\infty} (\alpha_k + i\beta_k) =
\sum\limits_{k = 1}^{\infty} \alpha_k +
i \sum\limits_{k = 1}^{\infty} \beta_k$,
то это приводит к рядам $\sum\limits_{k = 1}^{\infty} \alpha_k$ и
$\sum\limits_{k = 1}^{\infty} \beta_k$.
Конечный предел $\sigma_n$ существует тогда и только тогда, когда
существуют конечные пределы $\sum\limits_{k = 1}^{n} \alpha_k$ и
$\sum\limits_{k = 1}^{n} \beta_k$.
Т.~е. $\sum\limits_{k = 1}^{\infty} z_k$ сходится тогда и только тогда, когда
сходятся оба ряда $\sum\limits_{k = 1}^{\infty} \alpha_k$ и $
\sum\limits_{k = 1}^{\infty} \beta_k$. При этом сумма ряда равна
\[\sum\limits_{k = 1}^{\infty} z_k = \sum\limits_{k = 1}^{\infty}
(\alpha_k + i\beta_k) = \sum\limits_{k = 1}^{\infty} \alpha_k +
i\sum\limits_{k = 1}^{\infty} \beta_k.\]
Необходимое условие сходимости рядов в $\C$: $\lim\limits_{k \to \infty} z_k = 
0$.

\begin{thm}[критерий Коши] Ряд \eqref{30:?} сходится тогда и только тогда, 
когда

\[ \forall \eps > 0 \quad \exists \nu = \nu_\eps \quad \forall n, m \in \N : 
m > n \geq \nu \implies
\abs{\sum_{k=n+1}^m z_n} \leq \eps. \]
\end{thm}

\end{document}
