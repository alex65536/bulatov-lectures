\makeatletter
\def\input@path{{../../}}
\makeatother
\documentclass[../../main.tex]{subfiles}

\graphicspath{
	{../../img/}
	{../img/}
	{img/}
}

\begin{document}
	\section{Знакопеременные ряды}
	
	Рассмотрим ряд
	\begin{equation}
		\sum_{n = 1}^\infty a_n
		\label{29:1}
	\end{equation}
	без условия $a_n > 0$. Вместе с рядом \eqref{29:1} 
	рассмотрим также ряд
	\begin{equation}
		\sum_{n = 1}^\infty |a_n|.
		\label{29:2}
	\end{equation}
	\begin{definition}
		Если ряд \eqref{29:2} сходится, то говорят, что ряд \eqref{29:1} 
		\emph{сходится 
		абсолютно}.
	\end{definition}
	
	\begin{thm}\label{lec29:converge}
		Если ряд сходится абсолютно, то он сходится.
	\end{thm}
	\begin{proof}
		Пусть ряд \eqref{29:2} сходится. Тогда, по критерию Коши сходимости рядов, 
		имеем
		\[\forall \eps > 0\ \exists\nu = \nu_\eps\ \colon\ \forall n > 
		m \geq \nu_\eps \implies 
		\left|\sum_{k = m + 1}^n |a_k|\right| =
		\sum_{k = m + 1}^n |a_k|\leq\eps.\]
		
		Но тогда
		\[\forall \eps > 0\ \exists\nu = \nu_\eps\ \colon\ \forall n > 
		m 
		\geq \nu_\eps \implies \left|\sum_{k = m + 1}^n a_k\right|
		\leq\sum_{k = m + 1}^n |a_k|\leq\eps.\]
		Значит, ряд \eqref{29:1} сходится по критерию Коши.
	\end{proof}

\begin{remark}
	Заметим, что ряд \eqref{29:2} является положительным рядом, т.~е. к нему 
	применимы все признаки для положительных рядов.
\end{remark}
	
	\begin{example}
		Рассмотрим ряд
		\[\sum_{n = 1}^\infty \frac{\sin n}{n\sqrt{n + 1}}.\]
		
		Исследуем на абсолютную сходимость. Имеем:
		\[\left|\frac{\sin n}{n\sqrt{n + 1}}\right|\leq\frac{1}{n\sqrt{n + 
		1}}\leq\frac{1}{n^{3/2}}.\]
		
		Последний ряд сходится, так как является обобщённым гармоническим рядом с 
		показателем, большим единицы. 
		А значит, по \hyperref[lec26:comp_test_1]{признаку 1\textdegree}, 
		исходный ряд сходится абсолютно, а тогда, 
		по доказанной выше теореме \ref{lec29:converge}, он сходится.
	\end{example}

	При изучении знакопеременных рядов выделяют специальные ряды вида
	\begin{equation}
		\sum_{n = 1}^\infty (-1)^{n - 1}a_n\quad a_n > 0,
		\label{29:3}
	\end{equation}
	которые называют \emph{знакочередующимися рядами}. 
	Для этих рядов может быть использован \emph{признак Лейбница}.
	\begin{thm}[признак Лейбница сходимости знакочередующихся рядов]
		\label{29:leibnitz}
		Если $a_n$ стремится к нулю монотонно, то \eqref{29:3} сходится.
	\end{thm}
	\begin{proof}
		Рассмотрим частные суммы \eqref{29:3}, а именно
		\begin{gather*}
			s_{2n} = (a_1 - a_2) + (a_3 - a_4) + \ldots + (a_{2n - 3} - a_{2n - 2}) + 
			\underbrace{(a_{2n - 1} - a_{2n})}_{> 0} >\\
			> (a_1 - a_2) + (a_3 - a_4) + \ldots + (a_{2n - 3} - a_{2n - 2}) = s_{2n - 
			2}.
		\end{gather*}
		Получаем, что $s_{2n}\uparrow$.
		
		Вместе с тем,
		\[s_{2n} = a_1 - (a_{2} - a_3) - \ldots - (a_{2n - 2} - a_{2n - 1}) - a_{2n} 
		< a_1,\]
		где скобки положительны ввиду монотонности $(a_n)$.
		
		Т.~е. $s_{2n}\uparrow$ и $s_{2n} < a_1$, а, 
		значит, последовательность $s_{2n}$ сходится:
		\begin{equation}
			\exists S = \lim\limits_{n\to\infty}s_{2n}\in\R.
			\label{29:4}
		\end{equation}
		Также 
		\begin{equation}
			s_{2n + 1} = s_{2n} + a_{2n + 1}
			\appr{n\to\infty} S.
			\label{29:5}
		\end{equation}
		
		Из \eqref{29:4}, \eqref{29:5} получаем, что $\lim\limits_{n\to\infty} s_n = 
		S$, 
		т.~е. ряд 
		\eqref{29:3} сходится.
	\end{proof}

	В дополнение к признаку Лейбница можно 
	сформулировать следующее утверждение: если 
	выполнены все условия теоремы, то верна
	оценка сходимости ряда \eqref{29:3}:
	\begin{equation}
		\left|s_n - s\right|\leq a_{n + 1}.
		\label{29:6}
	\end{equation}
	\begin{proof}
		Действительно, по ходу доказательства признака Лейбница мы показали, что 
		$s_{2n}\uparrow s$. 
		Покажем теперь, что $s_{2n + 1}\downarrow s$:
		\[s_{2n + 1} = a_1 - (a_2 - a_3) - \ldots - \underbrace{(a_{2n} - a_{2n + 
		1})}_{\ge 0} < s_{2n - 
		1}.\]
		
		Тогда для $s_{2n}$ получаем
		\[s - s_{2n} \leq s_{2n + 1} - s_{2n} = a_{2n + 1}
		\implies |s_{2n} - s|\leq a_{2n + 1}.\]
		
		Для $s_{2n + 1}$ имеем
		\[|s_{2n + 1} - s| = s_{2n + 1} - s 
		\leq s_{2n + 1} - s_{2n + 3} = 
		a_{2n + 2} - a_{2n + 3}\leq a_{2n + 2}.\]
		Т.~е. $|s_n - s|\leq a_{n + 1}$, что соответствует \eqref{29:6}.
	\end{proof}

	\begin{example}
		Рассмотрим ряд
		\[\sum_{k = 1}^\infty\frac{(-1)^{k - 1}}{k}.\]
		Заметим, что ряд
		\[\sum_{k = 1}^\infty\abs{\frac{(-1)^{k - 1}}{k}} = 
		\sum_{k = 1}^\infty\frac{1}{k}\]
		расходится, значит, абсолютной сходимости нет.
		
		Однако $\cfrac{1}{k}\downarrow0$, 
		т.~е. выполнены все условия \hyperref[29:leibnitz]{признака Лейбница}. 
		Значит, 
		ряд сходится.
		
		Этот ряд называют \emph{рядом Лейбница}.
	\end{example}

\begin{definition}
	Если \eqref{29:1} сходится, а \eqref{29:2} расходится, то говорят, то 
	\eqref{29:1} \emph{сходится условно}.
\end{definition}
	
	\subsection{Признаки Дирихле и Абеля}
	
	Будем рассматривать сумму $\sum\limits_{i = 1}^n a_ib_i$. Обозначим
	\begin{gather*}
	 B_1 = b_1, \\
	 B_2 = b_1 + b_2, \\
	 \ldots \\
	 B_k = \sum_{i=1}^k b_i.
	\end{gather*}
	Из этих обозначений получаем
	\begin{gather*}
	 b_1 = B_1, \\
	 b_2 = B_2 - B_1, \\
	 b_3 = B_3 - B_2, \\
	 \ldots \\
	 b_n = B_n - B_{n - 1}.
	\end{gather*}
	Тогда
	\begin{gather*}
		\sum_{i = 1}^n a_ib_i = a_1b_1 + a_2b_2 + a_3b_3 + \ldots + a_nb_n = \\
		= a_1B_1 + a_2(B_2 - B_1) + a_3(B_3 - B_2) + \ldots + a_n(B_n - B_{n - 1}) = 
		\\
		= B_1(a_1 - a_2) + B_2(a_2 - a_3) + \ldots + B_{n - 1}(a_{n - 1} - a_n) 
		+ B_na_n.
	\end{gather*}
	Таким образом получаем, что
	\begin{equation}
		\sum_{i = 1}^na_ib_i = \sum_{i = 1}^{n - 1}B_i(a_i - a_{i + 1}) + a_nB_n
		\label{29:7}
	\end{equation}
	
	\begin{definition}
		\eqref{29:7} называется \emph{преобразованием Абеля}.
	\end{definition}
	
	\begin{lemma}[Абель]\label{29:lemma_abel}
		Пусть последовательность $(a_n)$ монотонна и 
		$\exists M\colon |B_i|\leq M$, $\forall i = \overline{1,n}$. 
		Тогда
		\[\left|\sum_{i = 1}^n a_ib_i\right|\leq M(|a_1| + 2|a_n|).\]
	\end{lemma}
	\begin{proof}
		Воспользуемся преобразованием Абеля \eqref{29:7}:
		\begin{gather*}
			\left|\sum_{i = 1}^n a_ib_i\right| = 
			\left|\sum_{i = 1}^{n - 1}B_i(a_i - a_{i + 1}) + a_nB_n\right|
			\leq \sum_{i = 1}^{n - 1}|B_i||a_{i} - a_{i + 1}| + |a_n||B_n| \leq \\
			\leq M\left(\sum_{i = 1}^{n - 1}|a_i - a_{i + 1}| + |a_n|\right) = *
		\end{gather*}
		
		Так как $(a_n)$ монотонна, то разности $a_i - a_{i + 1}$ имеют один и 
		тот же знак (либо <<$+$>>, когда $a_n\downarrow$, либо <<$-$>>, 
		когда $a_n\uparrow$).
		Тогда
		\[* = M|a_1 - \cancel{a_2} + \cancel{a_2} - \cancel{a_3} + \ldots + 
		\cancel{a_{n - 1}} + a_n| + M|a_n| = M|a_1 - a_n| + M|a_n|\leq
		 M(|a_1| + 2|a_n|).\qedhere\]
	\end{proof}

	Будем далее рассматривать ряды вида
	\begin{equation}
		\sum_{k = 1}^\infty a_kb_k.
		\label{29:8}
	\end{equation}
	
	\begin{thm}[признак Дирихле]\label{29:dirichle}
		Если
		\begin{enumerate}
			\item $a_k\downarrow 0$ при $k\to\infty$;
			\item $\exists M\colon |B_k|\leq M\iff 
			\left|\sum\limits_{i = 1}^k b_i\right|\leq M \quad \forall k \in \N$,
		\end{enumerate}
		то ряд \eqref{29:8} сходится.
	\end{thm}
	\begin{proof}
		Воспользуемся определением сходимости $a_n$ к нулю. 
		Выберем произвольное $\eps > 0$. Так как $a_k \downarrow 0$, то 
		$\exists \nu = \nu_\eps\colon \forall n\geq \nu_\eps\implies 
		a_n\leq\eps$.
		
		Зафиксируем произвольные $n > m\geq \nu_\eps$. Имеем
		\[\left|\sum_{k = m + 1}^n b_k\right|\leq 
		\left|\sum_{k = 1}^n b_k\right| + \left|\sum_{k = 1}^m b_k\right|
		\leq 2M.\]
		Тогда
		\[\left|\sum_{k = m + 1}^n a_kb_k\right|\leq
		\left[\text{используем \hyperref[29:lemma_abel]{лемму Абеля}}\,\right]\leq
		 2M(|a_{m + 1}| + 2|a_n|)\leq 6M\eps.\]
			
		Значит, \eqref{29:8} сходится по критерию Коши.
	\end{proof}

	\begin{example}
		Рассмотрим ряд
		\[\sum_{k = 1}^\infty\frac{\sin k\alpha}{k}.\]
		
		Если $\alpha = 0$ или $\alpha = \pi$, то все члены ряда равны нулю, 
		и ряд сходится. Пусть теперь $\alpha\in (0;\pi)$. 
		
		Пусть $a_k = \cfrac{1}{k}\downarrow 0$, $b_k = \sin k\alpha$. 
		Тогда получаем, что
		\begin{gather*}
			\sum_{k = 1}^n b_k = \sum_{k = 1}^n\sin k\alpha =
			\frac{1}{\sin\frac{\alpha}{2}}
			\sum_{k = 1}^n \sin k\alpha\cdot\sin\frac{\alpha}{2} = \\
			= \frac{1}{2\sin\frac{\alpha}{2}}
			\sum_{k = 1}^n\left(\cos\left(k\alpha - \frac{\alpha}{2}\right) - 
			\cos\left(k\alpha + \frac{\alpha}{2}\right)\right) = \\
			= \frac{1}{2\sin\frac{\alpha}{2}}\left(\cos\frac{\alpha}{2} - 
			\cancel{\cos\frac{3\alpha}{2}} + \cancel{\cos\frac{3\alpha}{2}} - 
			\ldots + \cancel{\cos\frac{(2n - 1)\alpha}{2}} - 
			\cos\frac{(2n + 1)\alpha}{2}\right) = \\
			= \frac{1}{2\sin\frac{\alpha}{2}}\left(\cos\frac{\alpha}{2} - 
			\cos\frac{(2n + 1)\alpha}{2}\right).
		\end{gather*}
		
		Поэтому
		\[\left|\sum_{k = 1}^n \sin k\alpha\right|\leq 
		\frac{1}{\left|\sin\frac{\alpha}{2}\right|} = M\quad \forall n \in \N.\]
		
		Выполнены все условия \hyperref[29:dirichle]{признака Дирихле}, т.~е. ряд 
		сходится.
	\end{example}

	\begin{thm}[признак Абеля]
		\label{29:abel}
		Пусть
		\begin{enumerate}
			\item последовательность $(a_n)$ монотонна и ограничена;
			\item ряд $\sum\limits_{k = 1}^\infty b_k$ сходится.
		\end{enumerate}
	
		Тогда ряд \eqref{29:8} сходится.
	\end{thm}
	\begin{proof}
		Так как $(a_n)$ монотонна и ограничена, то эта последовательность сходится, 
		т.~е. $\exists a = 
		\lim\limits_{n\to\infty} a_n\in \R$.
		
		Для определенности будем считать, что $a_n\uparrow$.
		Обозначим $c_k = a - a_k$, т.~е. $a_k = a - c_k$. 
		Тогда для частной суммы \eqref{29:8} имеем
		\begin{equation}
			\sum_{k = 1}^n a_kb_k = \sum_{k = 1}^n ab_k - \sum_{k = 1}^n c_kb_k.
			\label{29:9}
		\end{equation}
		
		Так как ряд $\sum\limits_{k = 1}^\infty b_k$ сходится, 
		то его частные суммы ограничены, т.~е. 
		\[\exists M\colon \left|\sum\limits_{k = 1}^n b_k\right|\leq M \quad
		\forall n \in \N.\]
		Также $c_k = a - a_k \downarrow 0$ и для ряда
		$\sum\limits_{k = 1}^\infty c_kb_k$
		выполнены все условия \hyperref[29:dirichle]{признака Дирихле}. 
		Значит, ряд сходится.
		
		Тогда в \eqref{29:9} при $n\to\infty$ 
		справа будет конечный предел, т.~е. для частных сумм ряда \eqref{29:8} также 
		существует конечный предел. Значит, ряд \eqref{29:8} сходится.
	\end{proof}
	\begin{example}
		Рассмотрим ряд
		\[\sum_{n = 2}^\infty\frac{\sin n\arctg n}{\ln n}.\]
		
		Обозначим $a_n = \arctg n$, $b_n = \cfrac{\sin n}{\ln n}$. 
		Имеем, что $a_n\uparrow$ и $|a_n|\leq \cfrac{\pi}{2}$.
		
		Для ряда \[\sum_{k = 2}^\infty b_k\] 
		имеем
		\[\frac{1}{\ln n}\downarrow0,\quad \left|\sum_{k = 2}^n \sin k\right|
		\leq \frac{1}{\sin\frac{1}{2}} \quad\forall n \in \N.\]
		Значит, ряд $\sum\limits_{k = 2}^\infty b_k$ сходится по 
		\hyperref[29:dirichle]{признаку Дирихле}, а отсюда получаем, 
		что и исходный ряд сходится по \hyperref[29:abel]{признаку Абеля}.
	\end{example}

	\subsection{Абсолютная и условная сходимость}
	
	Рассмотрим ряд \eqref{29:1} с элементами произовального знака. Наряду с ним 
	рассмотрим два ряда 
	\begin{equation}
		\sum\limits_{k = 1}^\infty b_k,\ \sum\limits_{k = 1}^\infty c_k,
		\label{29:10}
	\end{equation} 
	где
	\[b_k = \begin{cases}
		a_k,& a_{k} > 0\\
		0,& a_{k}\leq0
	\end{cases}\]
	\[c_k = \begin{cases}
		0,& a_{k} > 0\\
		-a_k,& a_{k}\leq 0
	\end{cases}\]
	
	Ясно, что $a_k = b_k - c_k$ $\forall k\in\N$.
	
	\begin{thm}[критерий абсолютной сходимости]\label{29:absolute_conv}
		Ряд $\sum\limits_{k = 1}^\infty a_k$ сходится абсолютно 
		тогда и только тогда, когда сходятся оба ряда 
		$\sum\limits_{k = 1}^\infty b_k$, $\sum\limits_{k = 1}^\infty c_k$, 
		причём
		\begin{equation}
			\sum_{k = 1}^\infty a_k = \sum_{k = 1}^\infty b_k - \sum_{k = 1}^\infty c_k.
			\label{29:11}
		\end{equation}
	\end{thm}
	\begin{proof}
		~
		
		\nec: Пусть ряд \eqref{29:1} сходится 
		абсолютно, т.~е. сходится ряд $\sum\limits_{k = 1}^\infty |a_k|$.
		Но тогда $b_k\leq |a_k|$, $c_k\leq |a_k|$, $\forall k\in\N$. 
		Тогда, по \hyperref[lec26:comp_test_1]{признаку 1\textdegree}, 
		сходятся ряды \eqref{29:10}.
		
		\suff: Пусть оба ряда \eqref{29:10} сходятся. 
		Тогда
		\[\sum_{k = 1}^\infty |a_k|\leq \sum_{k = 1}^\infty |b_k| + 
		\sum_{k = 1}^\infty|c_k| = \sum_{k = 1}^\infty b_k + 
		\sum_{k = 1}^\infty c_k.\]
		
		Так как оба ряда \eqref{29:10} сходятся, 
		то частные суммы ряда \eqref{29:2} ограничены и монотонны, т.~е. исходный 
		ряд сходится абсолютно.
		
		Докажем вторую часть. Так как 
		\[\sum_{k = 1}^n a_k = \sum_{k = 1}^n b_k - \sum_{k = 1}^n c_k,\]
		то при $n\to\infty$ получаем \eqref{29:11}.
	\end{proof}

	\begin{thm}[необходимое условие условной 
	сходимости]\label{29:conditional_conv}
		Если ряд \eqref{29:1} сходится условно, 
		то оба ряда \eqref{29:10} имеют бесконечные суммы.
	\end{thm}
	\begin{proof}
		Действительно, для частичных сумм \eqref{29:1} имеем 
		\[\sum_{k = 1}^n a_k = \sum_{k = 1}^n b_k - \sum_{k = 1}^n c_k.\]
		
		Пусть $n\to\infty$. Если оба ряда \eqref{29:10} сходятся, 
		то, по \hyperref[29:absolute_conv]{критерию абсолютной сходимости} 
		получаем, что \eqref{29:1} сходится абсолютно, что противоречит 
		условной сходимости \eqref{29:1}.
		
		Если справа одна из сумм имеет конечный предел, а вторая~--- бесконечный,
		 то и слева будет бесконечный предел, т.~е. \eqref{29:1} не сходится,
		  что противоречит сходимости \eqref{29:1}.
		
		Значит, оба ряда \eqref{29:10} имеют бесконечные суммы.
	\end{proof}
\end{document}
