\makeatletter
\def\input@path{{../../}}
\makeatother
\documentclass[../../main.tex]{subfiles}

\graphicspath{
	{../../img/}
	{../img/}
	{img/}
}

\begin{document}
\subsection{Признак Гаусса}

Пусть \[\dfrac{a_n}{a_{n+1}} = \lambda + \dfrac{\mu}{n} +
\dfrac{\theta_n}{n^{1 + \eps}},\]
где $\theta_n $ ограничена, $\eps > 0$

Тогда если:
\begin{itemize}
	 \item[] $\lambda > 1$, то ряд сходится
	 \item[] $\lambda < 1$ ряд расходится
	 \item[] $\lambda = 1
	 \begin{cases}
	 	\text{Если } \mu > 1 \text{ ряд сходится}\\
	 	\mu \leq 1 \text{ ряд расходится} 
	 \end{cases}$
\end{itemize}

\begin{example}
$ $

$\sum\limits_{k = 1}^{\infty}\left(\dfrac{(2k - 1)!!}{(2k)!!} \right)^p
\dfrac{1}{k^q}$
\[
\dfrac{a_{n}}{a_{n + 1}} = \left(\dfrac{(2n - 1)!!}{(2n)!!}\right)^p \cdot
\dfrac{1}{n^q} \cdot\left( \dfrac{(2n + 2)!!}{(2n + 1)!!}\right)^p \cdot
(n + 1)^q = \left(\dfrac{2n + 2}{2n + 1}\right)^p \cdot
\left(\dfrac{n+1}{n}\right)^q = \left(1 + \dfrac{1}{1 + 2n}\right)^p \cdot
\left(1 + \dfrac{1}{n}\right)^q =
\]
\[
= \left(1 + \dfrac{q}{n} + O\left(\dfrac{1}{n^2}\right)\right) \cdot
\left(1 + \dfrac{p}{2n + 1} + O\left(\dfrac{1}{n^2}\right)\right) =
\left(1 + \dfrac{q}{n} + O\left(\dfrac{1}{n^2}\right)\right) \cdot
\left(1 + \dfrac{p}{2n} \cdot \dfrac{1}{1 + \frac{1}{2n}}
+ O\left(\dfrac{1}{n^2}\right)\right) =
\]
\[
= \left(1 + \dfrac{q}{n} + O\left(\dfrac{1}{n^2}\right)\right) \cdot \left(1 +
\dfrac{p}{2n} \cdot \left(1 - \dfrac{1}{2n} + O\left(\dfrac{1}{n^2}
\right)\right)\right) =
\]
\[
= \left(1 + \dfrac{q}{n} + O\left(\dfrac{1}{n^2}\right)\right) \cdot \left(1 +
\dfrac{p}{2n} + O\left(\dfrac{1}{n^2}\right)\right) = 1 + \left(\dfrac{p}{2} +
q \right) \cdot \dfrac{1}{n} + O\left(\dfrac{1}{n^2}\right)
\]

Сходится при $\dfrac{p}{2} + q > 1$

Расходится при $\dfrac{p}{2} + q \leq 1$ 
\end{example}

\subsection{Формула Валлиса}

\[
\mathrm{I}_n = \int\limits_0^{\frac{\pi}{2}} \cos^n x dx, \;\;\;
\mathrm{I}_0 \dfrac{\pi}{2} \;\;\; \mathrm{I}_{1} = 1
\]
\[
a_n =  \int\limits_0^{\frac{\pi}{2}} \cos^{n - 1} x \cdot \cos x dx
\left[\begin{array}{l}
\upsilon = \cos^{n-1}x\\
d\upsilon = (n - 1) \cos^{n-1}x \sin x dx\\
d \nu = \cos dx\\
\nu = \sin x
\end{array}\right]
= \sin x \cos^{n-1} x \big|_0^{\frac{\pi}{2}} + 
\]
\[
+ (n - 1) \cdot \int\limits_{0}^{\frac{\pi}{2}} \cos^{n-2}x sin^2 x dx
= (n - 1) \int\limits_{0}^{\frac{\pi}{2}} \cos^{n-2}x \cdot (1 - \cos^{2}x)dx =
(n-1)\mathrm{I}_{n-2} - (n-1)\mathrm{I}_{n}
\]
\[\mathrm{I}_n = \dfrac{(n-1)}{n}\mathrm{I}_{n-2} \;\;\; n>1\]
$n = 2m$
\[
\mathrm{I}_{2m} = \dfrac{2m - 1}{2m} \cdot \mathrm{I}_{2m-2} = \dfrac{2m-1}{2m}
\cdot \dfrac{2m - 3}{2m-2} \cdot \mathrm{I}_{2m- 4} = \ldots =
\dfrac{(2m - 1)!!}{(2m)!!} \cdot \dfrac{\pi}{2}
\]
\[\mathrm{I}_{2m} = \dfrac{(2m - 1)!!}{(2m)!!} \cdot \dfrac{\pi}{2}\]
$n = 2m + 1$
\[
\mathrm{I}_{2m + 1} = \dfrac{2m}{2m + 1} \cdot \mathrm{I}_{2m-1} = \ldots =
\dfrac{(2m)!!}{(2m + 1)!!} \cdot \mathrm{I}_{1} = \dfrac{(2m)!!}{(2m + 1)!!} 
\]

\[\mathrm{I}_{2m + 1} = \dfrac{(2m)!!}{(2m + 1)!!}\]

\[
\mathrm{I}_{2m+2}\le \mathrm{I}_{2m+1}\le \mathrm{I}_{2m}
\]
\[
\dfrac{(2m+1)!!}{(2m+2)!!} \cdot \dfrac{\pi}{2} \le \dfrac{(2m)!!}{(2m+1)!!} 
\le \dfrac{\pi}{2} \cdot \dfrac{(2m-1)!!}{(2m)!!}
\]
Умножив неравенство на $\dfrac{(2m)!!}{(2m-1)!!}$, получим:
\[
\dfrac{2m+1}{2m+2} \cdot \dfrac{\pi}{2} \le \left( \dfrac{(2m)!!}
{(2m-1)!!}  \right)^2 \cdot \dfrac{1}{2m+1} \le \dfrac{\pi}{2}
\]
\[
\lim\limits_{m \to \infty} \left( \dfrac{(2m)!!} {(2m-1)!!}  \right)^2
\cdot \dfrac{1}{2m+1} = \dfrac{\pi}{2}
\]

\subsection{Формула Стирлинга}
Рассмотрим последовательность
\[\sigma_n = \dfrac{n! \cdot e^n}{n^n \cdot \sqrt{n}}\]
Обозначим $ S_n = \ln{\sigma_n},\; U_k = (S_k - S_{k-1})$ и рассмотрим ряд:
\begin{equation}
\label{stirling} S_1 + \sum\limits_{k=2}^\infty U_k
\end{equation}
Заметим, что частные суммы ряда:
\[ 
S_1 + U_2 + U_3 + \ldots + U_n = S_1 + \left(S_2 - S_1\right) + 
\left(S_3 - S_2\right) + \ldots + \left(S_n - S_{n - 1}\right) = S_n
\]
Изучим сходимость ряда \eqref{stirling}:
\[
U_n = S_n - S_{n - 1} = \ln\sigma_n - \ln\sigma_{n - 1} = 
\ln\dfrac{\sigma_n}{\sigma_{n - 1}} = \ln\dfrac{n! \cdot e^n}{n^n\sqrt{n}}
\cdot \dfrac{(n - 1)^{n - 1}\sqrt{n - 1}}{(n - 1)! \cdot e^{n -1}} =
\] 
\[
= \ln\dfrac{(n - 1)^{n - 1}}{n^{n - 1}} \cdot \dfrac{\sqrt{n - 1}}{\sqrt{n}}
\cdot e = \ln e \left(1 - \dfrac{1}{n} \right)^{n - \frac{1}{2}} = 
1 + \left(n - \dfrac{1}{2} \right) \ln \left( 1 - \dfrac{1}{n} \right) =
\]
\[
= 1 + \left( n - \dfrac{1}{2} \right)\left(-\dfrac{1}{n} - \dfrac{1}{2n^2}
- \dfrac{1}{3n^3} + O\left(\dfrac{1}{n^3} \right) \right) = 
1 - 1 - \dfrac{1}{2n} - \dfrac{1}{3n^2} + \dfrac{1}{2n} + \dfrac{1}{4n^2}
+ O\left( \dfrac{1}{n^3} \right) = 
\]
\[= -\dfrac{1}{12n^2} + O\left( \dfrac{1}{n^3}\right)\]
$U_n \thicksim -\dfrac{1}{12n^2} \implies$  \eqref{stirling} сходится

Сходимость ряда ~--- сходимость его частичных сумм $\implies$
сходится последовательность $\sigma_n$.

Т.~е. $ \exists s = \lim S_n \in \R$

Рассмотрим формулу Валлиса:
\[
\dfrac{\pi}{2} = \lim\limits_{n \to \infty} \left( \dfrac{(2m)!!}{(2m - 1)!!}
\right)^2 \dfrac{1}{2m + 1} = \dfrac{((2m)!!)^4}{((2m)!)^2} \cdot
\dfrac{1}{2m + 1} = \lim\limits_{n \to \infty} \dfrac{2^{4m}(m!)^4}{((2m)!)^2} 
\cdot \dfrac{1}{2m + 1}
\]
\[
\alpha_m = \dfrac{2^{2m}(m!)^2}{(2m)!\sqrt{2m + 1}} \to \sqrt{\dfrac{\pi}{2}}
\]
Т.~к. $ S_m \to S $, то $ \lim\limits_{n \to \infty} \sigma_n = e^s, \ 
\dfrac{\sigma_n^2}{\sigma_{2n}} \to \dfrac{e^{2s}}{e^s} = e^s $.
\[
\dfrac{\sigma_n^2}{\sigma_{2n}} = 
\dfrac{(n!)^2e^{2n}(2n)^{2n}\sqrt{2n}}{n^{2n}n(2n)!e^{2n}} =
\dfrac{(n!)^22^{2n}\sqrt{2n}}{n(2n)!} =
\dfrac{2^{2n}(n!)^2}{(2n)!\sqrt{2n + 1}} \cdot 
\dfrac{\sqrt{2n+1}\sqrt{2}}{\sqrt{n}} = 
\dfrac{\alpha_n\sqrt{2n + 1}\sqrt{2}}{\sqrt{n}} \to
2\sqrt{\dfrac{\pi}{2}} = \sqrt{2\pi}
\]
\[
\lim \sigma_n = \lim \dfrac{n!e^n}{n^n \sqrt{n}} = \sqrt{2\pi}
\]
\[
n! \thicksim \dfrac{n^n}{e^n}\sqrt{2\pi n}
\]
\begin{example}
$ $

$\dfrac{(n!)^2}{(2n)!}4^n$

$a_n = \dfrac{(n!)^2}{(2n)!}4^n = \dfrac{n^{2n}}{e^{2n}} 
\cdot 2\pi n \cdot \dfrac{e^{2n}}{(2n)^{2n}} \cdot 2 \sqrt{\pi n}
\cdot 4^n = 4 \cdot (\pi n)^{\frac{3}{2}}
$
\end{example}
\end{document}
