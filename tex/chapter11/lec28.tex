\makeatletter
\def\input@path{{../../}}
\makeatother
\documentclass[../../main.tex]{subfiles}

\graphicspath{
	{../../img/}
	{../img/}
	{img/}
}

\begin{document}
\subsection{Признак Гаусса}

\begin{thm}[признак Гаусса]
Пусть \[\dfrac{a_n}{a_{n+1}} = \lambda + \dfrac{\mu}{n} +
\dfrac{\theta_n}{n^{1 + \eps}},\]
где $\theta_n$ ограничена и $\eps > 0$.

Тогда:
\begin{itemize}
	 \item если $\lambda > 1$, то ряд сходится;
	 \item если $\lambda < 1$, то ряд расходится;
	 \item если $\lambda = 1$, то
	 $\begin{cases}
	 	\mu > 1 \implies \text{ряд сходится;}\\
	 	\mu \leq 1 \implies \text{ряд расходится.} 
	 \end{cases}$
\end{itemize}
\end{thm}

\begin{example} Исследовать на сходимость

\[\sum\limits_{k = 1}^{\infty}\left(\dfrac{(2k - 1)!!}{(2k)!!} \right)^p
\dfrac{1}{k^q}.\]

\begin{gather*}
\dfrac{a_{n}}{a_{n + 1}} = \left(\dfrac{(2n - 1)!!}{(2n)!!}\right)^p \cdot
\dfrac{1}{n^q} \cdot\left( \dfrac{(2n + 2)!!}{(2n + 1)!!}\right)^p \cdot
(n + 1)^q = \left(\dfrac{2n + 2}{2n + 1}\right)^p \cdot
\left(\dfrac{n+1}{n}\right)^q = 
\\
= \left(1 + \dfrac{1}{1 + 2n}\right)^p \cdot
\left(1 + \dfrac{1}{n}\right)^q
= \left(1 + \dfrac{q}{n} + O\left(\dfrac{1}{n^2}\right)\right) \cdot
\left(1 + \dfrac{p}{2n + 1} + O\left(\dfrac{1}{n^2}\right)\right) =
\\
=
\left(1 + \dfrac{q}{n} + O\left(\dfrac{1}{n^2}\right)\right) \cdot
\left(1 + \dfrac{p}{2n} \cdot \dfrac{1}{1 + \frac{1}{2n}}
+ O\left(\dfrac{1}{n^2}\right)\right) =
\\
= \left(1 + \dfrac{q}{n} + O\left(\dfrac{1}{n^2}\right)\right) \cdot \left(1 +
\dfrac{p}{2n} \cdot \left(1 - \dfrac{1}{2n} + O\left(\dfrac{1}{n^2}
\right)\right) + O\left(\dfrac{1}{n^2}\right) \right) =
\\
= \left(1 + \dfrac{q}{n} + O\left(\dfrac{1}{n^2}\right)\right) \cdot \left(1 +
\dfrac{p}{2n} + O\left(\dfrac{1}{n^2}\right)\right) = 1 + \left(\dfrac{p}{2} +
q \right) \cdot \dfrac{1}{n} + O\left(\dfrac{1}{n^2}\right).
\end{gather*}

Т.~е. ряд сходится при $\dfrac{p}{2} + q > 1$ и
расходится при $\dfrac{p}{2} + q \leq 1$.
\end{example}

\subsection{Формула Валлиса}

\begin{thm}[формула Валлиса]
\[\boxed{
\lim\limits_{m \to \infty} \left( \dfrac{(2m)!!} {(2m-1)!!}  \right)^2
\cdot \dfrac{1}{2m+1} = \dfrac{\pi}{2}.
}\]
\end{thm}
\begin{proof}
Рассмотрим интеграл
\[
\mathrm{I}_n = \int\limits_0^{\frac{\pi}{2}} \cos^n x \;dx.\]
При $n = 0$ и $n = 1$ имеем
\[\mathrm{I}_0 = \dfrac{\pi}{2}, \quad \mathrm{I}_{1} = 1.
\]
Вычислим $\mathrm{I}_n$:
\begin{gather*}
\mathrm{I}_n =  \int\limits_0^{\frac{\pi}{2}} \cos^{n - 1} x \cdot \cos x\;dx =
\left[\begin{array}{l}
u = \cos^{n-1}x\\
du = (n - 1) \cos^{n-2}x (-\sin x) \;dx\\
v = \sin x \\
dv = \cos x\;dx\\
\end{array}\right]
= \sin x \cos^{n-1} x \Big|_0^{\frac{\pi}{2}} + 
\\
+ (n - 1) \cdot \int\limits_{0}^{\frac{\pi}{2}} \cos^{n-2}x \sin^2 x\; dx
= (n - 1) \int\limits_{0}^{\frac{\pi}{2}} \cos^{n-2}x \cdot (1 - \cos^{2}x)dx 
= \\
= (n-1)\mathrm{I}_{n-2} - (n-1)\mathrm{I}_{n} \implies 
\mathrm{I}_n = \dfrac{(n-1)}{n}\cdot\mathrm{I}_{n-2}.
\end{gather*}

Т.~е. \[\mathrm{I}_n = \dfrac{(n-1)}{n}\cdot\mathrm{I}_{n-2}, \quad n > 1.\]

При $n = 2m$ получаем:
\[
\mathrm{I}_{2m} = \dfrac{2m - 1}{2m} \cdot \mathrm{I}_{2m-2} = \dfrac{2m-1}{2m}
\cdot \dfrac{2m - 3}{2m-2} \cdot \mathrm{I}_{2m- 4} = \ldots =
\dfrac{(2m - 1)!!}{(2m)!!} \cdot \dfrac{\pi}{2}.
\]
\[\mathrm{I}_{2m} = \dfrac{(2m - 1)!!}{(2m)!!} \cdot \dfrac{\pi}{2}.\]

При $n = 2m+1$:
\[
\mathrm{I}_{2m + 1} = \dfrac{2m}{2m + 1} \cdot \mathrm{I}_{2m-1} = \ldots =
\dfrac{(2m)!!}{(2m + 1)!!} \cdot \mathrm{I}_{1} = \dfrac{(2m)!!}{(2m + 1)!!}.
\]

\[\mathrm{I}_{2m + 1} = \dfrac{(2m)!!}{(2m + 1)!!}.\]

Учитывая, что $
\mathrm{I}_{2m+2}\le \mathrm{I}_{2m+1}\le \mathrm{I}_{2m}
$, имеем
\[
\dfrac{(2m+1)!!}{(2m+2)!!} \cdot \dfrac{\pi}{2} \le \dfrac{(2m)!!}{(2m+1)!!} 
\le \dfrac{\pi}{2} \cdot \dfrac{(2m-1)!!}{(2m)!!}.
\]
Умножив последнее неравенство на $\dfrac{(2m)!!}{(2m-1)!!}$, получим:
\[
\dfrac{2m+1}{2m+2} \cdot \dfrac{\pi}{2} \le \left( \dfrac{(2m)!!}
{(2m-1)!!}  \right)^2 \cdot \dfrac{1}{2m+1} \le \dfrac{\pi}{2}.
\]
Используя теорему о сжатой последовательности, получаем, что
\[
\lim\limits_{m \to \infty} \left( \dfrac{(2m)!!} {(2m-1)!!}  \right)^2
\cdot \dfrac{1}{2m+1} = \dfrac{\pi}{2}. \qedhere
\]
\end{proof}

\subsection{Формула Стирлинга}
\begin{thm}[формула Стирлинга]
\[
n! \sim \dfrac{n^n}{e^n}\cdot\sqrt{2\pi n}.
\]
\end{thm}

\begin{proof}
Рассмотрим последовательность
\[\sigma_n = \dfrac{n! \cdot e^n}{n^n \cdot \sqrt{n}}.\]
Обозначим $ S_n = \ln{\sigma_n}$, $U_k = S_k - S_{k-1}$ и рассмотрим ряд
\begin{equation}
\label{stirling} S_1 + \sum\limits_{k=2}^\infty U_k.
\end{equation}
Заметим, что частные суммы ряда:
\[ 
S_1 + U_2 + U_3 + \ldots + U_n = S_1 + \left(S_2 - S_1\right) + 
\left(S_3 - S_2\right) + \ldots + \left(S_n - S_{n - 1}\right) = S_n.
\]
Исследуем ряд \eqref{stirling} на сходимость:
\begin{gather*}
U_n = S_n - S_{n - 1} = \ln\sigma_n - \ln\sigma_{n - 1} = 
\ln\dfrac{\sigma_n}{\sigma_{n - 1}} = \ln\left(\dfrac{n! \cdot 
e^n}{n^n\sqrt{n}}
\cdot \dfrac{(n - 1)^{n - 1}\sqrt{n - 1}}{(n - 1)! \cdot e^{n -1}}\right) =
\\
= \ln\left(\dfrac{(n - 1)^{n - 1}}{n^{n - 1}} \cdot \dfrac{\sqrt{n - 
1}}{\sqrt{n}}
\cdot e\right) = \ln \left(e\cdot \left(1 - \dfrac{1}{n} \right)^{n - 
\frac{1}{2}}\right) = 
1 + \left(n - \dfrac{1}{2} \right) \cdot\ln \left( 1 - \dfrac{1}{n} \right) =
\\
= 1 + \left( n - \dfrac{1}{2} \right)\cdot\left(-\dfrac{1}{n} - \dfrac{1}{2n^2}
- \dfrac{1}{3n^3} + O\left(\dfrac{1}{n^4} \right) \right) = 
1 - 1 - \dfrac{1}{2n} - \dfrac{1}{3n^2} + \dfrac{1}{2n} + \dfrac{1}{4n^2}
+ O\left( \dfrac{1}{n^3} \right) = 
\\
= -\dfrac{1}{12n^2} + O\left( \dfrac{1}{n^3}\right).
\end{gather*}
Т.~е. $U_n \sim -\frac{1}{12n^2}$, откуда получаем, что ряд \eqref{stirling} 
сходится.
Из сходимости ряда \eqref{stirling} следует сходимость
последовательности его частичных сумм. Т.~е. $ \exists 
S = \lim\limits_{n\to\infty} S_n \in \R$.

Из формулы Валлиса:
\[
\dfrac{\pi}{2} = \lim\limits_{m \to \infty} \left( \dfrac{(2m)!!}{(2m - 1)!!}
\right)^2\cdot \dfrac{1}{2m + 1} =
\lim\limits_{m \to \infty} \dfrac{((2m)!!)^4}{((2m)!)^2} \cdot
\dfrac{1}{2m + 1} = \lim\limits_{n \to \infty} 
\underbrace{\dfrac{2^{4m}\cdot(m!)^4}{((2m)!)^2} 
\cdot \dfrac{1}{2m + 1}}_{\alpha_m^2}.
\]
\[
\alpha_m = \dfrac{2^{2m}(m!)^2}{(2m)!\sqrt{2m + 1}} \appr{m\to\infty} 
\sqrt{\dfrac{\pi}{2}}.
\]
Т.~к. $ S_m \appr{m\to\infty} S $, то $ \lim\limits_{n \to \infty} \sigma_n = 
e^s$, откуда 
$\dfrac{\sigma_n^2}{\sigma_{2n}} \appr{n\to\infty}  \dfrac{e^{2s}}{e^s} = e^s 
$.

Рассмотрим выражение $\dfrac{\sigma_n^2}{\sigma_{2n}}$:
\begin{gather*}
\dfrac{\sigma_n^2}{\sigma_{2n}} = 
\dfrac{(n!)^2\cdot e^{2n}\cdot (2n)^{2n}\sqrt{2n}}{n^{2n}\cdot n\cdot 
(2n)!\cdot e^{2n}} =
\dfrac{(n!)^2\cdot2^{2n}\sqrt{2n}}{n\cdot(2n)!} =
\dfrac{2^{2n}(n!)^2}{(2n)!\sqrt{2n + 1}} \cdot 
\dfrac{\sqrt{2n+1}\cdot\sqrt{2}}{\sqrt{n}} = 
\\ =
\dfrac{\alpha_n\cdot\sqrt{2n + 1}\cdot\sqrt{2}}{\sqrt{n}} \appr{n\to\infty}
2\sqrt{\dfrac{\pi}{2}} = \sqrt{2\pi} \implies
\left[\lim\limits_{n \to \infty} \sigma_n = \lim\limits_{n\to\infty} 
\dfrac{\sigma_n^2}{\sigma_{2n}} = e^s\right] \implies \\ \implies
\lim\limits_{n\to\infty} \sigma_n = \lim\limits_{n\to\infty} \dfrac{n!\cdot 
e^n}{n^n \sqrt{n}} = \sqrt{2\pi}.
\end{gather*}
Отсюда получаем, что
\[
n! \sim \dfrac{n^n}{e^n}\sqrt{2\pi n}. \qedhere
\]
\end{proof}
\begin{example}
Исследовать на сходимость
\[\sum\limits_{n=1}^\infty\dfrac{(n!)^2}{(2n)!}\cdot4^n.\]

$a_n = \dfrac{(n!)^2}{(2n)!}\cdot4^n \sim \dfrac{n^{2n}}{e^{2n}} 
\cdot 2\pi n \cdot \dfrac{e^{2n}}{(2n)^{2n}} \cdot \dfrac1{2 \sqrt{\pi n}}
\cdot 4^n = \sqrt{\pi n} \underset{n\to\infty}{\centernot\longrightarrow} 0
$, т.~е. ряд расходится.
\end{example}
\end{document}
