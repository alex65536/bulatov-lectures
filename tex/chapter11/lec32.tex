\makeatletter
\def\input@path{{../../}}
\makeatother
\documentclass[../../main.tex]{subfiles}

\graphicspath{
	{../../img/}
	{../img/}
	{img/}
}

\begin{document}
\section{Обобщенная сумма ряда}

\subsection{Суммирование рядов методом средних арифметических}

Рассмотрим ряд
\begin{equation}
 \sum\limits_{k=1}^\infty a_k.
 \label{32:1}
\end{equation}
Как мы рассматривали ранее, его суммой называется предел частных сумм: \[S = 
\lim\limits_{n\to\infty} \sum\limits_{k=1}^n a_k.\] Существуют другие способы 
суммирования рядов, в частности, метод средних арифметических. 
Обозначим 
\begin{gather*}
 \sigma_1 = S_1, \\
 \sigma_2 = \frac{S_1 + S_2}2, \\
 \sigma_3 = \frac{S_1 + S_2 + S_3}3, \\
 \dots\\
 \sigma_n = \frac1n\cdot\sum\limits_{k=1}^n S_k,
\end{gather*}
где $S_k = \sum\limits_{j=1}^k a_j$.

Предел $\lim\limits_{n\to\infty} \sigma_n = \sigma$ называется \emph{суммой} 
ряда \eqref{32:1} (обобщенной суммой ряда \eqref{32:1}). Такой метод 
суммирования называется \emph{методом средних арифметических} или 
\emph{методом Чезаро}.

\begin{exmp}
 Рассмотрим ряд
 \[\sum\limits_{k=1}^\infty (-1)^{k+1} = 1-1+1-1+\dots\]
 Ряд расходится по методу суммирования Коши. Рассмотрим его частичные суммы
 \begin{gather*}
  \begin{array}{lll}
   S_1 = 1,&& S_2 = 0, \\
   S_3 = 1,&& S_4 = 0, \\
   S_5 = 1,&& S_6 = 0, \\
  \end{array} \\
  \dots
 \end{gather*}
Тогда
\[
 \begin{array}{lll}
 \sigma_1 = 1,&&
 \sigma_2 = \frac{1+0}2 = \frac12, \\
 \sigma_3 = \frac{1+0+1}3 = \frac23,&&
 \sigma_4 = \frac12, \\
 \sigma_5 = \frac35,&&
 \sigma_6 = \frac12, \\
 \dots&& \dots \\
 \sigma_{2n-1} = \frac{n}{2n-1},&& \sigma_{2n}=\frac12.
 \end{array}
\]

Значит, по методу Чезаро ряд сходится, и его сумма равна $\frac12$.
\end{exmp}

Таким образом, метод Чезаро позволяет суммировать некоторые расходящиеся в 
смысле Коши ряды. Существует большое количество обощенных методов суммирования 
рядов. При построении таких методов выдвигаются два требования:
\begin{enumerate}
 \item Метод должен быть \emph{линейным}, т.~е. если ряд 
 $\sum\limits_{k=1}^\infty a_k$ имеет обобщенную сумму $\sigma_a$, а ряд 
 $\sum\limits_{k=1}^\infty b_k$ имеет обобщенную сумму $\sigma_b$, то линейная 
 комбинация этих рядов \[\sum\limits_{k=1}^\infty (\alpha a_k + \beta b_k)\] 
 имеет сумму $\alpha\sigma_a + \beta\sigma_b$.
 \item Метод должен быть \emph{регулярным}, т.~е. если ряд \eqref{32:1} имеет 
 по Коши сумму $S$, то и сумма по обобщенному методу должна быть равна $S$.
\end{enumerate}

Метод Чезаро, очевидно, является линейным.

\begin{exc}
 Убедиться в том, что метод Чезаро линеен.
\end{exc}

Прежде всего получим необходимое условие сходимости ряда по методу Чезаро. 
Пусть ряд \eqref{32:1} сходится и имеет сумму $\sigma$ по методу Чезаро. 
Рассмотрим выражение
\[\frac{n}{n-1}\sigma_n - \sigma_{n-1} \appr{n\to\infty} \sigma - \sigma = 
0.\] Т.~е. \[\frac{n\sigma_n - (n-1)\sigma_{n-1}}{n-1} \appr{n\to\infty} 0.\] 
Но $n\sigma_n = \sum\limits_{k=1}^n S_k$ и $(n-1)\sigma_{n-1} = 
\sum\limits_{k=1}^{n-1} S_k$. Тогда получаем, что
\[\frac{\sum\limits_{k=1}^n S_k - \sum\limits_{k=1}^{n-1} S_k}{n-1} = 
\frac{S_n}{n-1} \appr{n\to\infty} 0.\]
Полученное выражение уже можно считать необходимым условием, но мы пойдем 
дальше. Рассмотрим другое выражение: \[\frac{a_n}n = \frac{\sum\limits_{k=1}^n 
a_k - \sum\limits_{k=1}^{n-1} a_k}n = \frac{S_n - S_{n-1}}n  = 
\underbrace{\frac{n-1}n\cdot\frac{S_n}{n-1}}_{\to0} - 
\underbrace{\frac{S_{n-1}}{n-2}\cdot\frac{n-2}n}_{\to0} \appr{n\to\infty} 0-0 
= 0.\]

Таким образом, получаем необходимое условие сходимости ряда по Чезаро: 
\[\frac{a_n}n \appr{n\to\infty} 0.\]

\begin{thm}
Если \eqref{32:1} сходится по методу Коши и имеет сумму $S$, то он сходится и 
по методу Чезаро и имеет обобщенную сумму $\sigma = S$. Т.~е. метод Чезаро 
регулярен.
\end{thm}
\begin{proof}
 Пусть $\lim\limits_{n\to\infty} S_n = S \in\R$. Раз существует конечный 
 предел частных сумм, то \[\exists M : |S_n| \le M\quad\forall n\in\N,\] а 
 также
 \[\forall\eps>0 \ \ \exists N \in \N \quad \forall n \ge N: |S - S_n| \le 
 \eps.\]
Покажем, что $\sigma_n \appr{n\to\infty} S$ или, что то же самое, 
$\sigma_{N+p} \appr{p\to\infty} S$. Для этого рассмотрим разность
\begin{gather*}
|S - \sigma_{N+p}| = \abs{S - \frac1{N+p}\cdot\sum\limits_{k=1}^{N+p}S_k}
= \\ =
\abs{\left(S - \frac1p\cdot\sum\limits_{k=N+1}^{N+p}S_k\right) + 
\left(\frac1p\cdot\sum\limits_{k=N+1}^{N+p}S_k - 
\frac1{N+p}\cdot\sum_{k=N+1}^{N+p}S_k\right) - 
\frac1{N+p}\cdot\sum\limits_{k=1}^{N}S_k}
\le \\ \le
\abs{S - \frac1p\cdot\sum\limits_{k=N+1}^{N+p}S_k} + 
\abs{\frac1p\cdot\sum\limits_{k=N+1}^{N+p}S_k - 
\frac1{N+p}\cdot\sum\limits_{k=N+1}^{N+p}S_k} + 
\abs{\frac1{N+p}\cdot\sum\limits_{k=1}^N S_k} = A_1 + A_2 + A_3,
\end{gather*}
где
\begin{gather*}
A_1 = \abs{S - \frac1p\cdot\sum\limits_{k=N+1}^{N+p}S_k} = \frac1p\abs{S\cdot 
p - \sum\limits_{k=N+1}^{N+p}S_k} = \frac1p \abs{\sum\limits_{k=N+1}^{N+p}(S - 
S_k)} \le\\\le \frac1p\cdot\sum\limits_{k=N+1}^{N+p}|S - S_k| \le 
\frac1p\cdot(p\eps) = \eps, \\
A_2 = \abs{\frac1p\cdot\sum\limits_{k=N+1}^{N+p}S_k - 
\frac1{N+p}\cdot\sum\limits_{k=N+1}^{N+p}S_k} = \abs{\sum\limits_{k=N+1}^{N+p} 
\left(\frac1p - \frac1{N+p}\right)S_k} \le \frac N{p(N+p)}\cdot 
\sum\limits_{k=N+1}^{N+p} |S_k| \le\\\le \frac N{p(N+p)}\cdot pM = 
\frac{N\cdot M}{N + p} \le \eps \text{ (для достаточно больших $p$)}, \\
A_3 = \abs{\frac1{N+p}\cdot\sum\limits_{k=1}^{N}S_k} \le 
\frac1{N+p}\cdot\sum\limits_{k=1}^{N}|S_k| \le \frac{N\cdot M}{N + p} \le \eps 
\text{ (для достаточно больших $p$)}.
\end{gather*}

Поэтому для достаточно больших $p$ получаем, что $|S - \sigma_{N+p}| \le 
3\eps$. Таким образом, $\sigma_{N+p} \appr{p\to\infty} S$. Теорема 
доказана.
\end{proof}

\subsection{Суммирование рядов методом Абеля-Пуассона}

Рассмотрим ряд
\begin{equation}
 \label{32:2}
 \sum\limits_{k=0}^\infty a_k.
\end{equation}
Вместе с ним рассмотрим так называемый \emph{степенной ряд}
\[\sum\limits_{k=0}^\infty a_kx^k = a_0 + a_1x + a_2x^2 + \dots\]
\underline{Если} этот ряд сходится для всех $x \in [0;1[$ и имеет сумму 
$S(x)$, то обобщенной суммой ряда \eqref{32:2} по методу Абеля-Пуассона 
считается число \[P = \lim_{x\to1-0}S(x).\]

\begin{thm}
 Если ряд \eqref{32:2} сходится по методу Чезаро и имеет сумму $\sigma$, то он 
 сходится по методу Абеля-Пуассона и имеет сумму, равную $P = \sigma$.
\end{thm}

\begin{crl*}
 Если ряд сходится по Коши и имеет сумму $S$, то его сумма по методу 
 Абеля-Пуассона также равна $S$.
\end{crl*}

\begin{exmp}
Рассмотрим ряд \[1-2+3-4+5-\dots\] Для него построим соответствующий степенной 
ряд \[1-2x+3x^2-4x^3 + 5x^4 - \dots = \sum_{k=0}^\infty (-1)^k\cdot(k+1)x^k.\]
Исследуем этот ряд по \hyperref[lec27:cauchy]{признаку Коши}:
\[\sqrt[n]{\abs{(-1)^n\cdot(n+1)x^n}} \appr{n\to\infty} |x|.\]
По Коши ряд сходится по крайней мере для $|x| < 1$, т.~е. он сходится и для $x 
\in [0;1[$.

Найдем его сумму. Пусть она равна $S(x)$. Но тогда
\begin{gather*}
 S(x) = \sum\limits_{k=0}^\infty (-1)^k\cdot(k+1)x^k \implies \int S(x)dx + C 
 = \sum\limits_{k=0}^\infty \left(\int (-1)^k\cdot(k+1)x^k\;dx\right) = \\ =
 \sum\limits_{k=0}^\infty (-1)^k\cdot x^{k+1} =
 x - x^2 + x^3 - x^4 + \dots = \frac x{1+x} \implies S(x) = \left(\frac 
 x{1+x}\right)' = \frac 1{(1+x)^2}.
\end{gather*}

Т.~е. $P = \lim\limits_{x\to1-0} S(x) = \dfrac14$. Получаем, что ряд сходится 
по Абелю-Пуассону и имеет сумму $\dfrac14$.

Заметим, что этот ряд не сходится по Чезаро, т.~к. для него не выполняется 
необходимое условие сходимости: \[\frac{a_n}n = \frac{(-1)^n(n+1)}n 
\underset{n\to\infty}{\centernot\longrightarrow} 0.\]
\end{exmp}

\end{document}
