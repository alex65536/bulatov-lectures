\makeatletter
\def\input@path{{../../}}
\makeatother
\documentclass[../../main.tex]{subfiles}

\graphicspath{
	{../../img/}
	{../img/}
	{img/}
}

\begin{document}
	\subsection{Признаки Коши и Даламбера. Интегральный признак.}
	
	\begin{thm}[признак Коши]\label{lec27:cauchy} 	
		Пусть $\exists \lim\limits_{n \to \infty } \sqrt[n]{a_n}  = c$. Если
	\begin{enumerate}[label={\alph*)}]
			\item $c < 1$, то ряд сходится;
			\item $c > 1$, то ряд расходится.
	\end{enumerate}
	\end{thm}
	\begin{proof}
		
		~
		
	\begin{enumerate}[label={\alph*)}]
		 \item Пусть $c < 1$. Тогда $\exists q: c < q < 1$, и для достаточно больших 
		 $n$ выполнено
			\[ \sqrt[n]{a_n} \leq q \iff a_n \leq q^n. \]
			Ряд $\sum\limits_{n=1}^{\infty} q^n$ сходится, т.~к. $q < 1$. Тогда по 
			теореме \ref{lec26:comp_test_1}
			рассматриваемый ряд \eqref{lec_26, num_1} сходится.
			 
		\item Пусть $c>1$ . Тогда $\exists q : c > q > 1 $ и \[\exists N : \forall n 
		\geq N \quad
			 \sqrt[n]{a_n} \geq q \iff a_n \geq q^n.\] Значит, $\lim\limits_{n \to 
			 \infty} a_n 
			 \neq 0$, и не выполнено необходимое условие сходимости ряда. \qedhere
	\end{enumerate}
	\end{proof}

	\begin{examples}

	~

	\begin{enumerate}
	 \item
		\[ \sum_{k=1}^{\infty}  \frac{3^k + k}{5^k}.  \]
		\[ \sqrt[n]{a_n} = \sqrt[n]{\frac{3^n + n}{5^n}} = \frac{3}{5} \sqrt[n]{1 + 
		\frac{n}{3^n}}  \underset{n \rightarrow \infty}
		{\longrightarrow} \frac{3}{5} < 1.    \]
		Из \hyperref[lec27:cauchy]{признака Коши} получаем, что ряд  
		$\sum\limits_{k=1}^{\infty}  \frac{3^k + k}{5^k}$
		сходится.
	\item
		\[ \sum_{k=1}^{\infty} \left( \frac{k}{k+1}\right) ^{k^2}. \]
		\[ \sqrt[n]{a_n} =  \left( \frac{n}{n+1}\right) ^n  \appr{n \rightarrow 
		\infty} \frac{1}{e} < 1. \]
		Из \hyperref[lec27:cauchy]{признака Коши}, получаем, что ряд 
		$\sum\limits_{k=1}^{\infty} \left( \frac{k}{k+1}\right) ^{k^2}$ сходится.
	\end{enumerate}
	\end{examples} 	

	\begin{thm}[признак Даламбера]\label{lec27:dalamber}
		
		Пусть $\exists \lim\limits_{n \to \infty} \dfrac{a_{n+1}}{a_n} = d$. Тогда:
		\begin{enumerate}[label={\alph*)}]
		\item Из $d < 1$ следует, что ряд \eqref{lec_26, num_1} сходится;
		\item Из $d > 1$ следует, что ряд \eqref{lec_26, num_1} расходится.
		\end{enumerate}
	\end{thm}
		\begin{proof}
			
			~
			
			\begin{enumerate}[label={\alph*)}]
			\item Пусть $d < 1$, тогда $\exists q : d < q < 1$  и тогда \[\exists N \in 
			\N \quad
			\forall n \geq N:
			\frac{a_{n+1}}{a_n} \leq q = \frac{q^{n+1}}{q^n}.  \]
			Ряд $\sum\limits_{k=1}^{\infty} q^k$ сходится, т.~к. 
			$q<1$. Но тогда по теореме \ref{lec26:comp_test_3} сходится и 
			\eqref{lec_26, num_1}.
			\item Пусть $d > 1$, тогда $\exists q : 1 < q < d $ и \[\exists N \in \N 
			\quad \forall n 
			\geq N:
			\frac{a_{n+1}}{a_n} \geq \frac{q^{n+1}}{q^n} = q > 1.  \]
			Т.~к. все $a_k > 0$, то $\lim\limits_{k \to \infty} a_k \neq 0$, и не 
			выполнено необходимое условие сходимости.
			\qedhere
			\end{enumerate}
		\end{proof}
	\begin{examples}

	~

	\begin{enumerate}
	 \item
		\[ \sum_{k = 1}^{\infty} \frac{(k!)^2}{(2k!)^2}    \]
		\[ \frac{a_{n+1}}{a_n} = \frac{((n+1)!)^2 \cdot (2n)!}{(n!)^2 \cdot (2n+2)!} 
		= 
		\frac{n+1}{2(2n+1)}  \appr{n \rightarrow \infty} \frac{1}{4}. \]
		Т.~е. ряд $\sum\limits_{k = 1}^{\infty} \frac{(k!)^2}{(2k!)^2}$ сходится по 
		\hyperref[lec27:dalamber]{признаку Даламбера}.
	\item
		\[  \sum_{k = 1}^{\infty} \frac{k^k}{k! \cdot 2^k} \]
		\[ \frac{a_{n+1}}{a_n} = \frac{n! \cdot 2^n \cdot (n+1)^{n+1}}{(n+1)! \cdot 
		n^n \cdot 
		2^{n+1}} = \frac{(n+1)^{n}}{2\cdot n^n } \underset{n \rightarrow \infty} 
		{\longrightarrow}  \frac{e}{2} > 1. \]
		Т.~е. ряд $\sum\limits_{k = 1}^{\infty} \frac{k^k}{k! \cdot 2^k}$ расходится 
		по 
		\hyperref[lec27:dalamber]{признаку Даламбера}.
	\end{enumerate}
	\end{examples}

	\begin{remarks}
	
	~
	
	\begin{enumerate}
	\item В доказательствах мы использовали <<для достаточно больших $N$>>, 
	то есть условие должно выполнятся для некоторого остатка ряда, т.~к. из 
	сходимости/расходимости остатка следует сходимость/расходимость ряда.
	\item  В признаке Даламбера можно рассматривать
		\[  \lim_{n \to \infty} \frac{a_n}{a_{n+1}} = \delta = \frac{1}{d}. \]
		И тогда, соответственно:
		\begin{enumerate}[label={\alph*)}]
		\item $\delta > 1$~--- \eqref{lec_26, 
		num_1} сходится; \\
		\item $\delta < 1$~--- \eqref{lec_26, 
		num_1} расходится.
		\end{enumerate}
		
	\item При $d=1$ признаки Коши и Даламбера не дают точного результата 
	на сходимость ряда \eqref{lec_26, num_1}, поэтому требуются дополнительные 
	исследования.
	\end{enumerate}
	\end{remarks}
	
	Рассмотрим также интегральный признак Коши.
	
	Функцию $f(x)$, определенную на промежутке $[1, + \infty]$, называют 
		\emph{производящей} для ряда \eqref{lec_26, num_1}, если
		\[ f(k) = a_k, \ \forall k \in \N.\]
		Поскольку \eqref{lec_26, num_1} является положительным, то будем считать, что
		$f(x) > 0 \quad \forall x \in 
		[1, + \infty]$.
	
	\begin{thm}[интегральный признак] \label{lec27,integral_att}
		Если $f(x)$~--- монотонная производящая интегрируемая функция для ряда 
		\eqref{lec_26, num_1}, то \eqref{lec_26, num_1} сходится тогда и только 
		тогда, когда сходится последовательность \[b_n = \int\limits_1^n f(x)dx.\]
		\begin{proof}
			Рассмотрим
			\[ \int_{1}^{n} f(x) \, dx = b_n , \; n \in \N.  \]
			По условию $f(x)$ интегрируема и монотонна. Тогда для сходимости 
			необходимо, чтобы $f(x) \appr{x\to\infty}0$.
			Последовательность $b_n$ возрастает, т.~к.
			\[ b_{n+1} = \int\limits_{1}^{n+1} f(x) \, dx  = b_n + 
			\int\limits_{n}^{n+1} f(x) \, dx > 
			b_n. \]
			В силу того, что $f(x) > 0 \ \; \forall x \in \R$,
			имеем $a_{k+1} \le f(x) \le a_k, \; \forall x \in [k,k+1]$
			В силу монотонности ОИ получаем
			\[ a_{k+1} = \int\limits_{k}^{k+1}a_{k+1} \; dx  \le 
			\int\limits_{k}^{k+1}f(x) \, dx \le 
			\int\limits_{k}^{k+1}a_{k} \, dx  = a_{k} \implies a_{k+1} \le 
			\int\limits_{k}^{k+1}f(x) \, dx \le a_k. \]
			Тогда рассмотрим частную сумму $S_n$:
			\[ S_n = a_1 + a_2 + \dots + a_n \le a_1 +  \int\limits_{1}^{n}f(x) \, dx  
			= a_1 + 
			b_n. \]
			Если $b_n$ сходится, то т.~к. $b_n$ возрастающая, по критерию сходимости 
			монотонной последовательности $\exists M : b_n \le M \implies S_n \le a_1 + 
			M \ \; 
			\forall n \in \N$, т.~е. $S_n$ сходится по тому же критерию, т.~е. сходится 
			и ряд \eqref{lec_26, num_1}. С другой стороны,
			\[ S_n = a_1 + a_2 + \dots + a_n \ge \int\limits_{1}^{n}f(x) \, dx  + a_n = 
			a_n + b_n. \]
			Если $b_n$ расходится ($b_n$ возрастает и неограничена), то тогда и $S_n 
			\ge 
			b_n + a_n$ неограничена (т.~е. расходится), значит, расходится и 
			\eqref{lec_26, num_1}.
		\end{proof}
	\end{thm}

	\begin{iex} Рассмотрим \emph{обобщенный гармонический ряд}
		\begin{equation}  \sum_{k = 1}^{\infty} \frac{1}{k^p}. 
		\label{lec27,harmonic_row_generalized}  \end{equation}
		
		В качестве производящей функции возьмем $f(x) = \dfrac{1}{x^p}$.
		\[ b_n = \int\limits_{1}^{n} \frac{dx}{x^p} = \begin{cases} \frac{x^{-p + 
		1}}{-p + 
		1} , \; \;\; p \ne 1  \\  \ln{|x|}, \; \; \; p = 1 \end{cases} =  
		\begin{cases} \frac{n^{1-p}}{1-p } \underset{n \rightarrow \infty}
		{\longrightarrow} \frac{1}{1-p} ,& \; \;\; p > 1  \\  \ln{|x|} \underset{n 
		\rightarrow \infty}
		{\longrightarrow} \infty ,& \; \; \; p = 1 \\
		\frac{n^{1-p}}{1-p } \underset{n \rightarrow \infty}
		{\longrightarrow} \infty ,& \; \;\; p < 1  \\
		\end{cases} \]
		Значит, ряд \eqref{lec27,harmonic_row_generalized} 
		сходится при $p>1$ и расходится при $p \le 1$.
		\end{iex}
		
		Воспользуемся \hyperref[lec26:comp_test_2]{признаком 2\textdegree}, где 
		возьмем $b_n = \dfrac{1}{n^p}$. Из 
		\hyperref[lec26:comp_test_2]{признака 2\textdegree} следует, что если 
		$\dfrac{a_n}{b_n} \underset{n 
		\rightarrow \infty} {\longrightarrow} l $ и $0 < l < +\infty$, то ряды 
		либо одновременно сходятся, либо одновременно расходятся.
		
		\[\dfrac{a_n}{b_n} \underset{n \rightarrow \infty} {\longrightarrow} l \iff 
		a_n 
		\sim l b_n,\ 0 < l < +\infty.\] Если $b_n = \dfrac{1}{n^p}$, то $a_n \sim 
		\dfrac{l}{n^p}$.
		
		Получаем \emph{степенной признак сравнения}:
		если $a_n \sim \dfrac{l}{n^p}$, то при $p>1$ ряд сходится, при $p \le 1$ ряд 
		расходится.

	\begin{examples}
		~
	\begin{enumerate}[label=\arabic*)]	
		\item \[ \sum_{k = 1}^{\infty} \frac{k+1}{\sqrt[3]{2k^4 + 1}}  \]
		Воспользуемся степенным признаком сравнения:
		\[ a_n = \frac{n+1}{\sqrt[3]{2n^4 + 1}} \sim \frac{n}{\sqrt[3]{2} 
		\sqrt[3]{n^4}} = \frac{1}{\sqrt[3]{2} \sqrt[3]{n}}. \]
		
		$p = \frac{1}{3} < 1$, откуда следует, что ряд расходится. 
		\item \[ \sum_{k=1}^{\infty} \frac{1}{\sqrt[5]{k+1}}\cdot \sin{\frac{1}{k}}  
		\]
		\[  a_n =  \frac{1}{\sqrt[5]{k+1}}\cdot \sin{\frac{1}{k}} \sim 
		\frac{1}{n^{\frac{1}{5}} } \cdot \frac{1}{n} =  \frac{1}{n^{\frac{6}{5}} }.  
		\]
		Ряд сходится, т.~к. $p = \frac{6}{5} > 1$.
	\end{enumerate}	
	\end{examples}	

	\subsection{Признак Дюамеля (Дюамеля-Раабе)}
	
	Признак Даламбера в случае $\dfrac{a_{n+1}}{a_n} \appr{n\to\infty} 1$ не дает 
	ответа, сходится ли ряд. В таком случае будем изучать ряд дальше.
	
	\begin{thm}[признак Дюамеля-Раабе]\label{lec27,duamele_raabe}
		Пусть $\dfrac{a_n}{a_{n+1}} = 1 + \dfrac{\mu}{n} + o\left( 
		\dfrac{1}{n}\right) 
		$. Тогда при $\mu > 1$ ряд \eqref{lec_26, num_1} сходится, а при $\mu < 1$ 
		\--- расходится.
		\begin{proof}
			Воспользуемся \hyperref[lec26:comp_test_3]{признаком 3\textdegree}. В нем 
			\eqref{lec_26, num_1} сравним с 
			\begin{equation} \label{lec27, 3}
			\sum_{k = 1}^{\infty} b_k.
			\end{equation}
			Если \begin{equation}\label{lec27, 4}
			\frac{a_{n+1}}{a_n} \le \frac{b_{n+1}}{b_n},
			\end{equation}
			то из сходимости \eqref{lec27, 3} следует сходимость \eqref{lec_26, num_1}, 
			а 
			из расходимости  \eqref{lec_26, num_1}~--- расходимость \eqref{lec27, 3} . 
			\eqref{lec27, 4} перепишем в виде
			\begin{equation} \label{lec27,(5)}
			\frac{a_n}{a_{n+1}} \ge \frac{b_n}{b_{n+1}}.
			\end{equation}
			Пусть $\mu > 1$. Тогда $\exists p :  1 < p < \mu$. Рассмотрим ряд
			\[  \sum_{k=1}^{\infty} b_k = \sum_{k=1}^{\infty} \frac{1}{k^p},\]
			для которого
			\[ \frac{b_n}{b_{n+1}} = \frac{(n+1)^p}{n^p} = \left( 1 + \frac{1}{n} 
			\right)^p = 1 + \frac{p}{n} + \overline{o}\left( \frac{1}{n} \right).       
			\]
			Раз $1 < p < \mu$, то
			$\dfrac{b_n}{b_{n+1}} \le \dfrac{a_n}{a_{n+1}}$
			при достаточно больших $n$. Т.~к. $p > 1$, то ряд $\eqref{lec27, 3} = 
			\sum\limits_{k=1}^{\infty} \frac{1}{k^p}$ сходится. Но тогда по 
			\hyperref[lec26:comp_test_3]{признаку 3\textdegree} сходится и ряд 
			\eqref{lec_26, num_1}.
			
			Если $\mu < 1$, то $\exists p : 1 > p > \mu$, и для ряда \eqref{lec27, 3}:
			\[  \frac{a_n}{a_{n+1}} = 1 + \frac{\mu}{n} + \overline{o}\left( 
			\frac{1}{n} \right) \le  1 + \frac{p}{n} + \overline{o}\left( \frac{1}{n} 
			\right) = \frac{b_n}{b_{n+1}}                   \]
			Из неравенства выше получаем, что
			\[\frac{a_n}{a_{n+1}} \le \frac{b_n}{b_{n+1}} \iff \frac{a_{n+1}}{a_n} \ge 
			\frac{b_{n+1}}{b_n},\]
			откуда по \hyperref[lec26:comp_test_3]{признаку 3\textdegree} ряд 
			\eqref{lec_26, num_1} расходится, т.~к. 
			\eqref{lec27, 3} расходится.
		\end{proof}
	\end{thm}	
	
	
	\begin{example}
		\[ \sum_{k=1}^{\infty} \frac{(2k-1)!!}{(2k)!!}    \]
		\[ \frac{a_n}{a_{n+1}} = \frac{ (2n-1)!! \; (2n+2)!!}{ (2n+1)!! \;  (2n)!! } 
		= \frac{2n+2}{2n+1} = 1 + \frac{1}{2n+1} = 1 + \frac{1}{2n} + \frac{1}{2n+1} 
		- \frac{1}{2n} = 1 + \frac{1}{2n} + o\left( \frac{1}{n} \right). \]
		$\mu = \dfrac{1}{2} < 1$, следовательно, ряд $\sum\limits_{k=1}^{\infty} 
		\dfrac{(2k-1)!!}{(2k)!!}$ расходится.
	\end{example}	
\end{document}
