\makeatletter
\def\input@path{{../../}}
\makeatother
\documentclass[../../main.tex]{subfiles}

\graphicspath{
	{../../img/}
	{../img/}
	{img/}
}

\begin{document}
	
	Рассматриваем ряд $\sum_{n=1}^{\infty} a_n$ \ref{lec_26, num_1}
	
	\begin{thm}[ Признак Коши ]\label{lec27:cauchy} 
		
		Пусть $\exists \lim_{n \to \infty } \sqrt[n]{a_n}  = c$. Если
		
			а) $c < 1$, то ряд сходится 
			
			б) $c>1$, то ряд расходится
		
	\begin{proof}
		 а) Пусть $c < 1$. Тогда $\exists q, c < q < 1$ и для достаточно больших $n$
			\[ \sqrt[n]{a_n} \leq q <=> a_n \leq q^n \]
			Ряд $\sum_{n=1}^{\infty} q^n = \sum_{n=1}^{\infty} q_n$ сходится, тогда по признаку \ref{lec26:comp_test_1}
			 $c>1$, то рассматриваемый ряд \ref{lec_26, num_1} сходится \\
			 
			 б) $c>1$ . Тогда $\exists q : c > q > 1 $ и $\exists N : \forall n \geq N \sqrt[n]{a_n} \geq q <=> a_n \geq q^n$. Значит $\lim_{n \to \infty} a_n \neq 0$ и не выполнено основное условеи сходимости ряда.
		
		
	\end{proof}
	\end{thm}

	\begin{example}
		\[ \sum_{k=1}^{\infty}  \frac{3^k + k}{5^k}  \]
		\[ \sqrt[n]{a_n} = \sqrt[n]{\frac{3^n + n}{5^n}} = \frac{3}{5} \sqrt[n]{1 + \frac{n}{3^n}}  \underset{n \rightarrow \infty}
		{\longrightarrow} \frac{3}{5} < 1    \]
		Откуда из признака \ref{lec27:cauchy}, получаем, что сходится ряд $\sum_{k=1}^{\infty}  \frac{3^k + k}{5^k}$
		
	\end{example} 
 
	\begin{example}
		\[ \sum_{k}^{\infty} \left( \frac{k}{k+1}\right) ^{k^2}   \]
		\[ \sqrt[n]{a_n} =  \left( \frac{n}{n+1}\right) ^n  \underset{n \rightarrow \infty}
		{\longrightarrow} \frac{1}{e} < 1  \]
		Откуда из признака \ref{lec27:cauchy}, получаем, что сходится ряд $\sum_{k}^{\infty} \left( \frac{k}{k+1}\right) ^{k^2}$
		
		
	\end{example} 	



	\begin{thm}[Признак Даламбера ]\label{lec27:dalamber}
		
		Пусть $\exists \lim_{n \to \infty } \frac{a_{n+1}}{a_n} = d$. Тогда:
		
		а) Из $d < 1$ следует, что ряд \ref{lec_26, num_1} сходится
		
		б) Из $d > 1$ следует, что ряд \ref{lec_26, num_1} расходится
		
		\begin{proof}
			
			а) Пусть $d < 1$, тогда $\exists q : d < q < 1$  и тогда $\exists N : \forall n \geq N$:			
			\[ \frac{a_{n+1}}{a_n} \leq q = \frac{q^{n+1}}{q^n}   \]
			Ряд $\sum_{k=1}^{\infty} b_k = \sum_{k=1}^{\infty} q^k$ сходится, т.к. $q<1$. По свойству \ref{lec26:comp_test_3} сходится и \ref{lec_26, num_1} \\
			б) Пусть $d > 1$, тогда $\exists q : 1 < q < d $ и $\exists N : \forall n \geq N$:
			\[  \frac{a_{n+1}}{a_n} \geq \frac{q^{n+1}}{q^n} = q > 1    \]
			И так как все $a_k > 0$, то $\lim_{k \to \infty} a_k \neq 0 $ и не выполнено необходимое условие сходимости.
		\end{proof}
	\end{thm}			
	
	\begin{example}
		\[ \sum_{k = 1}^{\infty} \frac{(k!)^2}{(2k!)^2}    \]
		\[ \frac{a_{n+1}}{a_n} = \frac{((n+1)!)^2 \; (2n)!}{(n!)^2 \; (2n+2)!} = \frac{n+1}{2(2n+1)}  \underset{n \rightarrow \infty} {\longrightarrow} \frac{1}{4}    \]
		Т.е. ряд $\sum_{k = 1}^{\infty} \frac{(k!)^2}{(2k!)^2}$ сходится по признаку Даламбера.
		
	\end{example}

	\begin{example}
		\[  \sum_{k = 1}^{\infty} \frac{k^k}{k! \; 2^k} \]
		\[ \frac{a_{n+1}}{a_n} = \frac{n! \; 2^n \; (n+1)^{n+1}}{(n+1)! \; n^n \; 2^{n+1}} = \frac{(n+1)^{n}}{2 n^n } \underset{n \rightarrow \infty} {\longrightarrow}  \frac{e}{2} \]
		Т.е. ряд $\sum_{k = 1}^{\infty} \frac{k^k}{k! \; 2^k}$ расходится по признаку Даламбера.
	\end{example}	

	\begin{remark} В доказательствах мы использовали "для достаточно больших N", то есть условие должно выполнятся для некоторого остатка ряда, т.к. из сходимости/расходимости остатка следует сходимость/расходимость ряда.
		
	\end{remark}	

	\begin{remark} В признаке Даламбера можно рассматривать:
		
		\[  \lim_{n \to \infty} \frac{a_n}{a_{n+1}} = \delta = \frac{1}{d}      \]
		И тогда соответственно: $\begin{cases} 1)\delta > 1 \text{\ref{lec_26, num_1} сходится} \\ 
											   2)\delta < 1 \text{\ref{lec_26, num_1} расходится} \\
		 \end{cases}$
	\end{remark}

	\begin{remark} При $d=1$ признаки Коши и Даламбера не дают точного результата на сходимость ряда \ref{lec_26, num_1}, требуются дополнительные исследования.
		
	\end{remark}

	Рассмотрим еще один:
	
	\begin{thm}[Интегральный признак] \label{lec27,integral_att}
		Функцию $f(x)$ определенную на промежутке $[1, + \infty]$ называют производящей для ряда \ref{lec_26, num_1}, если:
		\[ f(k) = a_k, \; \forall k \in \mathbb{N}, \; f(x) > 0 \; \forall x \in \mathbb{R} \]
		Для сходимости  \ref{lec_26, num_1} необходимо, чтобы:
		\[  \lim_{k \to \infty} f(k) = 0    \]
		\begin{proof}
			Докажем, что:
			\[ \int_{1}^{n} f(x) \; dx = b_n , \; n \in \mathbb{N}  \]
			Последовательность $b_n$ возрастает, т.к. :
			\[ b_{n+1} = \int_{1}^{n+1} f(x) \; dx  = b_n + \int_{n}^{n+1} f(x) \; dx > b_n \]
			В силу того, что $f(x) > 0 \; \forall x \in \mathbb{R}$.\\			
			Имеем $a_{k+1} \le f(x) \le a_k, \; \forall x \in [k,k+1]$\\		
			В силу монотонности ОИ:
			\[ a_{k+1} = \int_{k}^{k+1}a_{k+1} \; dx  \le \int_{k}^{k+1}f(x) \; dx \le \int_{k}^{k+1}a_{k} \; dx  = a_{k}   \]
			Тогда, рассматривая остаток $S_n$:
			\[ S_n = a_1 + a_2 + \dots + a_n = a_1 +  \int_{1}^{n}f(x) \; dx  = a_1 + b_n \]
			Если $b_n$ сходится, то т.к. $b_n$ возрастающая, по критерию сходимости монотонной последовательности $\exists M : b_n \le M => S_n \le a_1 + M \;  \forall n => S_n$ сходится.
			\[ S_n = a_1 + a_2 + \dots + a_n \ge \int_{1}^{n}f(x) \; dx  + a_n   \]
			Если $b_n$ расходится ($b_n$  возрастает и неограничена), тогда и $S_n \ge b_n + a_n$ так же неограничена, а значит расходится, значит расходится и \ref{lec_26, num_1}
			\begin{exc}
				Доказать интегральный признак
			\end{exc}	
		\end{proof}
	\end{thm}

	\begin{iex}
		
		\begin{equation}  \sum_{k = 1}^{\infty} \frac{1}{k^p}  \label{lec27,harmonic_row_generalized}  \end{equation}
		Обобщенный гармонический ряд.
		
		Рассмотрим $f(x) = \frac{1}{x^p}$
		\[ b_n = \int_{1}^{n} \frac{dx}{x^p} = \begin{cases} \frac{x^{-p + 1}}{-p + 1} , \; \;\; p \ne 1  \\  \ln{|x|}, \; \; \; p = 1 \end{cases} =  \begin{cases} \frac{n^{1-p}}{1-p } \underset{n \rightarrow \infty}
		{\longrightarrow} \frac{1}{1-p} , \; \;\; p > 1  \\  \ln{|x|} \underset{n \rightarrow \infty}
		{\longrightarrow} \infty , \; \; \; p = 1 \end{cases} \]
		При $p < 1$, $\frac{n^{1-p} - 1}{1-p} \underset{n \rightarrow \infty}
		{\longrightarrow} \infty $, значит ряд \ref{lec27,harmonic_row_generalized} сходится при $p>1$, расходится при $p \le 1$.\\
		Воспользуемся \ref{lec26:comp_test_2}, где возьмем $b_n = \frac{1}{n^p}$. Из \ref{lec26:comp_test_2} следует, что  $\frac{a_n}{b_n} \underset{n \rightarrow \infty} {\longrightarrow} l $ и $0 < l < +\infty$, то ряды сходятся/расходятся одновременно.
		
		$\frac{a_n}{b_n} \underset{n \rightarrow \infty} {\longrightarrow} l <=> a_n \sim l b_n$, $0 < l < +\infty$, а если $b_n = \frac{1}{n^p}$, то $a_n \sim \frac{l}{n^p}$.\\		
		Получаем \emph{степенной признак сравнения}:\\		
		Если $a_n \sim \frac{l}{n^p}$, то при $p>1$ ряд сходится, при $p \le 1$ ряд расходится.
	\end{iex}	

	\begin{examples}
		~
	\begin{enumerate}[label=\arabic*)]	
		\item \[ \sum_{k = 1}^{\infty} \frac{k+1}{\sqrt[3]{2k^4 + 1}}  \]
		Воспользуемся степенным признаком сравнения:
		\[ a_n = \frac{n+1}{\sqrt[3]{2n^4 + 1}} \sim \frac{n}{\sqrt[3]{2} \sqrt[3]{n^4}} = \frac{1}{\sqrt[3]{2} \sqrt[3]{n}}          \]
		
		$p = \frac{1}{3} < 1$, откуда следует, что ряд расходится. 
		\item \[ \sum_{k=1}^{\infty} \frac{1}{\sqrt[5]{k+1}} \sin{\frac{1}{k}}  \]
		\[  a_n =  \frac{1}{\sqrt[5]{k+1}} \sin{\frac{1}{k}} \sim \frac{1}{n^{\frac{1}{5}} }  \frac{1}{n} =  \frac{1}{n^{\frac{6}{5}} }        \]
		Ряд сходится, т.к. $p = \frac{6}{5} > 1$
	\end{enumerate}	
	\end{examples}	

	\section{Признак Дюамеля (Дюамеля-Раабе)}
	
	\begin{thm}[Признак Дюамеля-Раабе]\label{lec27,duamele_raabe}
		Пусть $\frac{a_n}{a_{n+1}} = 1 + \frac{\mu}{n} + o\left( \frac{1}{n}\right) $. Тогда при $\mu > 1$, ряд \ref{lec_26, num_1} \--- сходится, $\mu < 1$ \--- расходится.
		\begin{proof}
			Воспользуемся \ref{lec26:comp_test_3}. В нем \ref{lec_26, num_1} сравним с \ref{lec27, (3) }
			Пусть \begin{equation} \label{lec27, (3) }
			\sum_{k = 1}^{\infty} b_k
			\end{equation}
			Если \begin{equation}\label{lec27, (4) }
			\frac{a_{n+1}}{a_n} \le \frac{b_{n+1}}{b_n}
			\end{equation}
			То из сходимости \ref{lec27, (3)} следует сходимость \ref{lec_26, num_1}, а из расходимости  \ref{lec_26, num_1}, расходимость \ref{lec27, (3)} . \ref{lec27, (4)} перепишем в виде:
			\begin{equation} \label{lec27,(5)}
			\frac{a_n}{a_{n+1}} \ge \frac{b_n}{b_{n+1}}
			\end{equation}
			Пусть $\mu > 1$. Тогда $\exists p :  1 < p < \mu$ и рассмотрим ряд:
			\[  \sum_{k=1}^{\infty} b_k = \sum_{k=1}^{\infty} \frac{1}{k^p}    \]
			Для него:
			\[ \frac{b_n}{b_{n+1}} = \frac{(n+1)^p}{n^p} = \left( 1 + \frac{1}{n} \right)^p = 1 + \frac{p}{n} + \overline{o}\left( \frac{1}{n} \right)       \]
			А тогда, если $\mu > 1$, $p < \mu$ и:
			\[ \frac{b_n}{b_{n+1}} \le \frac{a_n}{a_{n+1}}      \]
			При достаточно больших $n$ и ряд \ref{lec27, (3) } при $p > 1$ сходится. По \ref{lec26:comp_test_3} сходится и ряд \ref{lec_26, num_1}.
			
			Если $\mu < 1$, то $\exists p : 1 > p > \mu$, для ряда \ref{lec27, (3) }:
			\[  \frac{a_n}{a_{n+1}} = 1 + \frac{\mu}{n} + \overline{o}\left( \frac{1}{n} \right) \le  1 + \frac{p}{n} + \overline{o}\left( \frac{1}{n} \right) = \frac{b_n}{b_{n+1}}                   \]
			Откуда по \ref{lec26:comp_test_3} ряд \ref{lec_26, num_1} расходится, т.к. \ref{lec27, (3) } расходится.
		\end{proof}
	\end{thm}	
	
	
	\begin{example}
		\[ \sum_{k=1}^{\infty} \frac{(2k-1)!!}{(2k)!!}    \]
		\[ \frac{a_n}{a_{n+1}} = \frac{ (2n-1)!! \; (2n+2)!!}{ (2n+1)!! \;  (2n)!! } = \frac{2n+2}{2n+1} = 1 + \frac{1}{2n+1} = 1 + \frac{1}{2n} + \frac{1}{2n+1} - \frac{1}{2n} = 1 + \frac{1}{2n} + o\left( \frac{1}{n} \right)     \]
		$\mu = \frac{1}{2} < 1$, следовательно ряд $\sum_{k=1}^{\infty} \frac{(2k-1)!!}{(2k)!!}$ расходится.
		
	\end{example}	
\end{document}
