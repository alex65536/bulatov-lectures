\makeatletter
\def\input@path{{../../}}
\makeatother
\documentclass[../../main.tex]{subfiles}

\graphicspath{
	{../../img/}
	{../img/}
	{img/}
}

\begin{document}
	\subsection{Бесконечное произведение}

	Пусть задана последовательность $(p_k),\ p_k \in \R$. Отталкиваясь от нее, 
	построим новую последовательность:

	\[
	\begin{cases}
	P_1 = p_1 \\
	P_2 = p_1 \cdot p_2 \\
	P_3 = p_1 \cdot p_2 \cdot p_3 \\
	\ldots \\
	P_n = \prod\limits_{k = 1}^{n} p_k \\
	\ldots
	\end{cases} 
	\]

	Эту последовательность называют бесконечным произведением и записывают в виде:

	\begin{equation} \label{lec31, 1}
		\prod\limits_{k = 1}^{\infty} p_k
	\end{equation}
	
	Числа $p_k$ называются элементами (членами) произведения, а $P_n = 
	\prod\limits_{k = 1}^{n} p_k$ называют частичными произведениями.
	
	Если последовательность $P_n$ сходится (имеет предел $\lim\limits_{n \to 
	\infty}P_n$), то этот предел называют значением произведения \eqref{lec31, 
	1}. Если этот предел конечен и не равен нулю, то говорят, что \eqref{lec31, 
	1} сходится, иначе - расходится.
	
	\begin{example}
		\begin{enumerate}[label=\arabic*)]
			\item \[ \prod\limits_{k = 2}^{\infty} \left(1 - \frac{1}{k^2}\right) \]
			
			\[P_n = \prod\limits_{k = 2}^{n} \left(1 - \frac{1}{k^2}\right) = 
			\prod\limits_{k = 2}^{n} \frac{(k - 1)(k + 1)}{k^2} = \]
			
			\[= \frac{1 \cdot 3}{2 \cdot 2} \cdot \frac{2 \cdot 4}{3 \cdot 3} \cdot 
			\frac{3 \cdot 5}{4 \cdot 4} \cdot \ldots \cdot \frac{(n - 2) \cdot n}{(n - 
			1) \cdot (n - 1)} \cdot \frac{(n - 1) \cdot (n + 1)}{n \cdot n} = 
			\frac{1}{2} \cdot \frac{n + 1}{n} \appr{n \to \infty} \frac{1}{2}\]
			
			Поэтому бесконечное произведение имеет конечный предел, равный 
			$\frac{1}{2}$.
			
			\item  \[ \prod\limits_{k = 2}^{\infty} \left(1 - \frac{1}{k}\right) \]
			
			\[P_n = \prod\limits_{k = 2}^{n} \left(1 - \frac{1}{k}\right) = \frac{1}{2} 
			\cdot \frac{2}{3} \cdot \frac{3}{4} \cdot \ldots \cdot \frac{n - 1}{n} = 
			\frac{1}{n} \appr{n \to \infty} 0\]
			
			То есть бесконечная последовательность расходится.
			
			\item \[ \frac{2 \cdot 2}{1 \cdot 3} \cdot \frac{4 \cdot 4}{3 \cdot 5} 
			\cdot \frac{6 \cdot 6}{5 \cdot 7} \cdot \ldots \cdot \frac{2n \cdot 2n}{(2n 
			- 1) \cdot (2n + 1)} \cdot \ldots\]
			
			\[ P_n = \frac{2 \cdot 2}{1 \cdot 1} \cdot \frac{4 \cdot 4}{3 \cdot 3} 
			\cdot \frac{6 \cdot 6}{5 \cdot 5} \cdot \ldots \cdot \frac{2n \cdot 2n}{(2n 
			- 1) \cdot (2n + 1)} = \frac{2 \cdot 2}{1 \cdot 3} \cdot \frac{4 \cdot 4}{3 
			\cdot 5} \cdot \frac{6 \cdot 6}{5 \cdot 7} \cdot \ldots \cdot \frac{2n 
			\cdot 2n}{(2n - 1) \cdot (2n - 1)} \cdot \frac{1}{2n + 1} = \]
			
			\[= \frac{((2n)!!)^2}{((2n - 1)!!)^2 \cdot (2n + 1)} \appr{n \to \infty} 
			\frac{\pi}{2}\]
		\end{enumerate}
	\end{example}
	Пусть \eqref{lec31, 1} сходится, т.~е. $\exists \lim\limits_{n \to \infty} 
	\prod\limits_{k = 1}^{n} p_k = \pi \in \R,\ \pi \neq 0$. В этом случае $P_{n 
	- 1} = \prod\limits_{k = 1}^{n - 1}p_k,\ P_{n} = \prod\limits_{k = 
	1}^{n}p_k$. Тогда $p_n = \frac{P_n}{P_{n - 1}} \appr{n \to \infty} 
	\frac{\pi}{\pi} = 1$ - необходимое условие сходимости бесконечных 
	произведений.
		
	В дальнейшем будем это предполагать, т.~е. считаем, что $p_k$ примерно равно 
	1 и $p_k > 0$.
		
	Если у \eqref{lec31, 1} удалить несколько первых множителей (N штук), то 
	получим:
		
	\begin{equation} \label{lec31, 2}
	\prod\limits_{k = N + 1}^{\infty} p_k
	\end{equation}
		
	Возьмем любое $n > N$. При нем $P_n = \prod\limits_{k = 1}^{n} = 
	\left(\prod\limits_{k = 1}^{N}\right)\left(\prod\limits_{k = N + 
	1}^{n}\right)$
		
	При $n \to \infty$ конечный ненулевой предел слева существует тогда и только 
	тогда, когда предел справа конечен и не равен нулю. Т.~е. бесонечное 
	произведение \eqref{lec31, 1} сходится тогда и только тогда, когда сходится 
	бесконечное произведение \eqref{lec31, 2}. \eqref{lec31, 2} называется 
	остаточным произведением (остатком) \eqref{lec31, 1}.
		
	Получаем: 
		
	\begin{enumerate}[label={\alph*)}]
		\item Если сходится бесконечное произзведение, то и любой его остаток
			
		\item Если сходится какой-либо остаток произведения \eqref{lec31, 1}, то 
		сходится и \eqref{lec31, 1}.
		\end{enumerate}
	
	\subsection{Связь с рядами}
	
	\begin{equation}\label{lec31, 3}
	 \ln P_n = \ln \prod\limits_{k = 1}^{n} p_k = \sum\limits_{k = 1}^{n} \ln p_k
	\end{equation}
	
	Рассмотрим ряд
	\begin{equation}\label{lec31, 4}
	\sum\limits_{k = 1}^{\infty} \ln p_k
	\end{equation}
	
	При переходе в \eqref{lec31, 3} к пределу, конечные пределы слева и справа 
	могут существовать только одновременно. Это значит, что \eqref{lec31, 1} 
	сходится тогда и только тогда, когда сходится ряд \eqref{lec31, 4}.
	
	\begin{thm}\label{lec31:thm1} 
		Произведение $\prod\limits_{k = 1}^{\infty} p_k$ сходится тогда и только 
		тогда, когда сходится ряд $\sum\limits_{k = 1}^{\infty} \ln p_k$. При этом 
		если $s = \sum\limits_{k = 1}^{\infty} \ln p_k$, то $\prod\limits_{k = 
		1}^{\infty} p_k = e^s$
	\end{thm}

	\begin{example}
		\begin{enumerate}[label=\arabic*)]
			\item \[\prod\limits_{k = 1}^{\infty} \sqrt{\frac{k + 3}{k}}\]
			
			\[\sum\limits_{k = 1}^{\infty} \ln \sqrt{\frac{k + 3}{k}} = \frac{1}{2} 
			\sum\limits_{k = 1}^{\infty} \ln\left(1 + \frac{3}{k}\right) \]
			
			$a_k = \ln \left(1 + \frac{3}{k}\right) \sim \frac{3}{k}$ - расходится 
			$\implies$ расходится и бесконечное произведение.
			
			\item \[\prod\limits_{k = 1}^{\infty} e^{\frac{1}{k(k + 1)}} \]
			
			\[\ln P_n = \sum\limits_{k = 1}^{n} \ln \left(e^{\frac{1}{k(k + 1)}}\right) 
			= \sum\limits_{k = 1}^{n} \frac{1}{k(k + 1)}\] - сходится.
			
			\[\sum\limits_{k = 1}^{n} \frac{1}{k(k + 1)} = \sum\limits_{k = 1}^{n} 
			\left(\frac{1}{k} - \frac{1}{k + 1} \right) = 1 - \frac{1}{n + 1} \appr{n 
			\to \infty} 1 \]
			
			\[s = \sum\limits_{k = 1}^{\infty} \frac{1}{k(k + 1)} = 1 \implies 
			\prod\limits_{k = 1}^{\infty} e^{\frac{1}{k(k + 1)}} = e\]
		\end{enumerate}
	\end{example}

	Рассмотрим $\sum\limits_{k = 1}^{\infty} \ln p_k$ \eqref{lec31, 4}. Пусть 
	$p_k = 1 + \alpha_k,\ \forall k$. Из необходимого условия сходимости следует, 
	что $\alpha_k \appr{k \to \infty} 0$. Предположим, что все $\alpha_k > 0$, 
	тогда $\ln p_k = \ln (1 + \alpha_k) \sim \alpha_k$.
	
	Ряд \eqref{lec31, 4} и ряд $\sum\limits_{k = 1}^{\infty} \alpha_k$ оба 
	сходятся или оба расхоятся по \hyperref[lec26:comp_test_2]{признаку 
	2\textdegree}.
	
	\begin{thm}\label{lec31:thm2} 
		Произведение $\prod\limits_{k = 1}^{\infty} p_k = \prod\limits_{k = 
		1}^{\infty}(1 + \alpha_k),\ \alpha_k > 0$ сходится тогда и только тогда, 
		когда сходится ряд $\sum\limits_{k = 1}^{\infty} \alpha_k = \sum\limits_{k = 
		1}^{\infty} (p_k - 1)$
	\end{thm}		

	\begin{example}
		\begin{enumerate}
			\item $\sum\limits_{k = 1}^{\infty} \frac{1}{k^a}$ сходится при $a > 1$ и 
			расходится при $a \leq 1$, поэтому аналогичное верно и для ряда 
			$\prod\limits_{k = 1}^{\infty} \left(1 + \frac{1}{k^a}\right)$
			
			\item \[\prod\limits_{k = 1}^{\infty} \left(1 + \frac{x^{2k}}{3^k}\right)\]
			
			\[\sum\limits_{k = 1}^{\infty}\alpha_k = \sum\limits_{k = 1}^{\infty} 
			\frac{x^{2k}}{3^k}\]
			
			Исследуем сходимость по признаку Коши: $\sqrt[n]{\alpha_n} = 
			\sqrt[n]{\frac{x^{2n}}{3^n}} = \frac{x^2}{3} \implies$ ряд (и бесконечное 
			произведение также) сходится, если $\frac{x^2}{3} < 1$, т.~е. когда $x \in 
			]-\sqrt{3}; \sqrt{3}[$
			
			\item \[\prod\limits_{k = 1}^{\infty} \sqrt{\frac{k^2 + \sqrt{k}}{k^2}}\]
			
			Рассмотрим $\ln\sqrt{\frac{k^2 + \sqrt{k}}{k^2}} = \frac{1}{2}\ln\frac{k^2 
			+ \sqrt{k}}{k^2} \sim \frac{1}{2} \cdot \frac{1}{k^{3/2}}$. Т.~к. 
			$\frac{3}{2} > 1$, то ряд сходится.
			
			\item \[\prod\limits_{k = 1}^{\infty} \cos\frac{1}{k}\]
			
			$\sum\limits_{k = 1}^{\infty}(p_k - 1) = \sum\limits_{k = 1}^{\infty} 
			(\cos\frac{1}{k} - 1) = -\sum\limits_{k = 1}^{\infty}2\sin^2\frac{1}{2k}$
			
			$p_k - 1 = -2\sin^2\frac{1}{2k} \sim -\frac{1}{2k^2} \implies 
			\sum\limits_{k = 1}^{\infty}(p_k - 1)$ сходится.
		\end{enumerate}	
	\end{example}

	Рассмотрим произведение \eqref{lec31, 1}, где $p_k = 1 + \alpha_k$ и 
	$\alpha_k$ не сохраняет знак. Произведение называется абсолютно сходящимся, 
	если сходится
	\begin{equation} \label{lec31, 6}
	 \prod\limits_{k = 1}^{\infty} (1 + |\alpha_k|)
	\end{equation}
	
	\begin{thm}
		Если сходится \eqref{lec31, 6}, то и \eqref{lec31, 1} (т.~е. если 
		произведение сходится абсолютно, то оно сходится)
		\begin{proof}
			В \eqref{lec31, 6} $\textbar\alpha_k\textbar > 0$. Будем предполагать, что 
			\eqref{lec31, 6} сходится и воспользуемся критерием Коши для доказательства 
			сходимости \eqref{lec31, 1}. Сходимость \eqref{lec31, 1} равносильна 
			сходимости $\sum\limits_{k = 1}^{\infty} \ln (1 + \alpha_k)$.
			
			\[\forall \varepsilon > 0,\ \exists \nu = \nu_\varepsilon,\ \forall n, m 
			\geq \nu (n \geq m): \textbar\sum\limits_{k = m + 1}^{n} \ln(1 + 
			\alpha_k)\textbar \leq \varepsilon\].
			
			\[\textbar\sum\limits_{k = m + 1}^{n} \ln(1 + \alpha_k) \textbar \leq 
			\sum\limits_{k = m + 1}^{n} \textbar\ln(1 + \alpha_k)\textbar = (*)\]
			
			$\frac{\textbar\ln(1 + \alpha_k)\textbar}{\textbar\alpha_k\textbar} = 
			\textbar \frac{\ln(1 + \alpha_k)}{\alpha_k}\textbar \to 1$, т.~к. $\textbar 
			\ln(1 + \alpha_k) \textbar \sim \textbar \alpha_k \textbar \sim \ln(1 + 
			\textbar \alpha_k \textbar)$.
			
			Т.~к. \eqref(lec31, 6) сходится, то сходится и ряд $\sum\limits_{k = 
			1}^{\infty} \ln(1 + \textbar \alpha_k \textbar)$, а значит и 
			$\sum\limits_{k = 1}^{\infty} \textbar \alpha_k \textbar$, а значит 
			сходится ряд $\sum\limits_{k = 1}^{\infty} \textbar \ln(1 + \alpha_k) 
			\textbar$. Итак:
			
			\[\forall \varepsilon > 0,\ \exists \nu = \nu_\varepsilon,\ \forall n > m > 
			\nu: (*) \leq \varepsilon \implies\]
			по критерию Коши \eqref{lec31, 1} сходится.
		\end{proof}
	\end{thm}

	\begin{thm}
		Если сходятся оба ряда $\sum\limits_{k = 1}^{\infty} \alpha_k$ и 
		$\sum\limits_{k = 1}^{\infty} \alpha_k^2$, то и $\prod\limits_{k = 
		1}^{\infty} (1 + \alpha_k)$.
		
		\begin{proof}
			Рассмотрим 
			\begin{equation}\label{lec31, 7}
				\prod\limits_{k = 1}^{\infty} (1 + \alpha_k)
			\end{equation}
			
			Т.~к. $\ln(1 + \alpha_k) = \alpha_k - \frac{\alpha_k^2}{2} + o(\alpha_k^2) 
			= \alpha_k + b_k$
			
			Тогда если оба ряда в условии теоремы сходятся, то \eqref{lec31, 7} 
			сходится как сумма двух сдодящихся рядов $\sum\limits_{k = 1}^{\infty} 
			\alpha_k$ и $\sum\limits_{k = 1}^{\infty} b_k$, т.~к. $b_k \sim \alpha_k^2$.
		\end{proof}
	\end{thm}

	\begin{example}
		\[\prod\limits_{k = 1}^{\infty} \left(1 + \frac{\sin k}{k}\right)\]
		
		$\alpha_k = \frac{\sin k}{k}$ не сохраняет знак, но $\sum\limits_{k = 
		1}^{\infty} \frac{\sin k}{k}$ сходится по признаку Дирихле: $\textbar 
		\sum\limits_{k = 1}^{n} \sin k \textbar \leq \frac{1}{\sin \frac{1}{2}}$ - 
		ограничена, а $\frac{1}{k}$ монотонно стремится к 0.
		
		$\sum\limits_{k = 1}^{\infty} \left(\frac{\sin k}{k}\right)^2$ сходится, 
		поскольку $\left(\frac{\sin k}{k}\right)^2 \leq \frac{1}{k^2}$, поэтому 
		рассматриваемое произведение сходится.
	\end{example}

	Заметим, что из доказательства следует, что если один из рядов 
	$\sum\limits_{k = 1}^{\infty} \alpha_k$ и $\sum\limits_{k = 1}^{\infty} 
	\alpha_k^2$ сходится, а другой расходится, то произведение расходится.
	
	\begin{example}
		\[\prod\limits_{k = 1}^{\infty} \left(1 + \frac{(-1)^k}{k^a}\right)\]
		
		Ряд $\sum\limits_{k = 1}^{\infty} \frac{(-1)^k}{k^a}$ сходится при $a > 0$, 
		а ряд $\sum\limits_{k = 1}^{\infty} \left(\frac{(-1)^k}{k^a}\right)^2 = 
		\sum\limits_{k = 1}^{\infty} \frac{1}{k^{2a}}$ - сходится при $2a > 1$ и 
		расходится при $2a \leq 1$. Оба сходятся при $a > \frac{1}{2}$, а при $a 
		\leq \frac{1}{2}$ один расходится, поэтому $\prod\limits_{k = 1}^{\infty} 
		\left(1 + \frac{(-1)^k}{k^a}\right)$ сходится при $a > \frac{1}{2}$, 
		расходится при $a \leq \frac{1}{2}$.
	\end{example}
	
\end{document}
