\makeatletter
\def\input@path{{../../}}
\makeatother
\documentclass[../../main.tex]{subfiles}

\begin{document}

\section{Линейное евклидово пространство $\R^n$}

Через $\R^n$ будем обозначать множество упорядоченных наборов из $n$
действительных чисел. Эти наборы будем называть \emph{векторами},
а соответствующие им точки будем обозначать $M(x_1, x_2, \dots, x_n)$.

Если по определению для

\[
\begin{array}{c}
 \forall x = (x_1, x_2, \dots, x_n) \in \R^n, \\
 \forall y = (y_1, y_2, \dots, y_n) \in \R^n, \\
 \forall \lambda, \mu \in \R,
\end{array}
\]

положить

\[\lambda x + \mu y = (\lambda x_1 + \mu y_1, \lambda x_2 + \mu y_2, 
\dots, \lambda x_n + \mu y_n),\]

то относительно множества $\R^n$ мы получим линейное (векторное) 
пространство над полем $\R$. В общем случае, линейное (векторное) 
пространство $P$ над полем $\R$ называется \emph{евклидовым},
если в нем определено скалярное произведение, то есть  
задано отображение $\left<x, y\right>: P^2 \to \R$, 
ставящее в соответствие любым $x, y \in P$ число 
$\left<x, y\right>$, удовлетворяющее аксиомам:

\begin{enumerate}
 \item Неотрицательность:
 \[\forall x \in P \implies \left<x, x\right> \ge 0,\]
 причем
 \[\left<x, x\right> = 0 \iff x = \vec 0\]
 \item Симметричность:
 \[\forall x, y \in P \implies \left<x, y\right> = \left<y, x\right>\]
 \item Линейность:
 \[\forall \lambda, \mu \in \R, \quad \forall x, y, z \in P \implies
 \left<\lambda x + \mu y, z\right> = \lambda\left<x, z\right> + 
 \mu\left<y, z\right>\]
\end{enumerate}

Из последней аксиомы, в частности, следует \[ \lambda = \mu = 0
\implies \left<\vec{0}, z\right> = 0, \quad \forall z \in P\]

\begin{thm}[неравенство Коши-Буняковского]
 В любом линейном евклидовом пространстве $P$ над полем $\R$
 выполняется \emph{неравенство Коши-Буняковского}
 
 \begin{equation}
  \label{kosh-bun}
  (\left<x, y\right>)^2 \leq \left(\left<x, x\right>\right) 
  \cdot \left(\left<y, y\right>\right)
 \end{equation}
\end{thm}

\begin{proof}
 Для произвольного $t \in \R$ рассмотрим скалярную функцию
 
 \[f(t) = \left<x + ty, x + ty\right> \ge 0\]

 Используя линейность скалярного произведения, имеем
 
 \[f(t) = at^2 + 2bt + c,\]
 
 где
 
 \[
   \begin{array}{c}
    a = \left<y, y\right> \ge 0 \\
    b = \left<x, y\right> \\
    c = \left<x, x\right> \\
   \end{array}
 \]
 
 Если $a = 0$, то есть $\left<y, y\right> = 0$, то $y = \vec 0$; в этом 
 случае справа получаем $0$, а слева~--- тоже $0$, т.~е. выполнено 
 \eqref{kosh-bun}.
 
 Пусть $a > 0$. В этом случае имеем квадратичную функцию $f(t) \ge 0 
 \quad \forall t \in \R$. График этой функции~--- парабола, ветви 
 которой направлены вверх. Поэтому дискриминант данной функции должен 
 быть неположительным:
 
 \[
   D = (2b)^2 - 4ac \le 0 \iff b^2 \le ac \iff \eqref{kosh-bun} 
   \qedhere
 \]

\end{proof}

\begin{crl*}[неравенство Минковского]
 В линейном евклидовом пространстве $P$ для любых $x, y 
 \in P$ справедливо \emph{неравенство Минковского}
 
 \begin{equation}
  \label{mink}
  \sqrt{\left<x+y, x+y\right>} \le
  \sqrt{\left<x, x\right>} + \sqrt{\left<y, y\right>}
 \end{equation}

\end{crl*}

\begin{proof}
 Во-первых, отметим, что в силу аксиомы неотрицательности скалярного 
 произведения подкоренные выражения \eqref{mink} неотрицательны. 
 Далее, используя линейность скалярного произведения и неравенство 
 Коши-Буняковского \eqref{kosh-bun}, получаем:
 
 $\displaystyle
  \sqrt{\left<x+y, x+y\right>} = 
  \sqrt{\left<x, x\right> + 2\left<x, y\right> + \left<y, y\right>} 
  \le
  \Big[
    \left<x, y\right> \le \left|\left<x, y\right>\right|
    \stackrel{\eqref{kosh-bun}}{\le} \sqrt{\left<x, x\right> \cdot
    \left<y, y\right>}
  \Big]
  \le
  \sqrt{\left(\sqrt{\left<x, x\right>}\right)^2 +
        2\sqrt{\left<x, x\right> \cdot \left<y, y\right>} +
        \left(\sqrt{\left<y, y\right>}\right)^2} =
  \sqrt{\left(\sqrt{\left<x, x\right>} + \sqrt{\left<y, y\right>}\right)^2} =
  \sqrt{\left<x, x\right>} + \sqrt{\left<y, y\right>}
 $
\end{proof}

\begin{rem}
 В дальнейшем, как правило, в линейном пространстве 
 $\R^n$ над полем $\R$ будем 
 использовать евклидово скалярное произведение, т.~е. для

 \[
 \begin{array}{c}
  \forall x = (x_1, x_2, \dots, x_n) \in \R^n, \\
  \forall y = (y_1, y_2, \dots, y_n) \in \R^n, \\
  \forall \lambda, \mu \in \R
 \end{array}
 \]
 
 полагаем
 
 \begin{equation}
  \label{eucl}
  \left<x, y\right> = x_1y_1 + x_2y_2 + \dots + x_ny_n =
  \sum_{k=1}^n x_ky_k.
 \end{equation}

 Нетрудно проверить, что \eqref{eucl} удовлетворяет всем аксиомам 
 скалярного произведения (неотрицательность, симметричность, 
 линейность). В этом случае в линейном евклидовом пространстве
 $\R^n$ со скалярным произведением \eqref{eucl} неравенство
 Коши-Буняковского \eqref{kosh-bun} принимает вид
 
 \begin{equation}
  \label{kosh-bun.e}
  \left(x_1y_1 + x_2y_2 + \dots + x_ny_n\right)^2 \le
  \left(x_1^2 + x_2^2 + \dots + x_n^2\right)\cdot
  \left(y_1^2 + y_2^2 + \dots + y_n^2\right),
 \end{equation}
 
 а неравенство Минковского \eqref{mink} имеет вид
 
 \begin{equation}
  \label{mink.e}
  \begin{array}{c}
   \displaystyle
   \sqrt{\sum_{k=1}^n (x_k + y_k)^2} \le
   \sqrt{\sum_{k=1}^n x_k^2} + \sqrt{\sum_{k=1}^n y_k^2} \\
   \forall x_k, y_k \in \R,\ k = \overline{1, n}
  \end{array}
 \end{equation}
 
\end{rem}

\begin{exc}
 Выяснить, при каких условиях неравенства \eqref{kosh-bun.e} и 
 \eqref{mink.e} переходят в соответствующие равенства.
\end{exc}

% Ответы на упражнения заблокированы Роскомнадзором :)

% \begin{eans}
%  Пусть неравенство Коши \eqref{kosh-bun} перешло в равенство:
% 
%  \[ (\left<x, y\right>)^2 = \left(\left<x, x\right>\right) 
%  \cdot \left(\left<y, y\right>\right) \]
% 
%  Покажем, что это возможно тогда и только тогда, когда
%  $x = ay\ \lor\ y = ax, a \in \R$, т.~е. векторы линейно зависимы.
%  Нетрудно видеть, что этого условия достаточно для
%  выполнения равенства:
% 
%  \[ (\left<ay, y\right>)^2 = \left(\left<ay, ay\right>\right) 
%  \cdot \left(\left<y, y\right>\right) \iff \left[
%  \text{линейность}\right] \iff a^2 (\left<y, y\right>)^2 = 
%  a^2 (\left<y, y\right>)^2 \]
%  
%  Аналогично показывается, что равенство достигается при $y = ax$.
% 
%  Покажем, что оно необходимо. Случаи $x = \vec{0}$ и $y = \vec{0}$
%  тривиальны, поэтому пусть $x \neq \vec{0}$ и $y \neq \vec{0}$. 
%  Из равенства имеем:
% 
%  \[\left<x, x\right> = \frac{\left<x, y\right>^2}{\left<y, y\right>}
%  \implies \left<x, x\right> = \left(\frac{\left<x, y\right>}
%  {\left<y, y\right>}\right)^2 \left<y, y\right> \iff \left<x, x\right>
%  = \left<\left(\frac{\left<x, y\right>}{\left<y, y\right>}\right) y, 
%  \left(\frac{\left<x, y\right>}{\left<y, y\right>}\right) y\right> \]
% 
%  Пусть:
% 
%  \[a = \frac{\left<x, y\right>}{\left<y, y\right>}, a \in \R \]
%  Используя свойство линейности, получаем:
%  \[\left<x, x\right> - \left<ay, ay\right> = 0 \iff \left<x - ay,
%  x - ay\right> = 0 \iff x - ay = \vec{0} \implies x = ay\]
% 
%  Для того, чтобы неравенство Минковского \eqref{mink} перешло в
%  равенство, необходимо и достаточно, чтобы
%  $x = ay\ \lor\ y = ax, a \geq 0, a \in \R$.
%  Это следует из доказательства неравенства Минковского и условия равенства
%  в неравенстве Коши, которое используется при доказательстве
%  неравенства Минковского.
% \end{eans}

\section{Нормированные и метрические пространства $\R^n$}

Линейное пространство $\R^n$ будем называть 
\emph{нормированным}, если существует отображение

\begin{equation}
 \label{norm-disp}
 \lVert\cdot \rVert\colon \R^n \to \R,
\end{equation}

которое ставит в соответствие любому $x \in \R^n$ действительное 
число $\norm x$, для которого выполняются следующие аксиомы нормы:

\begin{enumerate}
 \item Невырожденность:
 \[\forall x \in \R^n, \quad \norm x = 0 \iff x = \vec 0 \in \R^n.\]
 \item Однородность:
 \[\forall x \in \R^n, \quad \forall \lambda \in \R \implies 
   \norm{\lambda x} = \abs\lambda \cdot \norm x.\]
 \item Неравенство треугольника:
 \[\forall x, y \in \R^n \implies \norm{x+y} \le \norm x + \norm y.\]
\end{enumerate}

($\norm{x}$~--- \emph{норма} вектора $x$)

\bigskip

Отметим, что в силу последней аксиомы при $y = -x \in \R^n$ получаем

$0 = \norm{\vec 0} = \norm{x - x} \le \norm x + \norm{-x} =
 \norm x + \abs{-1} \cdot \norm x = 2\norm x \implies \norm x \ge 0
 \quad \forall x \in \R^n$
 
Значит, как и скалярное произведение, норма $\norm x$ обладает
свойством неотрицательности.

\begin{thm}[о нормировании линейного евклидова пространства $\R^n$]
 Любое линейное евклидово пространство $\R^n$ с произвольным скалярным
 произведением нормируется с помощью естественной нормы
 
 \begin{equation}
  \label{rn-norm}
  \norm x = \sqrt{\left<x, x\right>} \quad \forall x \in \R^n
 \end{equation}
 
\end{thm}

\begin{proof}
 Во-первых, отметим, что \eqref{rn-norm} корректно определена в силу 
 неотрицательности подкоренного выражения. Во-вторых, проверим
 выполнимость всех аксиом нормы:
 
 \begin{enumerate}
  \item[а)]
  $\norm x = 0 \iff \sqrt{\left<x, x\right>} = 0 \iff
  \left<x, x\right> = 0 \implies x = \vec 0$ (в силу 
  неотрицательности скалярного произведения)
  
  \item[б)]
  Из линейности и симметричности скалярного произведения получаем
  
  $\forall \lambda \in \R \quad \forall x \in \R^n \quad
   \norm{\lambda x} \stackrel{\eqref{rn-norm}}{=}
   \sqrt{\left<\lambda x, \lambda x\right>} = \dots =
   \sqrt{\lambda^2\left<x, x\right>} =
   \abs\lambda \sqrt{\left<x, x\right>} \stackrel{\eqref{rn-norm}}{=}
   \abs\lambda \norm x$
 
  \item[в)]
  Используя неравенство Минковского, имеем
  
  $\norm{x+y} \stackrel{\eqref{rn-norm}}{=}
   \sqrt{\left<x+y, x+y\right>} \stackrel{\eqref{mink}}{\le}
   \sqrt{\left<x, x\right>} + \sqrt{\left<y, y\right>}
   \stackrel{\eqref{rn-norm}}{=} \norm x + \norm y$
 \end{enumerate}
 
\end{proof}

\begin{rem}
 В случае евклидова пространства $\R^n$ с евклидовым скалярным 
 произведением \eqref{eucl} естественной нормой будет
 
 \[\begin{array}{c}
    \norm x = \sqrt{x_1^2 + x_2^2 + \dots + x_n^2}, \\
    \forall x = (x_1, x_2, \dots, x_n) \in \R^n.
   \end{array}\]
   
 В этом случае для простоты будем писать $\abs x = \norm x = 
 \sqrt{x_1^2 + x_2^2 + \dots + x_n^2}$, а само евклидово скалярное
 произведение будем записывать в виде $x\cdot y = \sum\limits_{i=1}^k 
 x_ky_k$.

\end{rem}

\begin{exc}
  Доказать, что для $\forall x, y \in \R^n$ в случае естественной нормы
  имеем
  
  \[\Big|\abs x - \abs y\Big| \le \abs{x \pm y} \le 
  \abs x + \abs y\]
\end{exc}

% Ответы на упражнения заблокированы Роскомнадзором :)

% \begin{eans}
%  Не нарушая общности, предположим, что $x \ge y$.
%  
%  Докажем $\Big|\abs x - \abs y\Big| \le \abs{x + y} \le \abs x + \abs y$,
%  второй случай аналогичен. Правая часть неравенства $\abs{x + y} \le 
%  \abs x + \abs y$ верна в силу неравенства треугольника, левая часть
%  
%  $\Big|\abs x - \abs y\Big| = 
%  \abs x - \abs y \le \abs{x + y} \implies
%  \abs{(x + y) - y} \le \abs{x + y} + \abs y = \abs{x + y} + \abs{-y}$
%  
%  \noindent также верна в силу неравенства треугольника, что и 
%  требовалось доказать.
% \end{eans}

Линейное пространство $\R^n$ будем называть \emph{метрическим}, если 
имеется отображение 

\begin{equation}
 \label{metric}
 \rho: \R^n \times \R^n \to \R,
\end{equation}

ставящее в соответствие для любого $x, y \in \R^n$ единственное число 
$\rho(x, y) \in \R$, удовлетворяющее следующим аксиомам расстояния:

\begin{enumerate}[label=\Roman*.]
 \item Неотрицательность:
 \[\forall x, y \in \R^n \implies \rho(x, y) \ge 0\text{, причем }
   \rho(x, y) = 0 \iff x = y.\]
 
 \item Симметричность:
 \[\forall x, y \in \R^n \implies \rho(x, y) = \rho(y, x).\]
 
 \item Неравенство треугольника для расстояния:
 \[\forall x, y, z \in \R^n \implies \rho(x, y) \le \rho(x, z) + 
   \rho(z, y).\]
 
\end{enumerate}

В дальнейшем линейное метрическое пространство $\R^n$ с 
соответствующей метрикой \eqref{metric} кратко будем обозначать
$\left(\R^n, \rho\right)$.

\begin{thm}[о метризируемости произвольного линейного нормированного 
	пространства $\R^n$]
 Любое линейное нормированное пространство $\R^n$ с нормой
 \eqref{norm-disp} метризируется с помощью естественного расстояния

 \begin{equation}
  \label{nat-dist}
  \rho(x, y) = \norm{x-y},\quad \forall x, y \in \R^n.
 \end{equation}
 
\end{thm}

\begin{proof}
 В силу аксиомы неотрицательности нормы, во-первых, имеем $\rho(x, y) 
 = \norm{x-y} \ge 0 \quad \forall x, y \in \R^n$ и, во-вторых, получаем
 \[\rho(x, y) = 0 \stackrel{\eqref{nat-dist}}{\iff} \norm{x-y} = 0 
  \iff x = y\]
 
 Далее, используя аксиомы нормы, имеем
 
 \[\rho(x, y) \stackrel{\eqref{nat-dist}}{=} \norm{-(y-x)} =
   \abs{-1}\norm{y-x} = \norm{y-x} = \rho(y, x).\]
 
 В силу неравенства треугольника для нормы для
 
 $\forall x, y, z \in \R^n \implies \rho(x, y) \stk{nat-dist}{=} 
  \norm{x-y} = \norm{(x-z) + (z-y)} \le
  \norm{x-z} + \norm{z-y} \stk{nat-dist}{=} \rho(x, z) +  \rho(z, y).$
\end{proof}

\begin{rems} 

 \quad

 \begin{enumerate}
  \item  Для линейного евклидова пространства $\R^n$ с евклидовой
  нормой
  
  \begin{equation}
   \label{e-norm}
   \begin{array}{c}
    \abs x = \norm x = \sqrt{\sum\limits_{k=1}^n x_k^2}, \\
    \forall x = (x_1, x_2, \dots, x_n) \in \R^n
   \end{array}
  \end{equation}
  
  на основании \eqref{nat-dist} имеем \emph{евклидово 
  расстояние}, которое
  в дальнейшем будем обозначать $d(x, y) = \rho(x, y) \stk{e-norm}{=}
  \sqrt{\sum\limits_{k=1}^n (x_k - y_k)^2}$. Полученное линейное евклидово 
  метрическое пространство $\R^n$ с таким евклидовым расстоянием 
  будем записывать в виде $(\R^n, d)$.
  
  \item Любое линейное пространство $\R^n$ можно метризировать с
  помощью тривиальной метрики
  
  \[
   \rho_0(x, y) =
   \begin{cases}
    1, & \text{если } x \ne y, \\
    0, & \text{если } x = y. \\
   \end{cases}
  \]
  
  \item Иногда, наряду с пространствами $(\R^n, d)$ и $(\R^n, \rho_0)$
  удобнее использовать также метрические пространства $(\R^n, \rho_1)$
  и $(\R^n, \rho_2)$ соответственно с \emph{октаэдрической метрикой}
  \[\rho_1(x, y) = \sum_{i=1}^k \abs{x_k - y_k}\]
  и \emph{кубической метрикой}
  \[\rho_2(x, y) = \max_{1 \le k \le n} \abs{x_k - y_k},\]
  \[\begin{array}{c}
     \forall x = (x_1, x_2, \dots, x_k) \in \R^n, \\
     \forall y = (y_1, y_2, \dots, y_k) \in \R^n. \\
    \end{array}\]
 \end{enumerate}

\end{rems}

\end{document}
