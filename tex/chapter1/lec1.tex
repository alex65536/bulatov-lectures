\makeatletter
\def\input@path{{../../}}
\makeatother
\documentclass[../../main.tex]{subfiles}

\begin{document}

\section{Линейное евклидово пространство $\R^n$}

Через $R^n$ будем обозначать множество упорядоченных наборов из $n$
действительных чисел. Эти наборы будем называть \textit{векторами},
а соответствующие им точки будем обозначать $M(x_1, x_2, \dots, x_n)$.

Если по определению

\[
\begin{array}{c}
 \forall x = (x_1, x_2, \dots, x_n) \in \R^n \\
 \forall y = (y_1, y_2, \dots, y_n) \in \R^n \\
 \forall \lambda, \mu \in \R
\end{array}
\]

положить

\[\lambda x + \mu y = (\lambda x_1 + \mu y_1, \lambda x_2 + \mu y_2 + 
\dots + \lambda x_n + \mu y_n)\]

то относительно множества $\R^n$ мы получим линейное (векторное) 
пространство над полем $\R$. В общем случае, линейное (векторное) 
пространство $P$ над над полем $\R$ называется \textit{евклидовым},
если в нем определено скалярное произведение для любых $x, y \in P$, 
т. е. отображение $\left<x, y\right>: P^2 \to R$, ставящее в 
соответствие двум векторам $x$ и $y$ число $\left<x, y\right>$, 
удовлетворяющее аксиомам:

\begin{enumerate}
 \item Неотрицательность:
 \[\forall x \in P \implies \left<x, x\right> \ge 0,\]
 причем
 \[\left<x, x\right> = 0 \iff x = \vec 0\]
 \item Симметричность:
 \[\forall x, y \in P \implies \left<x, y\right> = \left<y, x\right>\]
 \item Линейность:
 \[\forall \lambda, \mu \in P \quad \forall x, y, z \in P \implies
 \left<\lambda x + \mu y, z\right>\]
\end{enumerate}

Из последней аксиомы, в частности, следует \[\left<0, z\right> = 0 
\quad \forall z \in P\]

\begin{thm}[Неравенство Коши-Буняковского]
 В любом линейном евклидовом линейном пространстве $P$ над полем $\R$
 выполняется неравенство
 
 \begin{equation}
  \label{kosh-bun}
  (\left<x, y\right>)^2 = \left<x, x\right>\cdot\left<y, y\right>
 \end{equation}
\end{thm}

\begin{proof}
 Для произвольного $t \in \R$ рассмотрим функцию
 
 \[f(t) = \left<x + ty, x + ty\right> \ge 0\]

 Используя линейность скалярного произведения, имеем
 
 \[f(t) = at^2 + 2bt + c,\]
 
 где
 
 \[
   \begin{array}{c}
    a = \left<y, y\right> \ge 0 \\
    b = \left<x, y\right> \\
    c = \left<x, x\right> \\
   \end{array}
 \]
 
 Если $a = 0$, т. е. $\left<y, y\right> = 0$, то $y = \vec 0$; в этом 
 случае справа имеем $0$, а слева~--- тоже $0$, т. е. выполнено 
 \eqref{kosh-bun}.
 
 Пусть $a > 0$. В этом случае имеем квадратичную функцию $f(t) \ge 0 
 \quad \forall t \in \R$. График этой функции~--- парабола, ветви 
 которой направлены вверх. Поэтому дискриминант должен быть 
 неположительным, т. е.
 
 \[
   D = (2b)^2 - 4ac \le 0 \iff b^2 \le 4ac \iff \eqref{kosh-bun} 
   \qedhere
 \]

\end{proof}

\begin{crl*}[Неравенство Минковского]
 В линейном евклидовом пространстве $P$ для любых $x, y \in P$ имеем:
 
 \begin{equation}
  \label{mink}
  \sqrt{\left<x+y, x+y\right>} \le \sqrt{\left<x, x\right>} + 
  \sqrt{\left<y, y\right>}
 \end{equation}

\end{crl*}

\begin{proof}
 Во-первых, отметим, что в силу аксиомы неотрицательности скалярного произведения подкоренное выражение \eqref{mink} неотрицательно.
\end{proof}




\end{document}
