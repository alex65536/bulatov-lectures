\makeatletter
\def\input@path{{../}}
\makeatother
\documentclass[../main.tex]{subfiles}


\begin{document}
	
	\section{Множества и окрестности в $\R^n$}
	Далее в метрическом пространстве $(\R^n, \rho)$ для записи его элементов будем использовать не только $a, b c, \dots x$, также $R, L, M, \cdots$, а также элементы будем называть точками.
	
	Открытым шаром $B_r(x_0)$ радиуса $r > 0$ с центром в $x_0 \in \R^n$ в пространстве ($\R^n, \rho$) будем называть множество 
	$$
	B_r(x_0) = \{x \in \R^n\ |\ \rho(x, x_0) < r\}, 
	$$
	А замкнутым шаром
	$$
	\overline{B_r}(x_0) = \{x \in \R^n\ |\ \rho(x, x_0) \leqslant r\}, 
	$$ 
	Очевидно, что $\overline{B_r}(x_0) = B_r(x_0) + S_r(x_0),$ где $ S_r(x_0)$~--- n-мерная сфера в  $(\R^n, \rho)$, то есть 
	$$
	S_r(x_0) = \{x \in \R^n\ |\ \rho(x, x_0) = r\}
	$$ 
	Шары $B_r, \overline{B_r}$ будем называть открытыми и замкнутыми окрестностями для $x_0 \in \R^n$. А в случае когда размеры не существенны или фиксированы будем их обозначать для краткости $B_r(x_0) = V(x_0)$ и $\overline{B_r}(x_0) = \overline{V}(x_0)$.
	
	\textbf{Примеры:}
	\begin{itemize}
	\item Для $n = 1 \Rightarrow d(x, x_0) = |x - x_0|, x, x_0 \in \R.$ В этом случае 
	$$
		B_r(x_0) = \{x \in \R\ |\ |x - x_0| < r\} =\ ] x_0 - r [,
	$$
	$$
		\overline{B_r}(x_0) = \{x \in \R\ |\ |x - x_0| \leqslant r\} =\ [ x_0 - r ].
	$$
	Геометрически имеем:
	
	...
	
	\item  Для $M(x, y) \in \R^2$ в ДПСК имеем для $M_0(x_0, y_0)$:
	$$
		d(M, M_0) = \sqrt{(x - x_0)^2 + (y - y_0)^2}.
	$$
	Поэтому открытым шаром от $M_0$ здесь будет открытый круг 
	$
		(x - x_0)^2 + (y - y_0)^2 < r^2,
	$ 
	а замкнутым шаром будет замкнутый круг 
	$
		(x - x_0)^2 + (y - y_0)^2 \leqslant r^2,
	$ 
	а сфера является окружностью 
	$
		(x - x_0)^2 + (y - y_0)^2 = r^2.
	$ 
	
	\item Аналогично для $n = 3$ для $M_0(a, b, c) \in \R^3$ и $M(x, y, z) \in \R^3$ в соответствующей ДПСК имеем  
	$$
		d(M, M_0) = \sqrt{(x - x_0)^2 + (y - y_0)^2 + (z - z_0)^2}.
	$$
	Поэтому:
	
	$B_r$~--- это открытый шар
	$$
		(x - x_0)^2 + (y - y_0)^2 + (z - z_0)^2 < r,
	$$
	$\overline{B_r}$~--- это замкнутый шар
	$$
	(x - x_0)^2 + (y - y_0)^2 + (z - z_0)^2 \leqslant r,
	$$	
	$S_r$~--- трехмерная сфера
	$$
	(x - x_0)^2 + (y - y_0)^2 + (z - z_0)^2 = r.
	$$
	\end{itemize}
	\begin{exc}
		Для $n = 1, 2, 3$  выяснить геометрический смысл $B_r(M_0), \overline{B_r}(M_0),  S_r(M_0)$ в пространствах $(\R^n, \rho_1)$ и $(\R^n, \rho_2)$  с соответствующими окторэдрической метрикой $\rho_1$ и кубической метрикой $\rho_2$. 
	\end{exc}
	\begin{rem}
		Кроме полных окрестностей $B_r(x_0), \overline{B_r}(x_0)$ в $(\R^n, \rho)$ будем рассматривать выколотые окрестности в $(\R^n, \rho)$ то есть множества
		$$
			\dot{B_r}(x_0) = B_r(x_0) \backslash \{x_0\},
		$$
		$$
			\dot{\bar{B_r}}(x_0) = \bar{B_r}(x_0) \backslash \{x_0\},
		$$
		Которые для краткости иногда будем обозначать $\dot{V}(x_0) $ и $\dot{\bar{V}}(x_0)$ соответственно. 
	\end{rem}
	\smallskip
	Точку $M_0 \in D,$ где $D \subseteq \R^n$ будем называть изолированной для множества $D$, если $\exists V(M_0) \subset \R^n$ в которой нет других точек из $D$, кроме $M_0$.
	
	Точку $M_0 \in \R^n$ будем называть граничной для $D \subseteq \R^n,$ если в любой окрестности $V(M_0) \subset \R^n$ есть точка из $D$, отличная от $M_0$. 
	
	Точку $M_0 \in D$ будем называть внутренней для $D \subseteq \R^n$, если $\exists V(M_0) \subset D$.
	
	Точку $M_0 \in D$ будем называть предельной для $D \subseteq \R^n$, если $\forall V(M_0) \subset \R^n$ есть точки как входящие, так и невходящие в $D$.
	
	Можно показать, что $M_0 \in \R^n$будет предельной для $D \subseteq \R^n$ тогда и только тогда, когда $M_0$ лтбо внутренняя для $D$, либо граничная.
	
	Очевидно, что любая изолированная точка не является предельной для $D$.
% \chapter{Тестовая}
% \subfile{tex/lec_test/test.tex}
% \subfile{tex/lec_test/template.tex}
\end{document}
