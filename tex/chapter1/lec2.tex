\makeatletter
\def\input@path{{../}}
\makeatother
\documentclass[../main.tex]{subfiles}


\begin{document}
	
	\section{Множества и окрестности в $\R^n$}
	Далее в метрическом пространстве $(\R^n, \rho)$ для записи его элементов будем использовать не только $a, b c, \dots x$, также $R, L, M, \cdots$, а также элементы будем называть точками.
	
	Открытым шаром $B_r(x_0)$ радиуса $r > 0$ с центром в $x_0 \in \R^n$ в пространстве ($\R^n, \rho$) будем называть множество 
	$$
	B_r(x_0) = \{x \in \R^n\ |\ \rho(x, x_0) < r\}, 
	$$
	А замкнутым шаром
	$$
	\overline{B_r}(x_0) = \{x \in \R^n\ |\ \rho(x, x_0) \leqslant r\}, 
	$$ 
	Очевидно, что $\overline{B_r}(x_0) = B_r(x_0) + S_r(x_0),$ где $ S_r(x_0)$~--- n-мерная сфера в  $(\R^n, \rho)$, то есть 
	$$
	S_r(x_0) = \{x \in \R^n\ |\ \rho(x, x_0) = r\}
	$$ 
	Шары $B_r, \overline{B_r}$ будем называть открытыми и замкнутыми окрестностями для $x_0 \in \R^n$. А в случае когда размеры не существенны или фиксированы будем их обозначать для краткости $B_r(x_0) = V(x_0)$ и $\overline{B_r}(x_0) = \overline{V}(x_0)$.
	
	\textbf{Примеры:}
	
	Для $n = 1 \Rightarrow d(x, x_0) = |x - x_0|, x, x_0 \in \R.$ В этом случае 
	$$
	B_r(x_0) = \{x \in \R\ |\ |x - x_0| < r\} =\ ] x_0 - r [ 
	$$
% \chapter{Тестовая}
% \subfile{tex/lec_test/test.tex}
% \subfile{tex/lec_test/template.tex}
\end{document}
