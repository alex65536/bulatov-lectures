\makeatletter
\def\input@path{{../../}}
\makeatother
\documentclass[../../main.tex]{subfiles}

\graphicspath{
	{../../img/}
	{../img/}
	{img/}
}

\begin{document}
\section{Поверхности в $\R^3$}

Пусть в некоторой области $D \subset \R^2$ определены функции 
$x = x(u,v),\ y =y(u,v),\ z = \\ = z(u,v),\ (u,v) \in D$. 
Когда $(u,v)$ пробегает $D$ в $\R^3$ 
будет определена некоторая поверхность $\Pi 
\iff$ задан вектор $\vec{r} = \vec{r}(u,v) = (x(u,v), y(u,v), z(u,v))$

\begin{center}
\includegraphics[scale = 0.5]{lec21_4.png}
\end{center}

Мы будем считать, что функции $x,\ y,\ z$ непрерывны, 
имеют непрерывные производные по $u$ и $v$ и что
\begin{equation}
\label{lec_21, num_2}
rank \dfrac{D(x,y,z)}{D(u,v)} = 
\begin{bmatrix}
x'_u & y'_u & z'_u \\
x'_v & y'_v & z'_v
\end{bmatrix} = 2
\end{equation}

Зафиксируем какое-то $u=u_0$. Мы получим вектор
$\vec{\rho_1} = \vec{r}(u_0,v)$. 
Он определяет кривую $l_1$, расположенную на поверхности $\Pi$.
Зафиксируем $v = v_0$. Получим $\vec{\rho_2} = \vec{r}(u,v_0)$. 
Это кривая $l_2$.

$l_1$ и $l_2$ называют координатными кривыми поверхности $\Pi$.

Вектор $(\vec{\rho_1})'_v$ --- касательный вектор к кривой $l_1$.
$(\vec{\rho_1})'_v = (x'_v,\ y'_v,\ z'_v) \bigg|_{(u_0,v_0)}$.
Вектор $(\vec{\rho_2})'_u = (x'_u,\ y'_u,\ z'_u) \bigg|_{(u_0,v_0)}$ --- 
касательный вектор кривой $l_2$. 

В точке $(u_0,\ v_0)$ определена пара касательных векторов.
Эти векторы линейно независимы в силу условия \eqref{lec_21, num_2}. 
Значит они определяют плоскость, проходящую через $(u_0,v_0)$.
Это касательная плоскость к $\Pi$. 

$\vec{r} = \vec{r}(u_0,v_0) + (\vec{\rho_1})'_v \ t +
(\vec{\rho_2})'_u \ s = 
\vec{r}(t,s)
$ --- уравнение касательной плоскости.

Вектор $\vec{N} = \pm \left[ r'_u, r'_v \right] = 
\pm \left[ (\rho_1)'_u, (\rho_2)'_v  \right]$ --- 
вектор нормали к плоскости 
(его называют также вектором нормали поверхности $\Pi$).
Он определяется с точностью до знака.

$
\vec{N} = 
\pm \begin{vmatrix}
\vec{i} & \vec{j} & \vec{k} \\
x'_u & y'_u & z'_u \\
x'_v & y'_v & z'_v
\end{vmatrix} = 
\pm (A,\ B,\ C)
$, где $A = \begin{vmatrix}
y'_u & z'_u \\
 y'_v & z'_v
\end{vmatrix},\ 
B = -\begin{vmatrix}
x'_u & z'_u \\
x'_v & z'_v
\end{vmatrix},\
C = \begin{vmatrix}
x'_u & y'_u \\
x'_v & y'_v
\end{vmatrix}
$.

В каждой точки поверхности $\Pi$ определён вектор нормали $\pm \vec{N}$.
Если выбран какой-то конкретный знак и в каждой точке будем вычислять 
$(A,\ B,\ C)$, то будет определён вектор $\vec{N} = (A,\ B,\ C), \
\vec{N} = (-A,\ -B,\ -C)$.
Соединим фиксированную точку поверхности с другой точкой.
Тогда выбрав в исходной точке знак $'+'$ или $'-'$ для вектора $\vec{N}$
мы определим  и одно значение вектора нормали других точек.

Рассмотрим на поверхности $\Pi$ замкнутую кривую, 
проходящую через точку $M_0$ и
считаем, что на $\Pi$ уже во всех точках заданы конкретные направления
вектора $\vec{N}$, определённые выбранным направлением точки $M_0$. 

Если при обходе кривой мы возвращаемся в точку $M_0$ с тем же
направлением вектора нормали, то поверхность называется двусторонней,
а если при таком обходе направление изменится на противоположное,
то поверхность называется односторонней





\end{document}