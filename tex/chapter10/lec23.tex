\makeatletter
\def\input@path{{../../}}
\makeatother
\documentclass[../../main.tex]{subfiles}

\graphicspath{
	{../../img/}
	{../img/}
	{img/}
}

\begin{document}
	Если $f$~--- непрерывная функция, а поверхность является гладкой, то такой 
	интеграл существует.

	Координаты центра тяжести материальной поверхности $\pi$ с плотностью $\rho =
	 \rho(x,y,z)$ могут быть найдены по формуле: 
	\begin{gather*}x_0 = \frac{\iint \limits_\pi \rho  x ds}{\iint \limits_\pi 
	\rho ds} =
	 \frac{M_{yz}}{m}, \\
	y_0 = \frac{\iint \limits_\pi \rho  y ds}{\iint \limits_\pi \rho ds} =
	 \frac{M_{xz}}{m}, \\
	z_0 = \frac{\iint \limits_\pi \rho  z ds}{\iint \limits_\pi \rho ds} =
	 \frac{M_{xy}}{m}.
	 \end{gather*}
	
	\subsection{Вычисление ПовИ-1}
	Пусть поверхность $\pi$ задана формулой \[\pi \ : \ \vec{r} =
	 \vec{r}(u, v) = (x(u,v), y(u,v), z(u, v)),\quad (u, v) \in D
	  \subset \R^2.\] $\pi$ является гладкой двусторонней
	   поверхностью.
	   
	Рассмотрим точку $M_k$:
	\[M_k(x_k, y_k, z_k) \leftrightarrow (u_k, v_k).\]
	Разбиение $\pi$ на части $\pi_k$ соответствует разбиению области $D$ на части 
	$D_k$, при 
	этом
	${M_k \in \pi_k} \iff (u_k, v_k) \in D_k$.
	 
	$\Delta s_k$ \--- площадь $\pi_k$. Тогда площадь $D_k$ равна $\Delta\sigma
	_k$. Интегральная сумма:
	\[\sum^{n}_{k=1} \ f(x_k, y_k, z_k)\Delta s_k = \sum^{n}_{k=1} \ f(x(u_k, 
	v_k),
	 y(u_k, v_k), z(u_k, v_k))\Delta s_k \]
	\[\Delta s_k = \iint \limits_{D_k}\sqrt{EG - F^2}dudv = *  \]
	Под корнем находится непрерывная от $u, v$ функция. Тогда найдется точка 
	$(\widetilde{u_k},
	 \widetilde{v_k}) \in D_k$ такая, что $* = \sqrt{EG -F^2} 
	 \big|_{(\widetilde{u}_k,
	  \widetilde{v}_k)} \cdot \Delta \sigma_k = (\sqrt{EG - F^2} \big|_{u_kv_k} 
	  +\alpha_k)
   \Delta \sigma_k$, где $\alpha_k \to 0$ при $\delta \to 0$ (это вытекает из 
   равномерной
    непрерывности). И тогда 
	\[ \sum_{k=1}^{n} f(x_k,y_k, z_k) \Delta s_k = \sum_{k=1}^{n} f(x(u_k, v_k),
	 y(u_k, v_k), z(u_k, v_k)) \sqrt{EG - F^2}\big|_{u_k v_k} \Delta \sigma_k
	  +\alpha,\]
	где $\alpha \appr{\delta \to 0} 0$. При $\delta \to 0$ получаем:
	\begin{equation}
	\label{lec23-1}
	\boxed{
	\iint \limits_{\pi} f(x,y,z)ds = \iint \limits_Df(x(u,v), y(u,v),z(u,v)) 
	\cdot \sqrt{EG - F^2}\, du\, dv.
	}
	\end{equation}
    В формуле слева находится ПовИ-1, а справа~--- двойной интеграл по плоской 
    области
     $D$.
    \begin{rem}
    Величину $ds = \sqrt{EG - F^2}\, du\, dv$ можно называть
     \emph{дифференциалом поверхности}. Ее можно заменить другим выражением:
    \[ds = \sqrt{EG - F^2}\,du\,dv = |[\vec r\,'_u, \vec r\,'_v]| 
    du\, dv =
     |\vec{N}| du\, dv = \sqrt{A^2 + B^2 + C^2}\, du\, dv. \]
    Пусть $\pi$ определена формулой $z= \phi(x,y)$, где $(x,y) \in D$ ($D$ \---
     замкнутая квадрируемая область). Тогда эту поверхность можно 
     параметризовать, положив
      \[\begin{cases} x=u, \\ y = v, \\ z= \phi(u,v),
        \end{cases}
        \ (u, v) \in D.\] В векторном виде получаем
      $\vec{r}(u,v) =
       (u, v, \phi(u,v))$. Имеем:
    \[\vec r\,'_u = (1, 0, \phi'_u),\] 
    \[\vec r\,'_v = (0, 1, \phi'_v),\]
	\[E =(\vec r\,'_u)^2 = 1 + (\phi'_u)^2,\]
	\[G =(\vec r\,'_v)^2 = 1 + (\phi'_v)^2,\]
	\[F = \left<\vec r\,'_u, \vec r\,'_v\right> = \phi'_u \cdot 
	\phi'_v,\]
	\[EG - F^2 = 1 + (\phi'_u)^2 + (\phi'_v)^2, \]
	\[\iint \limits_D f(x, y, z)ds = \iint \limits_D f(u,v, \phi(u,v)) \sqrt{1 +
		 (\phi'_u)^2 + (\phi'_v)^2} dudv = \] \[= \iint \limits_D f(x,y, \phi(x,y))
	  \sqrt{1 + (\phi'_x)^2 + (\phi'_y)^2}\, du\,dv. \]
	\end{rem}
	\begin{exmps}
	
	\,
	
	\begin{enumerate}
	\item
	\[ \iint \limits_\pi z^2 ds,\] где $\pi$ \--- часть конической 
	поверхности
	\[\pi: \vec{r} = (u\cos v\sin \alpha, u\sin v\sin\alpha, 
	u\cos\alpha) , \ 
	  \alpha\in \left[0, \frac{\pi}{2}\right], \ \alpha = const.\] 
	\[0 \leq v \leq 2\pi, \qquad 0 < u \leq a. \]
	\[\vec r\,'_u = (\cos v \sin \alpha, \sin v \sin \alpha, \cos \alpha), \]
	\[\vec r\,'_v = (-u \sin v \sin \alpha, u \cos v \sin \alpha, 0).  \]
	Тогда
	\begin{gather*}
	E = \left<\vec r\,'_u, \vec r\,'_u\right>= 1, \\ G = \left<\vec r\,'_v, \vec 
	r\,'_v\right> = u^2 \sin \alpha^2, \\
	F = \left<\vec r\,'_u, \vec r\,'_v\right> = -u \cos v \sin^2 \alpha \sin 
	v +
	 u \sin v \sin ^2 \alpha \cos v+0 =0, \\
	EG - F^2 = u^2 \sin^2 \alpha.
	\end{gather*}
	\begin{gather*}
	\iint \limits_\pi z^2 ds = \iint\limits_{0<u \leq \alpha} u^2 \cos^2 \alpha
	 \ u \sin \alpha \,du\, dv = \sin \alpha \cos ^2 \alpha \int \limits_0^\alpha 
	 du
	  \int \limits_0 ^{2\pi} u^3 dv = \\ = 2 \pi \sin \alpha \cos ^2 \alpha 
	  \frac{\alpha^4}{4} =
	   \frac{\pi \alpha^4}{2} \sin \alpha \cos^2 \alpha.
	\end{gather*}
	\item 
	$\displaystyle\iint \limits_\pi x ds$, где $\pi$ \--- часть плоскости
	$\pi : 2x+y+z-2=0$, расположенной в первом октанте.
	
	Уравнение плоскости в явном виде:
	\[z = 2 -2x -y.\]
	\begin{center}
	 \includegraphics[scale = 0.2]{lec23-2.jpg}
	\end{center}	
	При $z = 0$ вычислим пересечение $2 -x-y$ с осями $Ox$, $Oy$ :
	
	С осью $Ox$~--- $x = 0$, $y = 2$.
	
	С осью $Oy$~--- $y = 0$, $x = 1$.
	
	Получили, что поверхность задана на области $D$:
	
	\begin{center}
	 \includegraphics[scale = 0.2]{lec23-1.jpg}
	\end{center}

	Первая квадратичная форма равна $EG - F^2 = 1 + (z'_x)^2 + (z'_y)^2$.
	\[\iint \limits_\pi x ds = \iint \limits_D x \sqrt{1 + 4 +1} \  dx dy = 
	\sqrt{6}
	 \int \limits_0^1  dx \int \limits_0^{2 - 2x} x dy = \sqrt 6 \int \limits_0^1 
	 x(2 -2x) \ dx = \frac{\sqrt 6}{3}.\] 
	 \end{enumerate}
	\end{exmps}
	\subsection{Свойства ПовИ-1}
	\begin{enumerate}[label=\arabic*$^{\circ}$]
		\item 	ПовИ-1 не зависит от выбора стороны поверхности. В самом деле, 
		в интегральной сумме \[\sum\limits_{k} f(x_k, y_k, z_k) \Delta s_k\]
		 величина $\Delta s_k = \text{пл. } \pi_k$ не связана с выбором стороны 
		 поверхности.
		 
		\item
		 Для ПовИ-1 выполняются и другие свойства, присущие интегралам:
		 \begin{itemize}
		  \item линейность,
		  \item аддитивность,
		  \item монотонность.
		 \end{itemize}

		\item Основная оценка для ПовИ-1. Если \[m \leq f(x, y, z) \leq M, \quad (x, 
		y, z) \in \pi,\]
		  то \[mS \leq \iint \limits_\pi f(x, y, z) ds \leq MS,\] где $S$ \---
		   площадь $\pi$.
		\item \[\iint \limits_\pi ds  = S = \text{площадь }\pi.\]
		\item ПовИ-1 существует для тех функций $f$, для которых существует двойной
		 интеграл в формуле \eqref{lec23-1}. Для этого достаточно, чтобы функция была
		  кусочно-непрерывная и ограниченная на поверхности $\pi$.
		
		Вместо гладкой поверхности $\pi$ можно рассматривать кусочно-гладкие
		 поверхности, т.~е. такие, которые можно разбить на конечное множество
		  частей, каждая из которых \--- гладкая.
		\end{enumerate}
		
		\section{Поверхностный интеграл второго рода (ПовИ-2)}
		Пусть задана гладкая двусторонняя квадрируемая поверхность $\pi$. Выберем
		 какую-то сторону поверхности. Так как поверхность гладкая, в каждой точке
		  поверхности задан вектор нормали \[\vec{N} = \pm (A, B,C) = \pm
		   [\vec r\,'_u, \vec r\,'_v].\] Выбрать стороныу означает выбрать конкретный
		    знак вектора нормали (<<+>> или <<->>). 
		    Тогда во всех точках будет задан $\vec{N}$
		     единственным образом. Обозначим $\vec{n} =
		      \frac{\vec{N}}{|\vec{N}|}$, который имеет
		       единичную длину. Его координатами являются
		       направляющие косинусы
		\[\vec{n} = \frac{\pm(A,B,C)}{\sqrt{A^2 +B^2 + C^2}} = 
		(\cos \alpha, \cos \beta, \cos \gamma),\]
		где $\alpha$, $\beta$, $\gamma$~--- углы, которые $\vec n$ образует с осями 
		координат.
		\paragraph{Выбор стороны поверхности.}
		Если взять $\pm C$ с таким знаком, что $\pm C > 0$, то тогда $\cos \gamma
		 > 0$ и, значит, $\vec{n}$ и $\vec{N}$ образуют с осью
		  $Oz$ острый угол (можно говорить, что вектор нормали направлен вверх).
		   Это и дает определенную сторону поверхности.
		   
		Пусть задана поверхность $\pi$ \--- гладкая двусторонняя квадрируемая
	 поверхность, на которой определена функция $R(x,y,z)$. Разобьем поверхность 
	 $\pi$ на части	
	  $\pi_k$, при этом площадь $\pi_k$ равна $\Delta s_k$.
	  
	  Выберем некоторую сторону
	  поверхности $\pi$ (тогда говорят, что $\pi$ \emph{ориентирована}). Возьмем 
	  некоторую точку
      $M_k(x_k, y_k, z_k) \in \pi_k$. В этой точке есть вектор	
      $\vec{n} =
       (\cos \alpha, \cos \beta, \cos \gamma).$ Обозначим $\Delta \sigma_k = $
        площадь проекции части $\pi_k$ на $Oxy$. Введем в рассмотрение $(\pm
        \Delta\sigma_k),$ где стоит знак <<+>>, если $\cos \gamma > 0$ и 
        знак
         <<->>, если $\cos \gamma < 0$. Построим сумму 
         \[\sum\limits_{k=1}^n
         R(x_k, y_k, z_k)(\pm \Delta\sigma_k).\] Предел этой суммы обозначают
         \[\lim\limits_{\delta \to 0} \sum\limits_{k=1}^n R(x_k, y_k, z_k)(\pm
         \Delta\sigma_k) = \iint \limits_\pi R(x, y, z) dx dy\] и называют
         \emph{поверхностным интегралом второго рода} (ПовИ-2) по выбранной 
         стороне
         поверхности $\pi$. Аналогично определяют \[\iint \limits_\pi P(x, y, 
         z) 
         dydz\] и \[\iint \limits_\pi Q(x, y, z) dxdz .\]
         
         Обычно рассматривают
          \emph{ПовИ-2 общего вида}:
		\[\iint \limits_\pi P(x, y, z) dydz + Q(x, y, z) dxdz
		 + R(x, y, z) dxdy. \]
		 
		Сразу заметим, что ПовИ-2 зависит от выбора стороны поверхности (при выборе 
		другой
		 стороны знак поменяется на противоположный).
	\end{document}
