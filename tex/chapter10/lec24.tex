\makeatletter
\def\input@path{{../../}}
\makeatother
\documentclass[../../main.tex]{subfiles}

\graphicspath{
	{../../img/}
	{../img/}
	{img/}
}

\begin{document}
\section{Связь ПовИ-2 с ПовИ-2}

Рассмотрим ПовИ-2 вида $\iint\limits_\pi R(x, y, z)dxdy.$

На выбранной стороне поверхности $\pi$ проводится разбиение на части $\pi_k, k=\overline{1,n},$ где $\Delta s_k =\text{пл }\pi_k.$

Пусть $M_k(x_k, y_k, z_k)\in\pi_k$ --- некоторая точка на поверхности $\pi,$ которая проецируется на $Oxy,$ то есть $D_k = \text{пр}_{Oxy} \pi_k,$ где $\Delta\sigma_k = \text{пл }D_k.$

Определяем величину $(\pm\Delta\sigma_k):$ \textquotedblleft $+$\textquotedblright, если вектор нормали $\vec{n}$ с осью $Oz$ образует острый угол, и \textquotedblleft $-$\textquotedblright, если нет.

Составляем сумму $\sum\limits_k R(x_k, y_k, z_k)(\pm\Delta\sigma)$ и предел при $\delta\rightarrow 0$ называется ПовИ-2 от $R(x, y, z)$  по точке, получаем:

$$\iint\limits_\pi R(x, y, z)dxdy = \lim_{\delta\rightarrow 0}\sum\limits_k R(x_k, y_k, z_k)(\pm \Delta\sigma).$$

Так как $\vec{n}=(\cos\alpha_k, \cos\beta_k, \cos\gamma_k),$ где $\alpha, \beta, \gamma$ --- углы с положительными полуосями, имеем

$$(\pm\Delta\sigma) = \Delta s_k\cos\gamma_k.$$

Тогда 
$$\sum\limits_k R(x_k, y_k, z_k)(\pm \Delta\sigma) = \sum\limits_k  R(x_k, y_k, z_k) \cos\gamma_k \Delta s_k,$$
при переходе к пределу при $\delta\rightarrow 0$ получаем
$$\iint\limits_\pi R(x, y, z)dxdy = \iint\limits_\pi R(x, y, z) \cos\gamma ds,$$ где слева имеем ПовИ-2, справа --- ПовИ-1.

Аналогично получаем $$\iint\limits_\pi P(x, y, z)dydz = \iint\limits_\pi P(x, y, z) \cos\alpha ds, \iint\limits_\pi Q(x, y, z)dxdz = \iint\limits_\pi Q(x, y, z) \cos\beta ds.$$

Сложив, получаем для ПовИ-2 общего вида
\begin{equation}\label{lec24, num_1}\iint\limits_\pi Pdydz + Qdxdz + Rdxdy = \iint\limits_\pi(P\cos\alpha + Q\cos\beta + R\cos\alpha)ds\end{equation} --- формула, связывающая ПовИ-2 с ПовИ-1.

\section{Вычисление ПовИ-2}

Пусть поверхность $\pi$ определяется формулами $\vec{r} = \vec{r}(u,v) = (x(u,v), y(u,v), z(u,v)), (u, v)\in D\subset \R^2$ -- замкнутая область.

Как известно, $ds = \left|\left[ \vec r\,'_u,  \vec r\,'_v\right]\right|dudv = \sqrt{EG - F^2}dudv = \sqrt{A^2 + B^2 + C^2}dudv.$ В соответствии с этим вектор $\vec{n} = (\cos\alpha, \cos\beta, \cos\gamma) = \pm\frac{(A, B, C)}{\sqrt{A^2 + B^2 + C^2}}.$ Поэтому $$\cos\alpha = \frac{\pm A}{\sqrt{A^2 + B^2 + C^2}},\quad \cos\beta = \frac{\pm B}{\sqrt{A^2 + B^2 + C^2}},\quad \cos\gamma = \frac{\pm C}{\sqrt{A^2 + B^2 + C^2}}.$$

Используя выражения для косинусов и для $ds$ в \eqref{lec24, num_1}, получаем
\begin{equation}\label{lec24, num_2}
\iint\limits_\pi Pdydz + Qdxdz + Rdxdy = \pm \iint\limits_D(PA + QB + RC)dudv
\end{equation}
--- формула для вычисления ПовИ-2.

\begin{example}
	Вычислить интеграл $I = \iint\limits_\pi xdydz + ydxdz + zdxdy,$ где $\pi: x^2 + y^2 + z^2 = a^2$ --- внешняя сторона сферы.
	\smallskip
	
	\emph{1 способ.}
		
	Сферу можно задать параметрически:
	$\begin{cases}	x = a \cos u \cos v,\\
		y = a \cos u \sin v,\\
		z = a \sin u,
	\end{cases} $
	где $u \in [-\frac{\pi}{2}, \frac{\pi}{2}],$ $v \in [0, 2\pi).$
	
	То есть $\pi: \vec{r} = ( a \cos u \cos v,  a \cos u \sin v, a \sin u),$ тогда $$\vec r\,'_u = (-a \sin u \cos v, -a \sin u \sin v, a \cos u),\quad \vec r\,'_v = (-a \cos u \sin v, a \cos u \cos v, 0).$$
	
	Воспользуемся формулой \eqref{lec24, num_2}. Найдем $$A=\begin{vmatrix}
	y'_u & z'_u\\
	y'_v & z'_v
	\end{vmatrix} = \begin{vmatrix}
	-a \sin u \sin v & a \cos u\\
	 a \cos u \cos v & 0
	\end{vmatrix} = -a^2 \cos^2 u \cos v,$$ $$B=\begin{vmatrix}
	z'_u & x'_u\\
	z'_v & x'_v
	\end{vmatrix} = \begin{vmatrix}
	a \cos u & -a \sin u \cos v\\
	0 & -a \cos u \sin v
	\end{vmatrix} = -a^2 \cos^2 u \sin v,$$
	$$C=\begin{vmatrix}
	x'_u & y'_u\\
	x'_v & y'_v
	\end{vmatrix} = \begin{vmatrix}
	-a \sin u \cos v & -a \sin u \sin v\\
	-a \cos u \sin v & a \cos u \cos v
	\end{vmatrix} = -a^2 \sin u \cos u(**).$$
	
	Тогда $I = \int\limits_{-\frac{\pi}{2}}^{\frac{\pi}{2}}du\int\limits_0^{2\pi}(-a^3\cos^3 u \cos^2 v - a^3\cos^3 u \sin^2 v - a^3\sin^2 u \cos u)dv = *$
	
	Определим знак. Возьмем некоторую конкретную точку на сфере, в которой известно, куда направлен вектор $\vec{n}: u=\frac{\pi}{4}, v=\frac{\pi}{4}\Rightarrow \cos\gamma > 0$ (угол с $Oz$ острый), но $\cos\gamma =\\= \frac{\pm C}{\sqrt{A^2 + B^2 + C^2}}\Rightarrow \pm C>0,$ а $C\left(\frac{\pi}{4}, \frac{\pi}{4}\right)=\left[**\right]=-\frac{a^2}{2}<0\Rightarrow$ берем знак минус. Значит,
	\begin{multline*}*=-\int\limits_{-\frac{\pi}{2}}^{\frac{\pi}{2}}du\int\limits_0^{2\pi}(-a^3\cos^3 u \cos^2 v - a^3\cos^3 u \sin^2 v - a^3\sin^2 u \cos u)dv =\\= a^3 \int\limits_{-\frac{\pi}{2}}^{\frac{\pi}{2}}du\int\limits_0^{2\pi}(\cos^3 u + \sin^2 u \cos u)dv = a^3 \int\limits_{-\frac{\pi}{2}}^{\frac{\pi}{2}}du\int\limits_0^{2\pi}(\cos^3 u + (1-\cos^2 u) u \cos u)dv = \\ = a^3 \int\limits_{-\frac{\pi}{2}}^{\frac{\pi}{2}}du\int\limits_0^{2\pi}\cos u dv = 4\pi a^3.\end{multline*}
	
	\emph{2 способ.}
		
	Пусть $M(x, y, z)$ --- произвольная точка сферы, вектор нормали направлен по радиусу.
	
	\begin{center}
		\includegraphics[scale = 0.27]{lec24_1.jpg}
	\end{center}
	
	Этот вектор составляет следующие косинусы с соответствующими осями:
	
	$$\cos\alpha=\frac{x}{a},\quad \cos\beta=\frac{y}{a},\quad \cos\gamma=\frac{z}{a}.$$ 
	
	 Тогда по формуле \eqref{lec24, num_1}(переходим к ПовИ-1):
	$$I=\iint\limits_\pi\left(\frac{x^2}{a} + \frac{y^2}{a} + \frac{z^2}{a}\right)ds = a \iint\limits_\pi ds = 4\pi a^3.$$
\end{example}

\section{Связь ПовИ-2 с 3И}

Рассмотрим некоторое тело $V\subset \R^3,$ ограниченное поверхностью $\pi.$ Как и ранее, $\pi$ --- гладкая квадрируемая двусторонняя поверхность. Под $\pi$ понимают внешнюю сторону этой поверхности. Тогда 

\begin{equation}\label{lec24, num_3}
\iint\limits_\pi Pdydz +  Qdxdz + Rdxdy = \iiint\limits_V \left(\pderiv {P}{x}  + \pderiv {Q}{y} + \pderiv{R}{z}\right) dxdydz\end{equation} --- формула Гаусса-Остроградского, где слева --- ПовИ-2, справа --- 3И по области $V,$ для которой $\pi$ --- граница.

\begin{proof}
Вычислим
$$\iiint\limits_V \pderiv {R(x, y, z)}{z} dxdydz = *$$

Будем предполагать, что $V$ --- такое тело, что любая прямая параллельная $Oz$ пересекает поверхность $\pi$ не более, чем в двух точках.

Пусть $D$ --- проекция $V$ на плоскость $Oxy.$ На $D$ возьмем произвольную точку $(x, y)$ и восстановим перпендикуляр, и так для каждой точки из $D.$ Тогда поверхность $\pi$ разобьется на две части: одна будет содержать точку $z_{\text{н}},$ вторая --- $z_{\text{в}}.$

\begin{center}
	\includegraphics[scale = 0.3]{lec24_2.jpg}
\end{center}

Получим части $\pi_{\text{н}}$ и $\pi_{\text{в}}.$

Тогда \begin{multline*}*=\iint\limits_D dxdy\int\limits_{z_{\text{н}}(x,y)}^{z_{\text{в}}(x, y)}\pderiv{R(x, y, z)}{z}dz = \iint\limits_D dxdy R(x, y, z)\bigg|_{z=z_{\text{н}}(x, y)}^{z=z_{\text{в}}(x, y)} = \\= \iint\limits_{\pi_{\text{в}}^{\uparrow}} R(x, y, z)dxdy - \iint\limits_{\pi_{\text{н}}^{\uparrow}} R(x, y, z)dxdy =
\iint\limits_{\pi_{\text{в}}^{\uparrow}} R(x, y, z)dxdy + \iint\limits_{\pi_{\text{н}}^{\downarrow}} R(x, y, z)dxdy = *\end{multline*}

Второй интеграл --- интеграл по верхней стороне поверхности $\pi_{\text{н}},$ для которой нормаль направлена внутрь тела $V.$ Поменяв знак перед интегралом, получим интеграл по внешней стороне поверхности $\pi,$ соответствующий $\pi_{\text{н}}.$

$$* = \iint\limits_\pi R(x, y, z)dxdy.$$

Аналогично доказывается, что $$\iiint\limits_V \pderiv {P}{x} dxdydz = \iint Pdydz\quad \text{и}\quad \iiint\limits_V \pderiv {Q}{y} dxdydz = \iint Pdxdz.$$
В совокупности это и дает формулу Гаусса-Островского.
\end{proof}

\begin{example}
	Вычислить $$I = \iint\limits_\pi x^3dydz + y^3dxdz + z^3dxdy,$$ где $\pi$ --- внешняя сторона сферы $x^2 + y^2 + z^2 = a^2.$
	Воспользуемся формулой \eqref{lec24, num_3}:
	
	$I=\iint\limits_V (3x^2 + 3y^2 + 3z^2)dxdydz = \left[ \begin{cases}x= r\cos\phi \cos\psi,\\ y=r\sin\phi\cos\psi, \\ z=r\sin\psi,\end{cases} J =r^2\cos\psi,\\ \phi\in[0, 2\pi], \psi\in\left[-\frac{\pi}{2},\frac{\pi}{2}\right]\right]=\\=3\int\limits_0^{2\pi}d\phi\int\limits_{-\frac{\pi}{2}}^{\frac{\pi}{2}}d\psi\int\limits_0^a r^4\cos\psi d\psi=\frac{12}{5}a^5\pi.$
\end{example}

\end{document}