\makeatletter
\def\input@path{{../}}
\makeatother
\documentclass[../main.tex]{subfiles}

\begin{document}

{\Huge \bf Составители:}

\vspace{2em}

{
    \textbf{Лекторы:} Булатов В. И., Кастрица О. А.
    
    \smallskip
    
    \textbf{Руководитель проекта:} Керножицкий Александр 
    (\texttt{alex65536}).
    
    \smallskip
    
    \textbf{Компьютерная верстка:}
    \begin{itemize}
     \item Глава $1$~--- Керножицкий Александр, Соловей Михаил.
     \item Глава $2$~--- Соловей Михаил, Филипович Федор, Метлюк 
     Павел.
     \item Глава $3$~--- Метлюк Павел, Титенок Станислав, Шляго 
     Никита, Бакевич Алексей.
     \item Глава $4$~--- Бакевич Алексей, Коршакевич Дмитрий.
     \item Глава $5$~--- Коршакевич Дмитрий, Модзолевский 
     Валентин, Сверчков Алексей.
     \item Глава $6$~--- Сверчков Алексей, Ходор Иван.
     \item Глава $7$~--- Кохнович Алексей, Быченок Егор, Грибчук Даниил,
     Гаркавый Артем.
     \item Глава $8$~--- Гаркавый Артем, Диброва Екатерина, Сорока Егор.
     \item Глава $9$~--- Сорока Егор, Румак Данила, Сечко Никита, Тарайкович 
     Алеся, Боженков Никита.
     \item Глава $10$~--- Боженков Никита, Гармаза Александра, Мороз
     Екатерина, Сандрыгайло Янина, Румак Данила.
     \item Глава $11$~--- Кордияко Ян, Румак Данила, Грибчук Даниил, Шляго
     Никита, Басанец Михаил, Шакель Андрей.
    \end{itemize}
    
    \textbf{Исправление ошибок}: Керножицкий Александр, Неверо 
    Андрей, Шляго Никита.
    
    \medskip
    
    По мере добавления в конспект новых глав список людей, которые 
    внесли вклад в его разработку, будет пополняться.
    
    \vspace{1.5em}
    
    Используются материалы электронного конспекта, созданного в 
    2015 году Ильиным А. В., Линовым К. А., Павловичем В. В. и 
    Швец М. М.

    \smallskip
    
    Также выражаем благодарность Булатову Владимиру Ивановичу за 
    предложенные исправления и замечания.
    
    \smallskip
    
    Еще хочется поблагодарить создателей \LaTeX\ за замечательную 
    систему создания документов: без них этого электронного 
    конспекта не было бы :)
}

\pagebreak

\end{document}
