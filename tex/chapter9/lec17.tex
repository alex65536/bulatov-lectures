\makeatletter
\def\input@path{{../../}}
\makeatother
\documentclass[../../main.tex]{subfiles}

\graphicspath{
	{../../img/}
	{../img/}
	{img/}
}

\begin{document}
\section{Гладкие и кусочно-гладкие кривые в $\R^n$}

\emph{Линией (кривой)} $l \subset \R^n $ будем 
называеть произвольное отображение
\begin{equation}
  \label{lec17_2:1}
  f: [\alpha, \beta] \to \R^n,
\end{equation}
ставящее в соответствие для $\forall t \in [\alpha, \beta]$ 
единственную точку (образ) $M = f(t) \in \R^n$,
где $f(t) = (f_1(t), f_2(t), \ldots, f_n(t)) \in \R^n$ и
$\forall f_k(t)$~--- функция одной переменной $t \in 
[\alpha, \beta], k = \overline{1,n}$.

Используя координатную запись
$M = (x_1, x_2, \ldots, x_n) \in \R^n$, из \eqref{lec17_2:1} 
получаем \emph{параметрическое представление} кривой $l \subset \R^n:$
\begin{equation}
 \label{lec17_2:2}
 l: 
  \begin{cases}
    x_k = f_k(t),\ k = \overline{1,n},\\
    t \in \left[\alpha, \beta\right].
  \end{cases}
\end{equation}

В общем случае одна и та же кривая $l \subset \R^n$ 
как образ отображения \eqref{lec17_2:1} может иметь в $\R^n$ разные
параметрические представления \eqref{lec17_2:2}, при каждом из которых 
геометрически 
будут получаться одни и те же множества точек в $\R^n$.\\
Например, в $\R^2$ в ПДСК  $Oxy$ на единичной окружности мы можем 
использовать как параметрическое преставление
\[
  \begin{cases}
    x = \cos(t), \\
    y = \sin(t),
  \end{cases}
  t \in \left[0, 2\pi \right],
\] 
так и другое параметрическое представление
\[
  \begin{cases}
    x = \sin(t), \\
    y = \cos(t),
  \end{cases}
  t \in \left[0, 2\pi \right].
\] 
В результате, при каждом из этих отображений, 
получится один и тот же образ в $\R^2$.

В общем случае для \eqref{lec17_2:1}, \eqref{lec17_2:2},
точку $M_\alpha = f(\alpha) \in l$ будем считать \emph{начальной точкой},
а точку $M_\beta = f(\beta) \in l$ будем называть \emph{концевой точкой} для кривой 
$l$. В этом случае, при необходимости саму кривую $l$ будем записывать в виде
$l = \overbow{M_\alpha,\ M_\beta}$.

В дальнейшем будем также рассматривать \emph{ориентированные кривые} $l$, т.~е.
те, у которых задано направление движения.

Например:
  \begin{center}
    \begin{minipage}{.5\textwidth}
	  \usetikzlibrary{arrows.meta}
      \centering
        \begin{tikzpicture}				
           \draw (0, 0) arc(0:90:2);
           \draw[-{Latex[length=2.5mm]}] (0,0) arc(0:45:2);
				
           \fill [black] (0, 0) circle (2pt);
           \fill [black] (-2, 2) circle (2pt);
				
           \draw (-2,2.3) node[anchor=center] {$M_\beta$};
           \draw (0.5,0) node[anchor=center] {$M_\alpha$};
           \end{tikzpicture}
		\end{minipage}
     \end{center}
Ориентацию, соответствующую возрастанию параметра $t$, будем считать
положительной и записыввать в виде 
$l^+= \overbowright{M_\alpha,\ M_\beta}$.
Для этой же кривой с противоположенной ориентацией будем использовать запись
$l^-= \overbowleft{M_\alpha,\ M_\beta} =
\overbowright{M_\beta,\ M_\alpha}$.

Кривая $l \subset \R^n$ без самопересечений называется 
\emph{простой} кривой. 

Если у такой простой кривой начало
совпадает с концом, то получим \emph{замкнутую} кривую. В дальнейшем ориентированную соотвествующим образом простую замкнутую кривую
в $\R^n$ будем называть \emph{контуром}.

Если у кривой $l \subset \R^n$  существует параметрическое представление 
\eqref{lec17_2:2}, в котором используемые функции непрерывно дифференцируемы на
$\left[\alpha, \beta \right]$, причем в точках $\alpha$ и $\beta$
подразумевается соответствующая односторонняя дифференцируемость,
то $l$ называют \emph{гладкой} кривой.

Для гладкой кривой  $l \subset \R^n$ точка 
$M_0 = f(t_0),\ t_0 \in \left[\alpha, \beta \right]$
называется \emph{особой}, если $f'_k(t_0) = 0,\ \forall\,k = \overline{1,n}$.

Кривая $l \subset \R^n$ c представлением \eqref{lec17_2:2} называется 
\emph{кусочно-гладкой}, если 
\[
  \exists P = \left\{ t_k \right\},\ k = \overline{1,m},\    
  \alpha = t_0  < t_1 < t_2 < \dots < t_{m-1} < t_m = \beta
  ,
\]
такое разбиение
отрезка $\left[\alpha, \beta \right]$,
что, во-первых, любая $f_j(t),\ j = \overline{1, n}$ непрерывно дифференцируема 
на $t \in ]t_{k-1},\ t_k[,\ k = \overline{1, m}$,
причем либая из точек разбиения $t_j,\ j = \overline{1, m}$ не является
особой для $l \subset \R^n$, т.~е.
\[
  \sum\limits_{j = 1}^m(f'_j(t))^2 > 0,
\]
и, во-вторых, препологается, что на концах каждого из интервалов разбиения,
т.~е. в каждой из точек 
$t_j \in \left[ \alpha, \beta \right],\ j = \overline{1,m}$,
у любой $f_k(t)$ существуют конечные односторонние производные.

По той же схеме, как для плоских в $\R^2$ и пространственных в  $\R^3$ 
линий определяют спрямляемые кривые $l \subset \R^n$.
Если $l \subset \R^n$ спрямляема, то её длина, 
в соответствии c \eqref{lec17_2:2}
вычисляется по формуле
\begin{equation}
 \label{lec17_2:3}
 L = \text{длина}\ l = \int\limits_\alpha^\beta|f'(t)|dt =  
 \int\limits_\alpha^\beta\sqrt{(f'_1(t))^2 + (f'_2(t))^2 
 + \dots (f'_n(t))^2}dt
\end{equation}

Если рассматреть на отрезке $\alpha(t),\ t \in \left[\alpha, \beta\right]$,
длину $S$ соотвествующей части кривой $l \subset \R^n$, 
то в силу \eqref{lec17_2:3} имеем
\begin{equation}
\label{lec17_2:4}
 S = S(t) = \int\limits_\alpha^t(|f'(\tau)|)d\tau
\end{equation}
Отсюда, в случем непрервыности используемых в \eqref{lec17_2:2} 
функций  по т. Барроу следует:
\[
  \exists S'(t) = |f'(t)|>0
\]
для не особых точек $t \in \left[\alpha, \beta \right]$.

Поэтому получаемая функция $S = S(t)$ монотонно возрастает 
на $\left[\alpha, \beta \right]$ от начального значения  $S(\alpha) = 0$ 
до конечного значения $S(\beta) = L$.
$\forall s \in \left[0, L \right]$ в случае, когда $S'(t) > 0$  уравнение
$S(t) = S$ имеет единственное решение 
$t = t(S) \in \left[\alpha, \beta \right]$
Используюя это решение в \eqref{lec17_2:2} приходим к новой параметризации
рассматриваемой кривой
\begin{equation}
 \label{lec17_2:5}
 l:
 \left\{
   \begin{array}{cc}
      x_k = f_k(t(S)),\ k = \overline{1,n}\\
      S \in \left[0, L \right]
   \end{array}
 \right.
\end{equation}
В которой роль параметра играет длина $S$ в рассматриваемой части дуги $L$.
Параметр $S$ в \eqref{lec17_2:5} называется \emph{натуральным} параметром, 
а параметрическое представление \eqref{lec17_2:5} чрез натуральный параметр S
называется натуральным представлением кривой $l$ на $\left[ 0, L \right]$.

\end{document}
