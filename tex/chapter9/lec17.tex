\makeatletter
\def\input@path{{../../}}
\makeatother
\documentclass[../../main.tex]{subfiles}

\graphicspath{
	{../../img/}
	{../img/}
	{img/}
}

\begin{document}
\section{Гладике и кусочно-гладкие кривые в $\R^n$}

\emph{Линей(кривой)} $l \subset \R^n $ будем 
называеть произвольное отображение
\begin{equation}
  \label{lec17_2:1}
  f: [\alpha, \beta] \to \R^n,
\end{equation}
ставящее в соответствие для $\forall t \in [\alpha, \beta]$ 
единственную точку(образ) $M = f(t) \in \R^n$,
где $f(t) = (f_1(t), f_2(t), \ldots, f_n(t))$ и
$\forall f_k(t)$ ~--- Ф1П, $t \in [\alpha, \beta], k = \overline{1,n}$.\\
Испульзую координатную запись
$M = (x_1, x_2, \ldots, x_n) \in \R^n$ из \eqref{lec17_2:1} 
получаем параметрическое представление кривой $l \subset \R^n$
\begin{equation}
 \label{lec17_2:2}
 l: 
 \left\{
  \begin{array}{cc}
    x_k = f_k(t), k = \overline{1,n},\\
    t \in \left[\alpha, \beta\right].
  \end{array}
 \right.
\end{equation}

В общем случае одна и та же кривая $l \subset \R^n$, 
как образ отображения \eqref{lec17_2:1}, может иметь в $\R^n$ разные
параметрические представления \eqref{lec17_2:2} при каждом из которых 
геометрически 
будут получаться одни и те же множества точек в $\R^n$.\\
Например, в $R^2$ в ПДСК  $Oxy$ на единичной окружности мы можем 
использовать как параметрическое преставление
\[
\left\{
  \begin{array}{cc}
    x = \cos(t) \\
    y = \sin(t)
  \end{array}
  \right.
  t \in \left[0, 2\pi \right]
\] 
так и параметрическое представление:
\[
\left\{
  \begin{array}{cc}
    x = \sin(t) \\
    y = \cos(t)
  \end{array}
  \right.
  t \in \left[0, 2\pi \right]
\] 
в результате, при каждом из этих отображений, 
получится один и тот же образ в $\R^2$.\\
В общем случае для \eqref{lec17_2:1}, \eqref{lec17_2:2}
точку $M_\alpha = f(\alpha) \in l$ будем считать начальной точкой,
а точку $M_\beta = f(\beta) \in l$ будем называть концевой точкой для кривой 
$l$ и в этом случае, при необходимости саму кривую $l$ будем записывать в виде
$l = \buildrel\,\,\frown\over{M_\alpha,\ M_\beta}$.

В дальнейшем будем также рассматривать ориентированные кривые $l$, т.~е.
те, у которых задано направление движения.

Например:
  \begin{center}
    \begin{minipage}{.5\textwidth}
      \centering
        \begin{tikzpicture}				
           \draw (0, 0) arc(0:90:2);
           \draw[-<] (0,0) arc(0:45:2);
				
           \fill [black] (0, 0) circle (2pt);
           \fill [black] (-2, 2) circle (2pt);
				
           \draw (-2,2.3) node[anchor=center] {$M_\alpha$};
           \draw (0.5,0) node[anchor=center] {$M_\beta$};
           \end{tikzpicture}
		\end{minipage}
     \end{center}
Ориентацию, соответствующую возростанию параметра $t$ будем считать
положительной и записыввать в виде 
$l^+= \buildrel\,\,\frown\over{M_\alpha,\ M_\beta}$
(на дуге стрелка слева направо).\\
Для этой же кривой с противоположенной ориентацией будем использовать запись
$l^-= \buildrel\,\,\frown\over{M_\alpha,\ M_\beta} \text{(справа налево)} =
\buildrel\,\,\frown\over{M_\beta,\ M_\alpha} \text{(слева направо)}$.

Кривая $l \subset \R^n$ без \emph{самопересечений} называется 
\emph{простой} кривой.

Если у такой простой кривой начало
совпрадает с концом, то получим \emph{замкнутую} кривую.

В дальнейшем ориентированную соотвествующим образом простую замкнутую кривую
в $\R^n$ будем называть \emph{контуром}.

Если у кривой $l \subset \R^n$  существует параметрическое представление 
\eqref{lec17_2:2}, в котором используемые функции непрерывно дифференцируемы на
$\left[\alpha,\ \beta \right]$, причем в точках $\alpha$ и $\beta$
подразумевается соотвествующая односторонняя дифференцируемость,
то $l$ назывют \emph{гладкой} кривой.

Для гладкой кривой  $l \subset \R^n$ точка 
$M_0 = f(t_0),\ t_0 \in \left[\alpha,\ \beta \right]$
называется \emph{особой}, если $\forall f'_k(t_0)) = 0, k = \overline{1,n}$.

Кривая $l \subset \R^n$ c представлением \eqref{lec17_2:2} называется 
\emph{кусочно гладкой}, елси 
\[
  \exists P = \left\{ t_k \right\} , k = \overline{1,n}
\]
такое разбиение
отрезка $\left[\alpha, \beta \right]$,
\[
  \alpha = t_0  < t_1 < t_2 < \dots < t_{n-1} < t_n = \beta
\]
что во-первых: любое $f_j(t)$ непрерывно дифференцируемо 
для $t \in \left]t_k,\ t_{k-} \right[, k = \overline{1, m}$,
причём либая из точек разбиения $t_j, j = \overline{1, m}$ не является
особой для $l \subset \R^n$ т.~е.
\[
  \sum\limits_{j = 1}^m(f'_j(t))^2 > 0
\]
И во-вторых препологается, что на концах кадого из интервалов разбиения
т.~е. в каждой из точек 
$t_j \in \left[ \alpha, \beta \right], j = \overline{1,m}$
у любой $f_k(t)$ существуют конечные односторонние производные.

По той же схеме, как для плоских в $\R^2$ и пространственных в  $\R^3$ 
линий определяют спрямляемые кривые $l \subset \R^n$.
Если $l \subset \R^n$ спрямляема, то её длина, 
в соответствии c \eqref{lec17_2:2}
вычисляется по формуле
\begin{equation}
 \label{lec17_2:3}
 L = \text{длина}\ l = \int\limits_\alpha^\beta|f'(t)|dt =  
 \int\limits_\alpha^\beta\sqrt{(f'_1(t))^2 + (f'_2(t))^2 
 + \dots (f'_n(t))^2}dt
\end{equation}

Если рассматреть на отрезке $\alpha(t),\ t \in \left[\alpha, \beta\right]$,
длину $S$ соотвествующей части кривой $l \subset \R^n$, 
то в силу \eqref{lec17_2:3} имеем
\begin{equation}
\label{lec17_2:4}
 S = S(t) = \int\limits_\alpha^t(|f'(\tau)|)d\tau
\end{equation}
Отсюда, в случем непрервыности используемых в \eqref{lec17_2:2} 
функций  по т. Барроу следует:
\[
  \exists S'(t) = |f'(t)|>0
\]
для не особых точек $t \in \left[\alpha, \beta \right]$.

Поэтому получаемая функция $S = S(t)$ монотонно возрастает 
на $\left[\alpha, \beta \right]$ от начального значения  $S(\alpha) = 0$ 
до конечного значения $S(\beta) = L$.
$\forall s \in \left[0, L \right]$ в случае, когда $S'(t) > 0$  уравнение
$S(t) = S$ имеет единственное решение 
$t = t(S) \in \left[\alpha, \beta \right]$
Используюя это решение в \eqref{lec17_2:2} приходим к новой параметризации
рассматриваемой кривой
\begin{equation}
 \label{lec17_2:5}
 l:
 \left\{
   \begin{array}{cc}
      x_k = f_k(t(S)),\ k = \overline{1,n}\\
      S \in \left[0, L \right]
   \end{array}
 \right.
\end{equation}
В которой роль параметра играет длина $S$ в рассматриваемой части дуги $L$.
Параметр $S$ в \eqref{lec17_2:5} называется \emph{натуральным} параметром, 
а параметрическое представление \eqref{lec17_2:5} чрез натуральный параметр S
называется натуральным представлением кривой $l$ на $\left[ 0, L \right]$.






\end{document}
