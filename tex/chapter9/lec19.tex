\makeatletter
\def\input@path{{../../}}
\makeatother
\documentclass[../../main.tex]{subfiles}

\graphicspath{
	{../../img/}
	{../img/}
	{img/}
}

\begin{document}
\section{КрИ-1 в $\R^3$}

\par Если $\overbow{AB}$~--- пространственная кривая, то по той же схеме,
что и в $\R^2$, вводится КрИ-1 по $\overbow{AB}$:
\[
\int\limits_{\overbow{AB}} f(x,\, y,\, z)\, ds.
\]
В этом случае на $\overbow{AB}$ определена функция от 3-х переменных.

\par Если $\overbow{AB}$ задана в естественной параметризации, то:

\begin{equation}
\label{lec_19, num_1}
\int\limits_{\overbow{AB}} f(x,\, y,\, z)\, ds = \left[
\overbow{AB}: 
\begin{cases}
x = x \left( s \right)\\
y = y \left( s \right)\\
z = z \left( s \right)\\
a \leq s \leq b 
\end{cases} \right] = 
\int\limits_{a}^{b} f(x \left( s \right),\, y \left( s \right),\, z \left( s \right))\, ds.
\end{equation}

\par Если $\overbow{AB}$ задана в произвольной параметризации то:
\[
\int\limits_{\overbow{AB}} f(x,\, y,\, z)\, ds = \left[
\overbow{AB}: 
\begin{cases}
x = x \left( t \right)\\
y = y \left( t \right)\\
z = z \left( t \right)\\
\alpha \leq t \leq \beta 
\end{cases} \right] =
\]

\begin{equation}
\label{lec_19, num_2}
= \int\limits_{\alpha}^{\beta} 
f(x \left( t \right),\, y \left( t \right),\, z \left( t \right))
\sqrt{ \left( x' \left( t \right) \right)^2 + \left( y' \left( t \right) \right)^2 + \left( z' \left( t \right) \right)^2}\, dt.
\end{equation}

\par По аналогичной схеме приводят аналогичные формулы.

\begin{example}
Вычислим $\int\limits_{\overbow{AB}} x^2 + y^2\, ds,\,  \overbow{AB}:
\begin{cases}
x = a \cos{t}\\
y = a \sin{t}\\
z = b t\\
0 \leq t \leq 2\pi. 
\end{cases}
$

\par 1-й способ. По формуле \eqref{lec_19, num_2} имеем:
\[
I = \int\limits_{0}^{2\pi} \left(a^2 \cos^2{t} + a^2 \sin^2{t}\right)
 \sqrt{a^2 \sin^2{t} + a^2 \cos^2{t} + b^2}\, dt =
\]
\[ = \int\limits_{0}^{2\pi} a^2
 \sqrt{a^2 + b^2}\, dt = 2\pi a^2 \sqrt{a^2 + b^2}. 
\]

\includegraphics[scale = 0.5]{lec19_1.png}

\par 2-й способ. Попробуем задать в естественной параметризации:
\[s = \text{дл. }\overbow{A(0), A(t)} = 
\int\limits_{0}^{t} \sqrt{\left( x' \left( t \right) \right)^2 + \left( y' \left( t \right) \right)^2 + \left( z' \left( t \right) \right)^2}\, dt = 
t \sqrt{a^2 + b^2} \implies
\]
\[ \implies t = \frac{s}{\sqrt{a^2 + b^2}} \implies 
\overbow{AB}:
\begin{cases}
x = a \cos{\frac{s}{\sqrt{a^2 + b^2}}}\\
y = a \sin{\frac{s}{\sqrt{a^2 + b^2}}}\\
z = b \frac{s}{\sqrt{a^2 + b^2}}.\\
\end{cases}
\]

\par При этом, если $t = 2 \pi$, то получим точку $B$ и $s$ будет равно $2 \pi \sqrt{a^2 + b^2}$.
Тогда воспользовавшись формулой \eqref{lec_19, num_1} имеем:
\[
I = \int\limits_{0}^{2 \pi \sqrt{a^2 + b^2}} a^2\, ds = 2\pi a^2 \sqrt{a^2 + b^2}.
\]   
\end{example}

\section{КрИ-2 в $\R^2$}

\par Пусть $\overbow{AB}$ гладкая кривая в $\R^2$ на которой задано направление(ориентированная кривая). 
Пусть на $\overbow{AB}$ заданы функции $P \left( x,\, y \right),\, Q \left( x,\, y \right)$.
Разобьём $\overbow{AB}$ точками $A_k \left( x_k,\, y_k \right),\, k = \overline{0,n}$.
Тогда $\Delta x_k = x_k - x_{k - 1},\, \Delta y_k = y_k - y_{k - 1}$.
На каждой дуге $\overbow{A_{k - 1} A_{k}}$ возьмём произвольную точку $M_k(\widetilde{x}_k,\, \widetilde{y}_k)$.

\par Составим интегральные суммы:
\[
\begin{cases}
\sigma_1 = \sum\limits_{k = 1}^{n} P \left( \widetilde{x}_k,\, \widetilde{y}_k \right) \Delta x_k\\
\sigma_2 = \sum\limits_{k = 1}^{n} Q \left( \widetilde{x}_k,\, \widetilde{y}_k \right) \Delta y_k.
\end{cases}
\]

\par Пусть $\delta = \max{\text{дл. } \overbow{A_{k - 1} A_k}}$. Тогда:
\[
\left.
\begin{array}{c}
\lim\limits_{\delta \to 0} \sigma_1 = \int\limits_{\overbow{AB}} P \left( x,\, y \right) dx\\
\lim\limits_{\delta \to 0} \sigma_2 = \int\limits_{\overbow{AB}} Q \left( x,\, y \right) dy 
\end{array}
\right \} \text{ КрИ-2 по ориентированной кривой.}
\]

\par Рассматривают также КрИ-2 общего вида:
\[
\int\limits_{\overbow{AB}} P\left(x,\, y \right)\, dx + Q\left(x,\, y \right)\, dy.
\]

\par Если $\overbow{AB}$ является гладкой непрерывной и ограниченной кривой, то эти интегралы существуют.

\subsection{Физический смысл КрИ-2}

\par Представим, что на $\overbow{AB}$ находится точка с массой $m = 1$.
В каждой точке кривой определён вектор силы $\overrightarrow{F}(x,\, y)$ с координатами $\overrightarrow{F} \left( x,\, y \right) = \left( P \left( x,\, y \right),\, Q \left( x,\, y \right) \right)$.
Элемент кривой(при достаточно малом разбиении) можно заменить хордой. Тогда элементарная работа силы $F$ на этой хорде равна:
\[
\Delta_k A \approx \left| \overrightarrow{F} \right| \cdot \left| \overrightarrow{A_{k - 1} A_k} \right| \cdot \cos{\alpha} =
\langle \overrightarrow{F}, \overrightarrow{A_{k - 1}, A_{k}} \rangle = 
\langle \left( P,\, Q \right), \left( \Delta x_k,\, \Delta y_k \right) \rangle = P \Delta x_k + Q \Delta y_k.
\]

\par Суммируя элементарные работы мы получим приближённо работу $\overrightarrow{F}$ на $\overbow{AB}$:
\[
A \approx \sum\limits_k P \left( \widetilde{x}_k,\, \widetilde{y}_k \right) \Delta x_k + Q \left( \widetilde{x}_k,\, \widetilde{y}_k \right) \Delta y_k.
\]
Переходя к пределу при $\delta \to 0$ получим:
\[
A = \int\limits_{\overbow{AB}} P \left( x,\, y \right)\, dx + Q \left( x,\, y \right) dy.
\] 

\subsection{Вычисление КрИ-2}

\par Пусть $\overbow{AB}$~--- гладкая ориентированная кривая.
\[
\overbow{AB}:
\begin{cases}
x = x \left( t \right)\\
y = y \left( t \right)\\
\alpha \leq t \leq \beta.
\end{cases}
\]
Это равносильно заданию векторного вида такой кривой:
\[
\overrightarrow{r} = \overrightarrow{r}(t) =
\left( x \left( t \right),\, y \left( t \right) \right),\,
\alpha \leq t \leq \beta.
\]

\par Т.~к. кривая гладкая, то $x, y$ имеют непрерывные производные. При этом каждой точке разбиения $A_k$ соответствует значение $t_k$. Тогда:
\[
\Delta x_k = x \left( t_k \right) - x \left( t_{k - 1} \right) = \left[ \text{по формуле Лагранжа} \right] =
x' \left( \overline{t}_k \right) \Delta t_k,
\]
где $\overline{t}_k$ некоторая точка из $\left[ t_{k - 1}, t_k \right]$.

\par Пусть $M_k$ соответствует $\widetilde{t}_k$, тогда:
\[
\sigma_1 = \sum\limits_{k = 1}^{n} P \left( \widetilde{x}_k,\, \widetilde{y}_k \right) \Delta x_k =
\sum\limits_{k = 1}^{n} P \left( \widetilde{x}_k,\, \widetilde{y}_k \right) x' \left( \overline{t}_k \right) \Delta t_k = *
\]
Как и в рассуждениях для КрИ-1(см. прошлую лекцию)
$x' \left( \overline{t}_k \right) \Delta t_k = \left( x' \left( \widetilde{t}_k \right) + \alpha_k \right) \Delta t_k, \alpha_k \to 0,$ 
в силу равномерной непрерывности и формулы Лагранжа.
\[ 
* = \sum\limits_{k = 1}^{n} P \left( \widetilde{x}_k,\, \widetilde{y}_k \right) x' \left( \widetilde{t}_k \right) \Delta t_k +
\sum\limits_{k = 1}^{n} P \left( \widetilde{x}_k,\, \widetilde{y}_k \right) \alpha_k \Delta t_k = 
\]
\[
\sum\limits_{k = 1}^{n} P \left( \widetilde{x}_k,\, \widetilde{y}_k \right) x' \left( \widetilde{t}_k \right) \Delta t_k + \alpha,\, \alpha \to 0 \text{ при } \delta \to 0.
\]
Пусть $\delta \to 0$. Т.~к. Слева~--- интегральная сумма для КрИ-2, справа~--- интегральная сумма для 1И на отрезке $\left[\alpha, \beta\right]$ с разбиением $t_k$ и промежуточными точками $\widetilde{t}_k$. Таким образом получаем:
\[
\int\limits_{\overbow{AB}} P \left( x,\, y \right)\, dx = 
\int\limits_\alpha^{\beta} P \left( x(t),\, y(t) \right) \cdot x' \left( t \right)\, dt. 
\] 
Аналогично для $Q$:
\[
\int\limits_{\overbow{AB}} Q \left( x,\, y \right)\, dx = 
\int\limits_\alpha^{\beta} Q \left( x(t),\, y(t) \right) \cdot y' \left( t \right)\, dt. 
\]
Тогда КрИ-2 общего вида приобретает вид:

\begin{equation}
\label{lec_19, num_3}
\int\limits_{\overbow{AB}} P \left( x,\, y \right)\, dx + Q \left( x,\, y \right)\, dy =
\int\limits_{\alpha}^{\beta} \left( P \left( x \left( t \right),\, y \left( t \right) \right) \cdot x' \left( t \right)
+ Q \left( x \left( t \right),\, y \left( t \right) \right) \cdot y' \left( t \right) \right)\, dt. 
\end{equation}

\begin{example}

Вычислим $\int\limits_{\overbow{AB}} 2 x y\, dx + x^2\, dy,\, A = (0, 0),\, B = (1, 1)$.

\begin{enumerate}[label=\alph*)]

	\item $AB$~--- отрезок.

	\par По формуле \eqref{lec_19, num_3}:
	\[
	I = \left[ AB:	
	\begin{cases}
	x = t\\
	y = t\\
	0 \leq t \leq 1
	\end{cases}
	\right] = 
	\int\limits_{0}^{1} 2 t^{2} \cdot 1 + t^{2} \cdot 1\, dt = 
	\int\limits_{0}^{1} 3 t^{2}\, dt = 1.
	\]

	\item $y = x^{2}$.

	\par По формуле \eqref{lec_19, num_3}:
	\[
	I = \left[ AB:	
	\begin{cases}
	x = t\\
	y = t^{2}\\
	0 \leq t \leq 1
	\end{cases}
	\right] = 
	\int\limits_{0}^{1} 2 t^{3} + 2 t^{3}\, dt = 
	\int\limits_{0}^{1} 4 t^{3}\, dt = 1.
	\]

	\item $y = x^{10}$.

	\par По формуле \eqref{lec_19, num_3}:
	\[
	I = \left[ AB:	
	\begin{cases}
	x = t\\
	y = t^{10}\\
	0 \leq t \leq 1
	\end{cases}
	\right] = 
	\int\limits_{0}^{1} 2 t^{11} + 10 t^{11}\, dt = 
	\int\limits_{0}^{1} 12 t^{11}\, dt = 1.
	\]

\end{enumerate}

	\par Нетрудно показать, что вне зависимости от степени $x$ ответ всегда будет $1$.	

\end{example}

\subsection{Свойства КрИ-2}

\begin{enumerate}[label=\arabic*$^{\circ}$]
	\item $\int\limits_{\overbow{AB}} = -\int\limits_{\overbow{BA}}$

	\par Следует из того, что $\Delta x_k = x_k - x_{k - 1}, \, \Delta y_k = y_k - y_{k - 1}$
	при смене направлении меняют знак.
	\item Линейность (по аналогии с 1И)
	\item Аддитивность (по аналогии с 1И)
	\item Основная оценка

	\par Если $\left| P \left( x,\, y \right) \right| \leq M\ \forall\, \left( x,\, y \right) \in \overbow{AB}$,
	то $\int\limits_{\overbow{AB}} P \left( x,\, y \right)\, dx \leq M \cdot l.$

	\par Для доказательства достаточно оценить $\sigma_1$.
\end{enumerate}

\subsection{КрИ-2 в $\R^3$}

	\par Если $\overbow{AB}$~--- гладкая пространственная кривая на которой определены функции
	$P \left( x,\, y,\, z \right)$, $Q \left( x,\, y,\, z \right)$, $R \left( x,\, y,\, z \right)$,
	то по аналогии с КрИ-2 в $\R^2$ определяют КрИ-2 в $\R^3$:
	\[
	\int\limits_{\overbow{AB}} P \left( x,\, y,\, z \right)\, dx + Q \left( x,\, y,\, z \right)\, dy + R \left( x,\, y,\, z \right)\, dz. 
	\]

	\par КрИ-2 в $\R^3$ обладает теми же свойствами, что и КрИ-2 в $\R^2$.

	
	\par Если $\overbow{AB}$ задана в произвольной параметризации, 
	\[
	\overbow{AB}:
	\begin{cases}
	x = x \left( t \right)\\
	y = y \left( t \right)\\
	z = z \left( t \right)\\
	\alpha \leq t \leq \beta
	\end{cases}
	\] 
	или в векторном виде
	\[
	\overrightarrow{r} = \overrightarrow{r} \left( t \right) = \left( x \left( t \right),\, y \left( t \right),\, z \left( t \right) \right),\,
	\alpha \leq t \leq \beta,
	\]
	то КрИ-2 $\R^3$ принимает вид:
	
	\begin{equation}
	\label{lec_19, num_4}
	I = \int\limits_{\alpha}^{\beta} P \left( x \left( t \right),\, y \left( t \right),\, z \left( t \right) \right)\, dx +
	Q \left( x \left( t \right),\, y \left( t \right),\, z \left( t \right) \right)\, dy +
	R \left( x \left( t \right),\, y \left( t \right),\, z \left( t \right) \right)\, dz. 
	\end{equation}
	
	\begin{example}
	\par Вычислить: $\int\limits_{\overbow{AB}} yz\, dx + xz\, dy + xy\, dz,\, \overbow{AB}:
	\begin{cases}
	x = t\\
	y = t^2\\
	z = t^3\\
	0 \leq t \leq 1.
	\end{cases}$
	\[
	I = \int\limits_{0}^{1} \left( t^5 \cdot 1 + t^4 \cdot 2t + t^3 \cdot 3t^2 \right)\, dt = \int\limits_{0}^{1} 6 t^5\, dt = 1.
	\]
	\end{example}	 

\end{document}