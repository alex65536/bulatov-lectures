\makeatletter
\def\input@path{{../../}}
\makeatother
\documentclass[../../main.tex]{subfiles}

\graphicspath{
{../../img/}
{../img/}
{img/}
}

\begin{document}
    \section{Достаточное условие экстремальности стационарных точек ФНП}
    Пусть $u = f(x),\ x = (x_1, x_2, \ldots, x_n) \in D \subset \R^n$
    дважды непрерывно дифференцируема в некоторой окрестности $V(x_0) \subset 
    D$
    внутренней точки $x_0 \in D$. В этом случае
    \begin{equation}
        \exists \; d^2f(x_0) = \sum_{i, j = 1}^n\pderiv{^2f(x_0)}{x_i
        \partial x_j}dx_idx_j.
        \label{lec8:6}
    \end{equation}

    \eqref{lec8:6} можно рассмотреть как соответствующую квадратичную
    форму относительно дифференциалов независимой переменной $dx_i = \D x_i,
    i = \overline{1, n}$ c матрицей
    \[A = \left(\pderiv{^2f(x_0)} {x_i\partial x_j}\right), i,j =
    \overline{1,n}. \]

    В силу теоремы о равенстве смешанных производных второго порядка ФНП,
    эта матрица будет симметрической.

    \begin{thm}[достаточное условие локального экстремума ФНП]
        Если у дважды непрерывно дифференцируемой ФНП $u=f(x),\ x \in D 
        \subset \R^n$
        в некоторой окрестности $V(x_0) \subset D$ внутренней стационарной
        точки $x_0 \in D$ второй дифференциал \eqref{lec8:6} является
        знакопостоянной квадратичной формой относительно $dx_k,\ k =
        \overline{1, n}$, то стационарная точка $x_0$ будет точкой локального
        экстремума ФНП. При этом если квадратичная форма \eqref{lec8:6}
        положительно определена, то стационарная точка $x_0$ является точкой
        локального минимума. Если \eqref{lec8:6} является отрицательно
        определённой квадратичной формой, то $x_0$ является точкой локального
        максимума $f(x)$.
    \end{thm}

    \begin{proof}
        Для внутренней стационарной точки $x_0 = (x_1, x_2, \ldots, x_n)
        \in D \subset \R^n$ рассмотрим произвольное приращение $\D x =
        (\D x_1, \D x_2, \ldots, \D x_n)$ такое, что $x_0 + \D x \in V(x_0)
        \subset D$, где $f(x_0)$ дважды непрерывно дифференцируема. Далее, по
        формуле Тейлора-Пеано для ФНП, имеем
        \begin{equation}
            \begin{gathered}
                \D f(x_0) = f(x_0 + \D x) - f(x_0) = df(x_0) +
                \frac{d^2f(x_0)}2 + o(|dx|^2), \\|dx| = \sqrt{dx_1^2 + \ldots
                + dx_n^2} = \sqrt{\D x_1^2 + \ldots + \D x_n^2} = |\D x|.
            \end{gathered}
            \label{lec8:7}
        \end{equation}

        Учитывая, что в квадратичной форме \eqref{lec8:6} $\forall dx_k = \D
        x_k,\ k = \overline{1, n}$, в случае, когда $\D x \ne
        \vec{0}$, получаем
        \begin{equation}
            \begin{gathered}
                \D f(x_0) \stackrel{\eqref{lec8:7}}{=} [df(x_0) = 0] =
                \frac{d^2f(x_0)}2 + o(|\D x|^2) \stackrel{\eqref{lec8:6}}{=}
                \frac12 \sum_{i, j = 1}^n\pderiv{^2f(x_0)}{x_i\partial x_j}\D
                x_i \D x_j+ o(|\D x|^2) = \\ =\left(\frac12 \Phi(h) +
                o(1)\right) \cdot |\D x|^2,
            \end{gathered}
            \label{lec8:8}
        \end{equation}
        где
        \begin{equation}
            \forall h_k = \frac{\D x_k}{|\D x|}, \; k = \overline{1,n}, \;\;
            \Phi(h) =
                \sum_{i, j = 1}^n\pderiv{^2f(x_0)}{x_i\partial x_j}h_ih_j.
        \label{lec8:9}
        \end{equation}

        Для рассматриваемых $h = (h_1, \ldots, h_n) \in \R^n$ имеем
        $|h| = \left|\dfrac{\D x}{|\D x|}\right| = \dfrac{|\D x|}{|\D x|} = 1$,
        т.~е. $\forall h \in S_1(\vec{0})$~--- единичная сфера с
        центром в нуле. Если квадратичная форма \eqref{lec8:6} знакопостоянна,
        то знакопостоянной будет также и \eqref{lec8:9}, поэтому в силу леммы
        об оценке знакопостоянной квадратичной формы, $\exists C_0 = const >
        0$ такая, что $|\Phi(h)| \stackrel{\eqref{lec8:9}}{\geq} C_0 > 0$.

        Рассмотрим для простоты случай положительно определённой квадратичной
        формы \eqref{lec8:6}, значит, \eqref{lec8:9}. В этом случае $\Phi(h) =
        |\Phi(h)|$. Отсюда для достаточно малых
        приращений $\D x \ne 0$ в силу \eqref{lec8:8} будем иметь $\D f(x_0)
        \stackrel{\eqref{lec8:8}}{\geq} \dfrac12 (C_0 + o(1)) |\D x|^2 \geq 0$.
        То есть $\D f(x_0) \geq 0$ в соответствующей окрестности стационарной
        точки $x_0$, а значит, в этом случае $x_0 = x_{\min}$. Аналогично
        рассматривается случай, когда \eqref{lec8:6} является отрицательно
        определённой, и $x_0 = x_{\max}$.
    \end{proof}
    \begin{rem}
        Можно показать, что если \eqref{lec8:6} не является знакопостоянной
        относительно дифференциала независимой переменной, то $x_0$ не
        является экстремальной для $f(x)$. Если же \eqref{lec8:6} является
        вырожденной и либо положительно полуопределённой, либо отрицательно
        полуопределённой, то в этом случае нужны дополнительные исследования.
    \end{rem}
    
    \begin{iex}
        Рассмотрим дважды непрерывно дифференцируемую Ф2П $u = f(x, y), \;$
        $(x, y) \in D \subset \R^2$ в некоторой окрестности $V(M_0)$
        стационарной точки $M_0 (x_0, y_0) \in D$. Тогда, во-первых,
        $\begin{cases}
            f'_x(M_0) = 0, \\
            f'_y(M_0) = 0
        \end{cases}
        \implies df(M_0) = f'_x(M_0)dx + f'_y(M_0)dy = 0 
        \quad \forall(dx, dy)
        \in \R^2$, а, во-вторых,
        \begin{equation}
            d^2f(x_0) = Adx^2 + 2Bdxdy + Cdy^2,
            \label{lec8:10}
        \end{equation}
        где
        \[\begin{cases}
            A = f''_{x^2}(M_0), \\
            B = f''_{xy}(M_0), \\
            C = f''_{y^2}(M_0).
        \end{cases}\]
        Исследуем эту квадратичную форму относительно $dx, dy$ на
        знакопостоянство по критерию Сильвестра, используя главные угловые
        миноры матрицы
        $\begin{bmatrix}
            A & B \\
            B & C
        \end{bmatrix}$.
        Имеем $\D_1 = A,\:\: \D_2 =
        \begin{vmatrix}
            A & B \\
            B & C
        \end{vmatrix}
        = AC - B^2$. В результате получаем:
        \begin{enumerate}[label=\arabic*)]
            \item если $A = \D_1 > 0$, $D = AC-B^2 = \D_2 > 0$, то
            квадратичная форма \eqref{lec8:10} положительно определена,
            поэтому стационарная точка $M_0 = M_{\min}$;
            \item если $A = \D_1 < 0$, $D = AC-B^2 = \D_2 > 0$, то
            квадратичная форма \eqref{lec8:10} отрицательно определена,
            поэтому стационарная точка $M_0 = M_{\max}$;
            \item если $D = AC - B^2 < 0$, то стационарная точка $M_0$ не
            является экстремальной (здесь стационарную точку называют \emph{седловой});
            \item если $D = AC - B^2 = 0$, то требуются дополнительные
            исследования.
        \end{enumerate}
    \end{iex}

    \begin{exmps}
    ~
    \begin{enumerate}[label=\arabic*)]
        \item Рассмотрим Ф2П $u = x^3 + 3xy + y^3, (x,y) \in \R^2$. Для
        определения стационарных точек имеем систему
        \[
        \begin{cases}
            u'_x = 3x^2 + 3y = 0, \\
            u'_y = 3x + 3y^2 = 0.
        \end{cases}
        \]

        Вычитая одно уравнение системы из другого, получаем $3(x^2 - y^2) +
        3(y - x) = 0 \iff (x+y-1)(x-y)=0$.
        \begin{itemize}
            \item[а)] $x - y = 0 \implies x = y \implies 3x^2 + 3x = 0 \implies
            x_1 = 0 = y_1, \; x_2 = -1 = y_2$.
            Получили две стационарные точки $M_1(0,0)$ и $M_2(-1,-1)$.
            \item[б)] $x+y=1 \implies y = 1-x$.

            $3x^2 + 3(1-x) = 0 \iff x^2 - x + 1 = 0$. Нет решений.

            Используя предыдущий пример, имеем
            \begin{enumerate}[label=\arabic*.]
                \item $M_1(0,0)$.

                $A_1 = u''_{x^2} (M_1) = 6x \big|_{(0,0)} = 0$.

                $B_1 = u''_{yx} (M_1) = 3$.

                $C_1 = u''_{y^2} (M_1) = 6y \big|_{(0,0)} = 0$.

                Здесь $D_1 = A_1C_1 - B_1^2 = -9 < 0$. Поэтому стационарная
                точка $M_1$ не является экстремальной.

                \item $M_2 (-1, -1)$.

                $A_2 = 6x \big|_{(-1,-1)} = -6$;

                $B_2 = 3$;

                $C_2 = 6y \big|_{(-1,-1)} = -6$.

                Здесь $D_2 = A_2C_2 - B_2^2 =
                A_2C_2 - B_2^2 = 36 - 9 = 27 > 0$. Поэтому стационарная
                точка $M_2$ является экстремальной, причём, так как $A_2 = -6 <
                0$ и $D_2 > 0$, то $M_2 = M_{\max}$. В данном случае
                $u_{\max} = u(M_{\max}) = 1$.
            \end{enumerate}
        \end{itemize}
        \item Рассмотрим Ф3П $u = x^2 + 2y^2 + 3z^2 - 4x + 6y +12z-1$.
        Из системы
        $\begin{cases}
            u'_x = 2x-4 = 0, \\
            u'_y = 4y+6 = 0, \\
            u'_z = 6z+12 = 0
        \end{cases}$ находим стационарную точку $M_0\left(2, -\dfrac32,
        -2\right)$. Для исследования её на экстремум рассмотрим
        \begin{multline*}
            d^2u = d(du) = d((2x-4)dx+(4y+6)dy+(6z+12)dz) =
            \left[
                \begin{gathered}
                    x, y, z - \text{независимые} \\
                    dx, dy, dz = \text{fix}
                \end{gathered}
            \right] = \\ = 2dx^2 + 4dy^2 + 6dz^2 \geq 0,
        \end{multline*}
        причём, так как $d^2u(M_0)=0 \iff dx=dy=dz=0$, то эта квадратичная
        форма относительно $dx, dy, dz$ будет невырожденной, и $d^2u(M_0)>0
        \quad \forall (dx, dy, dz) \ne (0,0,0)$. В этом случае второй
        дифференциал является положительно определённой квадратичной формой,
        следовательно, $M_0 = M_{\min}$, $u_{\min} = u\left(2, -\dfrac32,
        -2\right) = -\dfrac{43}2$. В то же время эта точка является и точкой
        глобального минимума, так как после выделения полных квадратов
        \begin{multline*}
            u = (x^2-4x) + 2(y^2+3y) + 3(z^2+4z)-1 = (x-2)^2 - 4 + 2\left(y+
            \dfrac32\right)^2 - \dfrac92 + 3(z+2)^2 -\\- 12 - 1 = (x-2)^2 +
            2\left(y+\dfrac32\right)^2 + 3(z+2)^2 - \frac{43}2 \geq
            -\frac{43}2,
        \end{multline*}
        причём в точке $M_0$ полученная оценка достигается.
    \end{enumerate}
    \end{exmps}
\end{document}
