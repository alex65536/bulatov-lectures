\makeatletter
\def\input@path{{../../}}
\makeatother
\documentclass[../../main.tex]{subfiles}

\graphicspath{
  {../../img/}
  {../img/}
  {img/}
}

\begin{document}
	\section{Неявные ФНП}

	Рассмотрим функцию $F(x, u)$ от $(n+1)$ переменной, где $x = (x_1, x_2, \ldots,
	 x_n) \in \R^n, \; u \in \R$. В соответствии с этой функцией уравнение
	 \begin{equation}
		F(x,u) = 0, u \in \R \label{lec8.1:1}
	 \end{equation}
	 задаёт некоторую явную функцию $u = u(x_1, x_2, \ldots, x_n)$. Для таких 
	 уравнений в первую очередь важны условия их разрешимости относительно 
	 $u \in \R$, а также свойства непрерывности и дифференцируемости $u = u(x)$ в 
	 соответствующих окрестностях заданных точек $x_0 \in D \subset \R, u_0 \in I  
	 \subset \R$, для которых верно
	 \begin{equation}
		F(x_0, u_0) = 0. \label{lec8.1:2}
	 \end{equation}
	 В дальнейшем под решением ФУ \eqref{lec8.1:1} с начальным условием 
	 \eqref{lec8.1:2} будем подразумевать некоторую функцию $u(x)$, для которой 
	 верно: 
	 \begin{itemize}
	 	\item[1)] $u(x_0) = u_0$;
	 	\item[2)] в соответствующей окрестности $u_0 \in I$ справедливо тождество
		\begin{equation}
			F(x, u(x)) = 0. \label{lec8.1:3}
		\end{equation}
	 \end{itemize}
	 \begin{exmp}
		Пусть $F(x, u) = x^2 - u^2$

		/* неразборчивая каша, допишу позже */
	 \end{exmp}

	 \begin{thm}[Об однозначной разрешимости ФУ]
		Пусть задана функция от $(n+1)$ переменной $F(x, u), x = (x_1, x_2, \ldots,
	 	x_n) \in \R^n, \; u \in \R$, а также некоторые точки $x_0 \in D \subset \R, u_0
		\in I \subset \R$. Если $F(x,u)$ непрерывно дифференцируема по $u$, и 
		\begin{equation}
			F'_u(x_0, u_0) \ne 0, \label{lec8.1:4}
		\end{equation}
		то тогда ФУ \eqref{lec8.1:1} с начальным условием \eqref{lec8.1:2} имеет 
		единственное решение $u = u(x)$ в некоторой окрестности $V(x_0) \subset D$,
		удовлетворяющее
	 	\begin{equation}
			u(x_0) = u_0. \label{lec8.1:5}
		\end{equation}
	 \end{thm}
	 \begin{proof}
		Из условия \eqref{lec8.1:4} в силу непрерывности $F'_u$ в окрестности $x_0$
		следует, что $F'_u$ сохраняет один и тот же знак в рассматриваемых 
		окрестностях (по теореме о стабилизации знака непрерывных ФНП). Без 
		ограничения общности будем считать, что $\exists \delta > 0$, что $\forall u \in
		[u_0 - \delta, u_0 + \delta] \subset I \implies F'_u(x_0, u)>0$. В этом случае Ф1П
		 строго возрастает на $[u_0 - \delta, u_0 + \delta]$. Из этого $F(x_0, u_0 - 
		 \delta) < 0, F(x_0, u_0 + \delta) > 0$.

		 Отсюда в силу непрерывности
		 $F(x,u)$ по теореме о стабилизации  знака, уменьшив при необходимости
		 /* нет данных */, получаем, что $\widetilde V(x_0) \subset V(x_0) \subset D$, 
		 что $\forall x \in \widetilde V(x_0) \implies F(x_0, u_0 - \delta) < 0, F(x_0, u_0 + 
		 \delta) > 0$. Отсюда по теореме о промежуточных значениях ФНП следует,
		 что $\forall fix \; x \in \widetilde V(x_0) \exists! u \in [u_0 - \delta, u_0 + \delta]$ 
		 такое, что $u = u(x)$ удовлетворяет \eqref{lec8.1:3}. При этом в силу строгой 
		 монотонности $F'_u$ будет также выполняться \eqref{lec8.1:5} в силу 
		 начального условия \eqref{lec8.1:5}. 
	 \end{proof}
\end{document}