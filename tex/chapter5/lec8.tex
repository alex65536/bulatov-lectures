\makeatletter
\def\input@path{{../../}}
\makeatother
\documentclass[../../main.tex]{subfiles}

\graphicspath{
  {../../img/}
  {../img/}
  {img/}
}

\begin{document}
	\section{Неявные ФНП}

	Рассмотрим функцию $F(x, u)$ от $(n+1)$ переменной, где $x = (x_1, x_2, 
	\ldots,  x_n) \in \R^n, \; u \in \R$. В соответствии с этой функцией уравнение
	 \begin{equation}
		F(x,u) = 0 \label{lec8.1:1}
	 \end{equation}
	 относительно $u \in \R$ задаёт некоторую неявную ФНП $u = u(x_1, x_2 
	 \ldots, x_n)$. Далее \eqref{lec8.1:1} будем называть \emph{функциональным 
	 уравнением (ФУ)}. Для таких уравнений в первую очередь важны условия их 
	 разрешимости относительно $u \in \R$ при $x \in \R^n$, а также свойства 
	 непрерывности и дифференцируемости $u = u(x)$ в соответствующих 
	 окрестностях заданных точек $x_0 \in D \subset \R,\ u_0 \in I  \subset \R$, 
	 для которых выполняется
	 \begin{equation}
		F(x_0, u_0) = 0, \label{lec8.1:2}
	 \end{equation}
	 что задаёт для \eqref{lec8.1:1} соответствующее начальное условие на $u = 
	 u(x)$ так, чтобы $u(x_0) = u_0$. В дальнейшем под решением ФУ 
	 \eqref{lec8.1:1} с начальным условием \eqref{lec8.1:2} будем подразумевать 
	 некоторую функцию $u(x)$, для которой верно: 
	 \begin{enumerate}
	 	\item $u(x_0) = u_0$;
	 	\item в соответствующих окрестностях $V(x_0) \subset D$, $U(u_0) 
	 	\subset I$ справедливо тождество
		\begin{equation}
			F(x, u(x)) \equiv 0. \label{lec8.1:3}
		\end{equation}
	 \end{enumerate}
	 \begin{exmp}
		Пусть $F(x, u) = x^2 - u^2$.

		Рассмотрим уравнение $F(x, u) = x^2 - u^2 = 0$ в окрестности $x_0=0, 
		u_0=0$. В данном случае нет единственного решения, так как указанным 
		выше условиям удовлетворяют, например, $u=|x|$, $u=-|x|$, $u = x$  и т. д.
	 \end{exmp}

	 \begin{thm}[Об однозначной разрешимости ФУ]
		Пусть задана функция от $(n+1)$ переменной $F(x, u)$, непрерывная 
		относительно $x = (x_1, x_2, \ldots,
	 	x_n) \in \R^n$, $u \in \R$ в соответствующих окрестностях $x_0 \in D 
	 	\subset \R^n$, $u_0 \in I \subset \R$, причём $F(x_0,u_0)=0$. Если $F(x,u)$ 
	 	непрерывно дифференцируема по $u$, а также 
		\begin{equation}
			F'_u(x_0, u_0) \ne 0, \label{lec8.1:4}
		\end{equation}
		то тогда ФУ \eqref{lec8.1:1} с начальным условием \eqref{lec8.1:2} имеет 
		единственное решение $u = u(x)$ в некоторой окрестности $V(x_0) 
		\subset D$, удовлетворяющее
	 	\begin{equation}
			u(x_0) = u_0. \label{lec8.1:5}
		\end{equation}
	 \end{thm}
	 \begin{proof}
		 Из условия \eqref{lec8.1:4} в силу непрерывности $F'_u$ в 
		 соответствующих окрестностях $x_0, u_0$ следует, что $F'_u$ сохраняет 
		 один и тот же знак в рассматриваемых окрестностях (по теореме о 
		 стабилизации знака непрерывных ФНП). Без ограничения общности 
		 будем считать, что $\exists \delta > 0: \; 
		 \forall u \in [u_0 - \delta, 
		 u_0 + \delta] \subset I \implies F'_u(x_0, u)>0$. В этом случае Ф1П
		 строго возрастает на $[u_0 - \delta, u_0 + \delta]$. Из этого, в силу 
		 \eqref{lec8.1:2}, $F(x_0, u_0 - \delta) < 0$ и
		 $ F(x_0, u_0 + \delta) > 0$.

		 Отсюда, в силу непрерывности
		 $F(x,u)$, по теореме о стабилизации  знака непрерывных ФНП, 
		 уменьшив при необходимости
		 $\delta > 0$, получаем, что $\exists \widetilde V(x_0) \subset V(x_0) 
		 \subset D$, что $\forall x \in \widetilde V(x_0) \implies F(x, u_0 - 
		 \delta) < 0,\ F(x, u_0 + \delta) > 0$. Отсюда по теореме о промежуточных 
		 значениях ФНП следует, что $\forall\, \text{fix} 
		 \: x \in \widetilde V(x_0) \quad \exists! \,
		 u \in [u_0 - \delta, u_0 + \delta]$ такое, что $u = u(x)$ удовлетворяет 
		 \eqref{lec8.1:3}. При этом в силу строгой монотонности $F(x, u)$ по $u$ 
		 будет также выполняться \eqref{lec8.1:5} в силу начального условия 
		 \eqref{lec8.1:2}. 
	 \end{proof}

	 \begin{rems}
	 \;
		 \begin{itemize}
		 	 \item[1)] Можно показать, что при выполнении всех условий 
		 	 доказанной выше теоремы получаем единственное решение $u$, 
		 	 которое будет непрерывно в соответствующих окрестностях точек 
		 	 $x_0, u_0$.

			 \item[2)] Аналогичным образом, с использованием формулы Лагранжа 
			 конечных приращений, доказывается \emph{теорема о 
			 дифференцировании неявных ФНП}:
		\end{itemize}
	\end{rems}

    \begin{thm}[О дифференцировании неявных ФНП]
		Пусть наряду со всеми условиями предыдущей теоремы дополнительно 
		функция $F(x, u)$ непрерывно дифференцируема по $x$. Тогда ФУ 
		\eqref{lec8.1:1} с начальным условием \eqref{lec8.1:2} также разрешимо, 
		и его единственное решение в соответствующих окрестностях 
		рассматриваемых точек $x_0 \in D, u_0 \in I$ будет удовлетворять 
		начальному условию $u(x_0)=u_0$, а также будет дифференцируемо в 
		этой точке по $x$, а частные производные полученной ФНП $u = u(x)$ в 
		соответствующих окрестностях $x_0, u_0$ находятся по формуле
		\begin{equation}
			u'_{x_k} = -\frac{F'_{x_k}(x,u)}{F'_u(x,u)}. \label{lec8.1:6}
		\end{equation}
    \end{thm}
\end{document}
