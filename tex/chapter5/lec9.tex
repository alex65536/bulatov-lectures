\makeatletter
\def\input@path{{../../}}
\makeatother
\documentclass[../../main.tex]{subfiles}

\graphicspath{
  {../../img/}
  {../img/}
  {img/}
}

\begin{document}

\begin{exmp}
Зная производные и дифференциал 1-го порядка находятся последовательно
производные и дифференциалы высших порядков.


Рассмотрим функцию
\[\begin{cases}
    x + y + z = e^z \\
    z = z\left(x, y\right) \\
\end{cases}\]

Имеем:
\[
\begin{array}{l}
d\left(x + y + z\right) = d(e^z) \\
dx + dy + dz = e^zdz \implies 
\begin{cases}
dz = \frac{dx}{e^z - 1}
+ \frac{dy}{e^z - 1} = z'_xdx + z'_ydy \\
z'_x = \frac{1}{e^z - 1}\\
z'_y = \frac{1}{e^z - 1}
\end{cases}
\end{array}
\]

\[z''_x = \left(z'_x\right)'_x = \left(\frac{1}{e^z - 1}\right)'_x
= -\frac{\left(e^z\right)'_x}{\left(e^z - 1\right)^2} =
- \frac{e^zz'_x}{\left(e^z - 1\right)^2} = \left[z'_x
= \frac{1}{e^z - 1} \right] =
- \frac{e^z}{\left(e^z - 1\right)^3}\]

В силу симметрии \[\displaystyle z''_{y^2} = \dots = - 
\frac{e^z}{\left(e^z - 1\right)^3}\]

\[z''_{xy} = \left(z'_x\right)'_y = \left(\frac{1}{e^z - 1} \right)'_y =
\dots = - \frac{e^z}{\left(e^z - 1\right)^3}\]

Отсюда
\[d^2z = z''_{x^2}dx^2 + 2z''_{xy}dxdy + z''_{y^2}dy^2 =
-\frac{dx^2 + 2dxdy + dy^2}{\left(e^z - 1\right)^3} e^z = 
-\frac{\left(dx + dy\right)^2}{\left(e^z - 1\right)^3} e^z\]
и так далее.
\end{exmp}

\section{Системы функциональных уравнений (СФУ)}
Пусть имеется $m$ функций от $\left(n + m\right)$ переменных:
\[\begin{cases}
    F_k\left(x, u \right),\ k = \overline{1, m} \\
    x = \left(x_1, x_2, \dots, x_n\right) \in D \subset \R^n \\
    u = \left(u_1, u_2, \dots, u_m\right) \in G \subset \R^m
\end{cases}
\]

Под \emph{СФУ} будем подразумевать систему вида

\begin{equation}
    \label{lec_9.num_7}
    \begin{cases} 
        F_k\left(x, u\right) = 0, \\ 
        k = \overline{1, m} \\
    \end{cases}    
\end{equation}

Предполагается, что в \eqref{lec_9.num_7} $x$~--- независимая
переменная, а $u = u\left(x\right)$~--- искомая функция.

Наряду с \eqref{lec_9.num_7} будем рассматривать начальные условия
\begin{equation}
    \label{lec_9.num_8}
    \begin{cases}
        F_k\left(x_0, u_0\right) = 0, \\
        k = \overline{1, m}, \\
    \end{cases}
\end{equation}
в соответствии с которыми под решением системы
\eqref{lec_9.num_7}, \eqref{lec_9.num_8}
подразумевается $u = u(x_1, x_2, \dots, x_n)$,
удовлетворяющее условию
$u\left(x_0\right) = u_0$,
при этом в соответствующих окрестностях точек $x_0$, $u_0$
решение системы \eqref{lec_9.num_7}
с начальными условиями 
\eqref{lec_9.num_8} удовлетворяют тождеству
\[F(x, u(x)) \equiv 0\]

Пусть $F = \left(F_1, F_2, \dots,F_m\right) \in \R^m$,
тогда систему \eqref{lec_9.num_7}
с начальными условиями 
\eqref{lec_9.num_8} можно записать в виде

\begin{equation}
    \label{lec_9.num_9}
    \begin{cases}
     F\left(x, u\right) = \vec{0} \in \R^m\\
     F\left(x_0, u_0\right) = \vec{0} \in \R^m\\
    \end{cases}
\end{equation}


В дальнейшем для СФУ \eqref{lec_9.num_7} 
будем рассматривать \emph{матрицу Якоби}, которую будем обозначать:

\begin{equation}
    \label{lec_9.num_10}
    \pderiv{F}{u} =
    \begin{bmatrix}
        \pderiv{F_1}{u_1} & \pderiv{F_1}{u_2} & \dots & 
        \pderiv{F_1}{u_m} \\
        \pderiv{F_2}{u_1} & \pderiv{F_2}{u_2} &\dots & 
        \pderiv{F_2}{u_m} \\
        \vdots & \vdots & \ddots & \vdots \\
        \pderiv{F_m}{u_1} & \pderiv{F_m}{u_2} & \dots & 
        \pderiv{F_m}{u_m}
    \end{bmatrix}
\end{equation}

Определитель квадратной матрицы Якоби 
\eqref{lec_9.num_10} называется \emph{якобианом}
и записывается в виде:
\begin{equation}
    \label{lec_9.num_11}   
    I(x, u) = \left|\pderiv F u\right| = 
    \det\pderiv{F\left(x,u\right)}{u} =
    \pderiv{\left(F_1,F_2,\dots,F_m\right)}{\left(u_1, u_2,\dots, u_m\right)}
\end{equation}

\begin{thm}[об однозначной разрешимости СФУ]
Пусть функции $F_k\left(x, u\right)$, $k = \overline{1, m}$
от $(n+ m)$ переменных 
$x = \left(x_1, x_2, \dots, x_n\right) \in D \subset \R^n$,
$u = \left(u_1, u_2, \dots, u_m\right) \in G \subset \R^m$
непрерывны по $x$ и по $u$ в некоторых окрестностях точек
$x_0 = \left(x_{01}, x_{02}, \dots, x_{0n}\right) \in D 
\subset \R^n$,
$u_0 = \left(u_{01}, u_{02}, \dots, u_{0m}\right) \in G
\subset \R^m$. Если выполнено начальное условие
\begin{equation}
    \label{lec_9.num_12}
    F_k(x_0, u_0) = \vec{0},\ k = \overline{1, m},
\end{equation}
то в случае непрерывной дифференцируемости
\begin{equation}
    \label{lec_9.num_13}
    F = \left(F_1, F_2, \dots, F_m\right) \in \R^m
\end{equation}
по $u = \left(u_1, u_2, \dots, u_m\right) \in \R^m$,
то если якобиан
$I(x, u) = \det\pderiv{F\left(x, u\right)}{u}$
рассматриваемой СФУ $F\left(x, u\right) = \vec{0}$
в точке $\left(x_0, u_0\right)$ удовлетворяет условию
\begin{equation}
    \label{lec_9.num_14}
    I(x_0, u_0) \neq 0,
\end{equation}
тогда система \eqref{lec_9.num_7} будет иметь в 
соответствующих окрестностях точек $x_0\ \text{и}\ u_0$
единственное решение, т.е. $\exists!u = u(x)$ удовлетворяющая начальному
 условию $u(x_0) = u_0$.
\end{thm}

\begin{proof}
    Доказательство проведём, используя ММИ
    (метод математической индукции).
    
    Во-первых, при $m = 1$ СФУ \eqref{lec_9.num_7} 
    даёт ФУ \eqref{lec8.1:1}, для которого теорема об
    однозначности решения уже доказана.
    
    Во-вторых, предполагая, что теорема доказана
    при $m = k,\ k \in \N$, рассмотрим случай
    $m = k + 1$.
    
    Из условия \eqref{lec_9.num_14} в силу правила Лапласа
    вычисления определителя разложением по какой-либо
    строке (столбцу) следует, что в силу \eqref{lec_9.num_14} 
    в точках $x_0,\ u_0$ хотя бы один из миноров 
    $k$-ого порядка рассматриваемого якобиана ненулевой.
    Без ограничения общности (перестановкой строк, столбцов)
    считаем, что этот минор $k$-ого порядка
    \begin{equation}
        \label{lec_9.num_15}
        I_0 = \pderiv{\left(F_1, F_2, \dots, F_k\right)}
        {\left(u_1, u_2, \dots, u_k\right)}(x_0, y_0) \neq 0
    \end{equation}
    является главным угловым минором матрицы Якоби рассматриваемой
    СФУ. Тогда, во-первых, в силу индуктивного
    предположения, при $m = k$ СФУ
    \begin{equation}
        \label{lec_9.num_16}
        \begin{cases}
            F_i\left(x, v, u_{k+1}\right) = 0, \\
            i = \overline{1, k}
        \end{cases}
    \end{equation}
    для $\text{fix}\ u_{k+1}$ однозначно разрешена относительно 
    $v = \left(u_1, u_2, \dots, u_k\right)$, т.е. 
    $\exists!v = v\left(x, u_{k+1}\right)$, удовлетворяющее
    \eqref{lec_9.num_16} в соответствующих окрестностях
    рассматриваемых точек, и при этом выполнено начальное 
    условие
    $v(x_0) = v(x_0, u_{0,k+1}) = v_0$.
    
    Представляя это решение в последнем уравнении из
    \eqref{lec_9.num_7}, получим ФУ вида
    \begin{equation}
        \label{lec_9.num_17}
        H(x, u_{k+1}) = 0,
    \end{equation}
    где
    \begin{equation}
        \label{lec_9.num_18}
        H(x, u_{k+1}) = 
        F_{k+1}\left(x, v(x, u_{k+1}), 
        u_{k+1}\right).
    \end{equation}
    
    \eqref{lec_9.num_17} определяет некоторую функцию
    \begin{equation}
        \label{lec_9.num_19}
        \begin{cases}
        u_{k+1} = u_{k+1}(x)\\
        u_{k+1}(x_0) = u_{0,k+1}.
        \end{cases}
    \end{equation}
    
    Осталось показать, что \eqref{lec_9.num_18} с начальным
    условием \eqref{lec_9.num_19} имеет единственное решение относительно 
    $u_{k+1}$ в рассматриваемых окрестностях начальных
    точек. Для этого, в силу теоремы о однозначной 
    разрешимости СФУ
    достаточно проверить, что
    \begin{equation}
    \label{lec_9.num_20}
    \det\pderiv{H(x, u_{k+1})}{u_{k+1}} 
    \vline\underset{\left(x_0, u_{0,k+1}\right)}{} \neq 0
    \end{equation}
    
    Дифференцируя равенство
    \begin{equation}
    \label{lec_9.num_21}
    F_{k+1}\left(x, v(x, u_{k+1}), 
    u_{k+1}\right) \equiv 0
    \end{equation}
    по $u_{k+1}$ в соответствующих точках $x_0\ \text{и}\ u_0$,
    получаем
    \begin{equation}
        \label{lec_9.num_22}
        \sum_{j = 1}^{k}
        \pderiv{F_{k+1}}{v_j} \cdot \pderiv{v_j}{u_{k+1}} + 
        \pderiv{F}{u_{k+1}} \equiv 0
    \end{equation}
    
    Прибавляя к $\left(k + 1\right)$-му столбцу 
    якобиана соответствующий столбец, в результате
    получим, что 
    \[I\left(x_0, u_0\right)= \dots = 
    \det\pderiv{H\left(x_0,u_{k+1}\right)}{u_{k+1}} \neq 0,\]
    поэтому для полученного ФУ выполнены все условия 
    однозначной разрешимости ФУ и, значит, 
    $\exists! u_{k+1} = u_{k+1}\left(x\right)$, 
    удовлетворяющее, во-первых, начальному условию
    $u_{k+1} = u_{0,k+1}$ и, во-вторых, после подстановки
    заданной функции в
    $v = v\left(x, u_{k+1}\right)$ получим функцию,
    которая удовлетворяет на рассматриваемом этапе
    соответствующим условиям.
    При этом для $u = u\left(x\right)$ выполнено начальное
    условие $u\left(x_0\right) = u_0$.
\end{proof}

\begin{rems}

~

    \begin{enumerate}
        \item Как и для ФУ, при выполнении условий 
        теоремы об однозначной разрешимости ФУ, можно показать,
        что полученное решение $u = u\left(x\right)$
        будет непрерывно в соответствующих окрестностях
        точек $x_0$ и $u_0$.
        \item Если к условиям доказанной теоремы для
        СФУ присоединить условие непрерывной 
        дифференцируемости
        $F = \left(F_1, F_2, \dots, F_m\right)$ по
        $x = \left(x_1, x_2, \dots, x_n\right)$, т.е 
        \[\exists\:
        \pderiv{F_k\left(x, u\right)}{x_j},\ 
        j = \overline{1, n},\ k = \overline{1, m}\] то в некоторых окрестностях
        этих точек полученные решения 
        $u = u\left(x\right)$ будет не только 
        непрерывны, но и дифференцируемы.
        Для нахождения частных производных решения
        рассматриваемой СФУ надо продифференцировать
        заданные уравнения по правилу сложной функции и
        далее из полученной системы найти требуемые частные производные. 
    \end{enumerate}
\end{rems}

\begin{exmp}
    Рассмотрим $n = 2$ и $m = 2$ для системы
    \[
    \begin{cases}
        x + y + u_1 + u_2 = 0 \\
        x^2 + y^2 + u_1^2 + u_2^2 = 6
    \end{cases}
    \]
    которая разрешается относительно 
    $u_1 = u_1\left(x, y\right)$ и 
    $u_2 = u_2\left(x, y\right)$
    \[
    \begin{cases}
        u_1\left(1, 0\right) = 1\\
        u_2\left(0, 1\right) = -2
    \end{cases}
    \]
    
    Во-первых, рассматриваемые начальные условия для
    $x_0 = 1,\ y_0=0\ u_{01} = 1,\ u_{02} = -2$
    удовлетворяют рассматриваемой СФУ, т.е задача 
    поставлена корректно.
    Во-вторых, рассмотрим
    
    \[
    \begin{cases}
        F_1(x, y, u_1, u_2) = x + y + u_1 + u_2 \\
        F_2(x, y, u_1, u_2) = x^2 + y^2 + u_1^2 + u_2^2 - 6
    \end{cases}     
    \]
    
    Якобиан в начальной точке равен 
    \[
    I = \begin{bmatrix}
        \pderiv{F_1}{u_1} & \pderiv{F_1}{u_2} \\
        \pderiv{F_2}{u_1} & \pderiv{F_2}{u_2}
    \end{bmatrix}
    \Bigg|_{
    	\substack{
    		(x_0,y_0) = (1,0)\\
    		(u_{01},u_{02}) = (1,-2)
    	}} = \dots =  2(u_1 - u_2) \Big|_{(1, -2)} = -6 \neq 0,
    \]
    
    поэтому в силу непрерывной дифференцируемости рассматриваемых
    функций $F_1,\ F_2$ на основании
    предыдущей теоремы и замечания к ней 
    рассматриваемая СФУ имеет единственное 
    дифференцируемое решение
    
    \[
    \begin{cases}
        u_1 = u_1\left(x, y\right) \\
        u_2 = u_2\left(x, y\right), 
    \end{cases} 
    \]
    удовлетворяющее заданным начальным условиям.
    Для нахождения частных производных этого решения 
    в силу исходной СФУ имеем
    
    \[
    \begin{cases}
        d\left(x + y + u_1 + u_2\right) = 0 \\
        d\left(x^2 + y^2 + u_1^2 + u_2^2\right) = 0
    \end{cases} \implies
    \]
    \[
    \implies
    \begin{cases}
        du_1 + du_2 = -\left(dx + dy\right)\\
        2u_1du_1 + 2u_2du_2 = -\left(2xdx + 2ydy\right),
    \end{cases}
    \]
    откуда при $x = 1$, $y = 0$, $u_1 = 1$, $u_2 = -2$
    получаем
    
    \[
    \begin{array}{l}
    \begin{cases}
        du_1 = du_2 = -\left(dx + dy\right) \\
        2du_1 - 4du_2 = -2dx
    \end{cases} \implies \\ \implies
    \begin{cases}
     du_1 = du_2 = -\left(dx + dy\right) \\
     du_1 - 2du_2 = -dx
    \end{cases} \implies
    \left[\text{сначала разность,
    1-ю строку $\cdot$ 2 $+$ ко 2-й}\right] \implies \\ \implies
    \begin{cases}
        3du_2 = -dy \\
        3du_1 = -3dx + 2dy \\
    \end{cases} \implies \\ \implies
    \begin{cases}
        du_2 = -\frac{1}{3}dy \\
        du_1 = -dx-\frac{2}{3}dy
    \end{cases}
    \end{array}
    \]

    Отсюда для частных производных получаем
    \[
    \pderiv{u_1}{x} = -1 \ \ \ 
    \pderiv{u_2}{x} = 0\\
    \]
    
    \[
    \pderiv{u_1}{y} = \frac{2}{3} \ \ \
    \pderiv{u_2}{y} = -\frac{1}{3}
    \]
    
    В дальнейшем при необходимости аналогичным образом
    находим дифференциалы и частные производные 
    высших порядков.
\end{exmp}

\section{Зависимые и независимые CФ}
Пусть в некоторой общей области $D \subset \R^n$ 
заданы $m$ ФНП
$f_k\left(x\right),\ k = \overline{1, m},\ x \in D$.
Эта система функций называется \emph{зависимой}, 
если $\exists H$ от $(m - 1)$ переменной такая, что 
какая-то из функций $f_j,\ 1 \leq j \leq m$, 
выражается через другие:
\begin{equation}
\label{lec9-depend}
f_j = H(f_1, f_2, \dots, f_{j-1}, f_{j+1}, \dots, f_m)
\end{equation}

Система, не являющаяся зависимой, называется \emph{независимой}.

Будем рассматривать зависимость (независимость) ФНП
в соответствующих окрестностях внутренних точек
$V(x_0) \subset D$; $x_0 \in D$

В этом случае определена $m\times n$ 
прямоугольная матрица Якоби
\begin{equation}
 \label{lec_9,num_23}
 \begin{bmatrix}
        \pderiv{f_1}{x_1} 
        & \pderiv{f_1}{x_2} & \dots & 
        \pderiv{f_1}{x_n} \\
        \pderiv{f_2}{x_1} & \pderiv{f_2}{x_2} &\dots & 
        \pderiv{f_2}{x_n} \\
        \vdots & \vdots & \ddots & \vdots \\
        \pderiv{f_m}{x_1} & \pderiv{f_m}{x_2} & \dots & 
        \pderiv{f_m}{x_n} 
    \end{bmatrix}
\end{equation}

В дальнейшем будем рассматривать миноры $r-$ого порядка
для \eqref{lec_9,num_23}, т.е определители, 
составленные из элементов \eqref{lec_9,num_23}, 
стоящих на пересечении каких-либо $r$ строк или 
$r$ столбцов, где 
$r \leq \min\left\{ m, n\right\}$.

\end{document}
