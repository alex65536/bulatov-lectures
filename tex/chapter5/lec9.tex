\makeatletter
\def\input@path{{../../}}
\makeatother
\documentclass[../../main.tex]{subfiles}

\graphicspath{
  {../../img/}
  {../img/}
  {img/}
}

\begin{document}

\begin{exmp}
Зная производные и дифференциал 1-ого порядка находится последовательно
производная и дифференциал высших порядков. \\
Рассмотрим функцию:
$\begin{cases}
    x + y + z = e^z \\
    z = \left(x, y\right). \\
\end{cases}$

Имеем:
\[
\begin{array}{l}
d\left(x + y + z\right) = d(e^2) \\
dx + dx + dz = e^zdz \implies dz = \frac{dx}{e^z - 1}
+ \frac{dy}{e^z - 1} = z'_xdx + z'_ydy \\
z'_x = \frac{1}{e^z - 1};\ z'_y = \frac{1}{e^z - 1}
\end{array}
\]

\[z''_x = \left(z'_x\right)'_x = \left(\frac{1}{e^z - 1}\right)'_x
= -\frac{\left(e^z\right)'_x}{\left(e^z - 1\right)^2} =
- \frac{e^zz'x}{\left(e^z - 1\right)^2} = \left[z`_x
= \frac{1}{e^z - 1} \right] =
- \frac{e^z}{\left(e^z - 1\right)^3}\]

В силу того что $\displaystyle z''_{yz} = ... = - 
\frac{e^z}{\left(e^z - 1\right)^3}$:

\[z''_{yz} = \left(z'_x\right)'_y = \left(\frac{1}{e^z - 1} \right)'_x = ... =
- \frac{e^z}{\left(e^z - 1\right)^3}\]

Отсюда
\[d^2z = z''_{x^2}dx^2 + 2z''_{xy} + z''_{y^2}dy^2 =
-\frac{dx^2 + 2dxy + dy^2}{\left(e^z - 1\right)^3} e^z = 
-\frac{\left(dx + dy\right)^2}{\left(e^z - 1\right)^3} e^z\]
и так далее.
\end{exmp}

\section{Системы функциональных уравнений (СФУ)}
Пусть имеется $n$ функций от $\left(n + m\right)$ переменных:
\[\begin{cases}
    F_k\left(x, u \right),\ k = \overline{1, m} \\
    x = \left(x_1, x_2, ..., x_n\right) \in D \subset \R^n \\
    u = \left(u_1, u_2, ..., u_m\right) \in G \subset \R^m
\end{cases}
\]

Под \emph{СФУ} будем подразумевать систему вида

\begin{equation}
    \label{lec_9.num_7}
    \begin{cases} 
        F_k\left(x, u\right) = 0 \\ 
        k = \overline{1, m} \\
    \end{cases}    
\end{equation}

Предполагается, что в \eqref{lec_9.num_7}~--- независимая
переменная, а $u = u\left(x\right)$~--- искомые функции.

Наряду с \eqref{lec_9.num_7} будем рассматривать начальные условия
\begin{equation}
    \label{lec_9.num_8}
    \begin{cases}
        F_k\left(x_0, u_0\right) = 0, \\
        x = \overline{1, m}, \\
    \end{cases}
\end{equation}
в соответствии с которыми под решением системы
\eqref{lec_9.num_7}, \eqref{lec_9.num_8}
подразумевается $u(x_1, x_2, ..., x_n)$,
удовлетворяющее условию
$u\left(x_0\right) = u_0$

При этом в соответствующих окрестностях точек $x_0$, $y_0$
решение системы \eqref{lec_9.num_7}
с начальными условиями 
\eqref{lec_9.num_8} удовлетворяют тождеству:
\[F(x, u(x)) = 0\]

Пусть $F = \left(F_1, F_2, ...,F_m\right) \in \R^m$,
тогда система \eqref{lec_9.num_7}
с начальными условиями 
\eqref{lec_9.num_8} можно записать в виде:

\begin{equation}
    \label{lec_9.num_9}
    \begin{cases}
     F\left(x, u\right) = \vec{0} \in \R^m\\
     F\left(x_0, u_0\right) = \vec{0} \in \R^m\\
    \end{cases}
\end{equation}


В дальнейшем для системы СФУ \eqref{lec_9.num_7} 
будем рассматривать \emph{матрицу Якоби}, которую будем обозначать:

\begin{equation}
    \label{lec_9.num_10}
    \pderiv{F}{u} =
    \begin{bmatrix}
        \pderiv{F_1}{u_1} & \pderiv{F_1}{u_2} & \dots & 
        \pderiv{F_1}{u_m} \\
        \pderiv{F_2}{u_1} & \pderiv{F_2}{u_2} &\dots & 
        \pderiv{F_2}{u_m} \\
        \vdots & \vdots & \ddots & \vdots \\
        \pderiv{F_m}{u_1} & \pderiv{F_m}{u_2} & \dots & 
        \pderiv{F_m}{u_m}
    \end{bmatrix}
\end{equation}

Определитель квадратной матрицы Якоби 
    \eqref{lec_9.num_10} называется \emph{якобианом}
и записывается в виде:
\begin{equation}
    \label{lec_9.num_11}   
    I = det\pderiv{F\left(x,u\right)}{u} =
    \pderiv{\left(F_1,F_2,...,F_m\right)}{\left(u_1, u_2,..., u_m\right)}
\end{equation}

\begin{thm}[Теорема(об однозначности решения СФУ)]
Пусть $F_k\left(x, n\right)$, $k = \overline{1, n}$
по $(n+ m)$ переменным 
$x = \left(x_1, x_2, ..., x_n\right) \in D \in \R^n,
u = \left(u_1, u_2, ..., u_m\right) \in G \in \R^m$
непрерывны по $x и по u$. В некоторых окрестностях точек
$x_0 = \left(x_{01}, x_{02}, ..., x_{0n}\right) \in D 
\subset \R^n. $
$u_0 = \left(u_{01}, u_{02}, ..., u_{0n}\right) \in D
\subset \R^m$

Если выполнено первоначальное условие
\begin{equation}
    \label{lec_9.num_12}
    F_k(x_0, u_0) = \vec{0}, k = \overline{1, m},
\end{equation}
то в случае непрерывной дифференцируемости
\begin{equation}
    \label{lec_9.num_13}
    \left(F_1, F_2, ..., F_m\right) \in \R^m
\end{equation}

по $u = \left(u_1, u_2, ..., u_m\right) \in \R^m$,
то если якобиан
$I(x, u) = det\pderiv{F\left(x, u\right)}{u}$, 
рассматриваемой СФУ $F\left(x, u\right) = 0$
в точке $\left(x, y\right)$
\begin{equation}
    \label{lec_9.num_14}
    I(x, u) \neq 0 
\end{equation}

То тогда система \eqref{lec_9.num_7} будет иметь в 
соответствующих окрестностях точек $x_0\ \text{и}\ y_0$
единственное решение $\exists!\ u(x_0) = u_0$
\end{thm}

\begin{proof}
    Доказательство проведём используя ММИ.
    Во-первых, при $m = 1$ СФУ \eqref{lec_9.num_7} 
    даёт ФУ \eqref{lec8.1:1} для которой теорема об
    однозначности решения уже доказана.
    Во-вторых, предполагая, что теорема доказана
    при $m = k,\ k \in \N$, рассмотрим случай
    $m = k + 1$.
    Из условия \eqref{lec_9.num_14} в силу правила Лапласа
    вычисления определителя по разложению по какой-либо
    строке, столбцу следует, что в силу \eqref{lec_9.num_14} 
    в точках $x_0,\ y_0$ хотя бы один из миноров 
    $k$-ого порядка рассматриваемого якобиана не нулевой.
    Без ограничения общности,(перестановкой строк, столбцов)
    считаем, что этот минор $k$-ого порядка
    \begin{equation}
        \label{lec_9.num_15}
        I = \pderiv{\left(F_1, F_2, ..., F_k\right)}
        {\left(u_1, u_2, ..., u_k\right)},\ (x_0, y_0) \neq 0
    \end{equation}
    является угловым минором матрицы Якоби, рассматриваемой
    СФУ. Тогда, во-первых, в силу индуктивного
    предположения, при $m = k$
    \begin{equation}
        \label{lec_9.num_16}
        \begin{cases}
            F_i\left(x, v, u_{k+1}\right) = 0, \\
            i = \overline{1, k}
        \end{cases}
    \end{equation}
    
    Для $fix\ u_{k+1}$ разрешена относительно 
    $v = \left(u_1, u_2, ..., u_k\right)$, т.е. 
    $\exists!v = v\left(x, u_{k+1}\right)$, удовлетворяющее
    \eqref{lec_9.num_16} соответствующих окрестностях
    рассматриваемых точек и при этом выполнено условие\
    $v\left(x_0\right) = u\left(x_0, u_{0k+1}\right) = v_0$
    
    Представляя это решение в последнем уравнении из
    \eqref{lec_9.num_7} получим уравнение вида
    \begin{equation}
        \label{lec_9.num_17}
        H(x, u_{k+1}) = 0
    \end{equation}
    
    \begin{equation}
        \label{lec_9.num_18}
        H\left(x, u_{k+1}\right) = 
        F_{k+1}\left(x, v\left(x, u{k+1}\right), 
        u_{k+1}\right))
    \end{equation}
    
    \eqref{lec_9.num_17} определяет некоторую функцию
    \begin{equation}
        \label{lec_9.num_19}
        \begin{cases}
        F = u_{k+1}\left(x\right)\\
        u_{k+1}\left(x_0\right) = u_{0k+1}
        \end{cases}
    \end{equation}
    
    Осталось показать, что \eqref{lec_9.num_18} с начальным
    условием имеет единственное решение относительно 
    $u_{k=1}$ в рассматриваемых окрестностях начальных
    точек. Для этого, в силу теоремы о однозначности СФУ
    достаточно проверить, что
    \begin{equation}
    \label{lec_9.num_20}
    det\pderiv{H\left(x, u_{k+1}\right)}{u_{k+1}} 
    \vline\underset{\left(x_0, u_{0k+1}\right)}{} \neq 0
    \end{equation}
    
    Дифференцируя равенство
    \begin{equation}
    \label{lec_9.num_21}
    F_{k+1}\left(x, v\left(x, u_{k+1}\right), 
    u_{k+1}\right) \equiv 0
    \end{equation}
    
    по $u_{k+1}$ соответствующих точек $x_0\ \text{и}\ u_0$
    получаем
    \begin{equation}
        \label{lec_9.num_22}
        \sum_{j = 1}^{n} = 
        \pderiv{F_j}{v_j} \cdot \pderiv{v_j}{u_{k+1}} + 
        \pderiv{F}{u_{k+1}} \equiv 0
    \end{equation}
    
    Если прибавить к $\left(k + 1\right)$ столбцу 
    соответствующую ???? строку якобиана, то в результате
    получим, что для якобиана
    $I\left(x_0, u_0\right)= \dots = 
    \pderiv{H\left(x_0,u_{k+1}\right)}{u_{k+1}} \neq 0$,\\
    поэтому для полученного ФУ выполнены все условия для 
    однозначности решимости ФУ и значит 
    $\exists! u_{k+1} = u_{k+1}\left(x_0\right)$, 
    удовлетворяющее, во-первых, начальному условию
    $u_{k+1 = u_{0k+1}}$ и, во-вторых, после подстановки
    заданной функции получим 
    $v = v\left(x, u_{k=1}\right)$ получим функцию,
    которая удовлетворяет на рассматриваемом этапе
    соответствующему СФУ.
    При этом для $u = u\left(x\right)$ выполнено начальное
    условие $u\left(x_0\right) = u_0$.
\end{proof}

\begin{remark}

~

    \begin{enumerate}
        \item Как и для ФУ при выполнении условии 
        теоремы о однозначности решения ФУ можно показать,
        что полученное решение, равное $u\left(x_0\right)$,
        будет непрерывно в соответствующих окрестностях
        точек $x_0\ \text{и}\ u_0$
        \item Если к условиям доказанной теоремы для
        СФУ присоединить условие для непрерывной 
        дифференцируемости
        $F = \left(F_1, F_2, ..., F_m\right)$ по
        $x = \left(x_1, x_2, ..., x_n\right)$, т.е 
        существует
        $\pderiv{F_k\left(x, u\right)}{x_j}, 
        j = \overline{1, n}$ то в некторых окрестностях
        этих точек полученные решения 
        $u = u\left(x\right)$ будет не только 
        непрерывны, но и дифференцируемы.
        Для нахождения частных производных решения
        рассматриваемых СФУ надо продифференцировать
        заданные уравнения по правилу сложения функции.
        Далее из полученной системы найти требуемую частную производную. 
    \end{enumerate}
\end{remark}

\begin{exmp}
    Рассмотрим $n = 2$ и $m = 2$ для системы
    \[
    \begin{cases}
        x + y + u_1 + u_2 = 0 \\
        x^2 + y^2 + u_1^2 + u_2^2 = 6
    \end{cases}
    \]
    которая разрешается относительно 
    $u_1 = u\left(x, y\right)\ \text{и}
    u_2 = u\left(x, y\right)$
    \[
    \begin{cases}
        u_1\left(1, 0\right) = 1
        u_2\left(0, 1\right) = -2
    \end{cases}
    \]
    
    Во-первых, рассматриваемые первоначальные условия для
    $x_0 = 1,\ y_0=0\ u_{01} = 1,\ u_{02} = -2$
    удовлетворяет рассматриваемой СФУ, т.е задача 
    поставлена корректно.
    Во-вторых
    
    \[
    \begin{cases}
        F_1(x, y, u_1, u_2) = x + y + u_1 + u_2 \\
        F_2(x, y, u_1, u_2) = x^2 + y^2 + u_1^2 + u_2^2
    \end{cases}     
    \]
    для соответствующего якобиана в начальной точке 
    получаем
    \[
    I = \begin{bmatrix}
        \pderiv{F_1}{u_1} & \pderiv{F_1}{u_2} & \dots & 
        \pderiv{F_1}{u_m} \\
        \pderiv{F_2}{u_1} & \pderiv{F_2}{u_2} &\dots & 
        \pderiv{F_2}{u_m} \\
        \vdots & \vdots & \ddots & \vdots \\
        \pderiv{F_m}{u_1} & \pderiv{F_m}{u_2} & \dots & 
        \pderiv{F_m}{u_m} 
    \end{bmatrix}
    \vline{
    \begin{cases}
        x = 1\\
        y = 0\\
        u_{01} = 1\\
        u_{02} = -2
    \end{cases}
    }{} = \dots =  2(u_1 - u_2) = -6 \neq 0
    \]
    
    В силу непрерывной дифференцируемости рассматриваемых
    функций $F_1\text{,}\ F_2$ на основании
    предыдущей теоремы и замечаний к ней 
    рассматриваемая СФУ имеет единственное 
    дифференцируемое решение
    
    \[
    \begin{cases}
        u_1 = u_1\left(x, y\right) \\
        u_2 = u_2\left(x, y\right), 
    \end{cases} 
    \]
    удовлетворяющих заданным начальным условиям.
    Для нахождения частных производных этого решения 
    в силу исходной СФУ имеем
    
    \[
    \begin{cases}
        d\left(x + y + u_1 + u_2\right) = 0 \\
        d\left(x^2 + y^2 + u_1^2 + u_2^2\right) = 0
    \end{cases} \implies
    \]
    \[
    \implies
    \begin{cases}
        du_1 + du_2 = -\left(dx + dy\right)\\
        2u_1du_1 + 2u_2du_2 = -\left(2xdx + 2ydy\right)
    \end{cases}
    \]
    
    Откуда при $x = 1$, $y = 0$, $u_1 = 1$, $u_2 = -2$,
    получаем
    \[
    \begin{cases}
        du_1 = du_2 = -\left(dx + dy\right) \\
        2du_1 - 4du_2 = -2dx
    \end{cases}
    \]
    
    \[
    \begin{cases}
     du_1 = du_2 = -\left(dx + dy\right) \\
     du_1 - 2du_2 = -dx
    \end{cases}
    \vline{\text{сначала разность,
    1-ю строку * 2 + ко 2-й}}{}
    \]
    
    \[
    \begin{cases}
        3du_2 = -dy \\
        3du_1 = -3dx + 2dy \\
    \end{cases}
    \]
    
    \[
    \begin{cases}
        du_2 = -\frac{1}{3}dy \\
        du_1 = -dx-\frac{2}{3}dy
    \end{cases}
    \]

    Отсюда для частных производных получаем
    \[
    \pderiv{u_1}{x} = 1 \ \ \ 
    \pderiv{u_2}{x} = 0\\
    \]
    
    \[
    \pderiv{u_1}{y} = \frac{2}{3} \ \ \
    \pderiv{u_2}{y} = -\frac{1}{3}
    \]
    
    В дальнейшем при необходимости аналогичным образом
    находим дифференциалы и частные производные 
    высших порядков.
\end{exmp}

\section{Зависимые и независимые ФУ}
Пусть в некоторой общей области $D \subset \R^n$ 
заданы $m$ ФНП
$f_k\left(x\right), k = \overline{1, m}, x \in D$
Эта система функций называется зависимой, если 
\[
\exists H = H\left(f_1\left(x\right), 
f_2\left(x\right), ..., f_m\left(x\right)\right) = 0,
\ x \in D
\]

Система не являющаяся зависимой называется независимой.
Будем рассматривать зависимость (независимость) ФНП
в соответствующих окрестностях внутренних точек и 
соответствующих окрестностях 
$V(x_0) \in D$; $x_0 \in D$

В этом случае определена $\left(m*n\right)$ 
прямоугольная матрица Якоби
\begin{equation}
 \label{lec_9,num_23}
 \begin{bmatrix}
        \pderiv{f_1}{x_1} 
        & \pderiv{f_1}{x_2} & \dots & 
        \pderiv{f_1}{x_n} \\
        \pderiv{f_2}{x_1} & \pderiv{f_2}{x_2} &\dots & 
        \pderiv{f_2}{x_n} \\
        \vdots & \vdots & \ddots & \vdots \\
        \pderiv{f_m}{x_1} & \pderiv{f_m}{x_2} & \dots & 
        \pderiv{f_m}{x_n} 
    \end{bmatrix}
\end{equation}

В дальнейшем будем рассматривать миноры $r-$ого порядка
для \eqref{lec_9,num_23}, т.е определители, 
составленные из элементов \eqref{lec_9,num_23}, 
стоящих на пересечении каких-либо $r$ строк или 
$r$ столбцов, где 
$r \leq min\left\{ m, n\right\}$









\end{document}
