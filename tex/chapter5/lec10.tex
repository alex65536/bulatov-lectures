\makeatletter
\def\input@path{{../../}}
\makeatother
\documentclass[../../main.tex]{subfiles}

\graphicspath{
	{../../img/}
	{../img/}
	{img/}
}

\begin{document}
\begin{thm}[признак функциональной независимости системы функции]
		
	Пусть функции $f_k\left(x\right)$, $k = \overline{1,n}$,
	$x \in D \subset \R^n$ непрерывно дифференцируемы в некоторой 
	окрестности $V(x_0) \in D$ внутренней точки $x_0 \in D$. Если 
	для матрицы Якоби \ref{lec_9,num_23} в точке $x_0 \implies$ 
	\begin{equation} \label{lec_10, num_24}
		rankA = m
	\end{equation}
 	то тогда рассматриваемая функциональная система 
 	является независимой в $V(x_0)$ 
 	
\end{thm}

\begin{proof}
	От противного:
	
	Пусть для рассматриваемой функциональной системы (ФС) 
	выполнено \ref{lec_10, num_24} в точке $x_0$, но эта система 
	зависима в $V(x_0) \subset D$
	
	Из \ref{lec_10, num_24} $\implies$ хотя бы 1 минор $m$-ого 
	порядка матрицы Якоби в точке $x_0$ ненулевой.
	
	Предположим, что таким минором является якобиан:
	
	\begin{equation} \label{lec_10, num_25}
		I_0 = \pderiv{\left( f_1, \ldots, f_m \right)}{ \left(x_1, 
		\ldots, x_m \right) } = det 
		\begin{bmatrix}
		\pderiv{f_1}{x_1} & \pderiv{f_1}{x_2} & \ldots & \pderiv{f_1}{x_m} \\
		\pderiv{f_2}{x_1} & \pderiv{f_2}{x_2} & \ldots & \pderiv{f_2}{x_m}\\
		\vdots & \vdots & \ddots & \vdots \\
		\pderiv{f_m}{x_1} & \pderiv{f_m}{x_2} & \ldots & \pderiv{f_m}{x_m}
		\end{bmatrix} \Bigg|_{x_0}
		\neq 0
	\end{equation} 
	
	Из предположения зависимости ФУ в силу равенства:
	$\exists f_j = H\left( f_1, \ldots, f_{j-1}, f_{j+1}, \ldots, 
	f_n\right), 1 \le j \le m$
	После последовательного дифференцирования по $x_1, \ldots, 
	x_n$ в точке $x_0$ получим:
	\begin{equation} \label{lec_10, num26}
		\pderiv{f_j(x_0)}{x_i} = \pderiv{H}{t_1} =
		\pderiv{f_1(x_0)}{x_i} + \ldots + \pderiv{H}{ t_{j-1} } \cdot
		\pderiv{ f_{j-1}(x_0) }{x_i} + \pderiv{H}{ t_{j+1} } \cdot
		\pderiv{ f_{j+1}(x_0) }{x_i} + \ldots + \pderiv{H}{t_m} \cdot
	    \pderiv{ f_m(x_0) }{x_i}, i = \overline{1, m}
	\end{equation}
	
	где $H = H\left( t_1, \ldots, t_{j-1}, t_{j+1}, \ldots, t_m 
	\right)$ ~--- функция (n-1)-ой переменной
	
	Если вместо $j$-ой строки в якобиане \ref{lec_10, num_25}
	подставить равенство \ref{lec_10, num26}, то эта $j$-ая 
	строка будет линейной комбинацией строк с соответствующими 
	коэффициентами. А тогда по свойствам определителей, 
	используемый минор $I_0$ будет нулевым, что противоречит 
	\ref{lec_10, num_25}.
	
	Поэтому исходное предположение неверно.
\end{proof}

\begin{remarks}

	
	
	\begin{itemize}
		Полученные результаты обощаются в следующем виде:
		
		\item[1)]
		
		Если в матрице Якоби \ref{lec_9,num_23} есть ненулевой 
		минор $r$-ого порядка, а все миноры $(r+1)$-ого: нулевые 
		в рассматриваемой точке $x_0$ (т.е $rankA = r$), то в 
		рассматриваемой функциональной системе есть подмножество 
		из $r$ независимых функций в соответствующей окрестности 
		$V(x_0)$. А все остальные функции будут выражаться через 
		эти $r$ независимых функций
		
		\item[2)]
		
		Кроме общей функциональной зависимости и независимости 
		функций расссматривают линейно зависимые и независимые 
		системы функций. В этом случае используют функцию 
		$H\left( t_1, \ldots, t_{j-1}, t_{j+1}, \ldots, t_m 
		\right) = \alpha_1 t_1 + \ldots + \alpha_{j-1} t_{j-1} + 
		\alpha_{j+1} t_{j+1} + \ldots + \alpha_m t_m$ ~--- 
		линейная функция от $(m-1)$ переменной с действительными 
		коэффициентами: $\alpha_1, \ldots, \alpha_{j-1}, 
		\alpha_{j+1}, \ldots, \alpha_m \in \R$
	
		Нетрудно видеть, что из линейной зависимости следует 
		общая функциональная зависимость. А из функциональной 
		независимости следует линейная независимость.
	\end{itemize}
\end{remarks}

\begin{examples}
	\begin{itemize}
		\;
		\item[1)]
		
		Пусть $n = 1$, $m = 2$
		
		\[ \begin{cases}
		f_1(x) = \cos x \\
		f_2(x) = \sin x \\
		x \in \R
		\end{cases}  \]
		
		В данном случае эта система линейно зависима, так как 
		$\forall x \in \R \implies f_1^2 + f_2^2 = 1 $
		
		В тоже время система линейно независима, так как если 
		$\exists \alpha_1, \alpha_2 \in \R $, такие, что
		$\alpha_1 \cos x + \alpha_2 \sin x = 0, \forall x \in 
		\R$, то для $x = 0 \textrm{ и } x = \pi / 2$:
		
		\[\begin{cases}
		\alpha_1 * 1 + \alpha_2 * 0 = 0 \\
		\alpha_1 * 0 + \alpha_2 * 1 = 0 
		\end{cases} \implies \alpha_1 = \alpha_2 = 0 \] 
		
		Что гарантирует линейную независимость.
		
		\item[2)] 
		
		Пусть $n = 2, m = 2$	
		
		\[ \begin{cases}
		f_1 \left( x_1, x_2 \right) = x_1 - 2x_2 \\
		f_2 \left( x_1, x_2 \right) = 2x_1 + x_2
		\end{cases} \]
		
		В данном случае для $\forall x_1, x_2 \in \R^2$, для Якобиана получаем:
		
		\[ I = \det \begin{bmatrix}
		\pderiv{f_1}{x_1} & \pderiv{f_1}{x_2} \\
		
		\pderiv{f_2}{x_1} & \pderiv{f_2}{x_2}
		\end{bmatrix} = 5 \ne 0 \]
		
		Данная система в силу теоремы об однозначной разрешимости ФУ (?) 
		будет функционально независима,
		а так же в этой ситуации она будет линейно независима, 
		так как из условия $\alpha_1 f_1 + \alpha_2 f_2 = 0, 
		\forall x_1, x_2 \in \R^2 \implies \left(\alpha_1 + 
		2\alpha_2\right) x_1 + \left( -2\alpha_1 + \alpha_2 
		\right) x_2 = 0$
		Откуда при $x_1 = 1, x_2 = 0 \implies \alpha_1 + 2\alpha_2 = 0$
		
		При $x_1 =0, x_2 = 1 \implies -2\alpha_1 + \alpha_2 = 0$
		
		Решая систему
		\[ \begin{cases}
		\alpha_1 + 2\alpha_2 = 0 \\
		-2\alpha_1 + \alpha_2 = 0
		\end{cases} \implies \alpha_1 = \alpha_2 = 0 \implies \]
		
		$\implies$ линейная независимость$f_1 \text{ и } f_2$
		
	\end{itemize}
	
\end{examples}

\begin{remark}
	\;
	
	Доказанный признак (достаточное условие функциональной 
	независимости системы функций) 
	в случае линейной зависимости (независимости) будет также и 
	достаточным условием, т.е критерием линейной независимости.
\end{remark}


	
	
	
\end{document}