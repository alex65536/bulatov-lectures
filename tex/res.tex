\makeatletter
\def\input@path{{../}}
\makeatother
\documentclass[../main.tex]{subfiles}

\begin{document}\begin{thm}[неравенство Коши-Буняковского]
 В любом линейном евклидовом пространстве $P$ над полем $\R$
 выполняется неравенство
 
 \begin{equation}
  \label{kosh-bun}
  (\left<x, y\right>)^2 \leq \left(\left<x, x\right>\right) 
  \cdot \left(\left<y, y\right>\right)
 \end{equation}
\end{thm}
\begin{crl*}[неравенство Минковского]
 В линейном евклидовом пространстве $P$ для любых $x, y \in P$ имеем:
 
 \begin{equation}
  \label{mink}
  \sqrt{\left<x+y, x+y\right>} \le
  \sqrt{\left<x, x\right>} + \sqrt{\left<y, y\right>}
 \end{equation}

\end{crl*}
\begin{thm}[О нормируемости линейного евклидова пространства $\R^n$]
 Любое линейное евклидово пространство $\R^n$ с произвольным скалярным
 произведением нормируется с помощью естественной нормы
 
 \begin{equation}
  \label{rn-norm}
  \norm x = \sqrt{\left<x, x\right>} \quad \forall x \in \R^n
 \end{equation}
 
\end{thm}
\begin{thm}[О метризируемости произвольного линейного нормируемого 
	пространства $\R^n$]
 Любое линейное нормированное пространство $\R^n$ с нормой
 \eqref{norm-disp} метризируется с помощью естественного расстояния

 \begin{equation}
  \label{nat-dist}
  \rho(x, y) = \norm{x-y}\quad \forall x, y \in \R^n
 \end{equation}
 
\end{thm}
\begin{thm}[о непрерывности композиции ФНП]
    	Если $n$ функций $g_k(t),\ k = 
    	\overline{1,n}$ от $m$ переменных $t = (t_1, \ldots, t_m) \in D(g) 
    	\subset \R^m$ непрерывны в точке $t_0 = \left( t_{01}, \ldots, t_{0m} 
    	\right) \in D(g),$ где $g = (g_1, \ldots, g_n)$, a $f(x) = 
    	f(x_1, \ldots, x_n)$ непрерывна в точке $x_0 = g(t_0) \in D(f)$,
    	то, в случае существования композиции 
    	\begin{equation}
    	\label{composition}
    		h(t) = f(g(t)) = f(g_1(t), g_2(t), \ldots, g_n(t))
    	\end{equation}
        сложная функция \eqref{composition} будет непрерывна в точке 
        $t_0 \in D(g)$.
    \end{thm}
\begin{crl*}[о пределе композиции ФНП]
    	Пусть $\exists \lim\limits_{t \to t_0}g_k(t) = p_k, \ 
    	\forall k = \overline{1, n}, \  g(t) = (g_1(t), 
    	\ldots, g_m(t)),\ t = (t_1,
    	 \ldots, t_m) \in D(g) \subset \R^m,$ а $t_0 = 
    	 (t_{01}, \ldots, t_{0m})$~---
    	предельная точка для $D(g)$. Если функция $f(x)$ непрерывна в точке 
    	$x_0 = (p_1, \ldots, p_n) \in D \subset \R^n$, то в случае 
    	существования композиции $h(t) = f(g(t))$ соответствующих окрестностей
    	$t_0$ и $x_0$ 
    	\[
    	    \exists \lim\limits_{t \to t_0}h(t) = \lim\limits_{t \to t_0}f(g(t)) =
    		\left[f\text{~--- непрерывна}\right] = f(\lim\limits_{t \to t_0}g(t)) = 
    		\left[
    		\begin{array}{l}
    			x = g(t) \underset{t \to t_0}{\to} x_0 \\
    			x_0 = (p_1, \ldots, p_n) 
    		\end{array}
    		\right] = 
    	\]
    	\[
    		= f(p_1, \ldots, p_n) = f(x_0).
    	\]
    \end{crl*}
\begin{thm}[О промежуточном значении непрерывной на связном множестве ФНП]
		Пусть $f \in C(D), D \subset \R^n$~--- связное множество, то есть 
		множество, у которого две любые точки можно соединить линией  $\in D$.
		Если $f(x)$ принимает на $D$ значения $u_1, u_2$,то есть $\exists 
		a, b \in D: \ u_1 = f(a), u_2 = f(b),$ тогда $u = f(x)$ принимает
		на $D$ все промежуточные значения между $u_1$ и $u_2$.
	\end{thm}
\begin{crl*}[О прохождении непрерывной ФНП через ноль]
			Если $f(x)$ непрерывна на $D \subset \R^n$ и $\exists a,b \in D: 
			f(a) \cdot f(b) > 0$, то $\exists x_0 \in D : f(x_0) = 0.$
	\end{crl*}
\begin{thm}[Вейерштрасс]
		Если $f(x)$ непрерывна на компакте $D \subset \R^n$(ограниченном 
		замкнутом множестве), то $\exists\ \overline{x}, \widetilde{x} \in D:$
		\[ 
		\begin{cases}
		\min f(x) = f(\overline{x}),& x \in D, \\
		\max f(x) = f(\widetilde{x}),& x \in D
		\end{cases}
		\]
	\end{thm}
\begin{thm}[Кантор]
		Если $f(x)$ непрерывна на компакте $D \subset \R^n$, то $f(x)$~---
		равномерно непрерывна на $D$.
	\end{thm}
\begin{thm}[Критерий сходимости n-мерных последовательностей]
    Последовательность $M_k = (x_{k1}, \dots, x_{kn}) \in \R^n, k 
    \in \N$ сходится в линейном метрическом пространстве $(\R^n, d)$
    с евклидовой метрикой $d$ к точке $M_0 = (x_{01}, \dots, x_{0n})
    \in \R^n$ тогда и только тогда, когда
    \begin{equation}
    \label{krit-posl}
      \forall fix\ j = \overline{1, n} \implies
      x_{kj}\underset{k\rightarrow\infty}{\longrightarrow} x_{0j}
    \end{equation}
  \end{thm}
\begin{thm}[О равенстве смешанных производных второго порядка ФНП]
		\label{secdiffeq}
		Если ФНП $u = f(x)$, $x = \left(x_1, x_2, \dots, x_n\right)\in 
		D\subset\R^n$, дифференцируема в некоторой окрестности внутренней 
		точки $x_0 = \left(x_{01}, x_{02}, \dots, x_{0n}\right)\in D$ и имеет 
		непрерывные смешанные производные $u''_{x_ix_j}(x_0)$, 
		$u''_{x_jx_i}(x_0)$,\ \ $k\ne j$,\ \ $k, j = \overline{1, n}$, то тогда, 
		в силу непрерывности этих производных в окрестности $x_0$, выполняется
		\begin{equation}
			u''_{x_ix_j} = u''_{x_jx_i}
			\label{11:diff}
		\end{equation}
	\end{thm}
\begin{thm}[достаточное условие дифференцируемости ФНП]
Пусть $f(x)$ определена в области $D \subset \R^n$. Если её
частные производные в некоторой окрестности $V(M_0)\subset D$ точки
$M_0$ существуют и непрерывны, то $f(x)$ дифференцируема в $M_0$.
\end{thm}
\begin{thm}[о дифференцировании сложных ФНП]
Пусть $g_k(t),k=\overline{1,n}$~--- функции, определённые на 
$D\subset\R^m$ и непрерывно дифференцируемые в некоторой окрестности 
точки $t_0\in D$.

Пусть также $f(x)$~--- функция, определённая на $E\subset\R^n$ и
непрерывно дифференцируемая в соответствующей окрестности точки $x_0 = 
(g_1(t_0),g_2(t_0),\dots,g_n(t_0))\in E$.

Тогда, если существует их композиция $h(t) = f(g(t)) =
f(g_1(t),g_2(t),\dots,g_n(t))$, то она также является непрерывно 
дифференцируемой в окрестности $t_0$, причём
\[\pderiv{h(t_0)}{t_j}=\sum_{i=1}^{n}
\pderiv{f(x_0)}{x_i}\cdot
\pderiv{g_i(t_0)}{t_j},\ j=\overline{1,m}\]
\end{thm}
\begin{crl}[инвариантность формы 1-го дифференциала ФНП]
Для $h(t) = f(g(t))$ в случае непрерывно дифференцируемых в 
соответствующих окрестностях $f(x), g(x)$ имеем
\[dh=\sum_{j=1}^m\pderiv{h(t)}{t_j}dt_j=
\sum_{i=1}^n\pderiv{f(x)}{x_i}dx_i,\]
где $dt_j$~--- приращение независимой переменной, а $dx_i=dg_i(t)$ --
произвольное приращение $g_i(t)$.
\end{crl}
\begin{thm}[Достаточное условие локального экстремума ФНП]
        Если у дважды дифференцируемой ФНП $u=f(x),\ x \in D \subset \R^n$
        в некоторой окрестности $V(x_0) \subset D$ внутренней стационарной
        точки $x_0 \in D$ второй дифференциал \eqref{lec8:6} является
        знакопостоянной квадратичной формой относительно $dx_k,\ k =
        \overline{1, n}$, то стационарная точка $x_0$ будет точкой локального
        экстремума ФНП. При этом если квадратичная форма \eqref{lec8:6}
        положительно определена, то стационарная точка $x_0$ является точкой
        локального минимума. Если \eqref{lec8:6} является отрицательно
        определённой квадратичной формой, то $x_0$ является точкой локального
        максимума $f(x)$.
    \end{thm}
\begin{thm}[Необходимое условие ЛЭФНП]
		Пусть ФНП
		\[
		\begin{cases}
			u = f (x), \\
			x = (x_1, \ldots, x_n) \in D \subset \R^n; \\
		\end{cases}
		\]
		дифференцируема в некоторой окрестности $V (x_0) \subset D$
		внутренней точки $x_0 \in D$. Если эта $x_0$ является экстремальной
		для $f (x)$, то $x_0$ --- стационарная точка
		для $f (x)$, т.е. для произвольных допустимых
		\begin{equation}
		\label{nec cond loc extr}
			\forall dx = \D x \in \R^n \implies df (x_0) = 0.
		\end{equation}
	\end{thm}
\begin{thm}[Об однозначной разрешимости ФУ]
		Пусть задана функция от $(n+1)$ переменной $F(x, u)$, непрерывная 
		относительно $x = (x_1, x_2, \ldots,
	 	x_n) \in \R^n$, $u \in \R$ в соответствующих окрестностях $x_0 \in D 
	 	\subset \R^n$, $u_0 \in I \subset \R$, причём $F(x_0,u_0)=0$. Если $F(x,u)$ 
	 	непрерывно дифференцируема по $u$, а также 
		\begin{equation}
			F'_u(x_0, u_0) \ne 0, \label{lec8.1:4}
		\end{equation}
		то тогда ФУ \eqref{lec8.1:1} с начальным условием \eqref{lec8.1:2} имеет 
		единственное решение $u = u(x)$ в некоторой окрестности $V(x_0) 
		\subset D$, удовлетворяющее
	 	\begin{equation}
			u(x_0) = u_0. \label{lec8.1:5}
		\end{equation}
	 \end{thm}
\begin{thm}[О дифференцировании неявных ФНП]
		Пусть наряду со всеми условиями предыдущей теоремы дополнительно 
		функция $F(x, u)$ непрерывно дифференцируема по $x$. Тогда ФУ 
		\eqref{lec8.1:1} с начальным условием \eqref{lec8.1:2} также разрешимо, 
		и его единственное решение в соответствующих окрестностях 
		рассматриваемых точек $x_0 \in D, u_0 \in I$ будет удовлетворять 
		начальному условию $u(x_0)=u_0$, а также будет дифференцируемо в 
		этой точке по $x$, а частные производные полученной ФНП $u = u(x)$ в 
		соответствующих окрестностях $x_0, u_0$ находятся по формуле
		\begin{equation}
			u'_{x_k} = -\frac{F'_{x_k}(x,u)}{F'_u(x,u)}. \label{lec8.1:6}
		\end{equation}
    \end{thm}
\begin{thm}[признак функциональной независимости системы функций]
	Пусть функции $f_k\left(x\right)$, $k = \overline{1,m}$,
	$x \in D \subset \R^n$ непрерывно дифференцируемы в некоторой 
	окрестности $V(x_0) \subset D$ внутренней точки $x_0 \in D$. Если 
	для матрицы Якоби \eqref{lec_9,num_23} в точке $x_0 \implies$ 
	\begin{equation} \label{lec_10, num_24}
		\rank A = m,
	\end{equation}
 	то тогда рассматриваемая функциональная система 
 	является независимой в $V(x_0)$.
\end{thm}
\begin{thm}[об однозначной разрешимости СФУ]
Пусть функции $F_k\left(x, u\right)$, $k = \overline{1, m}$
от $(n+ m)$ переменных 
$x = \left(x_1, x_2, \dots, x_n\right) \in D \subset \R^n$,
$u = \left(u_1, u_2, \dots, u_m\right) \in G \subset \R^m$
непрерывны по $x$ и по $u$ в некоторых окрестностях точек
$x_0 = \left(x_{01}, x_{02}, \dots, x_{0n}\right) \in D 
\subset \R^n$,
$u_0 = \left(u_{01}, u_{02}, \dots, u_{0m}\right) \in G
\subset \R^m$. Если выполнено начальное условие
\begin{equation}
    \label{lec_9.num_12}
    F_k(x_0, u_0) = \vec{0},\ k = \overline{1, m},
\end{equation}
то в случае непрерывной дифференцируемости
\begin{equation}
    \label{lec_9.num_13}
    F = \left(F_1, F_2, \dots, F_m\right) \in \R^m
\end{equation}
по $u = \left(u_1, u_2, \dots, u_m\right) \in \R^m$,
то если якобиан
$I(x, u) = \det\pderiv{F\left(x, u\right)}{u}$
рассматриваемой СФУ $F\left(x, u\right) = \vec{0}$
в точке $\left(x_0, u_0\right)$ удовлетворяет условию
\begin{equation}
    \label{lec_9.num_14}
    I(x_0, u_0) \neq 0,
\end{equation}
тогда система \eqref{lec_9.num_7} будет иметь в 
соответствующих окрестностях точек $x_0\ \text{и}\ u_0$
единственное решение, т.е. $\exists!u = u(x)$ удовлетворяющая начальному
 условию $u(x_0) = u_0$.
\end{thm}
\begin{thm}[об интегрируемости непрерывных ФНП]
	Если $f \in C(D)$, т.~е. $f$ непрерывна на измеримом компакте
	 $D \subset \R^n$, то $f \in R(D)$,
	т.~е. интегрируема на $D$.
\end{thm}
\begin{crl*}[теорема о среднем для ФНП]
	Пусть $f \in C(D)$, а $g \subset R(D)$, где $D \subset \R^n$~---
	 измеримый компакт. Если функция $g(x)$ сохраняет один и 
	 тот же знак $\forall x \in D $, то тогда $\exists 
	 x_0 \in D \subset \R^n \implies$
	
	\begin{equation}
	\label{integral-formula}
	\int\limits_D f(x)g(x)dx =
	f(x_0)\int\limits_Dg(x)dx.
	\end{equation}
\end{crl*}

\end{document}